%\documentclass{article}
\documentclass[UTF8]{ctexart}
\usepackage{xeCJK}
\usepackage[margin=2cm]{geometry}
\usepackage[skins]{tcolorbox}
\usepackage{tikz}
\tcbuselibrary{breakable}
% \setmainfont{Times New Roman}
\usepackage{pgfplots}    % 绘图包
\usepackage{setspace}
\usepackage{fontawesome5} % 用于显示信息图标
\usepackage{multicol}
\usepackage{enumitem}
\usepackage{fontspec}



%\setCJKmainfont[
%    Path=/System/Library/Fonts/Supplemental/,
%    UprightFont=Songti.ttc,
%    BoldFont=Songti.ttc,
%    UprightFeatures={FontIndex=5},  % 把原来的 Bold 样式用于普通文本
%    BoldFeatures={FontIndex=2}      % 把原来的 Regular 样式用于加粗文本
%]{Songti SC}
%\setmainfont[
%    Path=/System/Library/Fonts/MSYH.TTC,
%    UprightFont=* W3,
%    BoldFont=* W6
%]{Microsoft YaHei UI}
%\setCJKmainfont{Hiragino Sans GB}
\setCJKmainfont{Microsoft YaHei UI}
\setCJKmonofont{Microsoft YaHei}
\setCJKsansfont{Microsoft YaHei}
\setmainfont{Microsoft YaHei}
%\setCJKmainfont{PingFang SC Light}
%\setmainfont{PingFangSC-Regular}
%\setmainfont{PingFang SC}
%\setCJKmainfont{Microsoft YaHei}
%\setCJKmainfont[Weight=Regular]{Microsoft YaHei}

% 定义颜色
\definecolor{boxborder}{RGB}{220, 220, 220} % 边框颜色
\definecolor{boxbg}{RGB}{248, 248, 248}     % 背景颜色
\definecolor{green}{RGB}{34, 139, 34}       % 绿色
\definecolor{blue}{RGB}{0, 102, 204}        % 蓝色
\definecolor{red}{RGB}{204, 0, 0}           % 红色
\definecolor{customred}{RGB}{255, 99, 71}

\definecolor{lightpurple}{RGB}{200, 162, 200} % 浅紫色
\definecolor{lightgreen}{RGB}{162, 200, 162} % 浅绿色

\definecolor{customTeal}{RGB}{0,178,180}
\definecolor{customTealBg}{RGB}{235,248,248}  % 创建专门的背景色
\definecolor{customTealBgDeep}{RGB}{200,230,230}
\definecolor{customSageGreen}{RGB}{189,236,182}

% 定义颜色
\definecolor{recommendgreen}{RGB}{92,184,92}
\definecolor{cautionred}{RGB}{217,83,79}
\definecolor{lightgreen}{RGB}{242,255,242}
\definecolor{lightred}{RGB}{255,242,242}

% 定义食物卡片环境
\newtcolorbox{greenbox}{
colback=lightgreen,
colframe=recommendgreen,
boxrule=0.5pt,
arc=5pt,
width=\linewidth
}

\newtcolorbox{redbox}{
colback=lightred,
colframe=cautionred,
boxrule=0.5pt,
arc=5pt,
width=\linewidth
}















\usetikzlibrary{shadows} % 添加shadows库
\usepackage{booktabs}
\usepackage{makecell}
\usepackage[table, dvipsnames]{xcolor}
\usepackage{multirow}
%\usepackage{tcolorbox}
\usepackage{longtable}
\usepackage{pdflscape}
\usepackage{colortbl}
\usepackage{caption}
\usepackage[UTF8]{ctex}
%\usepackage[dvipsnames]{xcolor}
\tcbuselibrary{raster}

\definecolor{customred}{HTML}{fd5c63}

\definecolor{customRed}{HTML}{FF6B6B}  % 可以换成其他暖色系颜色
\definecolor{customGreen}{HTML}{4CAF50}  % 一个适中的绿色

% 定义颜色
\definecolor{vitaminPurple}{RGB}{149,117,205} % 维生素紫色
\definecolor{mineralBlue}{RGB}{100,149,237} % 矿物质蓝色
\definecolor{aminoPink}{RGB}{219,112,147} % 氨基酸粉色
\definecolor{acidGreen}{RGB}{102,205,170} % 有机酸绿色

% 定义状态颜色
\definecolor{normalColor}{RGB}{16,185,129} % 正常-绿色
\definecolor{insufficientColor}{RGB}{245,158,11} % 不足-橙色
\definecolor{deficientColor}{RGB}{239,68,68} % 缺乏-红色
\definecolor{excessColor}{RGB}{139,92,246} % 过量-紫色

% 在导言区定义参考文献样式命令
\newcommand{\reference}[5]{%
    \vspace{-8pt}% 添加负的垂直间距
    \begin{tcolorbox}[
        enhanced,
        colback=customTealBg,
        colframe=customTealBg,
        arc=3mm,
        boxrule=0pt,
        width=\textwidth,
        top=1.5pt,    % 减小上边距
        bottom=1.5pt, % 减小下边距
        left=12pt,
        right=12pt
    ]
    \textcolor{black}{\textbf{\small #1. #2}}

    {\scriptsize \textcolor{blue!60!black}{#3} \textcolor{gray!40}{|} \textcolor{purple!70!black}{#4} \textcolor{gray!40}{|} \textcolor{teal!70!black}{(#5)}}
    \end{tcolorbox}
}

% 自定义营养卡片命令
\newcommand{\nutritioncard}[9]{%
% #1: 背景色
% #2: 分类名称中文
% #3: 总项数
% #4: 正常项数
% #5: 不足项数
% #6: 缺乏项数
% #7: 过量项数
% #8: 正常率
% #9: 具体项目列表
\begin{tikzpicture}
% 主背景卡片
\fill[#1, rounded corners=15pt] (0,0) rectangle (\textwidth,-2.8);

    % 左侧分类名称
    \node[text=white, font=\Large] at (1.5,-0.8) {#2};
    \node[text=white!80, font=\small] at (1.5,-1.4) {共 #3 项};

    % 中间状态列表
    \begin{scope}[shift={(3.5,0)}]
        % 正常状态
        \node[text=white, anchor=west] at (0,-0.4) {正常};
        \draw[white, line width=1pt] (1,-0.4) -- (2,-0.4);
        \draw[white!30, dotted] (2.2,-0.4) -- (9,-0.4);
        \node[text=white] at (9.5,-0.4) {#4 项};

        % 不足状态
        \node[text=white, anchor=west] at (0,-0.9) {不足};
        \draw[white, line width=1pt] (1,-0.9) -- (2,-0.9);
        \draw[white!30, dotted] (2.2,-0.9) -- (9,-0.9);
        \node[text=white] at (9.5,-0.9) {#5 项};

        % 缺乏状态
        \node[text=white, anchor=west] at (0,-1.4) {缺乏};
        \draw[white, line width=1pt] (1,-1.4) -- (2,-1.4);
        \draw[white!30, dotted] (2.2,-1.4) -- (9,-1.4);
        \node[text=white] at (9.5,-1.4) {#6 项};

        % 过量状态
        \node[text=white, anchor=west] at (0,-1.9) {过量};
        \draw[white, line width=1pt] (1,-1.9) -- (2,-1.9);
        \draw[white!30, dotted] (2.2,-1.9) -- (9,-1.9);
        \node[text=white] at (9.5,-1.9) {#7 项};
    \end{scope}

    % 右侧圆环
    \begin{scope}[shift={(\textwidth-2,-1.4)}]
        % 背景圆
        \draw[white!30, line width=2pt] (0,0) circle (1);
        % 进度圆弧
        \pgfmathsetmacro{\angle}{#8 * 3.6}
        \draw[white, line width=2pt] (1,0) arc (0:\angle:1);
        % 百分比文字
        \node[text=white, font=\Large] at (0,0) {#8\%};
        \node[text=white!80, font=\small] at (0,-0.4) {正常率};
    \end{scope}

    % 项目列表
    \node[text=white, align=left, font=\small] at (7,-2.4) {#9};
\end{tikzpicture}
\vspace{0.5em}
}

% 定义颜色
\definecolor{normalColor}{RGB}{67,160,71}    % 绿色
\definecolor{abnormalColor}{RGB}{211,47,47}  % 红色
\definecolor{coreColor}{RGB}{25,118,210}     % 蓝色
\definecolor{beneficialColor}{RGB}{76,175,80}% 绿色
\definecolor{harmfulColor}{RGB}{211,47,47}   % 红色

\usetikzlibrary{shadows.blur}
\usetikzlibrary{calc}  % 为了使用坐标计算

\definecolor{titleblue}{RGB}{49, 130, 206}
\definecolor{normalgreen}{RGB}{48, 140, 122}
\definecolor{abnormalred}{RGB}{229, 62, 62}

% 定义统一的box样式
\newtcolorbox{mybox}[1][]{
    enhanced,
    colback=white,
    colframe=gray!20,
    boxrule=0.5pt,
    arc=8pt,
    outer arc=8pt,
    drop fuzzy shadow,
    height=5cm,  % 修正: height fill 改为 height
    #1
}

% 定义颜色
\definecolor{pieGreen}{RGB}{144,198,149}    % 柔和的绿色
\definecolor{pieOrange}{RGB}{255,183,77}    % 温暖的橙色
\definecolor{pieRed}{RGB}{255,138,128}      % 柔和的红色

\definecolor{headerblue}{RGB}{75, 172, 198}

% 设置表格标题的颜色
\newcommand{\headercolor}{\cellcolor{headerblue}\color{white}}



% % 配色方案
\definecolor{primary}{RGB}{73, 175, 170}    % 主色调
\definecolor{success}{RGB}{46, 204, 113}    % 正常状态色
\definecolor{warning}{RGB}{231, 76, 60}     % 异常状态色
\definecolor{background}{RGB}{247, 248, 249} % 背景色

\pgfplotsset{compat=1.18}

% 定义圆角柱状图样式
\pgfplotsset{
    /pgfplots/bar cycle list/.style={/pgfplots/cycle list={
        {red!60,fill=red!60,mark=none},
        {cyan!60,fill=cyan!60,mark=none}
    }},
}

% 进度条命令
\newcommand{\healthbar}[4]{
\begin{tikzpicture}
    % 背景条
    \fill[gray!20] (0,0) rectangle (4,0.2);
    % 正常范围指示
    \fill[success!20] ({4*#4/#2},0) rectangle (4,0.2);
    % 进度条
    \fill[#3] (0,0) rectangle ({4*#1/#2},0.2);
    % 数值和参考范围
    \node[anchor=west] at (4.3,0.1) {\small #1 \ \ (#2)};
\end{tikzpicture}
}

% 状态标签命令
\newcommand{\statuslabel}[2]{
\begin{tcolorbox}[
    enhanced,
    colback=#2!10,
    colframe=#2!10,
    arc=1mm,
    boxrule=0pt,
    width=1.2cm,
    left=2pt,
    right=2pt,
    top=1pt,
    bottom=1pt
]
{\footnotesize\color{#2}\centering #1}
\end{tcolorbox}
}

\usepackage{array}

% 定义颜色
% \definecolor{primary}{RGB}{20,184,166}
% \definecolor{lightbg}{RGB}{240,253,250}

% 定义新的标题样式
\renewcommand{\maketitle}{%
    \begin{tcolorbox}[
        enhanced,
        colback=primary,
        colframe=primary,
        boxrule=0pt,
        arc=5pt,
        top=10pt,
        bottom=10pt
    ]
    \begin{center}
        \color{white}\Huge \textbf{宏基因组检测报告}
    \end{center}
    \end{tcolorbox}
}

% 定义信息框样式
\newcommand{\infobox}[2]{%
    \begin{tcolorbox}[
        enhanced,
        colback=lightbg,
        colframe=primary!20,
        boxrule=1pt,
        arc=5pt,
        title=#1,
        fonttitle=\bfseries
    ]
    #2
    \end{tcolorbox}
}



\usepackage{graphicx}
\usepackage{calc}
\usepackage{amsmath}

% 定义颜色
% \definecolor{primary}{RGB}{16, 185, 129}    % 绿色主色调
\definecolor{secondary}{RGB}{255, 255, 255}  % 浅灰背景色
% \definecolor{textgray}{RGB}{107, 114, 128}   % 次要文字颜色

% 自定义字体大小
\newcommand{\hugevalue}{\fontsize{32}{38}\selectfont}
\newcommand{\largevalue}{\fontsize{24}{29}\selectfont}

% 页面背景
\pagecolor{secondary}

% 定义新的标题样式
\renewcommand{\maketitle}{%
    \begin{tcolorbox}[
        enhanced,
        colback=white,
        colframe=white,
        boxrule=0pt,
        arc=15pt,
        left=0pt,
        right=0pt,
        top=20pt,
        bottom=20pt,
        width=\textwidth,
        shadow={2pt}{2pt}{10pt}{black!10},
    ]
        % 标题部分
        \begin{center}
%            \textcolor{customTeal}{\textbf{\fontsize{36}{43}\selectfont 宏基因组检测报告}}
            \textcolor{customTeal}{\fontsize{60}{48}\selectfont\textbf{宏基因组检测报告}}

            \vspace{1cm}
            % 分隔线
            \textcolor{customTeal}{\rule{\textwidth}{2pt}}  % 设置为2pt,可以根据需要调整粗细

            \vspace{1cm}
        \end{center}





        % 基本信息网格
        \begin{tabular*}{\textwidth}{@{\extracolsep{\fill}}ccccc@{}}
            \tcbox[
                enhanced,
                colback=customTealBg,
                colframe=customTealBg,
                boxrule=0pt,
                arc=5pt,
                top=8pt,
                bottom=8pt,
                left=12pt,
                right=12pt,
                width=0.19\textwidth
            ]{\begin{minipage}{0.17\textwidth}
                {\small\textcolor{textgray}{姓名}}\\
                {\large\textbf{张三}}
            \end{minipage}} &
            \tcbox[
                enhanced,
                colback=customTealBg,
                colframe=customTealBg,
                boxrule=0pt,
                arc=5pt,
                top=8pt,
                bottom=8pt,
                left=12pt,
                right=12pt,
                width=0.19\textwidth
            ]{\begin{minipage}{0.17\textwidth}
                {\small\textcolor{textgray}{年龄}}\\
                {\large\textbf{3岁0月}}
            \end{minipage}} &
            \tcbox[
                enhanced,
                colback=customTealBg,
                colframe=customTealBg,
                boxrule=0pt,
                arc=5pt,
                top=8pt,
                bottom=8pt,
                left=12pt,
                right=12pt,
                width=0.19\textwidth
            ]{\begin{minipage}{0.17\textwidth}
                {\small\textcolor{textgray}{样本编号}}\\
                {\large\textbf{64XXXXX09}}
            \end{minipage}} &
            \tcbox[
                enhanced,
                colback=customTealBg,
                colframe=customTealBg,
                boxrule=0pt,
                arc=5pt,
                top=8pt,
                bottom=8pt,
                left=12pt,
                right=12pt,
                width=0.19\textwidth
            ]{\begin{minipage}{0.17\textwidth}
                {\small\textcolor{textgray}{报告日期}}\\
                {\large\textbf{2024-04-05}}
            \end{minipage}}
%            &
%            \tcbox[
%                enhanced,
%                colback=primary!5,
%                colframe=primary!10,
%                boxrule=0pt,
%                arc=5pt,
%                top=8pt,
%                bottom=8pt,
%                left=12pt,
%                right=12pt,
%                width=0.19\textwidth
%            ]{\begin{minipage}{0.17\textwidth}
%                {\small\textcolor{textgray}{备注}}\\
%                {\large\textbf{-}}
%            \end{minipage}}
        \end{tabular*}
    \end{tcolorbox}
}

% 定义结果卡片样式
% 定义颜色
\definecolor{primary}{RGB}{16, 185, 129}    % 蓝绿主色调
\definecolor{cardbg}{RGB}{236, 253, 245}    % 浅薄荷绿背景色
\definecolor{textgray}{RGB}{107, 114, 128}   % 次要文字颜色
\definecolor{cardframe}{RGB}{209, 250, 229}  % 卡片边框色

\newcommand{\resultcard}[3]{%
    \begin{tcolorbox}[
        enhanced,
%        colback=cardbg,  % 使用浅薄荷绿背景
        colback=customTealBg!10,
        colframe=cardframe,  % 稍深的边框色
        boxrule=0pt,
        arc=12pt,
        top=15pt,
        bottom=15pt,
        left=20pt,
        right=20pt,
        shadow={2pt}{2pt}{10pt}{black!10},
        interior style={
%            top color=cardbg,
%            bottom color=cardbg!102,  % 略微深一点的底色
%            middle color=cardbg
            top color=customTealBg,
            bottom color=customTealBg,  % 略微深一点的底色
            middle color=customTealBg
        }
    ]
    {\Large\textbf{\textcolor{primary!80}{#1}}}
    \vspace{0.3cm}

    \begin{minipage}{\textwidth}
        {\hugevalue\textcolor{customTeal}{#2}}
        \vspace{0.2cm}

        {\normalsize\textcolor{textgray}{#3}}
    \end{minipage}
    \end{tcolorbox}
    \vspace{0.5cm}
}


% 在导言区添加命令定义
\newcommand{\bacteriaCard}[9]{%
    % #1: 菌门中文名 (e.g., 厚壁菌门)
    % #2: 菌门英文名 (e.g., Firmicutes)
    % #3: 状态标签 (e.g., 超标)
    % #4: 检测丰度值 (e.g., 94.36370\%)
    % #5: 正常范围 (e.g., 28.594\%-83.961\%)
    % #6: 人群百分位 (e.g., 90\%)
    % #7: 肥胖相关性 (e.g., 正相关)
    % #8: 相关性星级 (e.g., \faStar\hspace{0.1em}\faStar\hspace{0.1em}\faStar)
    % #9: 描述文本
\begin{center}
\begin{tcolorbox}[
    enhanced,
    colback=red!80!white!5,
    colframe=red!80!white!5,
    arc=3mm,
    boxrule=0.8pt,
    width=1.05\textwidth,
    top=8pt,
    bottom=8pt,
    center,
    nobeforeafter
]
\setstretch{0.85}
\large \textbf{\textit{#1}} \textcolor{gray}{\small \textsc{#2}}
\hspace{0.1em}\hfill
\raisebox{0.25ex}{
\tcbox[colback=customRed, colframe=green!70!black, arc=1.5mm,
       boxrule=0pt, left=1.5mm, right=1.5mm, top=0.1mm, bottom=0.1mm,
       rounded corners=south, on line]
{\small\textcolor{white}{#3}}
}

\vspace{0.5em}

\raisebox{0pt}{\scriptsize\tcbox[
    enhanced,
    frame hidden,
    interior hidden,
    boxrule=0pt,
    arc=15pt,
    boxsep=0pt,
    left=0pt,
    right=0pt,
    top=0pt,
    bottom=0pt,
    height=1.5em
]{%
\colorbox{customTeal}{%
    \strut\hspace{0.6em}\textcolor{white}{\raisebox{-0.1em}{\tiny\faBacteria}\hspace{0.2em}检测丰度}\hspace{0.6em}%
}%
\colorbox{gray!50}{%
    \strut\hspace{0.4em}\textcolor{white}{#4 \textcolor{red}{$\uparrow$}}\hspace{0.4em}%
}%
}}\hfill
\raisebox{0pt}{\scriptsize\tcbox[
    enhanced,
    frame hidden,
    interior hidden,
    boxrule=0pt,
    arc=15pt,
    boxsep=0pt,
    left=0pt,
    right=0pt,
    top=0pt,
    bottom=0pt,
    height=1.5em
]{%
\colorbox{customTeal}{%
    \strut\hspace{0.6em}\textcolor{white}{\raisebox{-0.1em}{\tiny\faChartBar}\hspace{0.2em}正常范围}\hspace{0.6em}%
}%
\colorbox{gray!50}{%
    \strut\hspace{0.4em}\textcolor{white}{#5}\hspace{0.4em}%
}%
}}\hfill
\raisebox{0pt}{\scriptsize\tcbox[
    enhanced,
    frame hidden,
    interior hidden,
    boxrule=0pt,
    arc=15pt,
    boxsep=0pt,
    left=0pt,
    right=0pt,
    top=0pt,
    bottom=0pt,
    height=1.5em
]{%
\colorbox{customTeal}{%
    \strut\hspace{0.6em}\textcolor{white}{\raisebox{-0.1em}{\tiny\faUsers}\hspace{0.2em}人群百分位}\hspace{0.6em}%
}%
\colorbox{gray!50}{%
    \strut\hspace{0.4em}\textcolor{white}{#6}\hspace{0.4em}%
}%
}}\hfill
\raisebox{0pt}{\scriptsize\tcbox[
    enhanced,
    frame hidden,
    interior hidden,
    boxrule=0pt,
    arc=15pt,
    boxsep=0pt,
    left=0pt,
    right=0pt,
    top=0pt,
    bottom=0pt,
    height=1.5em
]{%
\colorbox{customTeal}{%
    \strut\hspace{0.6em}\textcolor{white}{\raisebox{-0.1em}{\tiny\faUserCircle}\hspace{0.2em}肥胖相关}\hspace{0.6em}%
}%
\colorbox{gray!50}{%
    \strut\hspace{0.4em}\textcolor{white}{#7}\hspace{0.4em}%
}%
}}\hfill\raisebox{0pt}{\scriptsize\tcbox[
    enhanced,
    frame hidden,
    interior hidden,
    boxrule=0pt,
    arc=15pt,
    boxsep=0pt,
    left=0pt,
    right=0pt,
    top=0pt,
    bottom=0pt,
    height=1.5em
]{%
\colorbox{customTeal}{%
    \strut\hspace{0.6em}\textcolor{white}{\raisebox{-0.1em}{\tiny\faSearchPlus}\hspace{0.2em}相关性依据}\hspace{0.6em}%
}%
\colorbox{gray!50}{%
    \strut\hspace{0.4em}\textcolor{white}{\tiny #8}\hspace{0.4em}%
}%
}}

\vspace{0.8em}
{\footnotesize
#9
}
\end{tcolorbox}
\end{center}
}


\definecolor{softBg}{HTML}{F8FAFC}
\definecolor{softHeader}{HTML}{334155}
\definecolor{softNormal}{HTML}{10B981}
\definecolor{softMetric}{HTML}{475569}
\definecolor{softValue}{HTML}{64748B}

%% 定义新的颜色
%\definecolor{softBg}{HTML}{F8FAFC}
%\definecolor{softHeader}{HTML}{334155}
%\definecolor{softNormal}{HTML}{10B981}
%\definecolor{softWarning}{HTML}{EF4444}
%\definecolor{softMetric}{HTML}{475569}
%\definecolor{softValue}{HTML}{64748B}
%\definecolor{softText}{HTML}{1E293B}

% 定义医疗简约风配色
\definecolor{medBg}{HTML}{FAFAFA}
\definecolor{medHeader}{HTML}{0F766E}
\definecolor{medNormal}{HTML}{059669}
\definecolor{medWarning}{HTML}{DC2626}
\definecolor{medMetric}{HTML}{0D9488}
\definecolor{medValue}{HTML}{2DD4BF}
\definecolor{medText}{HTML}{334155}

\newcommand{\bacteriaCardNormal}[9]{%
    % #1: 菌门中文名 (e.g., 厚壁菌门)
    % #2: 菌门英文名 (e.g., Firmicutes)
    % #3: 状态标签 (e.g., 超标)
    % #4: 检测丰度值 (e.g., 94.36370\%)
    % #5: 正常范围 (e.g., 28.594\%-83.961\%)
    % #6: 人群百分位 (e.g., 90\%)
    % #7: 肥胖相关性 (e.g., 正相关)
    % #8: 相关性星级 (e.g., \faStar\hspace{0.1em}\faStar\hspace{0.1em}\faStar)
    % #9: 描述文本
\begin{center}
\begin{tcolorbox}[
    enhanced,
    colback=medBg,
    colframe=medHeader!20,
    arc=3mm,
    boxrule=0.8pt,
    width=\textwidth,
    top=8pt,
    bottom=8pt,
    center,
    nobeforeafter
]
\setstretch{0.75}
\large \textbf{\textit{#1}} \textcolor{medHeader!80}{\small \textsc{#2}}
\hspace{0.1em}\hfill
\raisebox{0.25ex}{
\tcbox[
    colback=\ifnum\pdfstrcmp{#3}{正常}=0 medNormal\else medWarning\fi,
    colframe=\ifnum\pdfstrcmp{#3}{正常}=0 medNormal!70!black\else medWarning!70!black\fi,
    arc=1.5mm,
    boxrule=0pt,
    left=1.5mm,
    right=1.5mm,
    top=0.1mm,
    bottom=0.1mm,
    rounded corners=south,
    on line
]
{\small\textcolor{white}{#3}}
}

\vspace{0.5em}

\raisebox{0pt}{\scriptsize\tcbox[
    enhanced,
    frame hidden,
    interior hidden,
    boxrule=0pt,
    arc=15pt,
    boxsep=0pt,
    left=0pt,
    right=0pt,
    top=0pt,
    bottom=0pt,
    height=1.5em
]{%
\colorbox{medMetric}{%
    \strut\hspace{0.6em}\textcolor{white}{\raisebox{-0.1em}{\tiny\faBacteria}\hspace{0.2em}检测丰度}\hspace{0.6em}%
}%
\colorbox{medValue}{%
    \strut\hspace{0.4em}\textcolor{white}{#4}\hspace{0.4em}%
}%
}}\hfill
\raisebox{0pt}{\scriptsize\tcbox[
    enhanced,
    frame hidden,
    interior hidden,
    boxrule=0pt,
    arc=15pt,
    boxsep=0pt,
    left=0pt,
    right=0pt,
    top=0pt,
    bottom=0pt,
    height=1.5em
]{%
\colorbox{medMetric}{%
    \strut\hspace{0.6em}\textcolor{white}{\raisebox{-0.1em}{\tiny\faChartBar}\hspace{0.2em}正常范围}\hspace{0.6em}%
}%
\colorbox{medValue}{%
    \strut\hspace{0.4em}\textcolor{white}{#5}\hspace{0.4em}%
}%
}}\hfill
\raisebox{0pt}{\scriptsize\tcbox[
    enhanced,
    frame hidden,
    interior hidden,
    boxrule=0pt,
    arc=15pt,
    boxsep=0pt,
    left=0pt,
    right=0pt,
    top=0pt,
    bottom=0pt,
    height=1.5em
]{%
\colorbox{medMetric}{%
    \strut\hspace{0.6em}\textcolor{white}{\raisebox{-0.1em}{\tiny\faUsers}\hspace{0.2em}人群百分位}\hspace{0.6em}%
}%
\colorbox{medValue}{%
    \strut\hspace{0.4em}\textcolor{white}{#6}\hspace{0.4em}%
}%
}}\hfill
\raisebox{0pt}{\scriptsize\tcbox[
    enhanced,
    frame hidden,
    interior hidden,
    boxrule=0pt,
    arc=15pt,
    boxsep=0pt,
    left=0pt,
    right=0pt,
    top=0pt,
    bottom=0pt,
    height=1.5em
]{%
\colorbox{medMetric}{%
    \strut\hspace{0.6em}\textcolor{white}{\raisebox{-0.1em}{\tiny\faUserCircle}\hspace{0.2em}肥胖相关}\hspace{0.6em}%
}%
\colorbox{medValue}{%
    \strut\hspace{0.4em}\textcolor{white}{#7}\hspace{0.4em}%
}%
}}\hfill\raisebox{0pt}{\scriptsize\tcbox[
    enhanced,
    frame hidden,
    interior hidden,
    boxrule=0pt,
    arc=15pt,
    boxsep=0pt,
    left=0pt,
    right=0pt,
    top=0pt,
    bottom=0pt,
    height=1.5em
]{%
\colorbox{medMetric}{%
    \strut\hspace{0.6em}\textcolor{white}{\raisebox{-0.1em}{\tiny\faSearchPlus}\hspace{0.2em}相关性依据}\hspace{0.6em}%
}%
\colorbox{medValue}{%
    \strut\hspace{0.4em}\textcolor{white}{\tiny #8}\hspace{0.4em}%
}%
}}

\vspace{0.8em}
{\footnotesize\textcolor{medText}{#9}}
\end{tcolorbox}
\end{center}
}

\begin{document}



\newpage






\maketitle

% 健康总分
\resultcard{\textcolor{customTeal}{健康总分}}{\textbf{64}}{健康总分64分,处于亚健康及营养饮食不合理状态,表明目前健康状况处于中等偏下水平,主要原因是饮食营养不均衡,需要通过调整饮食结构和生活习惯来改善健康状况。}

% 肠道年龄
\resultcard{\textcolor{customTeal}{肠道年龄}}{\textcolor{customTeal}{\textbf{5.47岁}}}{当肠道菌群反映的年龄大幅偏离真实年龄时通常代表您的肠道菌群出现了紊乱、存在疾病风险或发育滞后}

% 肠型
\resultcard{\textcolor{customTeal}{肠型}}{\textcolor{customTeal}{\textbf{普雷沃氏菌型}}}{肠型为“普雷沃氏菌”,意味着在您的肠道菌群中,普雷沃氏菌是主要优势菌。这种肠型通常与高碳水化合物、高纤维的饮食方式相关,常见于偏好植物性食物的人群中。}

\newpage

\begin{tcolorbox}[
    enhanced,
    colback=white,
    colframe=customTeal,
    arc=2mm,
    boxrule=1pt,
    left=20pt,
    right=20pt,
    top=12pt,
    bottom=12pt,
    width=\textwidth,
    fontupper=\sffamily,
    overlay={
    \draw[customTeal, line width=2pt]
    ([xshift=15pt]frame.south west) -- ([xshift=-15pt]frame.south east);
    }
]
{\textcolor{customTeal}{\Huge \textbf{宏基因组报告阅读指南}}}
\end{tcolorbox}

\begin{tcolorbox}[
    enhanced,
    colback=customTealBg,
    colframe=gray!5,
    arc=3mm,
    boxrule=0pt,
    width=\textwidth,
    top=8pt,
    bottom=8pt
]
{\normalsize{\color{customTeal}
\faInfoCircle} 本宏基因组报告旨在提供对您肠道中微生物群落的全面分析与理解。以下是一些基本概念技术解析和注意事项。
}
\end{tcolorbox}

\begin{tcolorbox}[
    enhanced,
    colback=lightpurple!10, % 卡片底色
    colframe=white,  % 边框颜色
    arc=3mm,
    boxrule=0.5pt,
    width=\textwidth,
    top=8pt,
    bottom=8pt
]
{\normalsize{\color{lightpurple}\faQuestionCircle}\quad \textbf{什么是肠道菌群的宏基因组?和16s有什么区别?}\\
{\color{orange!50}\faComments}\quad 肠道菌群的宏基因组是指从肠道微生物群体中提取的所有微生物(如细菌、古菌、真菌和病毒等)的遗传物质(DNA)进行分析和研究的领域。宏基因组报告提供更为详尽和全面的微生物分析,可以揭示微生物的功能和相互作用,而16S rRNA报告则更侧重于分类学上的信息,适合用作微生物群落的初步分析。
}
\end{tcolorbox}

\begin{tcolorbox}[
    enhanced,
    colback=lightpurple!10, % 卡片底色
    colframe=white,  % 边框颜色
    arc=3mm,
    boxrule=0.5pt,
    width=\textwidth,
    top=8pt,
    bottom=8pt
]
{\normalsize{\color{orange}\faExclamationTriangle}
\textbf{特别注意}:本报告采用高通量测序对肠道菌群进行宏基因组检测,然后使用大数据和人工智能技术对各项指标进行评估,以下是您在阅读报告时要注意的事项:
\begin{itemize}

    \item \textbf{结果的解读并非绝对}:
    \begin{itemize}
        \item 报告中的数据和分析结果只是对您肠道菌群的一个快照,它们可能会受到多种因素的影响,如饮食、生活方式、地理环境等。因此,请保持对结果的审慎态度,不要将其视为绝对的健康指标。
    \end{itemize}
    \item \textbf{个体差异}:
    \begin{itemize}
        \item 肠道菌群的构成因人而异,个体差异可能会导致相同的微生物组成在不同人群中的健康影响截然不同。因此,报告中某些相关性的普遍性可能无法适用于每个人。
    \end{itemize}
    \item \textbf{技术和方法的限制}:
    \begin{itemize}
        \item 尽管高通量测序和人工智能技术能提供强大的分析能力,但这些技术也有局限性。例如,某些微生物可能在样本处理中丢失,或存在序列的拼接错误,这可能导致结果的不准确性或偏差。
    \end{itemize}
    \item \textbf{功能预测的局限性}:
    \begin{itemize}
        \item 报告中对微生物功能的预测基于已有数据库和算法,这些预测并不一定能准确反映微生物的实际生理功能。请谨慎对待这些预测,尤其是在临床决策时。
    \end{itemize}
    \item \textbf{临床相关性并不确定}:
    \begin{itemize}
        \item 报告可能会提到与某些健康状况的相关性,但关联并不等于因果关系。个别结果需结合您的整体健康状况和其他临床因素进行判断,而不是单一指标决定健康状况。
    \end{itemize}
    \item \textbf{避免过度解读}:
    \begin{itemize}
        \item 有些微小的变化可能并不具有临床意义,因此请避免对微小差异进行过度解读。务必要将结果视为整体趋势,而非孤立的指标。
    \end{itemize}
    \item \textbf{跟踪研究的重要性}:
    \begin{itemize}
        \item 肠道菌群是动态的,定期监测和跟踪会提供更好的健康状态评估。单次测试结果的意义有限,尤其是在没有长期数据对比的情况下。
    \end{itemize}
\end{itemize}
}
\end{tcolorbox}

\newpage

\begin{tcolorbox}[
    enhanced,
    colback=white,
    colframe=customTeal,
    arc=2mm,
    boxrule=1pt,
    left=20pt,
    right=20pt,
    top=12pt,
    bottom=12pt,
    width=\textwidth,
    fontupper=\sffamily,
    overlay={
    \draw[customTeal, line width=2pt]
    ([xshift=15pt]frame.south west) -- ([xshift=-15pt]frame.south east);
    }
]
$\fontsize{30}{31}\selectfont\textbf{肠道菌群整体评估}$
\end{tcolorbox}

\begin{tcolorbox}[
    enhanced,
    colback=customTealBg,
    colframe=gray!5,
    arc=3mm,
    boxrule=0pt,
    width=\textwidth,
    top=8pt,
    bottom=8pt
]
{\small{\color{customTeal}\faInfoCircle} 本部分总结了您本次肠道菌群检测的各项整体关键指标,主要包含三个核心维度:
\begin{itemize}
    \item \textbf{肠道基础功能}: 评估肠道的基础生理功能,包括蛋白质发酵能力、消化吸收效率、肠道产气情况、肠道屏障完整性以及肠道炎症状态。

    \item \textbf{代谢相关指标}: 反映肠道菌群的代谢活性,评估包括代谢健康水平和特定物质(如草酸盐)的代谢能力。

    \item \textbf{菌群分析}: 全面评估肠道菌群的整体状况和菌群微环境状况:
    \begin{itemize}
        \item \textbf{菌群整体评估}:评估包括菌群数量、多样性、群落平衡性以及菌群的恢复能力等关键特征。
        \item \textbf{菌群微环境评估}:评估包含有益菌,有害菌,革兰氏阴性菌和好氧菌具有代表意义的菌群评估。
    \end{itemize}
\end{itemize}
}
\end{tcolorbox}

\begin{tcolorbox}[
    enhanced,
    colback=lightpurple!10, % 卡片底色
    colframe=white,  % 边框颜色
    arc=3mm,
    boxrule=0.5pt,
    width=\textwidth,
    top=8pt,
    bottom=8pt
]
{\small{\color{lightpurple}\faQuestionCircle}\quad \textbf{健康的肠道菌群整体是怎么样的?}\\
{\color{orange!50}\faComments}\quad 健康的肠道菌群应当具备以下特征:
\begin{itemize}
    \item \textbf{菌群构成合理}:健康的肠道菌群包含多种有益的细菌,它们能够在相互配合中发挥协同作用,促进消化与营养吸收。这种合理的构成确保了肠道功能的正常运作,并可以抵御外来病原体的侵袭。
    \item \textbf{物种多样性丰富}:肠道菌群的多样性是健康的重要指标。一种多样化的菌群能有效增加代谢产物的种类,为宿主提供多种营养,增强免疫系统的功能。同时,物种多样性能够提高耐受性,使肠道更能适应不同的饮食和环境变化,从而减少疾病发生的风险。
    \item \textbf{有益菌占优,有害菌较少}:健康的肠道菌群中,有益菌的数量通常占主导,能够抑制病原菌的生长,并促进良好的代谢活动。有害菌的数量相对较少,避免了对肠道和宿主健康的负面影响。这种平衡状态确保了肠道的健康和整体营养的吸收。
\end{itemize}
}
\end{tcolorbox}

\definecolor{recommendgreen}{RGB}{60,179,113}

\begin{center}
\begin{tikzpicture}
    % 定义基础尺寸
    \def\totalw{\textwidth}         % 总宽度
    \def\colw{0.49\textwidth}       % 列宽
    \def\gap{0.01\textwidth}        % 间隔

    % 左侧上方卡片
    \draw[gray!50, rounded corners] (0,3.7) rectangle (\colw,10);

    \node[customTeal, font=\large, anchor=north west] at (0.3,9.8) {\textbf{肠道基础功能}};

    % 代谢健康
    \node[anchor=west] at (0.3,8.8) {\small 蛋白发酵};
    \draw[draw=none, fill=gray!20, rounded corners=3pt] (0.4,8.3) rectangle (\colw-35,8.5);
    \draw[draw=none, fill=green!40, rounded corners=3pt] (0.4,8.3) rectangle (0.6,8.5);
    \node[anchor=west] at (\colw-30,8.4) {\footnotesize 85\%};

    % 草酸盐代谢
    \node[anchor=west] at (0.3,7.8) {\small 消化效率};
    \draw[draw=none, fill=gray!20, rounded corners=3pt] (0.4,7.3) rectangle (\colw-35,7.5);
    \draw[draw=none, fill=green!40, rounded corners=3pt] (0.4,7.3) rectangle (\colw-32,7.5);
    \node[anchor=west] at (\colw-30,7.4) {\footnotesize 75\%};

    % 代谢健康
    \node[anchor=west] at (0.3,6.8) {\small 肠道产气};
    \draw[draw=none, fill=gray!20, rounded corners=3pt] (0.4,6.3) rectangle (\colw-35,6.5);
    \draw[draw=none, fill=green!40, rounded corners=3pt] (0.4,6.3) rectangle (2.8,6.5);
    \node[anchor=west] at (\colw-30,6.4) {\footnotesize 85\%};

    % 草酸盐代谢
    \node[anchor=west] at (0.3,5.8) {\small 肠道屏障};
    \draw[draw=none, fill=gray!20, rounded corners=3pt] (0.4,5.3) rectangle (\colw-35,5.5);
    \draw[draw=none, fill=green!40, rounded corners=3pt] (0.4,5.3) rectangle (1.6,5.5);
    \node[anchor=west] at (\colw-30,5.4) {\footnotesize 75\%};

    % 草酸盐代谢
    \node[anchor=west] at (0.3,4.8) {\small 肠道炎症};
    \draw[draw=none, fill=gray!20, rounded corners=3pt] (0.4,4.3) rectangle (\colw-35,4.5);
    \draw[draw=none, fill=green!40, rounded corners=3pt] (0.4,4.3) rectangle (1.4,4.5);
    \node[anchor=west] at (\colw-30,4.4) {\footnotesize 75\%};



    % 左侧下方卡片
    \draw[gray!50, rounded corners] (0,0) rectangle (\colw,3.5);
    \node[customTeal, font=\large, anchor=north west] at (0.3,3.2) {\textbf{代谢相关指标}};

    % 代谢健康
    \node[anchor=west] at (0.3,2.1) {\small 代谢健康};
    \draw[draw=none, fill=gray!20, rounded corners=3pt] (0.4,1.5) rectangle (\colw-35,1.7);
    % 将4改为一个非常接近起始点的值,比如0.4加上总长度的1%
    \draw[draw=none, fill=red!40, rounded corners=3pt] (0.4,1.5) rectangle (0.55,1.7);
    \node[anchor=west] at (\colw-30,1.6) {\footnotesize 1\%};

    % 草酸盐代谢
    \node[anchor=west] at (0.3,1.0) {\small 草酸盐代谢};
    \draw[draw=none, fill=gray!20, rounded corners=3pt] (0.4,0.4) rectangle (\colw-35,0.6);
    \draw[draw=none, fill=red!40, rounded corners=3pt] (0.4,0.4) rectangle (0.55,0.6);
    \node[anchor=west] at (\colw-30,0.5) {\footnotesize 1\%};

    % 右侧卡片
    \draw[gray!50, rounded corners] (\colw+\gap,0) rectangle (\colw+\gap+\colw,10);
    \node[customTeal, font=\large, anchor=north west] at (\colw+\gap+8,9.8) {\textbf{菌群分析}};

    \node[anchor=west] at (\colw+\gap+8,8.8) {\small 菌种数量};
    \draw[draw=none, fill=gray!20, rounded corners=3pt] (\colw+\gap+12,8.3) rectangle (\colw+\gap+\colw-35,8.5);
    \draw[draw=none, fill=red!40, rounded corners=3pt] (\colw+\gap+12,8.3) rectangle (\colw+\gap+\colw-35,8.5);
    \node[anchor=west] at (\colw+\gap+\colw-30,8.4) {\footnotesize 120\%};

    \node[anchor=west] at (\colw+\gap+8,7.8) {\small 菌群多样性};
    \draw[draw=none, fill=gray!20, rounded corners=3pt] (\colw+\gap+12,7.3) rectangle (\colw+\gap+\colw-35,7.5);
    \draw[draw=none, fill=green!40, rounded corners=3pt] (\colw+\gap+12,7.3) rectangle (\colw+\gap+40,7.5);
    \node[anchor=west] at (\colw+\gap+\colw-30,7.4) {\footnotesize 37\%};

    \node[anchor=west] at (\colw+\gap+8,6.8) {\small 肠道菌群平衡};
    \draw[draw=none, fill=gray!20, rounded corners=3pt] (\colw+\gap+12,6.3) rectangle (\colw+\gap+\colw-35,6.5);
    \draw[draw=none, fill=green!40, rounded corners=3pt] (\colw+\gap+12,6.3) rectangle (\colw+\gap+\colw-45,6.5);
    \node[anchor=west] at (\colw+\gap+\colw-30,6.4) {\footnotesize 94\%};

    \node[anchor=west] at (\colw+\gap+8,5.8) {\small 菌群恢复力};
    \draw[draw=none, fill=gray!20, rounded corners=3pt] (\colw+\gap+12,5.3) rectangle (\colw+\gap+\colw-35,5.5);
    \draw[draw=none, fill=red!40, rounded corners=3pt] (\colw+\gap+12,5.3) rectangle (\colw+\gap+20,5.5);
    \node[anchor=west] at (\colw+\gap+\colw-30,5.4) {\footnotesize 8\%};

    \node[anchor=west] at (\colw+\gap+8,4.8) {\small 有益菌};
    \draw[draw=none, fill=gray!20, rounded corners=3pt] (\colw+\gap+12,4.3) rectangle (\colw+\gap+\colw-35,4.5);
    \draw[draw=none, fill=green!40, rounded corners=3pt] (\colw+\gap+12,4.3) rectangle (\colw+\gap+\colw-45,4.5);
    \node[anchor=west] at (\colw+\gap+\colw-30,4.4) {\footnotesize 94\%};

    \node[anchor=west] at (\colw+\gap+8,3.8) {\small 有害菌};
    \draw[draw=none, fill=gray!20, rounded corners=3pt] (\colw+\gap+12,3.3) rectangle (\colw+\gap+\colw-35,3.5);
    \draw[draw=none, fill=green!40, rounded corners=3pt] (\colw+\gap+12,3.3) rectangle (\colw+\gap+\colw-80,3.5);
    \node[anchor=west] at (\colw+\gap+\colw-30,3.4) {\footnotesize 40\%};

    \node[anchor=west] at (\colw+\gap+8,2.8) {\small 革兰氏阴性菌};
    \draw[draw=none, fill=gray!20, rounded corners=3pt] (\colw+\gap+12,2.3) rectangle (\colw+\gap+\colw-35,2.5);
%    \draw[draw=none, fill=red!40, rounded corners=3pt] (\colw+\gap+12,2.3) rectangle (\colw+\gap,2.5);
    \node[anchor=west] at (\colw+\gap+\colw-30,2.4) {\footnotesize 0\%};

    \node[anchor=west] at (\colw+\gap+8,1.8) {\small 好氧菌};
    \draw[draw=none, fill=gray!20, rounded corners=3pt] (\colw+\gap+12,1.3) rectangle (\colw+\gap+\colw-35,1.5);
%    \draw[draw=none, fill=red!40, rounded corners=3pt] (\colw+\gap+12,1.3) rectangle (\colw+\gap+10,1.5);
    \node[anchor=west] at (\colw+\gap+\colw-30,1.4) {\footnotesize 0\%};

\end{tikzpicture}
\end{center}

\newpage

\begin{tcolorbox}[
    enhanced,
    colback=white,
    colframe=white,
    boxrule=0pt,
%    rounded corners=6pt, % 设置圆角半径
    width=\textwidth,
    left=0pt,
    right=0pt,
    top=0pt,
    bottom=25pt
]
    \begin{tikzpicture}
        % 主背景
        \fill[customTeal] (0,0) rectangle (\textwidth,1.5);
        % 装饰色块
        \fill[customTeal!70] (0,0) rectangle (1,1.5);

        % 标题文字
        \node[anchor=west, text=white] at (1.5,0.75) {
            \fontsize{20}{31}\selectfont\textbf{肠道基础功能评估}
        };

        % 右侧图标
        \node[anchor=east, text=white] at (\textwidth-0.5,0.75) {
            \Large\faDna
        };
    \end{tikzpicture}
\end{tcolorbox}

\vspace{-1.0cm}

\begin{tcolorbox}[
    enhanced,
    colback=customTealBg,
    colframe=gray!3,
    arc=3mm,
    boxrule=0pt,
    width=\textwidth,
    top=8pt,
    bottom=8pt
]
{\small{\color{customTeal}\faInfoCircle} 肠道基础功能是反映消化系统健康状况的重要指标,包括消化吸收、代谢转化和防御屏障等核心功能。通过这些指标的评估,可以全面了解肠道的工作状态和功能完整性。
}
\end{tcolorbox}

\begin{tcolorbox}[
    enhanced,
    colback=lightpurple!10, % 卡片底色
    colframe=white,  % 边框颜色
    arc=3mm,
    boxrule=0.5pt,
    width=\textwidth,
    top=8pt,
    bottom=8pt
]
{\small{\color{lightpurple}\faQuestionCircle}\quad \textbf{肠道基础功能评估有什么临床意义?}\\
{\color{orange!50}\faComments}\quad 肠道基础功能评估的临床意义包括但不限于以下方面:
\begin{itemize}
    \item 蛋白发酵:评估蛋白质在肠道内的代谢过程及其产物的影响可以指导优化蛋白质摄取,调节饮食结构。
    \item 消化效率:反映机体消化吸收能力,可以指导饮食调整(如低脂饮食)或酶补充治疗。
    \item 肠道产气:评估微生物发酵代谢导致气体生成情况,识别肠道菌群异常与气体代谢问题。
    \item 肠道屏障:反映肠道通透性及防御功能,可以提示全身性炎症和代谢紊乱(如肥胖、糖尿病)。
    \item 肠道炎症:评估肠道免疫系统活性和炎症水平,诊断炎症性疾病,动态监测治疗效果。
\end{itemize}
}
\end{tcolorbox}
\vspace{-0.7cm}
\begin{center}
\begin{tikzpicture}[
    font=\small,
    title/.style={font=\small\bfseries\color{white}},
    value/.style={font=\small},
    reference/.style={font=\small},
    cell/.style={anchor=west, text width=5.0cm},
    note/.style={anchor=west, text width=4.5cm, align=left}
]
    \def\cardwidth{\textwidth}
    \def\cardheight{9.85}
    \def\barheight{0.25}
    \def\barwidth{1.5}
    \def\valuebarspace{0.4}

    % 容器和标题栏背景
    \draw[rounded corners=5, fill=white, draw=gray!20]
        (0,0) rectangle (\cardwidth,-\cardheight);
    \path[fill=customTeal]
        (0,0) [rounded corners=5] -- (\cardwidth,0) --
        (\cardwidth,0.8) -- (0,0.8) -- cycle;

    % 表头
    \node[title, anchor=west] at (0.5,0.4) {\textbf{菌种名称}};
    \node[title] at (7, 0.4) {\textbf{正常范围}};
    \node[title] at (11.5, 0.4) {\textbf{检测丰度}};
    \node[title] at (16, 0.4) {\textbf{结果评价}};

    % 初始化位置计数器
    \def\currentpos{0.25}

    % 数据行和卡片
    \foreach \item/\enitem/\value/\range/\percentile/\detection/\status/\intro/\suggestion/\index/\category in {
        {蛋白发酵}/{Protein fermentation}/3/{0-70}/67\%/99.52\%/正常/{反映肠道菌群对蛋白质的分解能力,影响氨基酸的吸收和利用效率。}/{}/\currentpos/{噬菌体},
        {消化效率}/{Digestive Efficiency}/99/{25-100}/2\%/99.04\%/正常/{评估肠道对各类营养物质的消化吸收能力,包括碳水化合物、脂肪和蛋白质的处理效率。}/{}/\currentpos/{噬菌体},
        {肠道产气}/{Intestinal Gas Production}/47/{0-70}/95\%/98.56\%/正常/{反映肠道菌群发酵过程的活跃程度,与肠道微生态平衡密切相关。}/{}/\currentpos/{噬菌体},
        {肠道屏障}/{Intestinal Barrier}/28/{25-100}/50\%/99.52\%/正常/{评估肠黏膜的完整性和防御功能,是阻止有害物质进入体内的重要屏障。}/{}/\currentpos/{植物性病毒},
        {肠道炎症}/{Intestinal Inflammation}/18/{0-75}/44\%/99.52\%/正常/{反映肠道免疫状态和炎症反应水平,是肠道健康的重要指标。}/{}/\currentpos/{内源性病毒}
    }
    {
        % 计算当前行的基础位置
        \pgfmathsetmacro{\basepos}{-2.8*\currentpos}

        % 菌种名称
        \node[cell, align=left] at (0.5,\basepos) {
            \small\textbf{\item}\\[-0.2em]
            {\color{gray}\small\enitem}
        };

        % 正常范围
        \node[reference] at (7,\basepos) {\footnotesize\range};

        % 进度条相关
        \pgfmathsetmacro{\barypos}{\basepos-\valuebarspace+0.1}
        \def\barstart{10.75}

        % 进度条背景
        \fill[gray!10, rounded corners=2] (\barstart,\barypos)
            rectangle (\barstart+\barwidth,\barypos+\barheight);

        % 检测丰度值
        \node[value] at (11.5, {\basepos-\valuebarspace+0.6}) {\footnotesize\value};

        % 解析范围并计算进度条长度
        \def\parserange#1-#2\endparse{\def\minval{#1}\def\maxval{#2}}
        \expandafter\parserange\range\endparse

        % 计算进度条长度和颜色
        \pgfmathsetmacro{\progress}{min(\value/\maxval, 1.0)}
        \pgfmathparse{\value > \maxval ? "customred" : (\value < \minval ? "customred" : "green!50")}
        \let\barcolor=\pgfmathresult

        % 进度条显示
        \ifnum\pdfstrcmp{\status}{超标}=0
            \fill[customred, rounded corners=2] (\barstart,\barypos)
                rectangle (\barstart+\barwidth,\barypos+\barheight);
        \else
            \fill[\barcolor, rounded corners=2] (\barstart,\barypos)
                rectangle (\barstart+\barwidth*\progress,\barypos+\barheight);
        \fi

        % 结果评价
        \ifnum\pdfstrcmp{\status}{超标}=0
            \node[value, text=customRed] at (16,\basepos) {\footnotesize\textbf{\status}};
        \else
            \node[value, text=customGreen] at (16,\basepos) {\footnotesize\textbf{\status}};
        \fi


        % 添加卡片
        \pgfmathsetmacro{\cardypos}{\basepos-0.5}
        \begin{scope}[shift={(0,\cardypos)}]
            % 卡片背景
            \pgfmathsetmacro{\cardheight}{
                \ifnum\pdfstrcmp{\status}{超标}=0
                    1.0  % 两行内容时的高度
                \else
                    0.6  % 一行内容时的高度
                \fi
            }

            \fill[rounded corners=5pt, customTeal!5, draw=gray!5]
                (0.3,-\cardheight) rectangle (17.3,0);

            % 菌群简介图标和内容
            \node[anchor=west] at (0.5,-0.3) {
                \textbf{\color{gray!90}\footnotesize \textcolor{customTeal}{\faInfoCircle}}
            };
            \node[anchor=west, text width=16cm] at (0.9,-0.3) {
                {\small\color{gray}\footnotesize \intro}
            };

            % 异常解读标题和内容
            \ifnum\pdfstrcmp{\status}{超标}=0
                \node[anchor=west] at (0.55,-0.8) {
                    \textbf{\color{customRed}\footnotesize \textcolor{customRed}{\faBell}}
                };
                \node[anchor=west, text width=16cm] at (0.9,-0.8) {
                    {\small\color{gray}\footnotesize \suggestion}
                };
            \fi
        \end{scope}

        % 分割线
        \pgfmathsetmacro{\linepos}{
            \ifnum\pdfstrcmp{\status}{超标}=0
                \basepos-1.7  % 超标时的分割线位置
            \else
                \basepos-1.3  % 正常时的分割线位置
            \fi
        }
        \draw[gray!20] (0.2,\linepos) -- (\cardwidth-0.2,\linepos);

        % 根据当前行的状态计算下一行的位置增量
        \ifnum\pdfstrcmp{\status}{超标}=0
            \pgfmathsetmacro{\increment}{0.85}  % 超标行(两行内容)需要更大的增量
        \else
            \pgfmathsetmacro{\increment}{0.7}  % 正常行(一行内容)使用较小的增量
        \fi

        % 更新位置计数器
        \pgfmathsetmacro{\nextpos}{\currentpos+\increment}
        \xdef\currentpos{\nextpos}
    }

    % 最后一行的处理,消除多余的空白
    \pgfmathsetmacro{\lastincrement}{2}  % 最后一行的增量
    \pgfmathsetmacro{\nextpos}{\currentpos+\lastincrement}
    \xdef\currentpos{\nextpos}

\end{tikzpicture}
\end{center}

\begin{tcolorbox}[
    enhanced,
    colback=gray!3,
    colframe=gray!3,
    arc=3mm,
    boxrule=0pt,
    width=\textwidth,
    top=8pt,
    bottom=8pt
]
{\small \textcolor{customGreen}{\faBell}\quad 所有指标的评估值都在参考范围之内。总体来看,您的消化吸收功能较强,肠道炎症水平较低,但需要关注蛋白发酵偏低和肠道屏障功能偏低的情况。建议在保持良好饮食习惯的同时,可以考虑适当补充有益菌,增强肠道菌群活性和屏障功能。
}
\end{tcolorbox}

\newpage

\begin{tcolorbox}[
    enhanced,
    colback=white,
    colframe=white,
    arc=2mm,
    boxrule=0pt,
    width=\textwidth,
    left=15pt,
    right=15pt,
    top=10pt,
    bottom=10pt,
    drop shadow={
        opacity=0.2,
        color=customTeal
    },
    borderline west={5pt}{0pt}{customTeal}
]
\textcolor{customTeal}{\Large\textbf{代谢功能评估}}
\end{tcolorbox}

\vspace{0.05cm}

\begin{tcolorbox}[
enhanced,
colback=customTealBg,
colframe=gray!3,
arc=3mm,
boxrule=0pt,
width=\textwidth,
top=8pt,
bottom=8pt
]
{\small{\color{customTeal}\faInfoCircle} 肠道代谢功能评估是反映消化系统代谢能力的重要指标组合,主要包括代谢健康和草酸盐代谢两个核心维度。通过这些指标的评估,可以深入了解肠道的代谢状态和功能完整性。
}
\end{tcolorbox}
\begin{tcolorbox}[
    enhanced,
    colback=lightpurple!10, % 卡片底色
    colframe=white,  % 边框颜色
    arc=3mm,
    boxrule=0.5pt,
    width=\textwidth,
    top=8pt,
    bottom=8pt
]
{\small{\color{lightpurple}\faQuestionCircle}\quad \textbf{代谢功能评估有什么临床意义?}\\
{\color{orange!50}\faComments}\quad 代谢功能评估的临床意义包括但不限于以下方面:
\begin{itemize}
    \item 代谢功能评估:通过指标判断肠道代谢效率和能量转化能力。
    \item 结石风险:评估草酸盐代谢情况,预测肾结石等疾病风险。
    \item 营养吸收:反映机体对营养物质的利用效率。
\end{itemize}
}
\end{tcolorbox}
\vspace{-0.7cm}
\begin{center}
\begin{tikzpicture}[
    font=\small,
    title/.style={font=\small\bfseries\color{white}},
    value/.style={font=\small},
    reference/.style={font=\small},
    cell/.style={anchor=west, text width=5.0cm},
    note/.style={anchor=west, text width=4.5cm, align=left}
]
    \def\cardwidth{\textwidth}
    \def\cardheight{4}
    \def\barheight{0.25}
    \def\barwidth{1.5}
    \def\valuebarspace{0.4}

    % 容器和标题栏背景
    \draw[rounded corners=5, fill=white, draw=gray!20]
        (0,0) rectangle (\cardwidth,-\cardheight);
    \path[fill=customTeal]
        (0,0) [rounded corners=5] -- (\cardwidth,0) --
        (\cardwidth,0.8) -- (0,0.8) -- cycle;

    % 表头
    \node[title, anchor=west] at (0.5,0.4) {\textbf{菌种名称}};
    \node[title] at (7, 0.4) {\textbf{正常范围}};
    \node[title] at (11.5, 0.4) {\textbf{检测丰度}};
    \node[title] at (16, 0.4) {\textbf{结果评价}};

    % 初始化位置计数器
    \def\currentpos{0.25}

    % 数据行和卡片
    \foreach \item/\enitem/\value/\range/\percentile/\detection/\status/\intro/\suggestion/\index/\category in {
        {代谢健康}/{Metabolic health}/1/{25-100}/67\%/99.52\%/缺乏/{反映肠道整体的代谢状态和效率,是评估机体能量代谢和营养物质转化的重要指标。}/{}/\currentpos/{噬菌体},
        {草酸盐代谢}/{Oxalate metabolism}/1/{25-100}/2\%/99.04\%/缺乏/{评估肠道对草酸盐的处理能力,与肾结石风险和钙吸收密切相关。}/{}/\currentpos/{噬菌体}
    }
    {
        % 计算当前行的基础位置
        \pgfmathsetmacro{\basepos}{-2.8*\currentpos}

        % 菌种名称
        \node[cell, align=left] at (0.5,\basepos) {
            \small\textbf{\item}\\[-0.2em]
            {\color{lightgray}\small\enitem}
        };

        % 正常范围
        \node[reference] at (7,\basepos) {\footnotesize\range};

        % 进度条相关
        \pgfmathsetmacro{\barypos}{\basepos-\valuebarspace+0.1}
        \def\barstart{10.75}

        % 进度条背景
        \fill[gray!10, rounded corners=2] (\barstart,\barypos)
            rectangle (\barstart+\barwidth,\barypos+\barheight);

        % 检测丰度值
        \node[value] at (11.5, {\basepos-\valuebarspace+0.6}) {\footnotesize\value};

        % 解析范围并计算进度条长度
        \def\parserange#1-#2\endparse{\def\minval{#1}\def\maxval{#2}}
        \expandafter\parserange\range\endparse

        % 计算进度条长度和颜色
        \pgfmathsetmacro{\progress}{min(\value/\maxval, 1.0)}
        \pgfmathparse{\value > \maxval ? "customred" : (\value < \minval ? "customred" : "green!50")}
        \let\barcolor=\pgfmathresult

        % 进度条显示
        \ifnum\pdfstrcmp{\status}{超标}=0
            \fill[customred, rounded corners=2] (\barstart,\barypos)
                rectangle (\barstart+\barwidth,\barypos+\barheight);
        \else
            \fill[\barcolor, rounded corners=2] (\barstart,\barypos)
                rectangle (\barstart+\barwidth*\progress,\barypos+\barheight);
        \fi

        % 结果评价
        \ifnum\pdfstrcmp{\status}{正常}=0
            \node[value, text=customRed] at (16,\basepos) {\footnotesize\textbf{\status}};
        \else
            \node[value, text=customGreen] at (16,\basepos) {\footnotesize\textbf{\status}};
        \fi


        % 添加卡片
        \pgfmathsetmacro{\cardypos}{\basepos-0.5}
        \begin{scope}[shift={(0,\cardypos)}]
            % 卡片背景
            \pgfmathsetmacro{\cardheight}{
                \ifnum\pdfstrcmp{\status}{超标}=0
                    1.0  % 两行内容时的高度
                \else
                    0.6  % 一行内容时的高度
                \fi
            }

            \fill[rounded corners=5pt, customTeal!5, draw=gray!5]
                (0.3,-\cardheight) rectangle (17.3,0);

            % 菌群简介图标和内容
            \node[anchor=west] at (0.5,-0.3) {
                \textbf{\color{gray!90}\footnotesize \textcolor{customTeal}{\faInfoCircle}}
            };
            \node[anchor=west, text width=16cm] at (0.9,-0.3) {
                {\small\color{gray}\footnotesize \intro}
            };

            % 异常解读标题和内容
            \ifnum\pdfstrcmp{\status}{超标}=0
                \node[anchor=west] at (0.55,-0.8) {
                    \textbf{\color{customRed}\footnotesize \textcolor{customRed}{\faBell}}
                };
                \node[anchor=west, text width=16cm] at (0.9,-0.8) {
                    {\small\color{gray}\footnotesize \suggestion}
                };
            \fi
        \end{scope}

        % 分割线
        \pgfmathsetmacro{\linepos}{
            \ifnum\pdfstrcmp{\status}{超标}=0
                \basepos-1.7  % 超标时的分割线位置
            \else
                \basepos-1.3  % 正常时的分割线位置
            \fi
        }
        \draw[gray!20] (0.2,\linepos) -- (\cardwidth-0.2,\linepos);

        % 根据当前行的状态计算下一行的位置增量
        \ifnum\pdfstrcmp{\status}{超标}=0
            \pgfmathsetmacro{\increment}{0.85}  % 超标行(两行内容)需要更大的增量
        \else
            \pgfmathsetmacro{\increment}{0.7}  % 正常行(一行内容)使用较小的增量
        \fi

        % 更新位置计数器
        \pgfmathsetmacro{\nextpos}{\currentpos+\increment}
        \xdef\currentpos{\nextpos}
    }

    % 最后一行的处理,消除多余的空白
    \pgfmathsetmacro{\lastincrement}{2}  % 最后一行的增量
    \pgfmathsetmacro{\nextpos}{\currentpos+\lastincrement}
    \xdef\currentpos{\nextpos}

\end{tikzpicture}
\end{center}

\begin{tcolorbox}[
    enhanced,
    colback=gray!3,
    colframe=gray!3,
    arc=3mm,
    boxrule=0pt,
    width=\textwidth,
    top=8pt,
    bottom=8pt
]
{\small{\textcolor{customRed}{\faBell}}\quad 代谢指标严重偏低需要重点关注:
\begin{itemize}
\item 代谢健康(1,参考范围25-100):这个关键指标严重低于正常范围,表明肠道代谢功能显著受损,就像一台运转效率极低的发动机,需要及时干预和调节。
\item 草酸盐代谢(1,参考范围25-100):同样处于极低水平,提示机体对草酸盐的处理能力严重不足,可能增加肾结石等健康风险。
\end{itemize}

{\textcolor{customGreen}{\faBell}}\quad 建议:
\begin{itemize}
\item 及时就医进行专业评估和干预
\item 调整饮食结构,避免高草酸食物
\item 可能需要相关营养补充和代谢调节治疗
\end{itemize}
}
\end{tcolorbox}



\newpage

\begin{tcolorbox}[
    enhanced,
    colback=white,
    colframe=white,
    arc=2mm,
    boxrule=0pt,
    width=\textwidth,
    left=15pt,
    right=15pt,
    top=10pt,
    bottom=10pt,
    drop shadow={
        opacity=0.2,
        color=customTeal
    },
    borderline west={5pt}{0pt}{customTeal}
]
\textcolor{customTeal}{\Large\textbf{菌群整体评估}}
\end{tcolorbox}

\vspace{0.05cm}

\begin{tcolorbox}[
    enhanced,
    colback=customTealBg,
    colframe=customTealBg,
    arc=3mm,
    boxrule=0pt,
    width=\textwidth,
    top=8pt,
    bottom=8pt
]
{\small{\color{customTeal}\faInfoCircle} 菌群整体评估是评估肠道微生态系统的核心指标,通过对菌群的数量、类型和功能特征的分析,可以全面了解肠道微生态的健康状况。
}
\end{tcolorbox}

\begin{tcolorbox}[
    enhanced,
    colback=lightpurple!10, % 卡片底色
    colframe=white,  % 边框颜色
    arc=3mm,
    boxrule=0.5pt,
    width=\textwidth,
    top=8pt,
    bottom=8pt
]
{\small{\color{lightpurple}\faQuestionCircle}\quad \textbf{菌群整体评估有什么临床意义?}\\
{\color{orange!50}\faComments}\quad 菌群整体评估的临床意义包括但不限于以下方面:
\begin{itemize}
    \item 微生态评估:全面了解肠道菌群结构和功能状态。
    \item 健康预警:及早发现微生态失衡风险。
    \item 治疗指导:为精准用药和菌群调节提供依据。
\end{itemize}
}
\end{tcolorbox}

\vspace{-0.7cm}

\begin{center}
\begin{tikzpicture}[
    font=\small,
    title/.style={font=\small\bfseries\color{white}},
    value/.style={font=\small},
    reference/.style={font=\small},
    cell/.style={anchor=west, text width=5.0cm},
    note/.style={anchor=west, text width=4.5cm, align=left}
]
    \def\cardwidth{\textwidth}
    \def\cardheight{8.5}
    \def\barheight{0.25}
    \def\barwidth{1.5}
    \def\valuebarspace{0.4}

    % 容器和标题栏背景
    \draw[rounded corners=5, fill=white, draw=gray!20]
        (0,0) rectangle (\cardwidth,-\cardheight);
    \path[fill=customTeal]
        (0,0) [rounded corners=5] -- (\cardwidth,0) --
        (\cardwidth,0.8) -- (0,0.8) -- cycle;

    % 表头
    \node[title, anchor=west] at (0.5,0.4) {\textbf{菌种名称}};
    \node[title] at (7, 0.4) {\textbf{正常范围}};
    \node[title] at (11.5, 0.4) {\textbf{检测丰度}};
    \node[title] at (16, 0.4) {\textbf{结果评价}};

    % 初始化位置计数器
    \def\currentpos{0.25}

    % 数据行和卡片
    \foreach \item/\enitem/\value/\range/\percentile/\detection/\status/\intro/\suggestion/\index/\category in {
        {菌种数量}/{Gut microbial species count}/101/{0-100}/67\%/99.52\%/超标/{反映肠道微生物的总体丰度,是评估肠道微生态稳定性的基础指标。}/{可通过增加膳食纤维、限制糖分和加工食品、摄入发酵食品、保持规律作息和锻炼、管理压力,以及考虑益生菌补充剂来实现。}/\currentpos/,
        {菌种多样性}/{Gut microbial diversity}/36/{15-95}/2\%/99.04\%/正常/{表征肠道菌群的物种丰富度和均匀度,高多样性通常预示着更稳定的微生态系统。}/{}/\currentpos/,
        {菌群平衡}/{Gut microbiota balance}/94/{15-100}/95\%/98.56\%/正常/{评估有益菌与有害菌的比例关系,反映肠道微生态的协调程度。}/{}/\currentpos/,
        {菌群恢复力}/{Gut microbial resilience}/8/{25-100}/50\%/99.52\%/缺乏/{衡量肠道菌群对外界干扰的抵抗能力和自我修复能力。}/{}/\currentpos/
    }
    {
        % 计算当前行的基础位置
        \pgfmathsetmacro{\basepos}{-2.8*\currentpos}

        % 菌种名称
        \node[cell, align=left] at (0.5,\basepos) {
            \small\textbf{\item}\\[-0.2em]
            {\color{gray}\small\enitem}
        };

        % 正常范围
        \node[reference] at (7,\basepos) {\footnotesize\range};

        % 进度条相关
        \pgfmathsetmacro{\barypos}{\basepos-\valuebarspace+0.1}
        \def\barstart{10.75}

        % 进度条背景
        \fill[gray!10, rounded corners=2] (\barstart,\barypos)
            rectangle (\barstart+\barwidth,\barypos+\barheight);

        % 检测丰度值
        \node[value] at (11.5, {\basepos-\valuebarspace+0.6}) {\footnotesize\value};

        % 解析范围并计算进度条长度
        \def\parserange#1-#2\endparse{\def\minval{#1}\def\maxval{#2}}
        \expandafter\parserange\range\endparse

        % 计算进度条长度和颜色
        \pgfmathsetmacro{\progress}{min(\value/\maxval, 1.0)}
        \pgfmathparse{\value > \maxval ? "customred" : (\value < \minval ? "customred" : "green!50")}
        \let\barcolor=\pgfmathresult

        % 进度条显示
        \ifnum\pdfstrcmp{\status}{超标}=0
            \fill[customred, rounded corners=2] (\barstart,\barypos)
                rectangle (\barstart+\barwidth,\barypos+\barheight);
        \else
            \fill[\barcolor, rounded corners=2] (\barstart,\barypos)
                rectangle (\barstart+\barwidth*\progress,\barypos+\barheight);
        \fi

        % 结果评价
        \ifnum\pdfstrcmp{\status}{超标}=0
            \node[value, text=customRed] at (16,\basepos) {\footnotesize\textbf{\status}};
        \else
            \node[value, text=customGreen] at (16,\basepos) {\footnotesize\textbf{\status}};
        \fi


        % 添加卡片
        \pgfmathsetmacro{\cardypos}{\basepos-0.5}
        \begin{scope}[shift={(0,\cardypos)}]
            % 卡片背景
            \pgfmathsetmacro{\cardheight}{
                \ifnum\pdfstrcmp{\status}{超标}=0
                    1.0  % 两行内容时的高度
                \else
                    0.6  % 一行内容时的高度
                \fi
            }

            \fill[rounded corners=5pt, customTeal!5, draw=gray!5]
                (0.3,-\cardheight) rectangle (17.3,0);

            % 菌群简介图标和内容
            \node[anchor=west] at (0.5,-0.3) {
                \textbf{\color{gray!90}\footnotesize \textcolor{customTeal}{\faInfoCircle}}
            };
            \node[anchor=west, text width=16cm] at (0.9,-0.3) {
                {\small\color{gray}\footnotesize \intro}
            };

            % 异常解读标题和内容
            \ifnum\pdfstrcmp{\status}{超标}=0
                \node[anchor=west] at (0.55,-0.8) {
                    \textbf{\color{customRed}\footnotesize \textcolor{customRed}{\faBell}}
                };
                \node[anchor=west, text width=16cm] at (0.9,-0.8) {
                    {\small\color{gray}\footnotesize \suggestion}
                };
            \fi
        \end{scope}

        % 分割线
        \pgfmathsetmacro{\linepos}{
            \ifnum\pdfstrcmp{\status}{超标}=0
                \basepos-1.7  % 超标时的分割线位置
            \else
                \basepos-1.3  % 正常时的分割线位置
            \fi
        }
        \draw[gray!20] (0.2,\linepos) -- (\cardwidth-0.2,\linepos);

        % 根据当前行的状态计算下一行的位置增量
        \ifnum\pdfstrcmp{\status}{超标}=0
            \pgfmathsetmacro{\increment}{0.85}  % 超标行(两行内容)需要更大的增量
        \else
            \pgfmathsetmacro{\increment}{0.7}  % 正常行(一行内容)使用较小的增量
        \fi

        % 更新位置计数器
        \pgfmathsetmacro{\nextpos}{\currentpos+\increment}
        \xdef\currentpos{\nextpos}
    }

    % 最后一行的处理,消除多余的空白
    \pgfmathsetmacro{\lastincrement}{2}  % 最后一行的增量
    \pgfmathsetmacro{\nextpos}{\currentpos+\lastincrement}
    \xdef\currentpos{\nextpos}

\end{tikzpicture}
\end{center}

\begin{tcolorbox}[
enhanced,
colback=gray!3,
colframe=gray!3,
arc=3mm,
boxrule=0pt,
width=\textwidth,
top=8pt,
bottom=8pt
]
{\small{\textcolor{customRed}{\faBell}}\quad 菌群指标异常需要重点关注:
\begin{itemize}
\item 菌种数量(101,参考范围0-100):略微超出正常范围上限,表明肠道菌群总量偏多,需要适度关注。
\item 菌群恢复力(8,参考范围25-100):该指标严重低于正常范围,说明肠道微生态系统的自我修复和调节能力显著减弱,就像免疫系统失去了自我修复的能力,需要及时干预。
\end{itemize}

{\textcolor{customGreen}{\faBell}}\quad 其他指标情况:
\begin{itemize}
\item 菌群多样性(36,参考范围15-95):处于正常范围内,但偏低,表明肠道菌群种类相对单一。
\item 菌群平衡(94,参考范围15-100):接近正常范围上限,显示各类菌群比例分布较为合理。
\end{itemize}
}
\end{tcolorbox}



\newpage

\begin{tcolorbox}[
    enhanced,
    colback=white,
    colframe=white,
    arc=2mm,
    boxrule=0pt,
    width=\textwidth,
    left=15pt,
    right=15pt,
    top=10pt,
    bottom=10pt,
    drop shadow={
        opacity=0.2,
        color=customTeal
    },
    borderline west={5pt}{0pt}{customTeal}
]
\textcolor{customTeal}{\Large\textbf{菌群微环境评估}}
\end{tcolorbox}

\vspace{0.05cm}

\begin{tcolorbox}[
    enhanced,
    colback=customTealBg,
    colframe=customTealBg,
    arc=3mm,
    boxrule=0pt,
    width=\textwidth,
    top=8pt,
    bottom=8pt
]
{\normalsize{\color{customTeal}\faInfoCircle} 肠道菌群微环境评估是反映肠道生态系统健康状况的重要指标组合,通过对不同类型菌群的分析,可以全面了解肠道微生态的平衡状态。
}
\end{tcolorbox}

\begin{tcolorbox}[
    enhanced,
    colback=lightpurple!10, % 卡片底色
    colframe=white,  % 边框颜色
    arc=3mm,
    boxrule=0.5pt,
    width=\textwidth,
    top=8pt,
    bottom=8pt
]
{\normalsize{\color{lightpurple}\faQuestionCircle}\quad \textbf{菌群微环境评估有什么临床意义?}\\
{\color{orange!50}\faComments}\quad 菌群微环境评估的临床意义包括但不限于以下方面:
\begin{itemize}
    \item 有益菌:评估有益菌的临床意义在于维护肠道平衡、降低疾病风险、提升免疫力和改善整体健康。
    \item 有害菌:评估有害菌的临床意义在于识别和控制感染、预防疾病传播、指导抗生素使用及评估治疗效果。
    \item 革兰氏阴性菌:评估革兰氏阴性菌的临床意义在于识别其引起的感染(如肺炎、尿路感染和败血症)、了解其细胞壁结构薄且具有外膜特性导致抗生素耐药性增加,从而指导临床治疗和优化感染控制措施。
    \item 好氧菌:评估好氧菌的临床意义在于识别其引起的感染(如肺炎和败血症)、指导抗生素选择以有效治疗感染、了解其生长环境及代谢特征以便于监测和预防感染的发生。
\end{itemize}
}
\end{tcolorbox}
\vspace{-0.5cm}
\begin{center}
\begin{tikzpicture}[
    font=\normalsize,
    title/.style={font=\normalsize\bfseries\color{white}},
    value/.style={font=\normalsize},
    reference/.style={font=\normalsize},
    cell/.style={anchor=west, text width=5.0cm},
    note/.style={anchor=west, text width=4.5cm, align=left}
]
    \def\cardwidth{\textwidth}
    \def\cardheight{8.5}
    \def\barheight{0.25}
    \def\barwidth{1.5}
    \def\valuebarspace{0.4}

    % 容器和标题栏背景
    \draw[rounded corners=5, fill=white, draw=gray!20]
        (0,0) rectangle (\cardwidth,-\cardheight);
    \path[fill=customTeal]
        (0,0) [rounded corners=5] -- (\cardwidth,0) --
        (\cardwidth,0.8) -- (0,0.8) -- cycle;

    % 表头
    \node[title, anchor=west] at (0.5,0.4) {\textbf{菌种名称}};
    \node[title] at (7, 0.4) {\textbf{正常范围}};
    \node[title] at (11.5, 0.4) {\textbf{检测丰度}};
    \node[title] at (16, 0.4) {\textbf{结果评价}};

    % 初始化位置计数器
    \def\currentpos{0.25}

    % 数据行和卡片
    \foreach \item/\enitem/\value/\range/\percentile/\detection/\status/\intro/\suggestion/\index/\category in {
        {有益菌}/{Beneficial bacteria}/94/{16-100}/67\%/99.52\%/正常/{包括双歧杆菌、乳酸菌等对肠道健康有利的菌群,能促进营养物质吸收和免疫功能。}/{}/\currentpos/,
        {有害菌}/{Pathogenic bacteria}/34/{0-84}/2\%/99.04\%/正常/{指可能导致肠道功能紊乱的菌群,过高水平可能引起消化问题。}/{}/\currentpos/,
        {革兰氏阴性菌}/{Gram-negative bacteria}/0/{0-0.4}/95\%/98.56\%/正常/{这类细菌的细胞壁结构特殊,与某些疾病风险相关。}/{}/\currentpos/,
        {好氧菌}/{Aerobic bacteria}/0/{0-0.3}/50\%/99.52\%/正常/{需要氧气生存的菌群,反映肠道氧化还原环境状态。}/{}/\currentpos/
    }
    {
        % 计算当前行的基础位置
        \pgfmathsetmacro{\basepos}{-2.8*\currentpos}

        % 菌种名称
        \node[cell, align=left] at (0.5,\basepos) {
            \normalsize\textbf{\item}\\[-0.2em]
            {\color{gray}\normalsize\enitem}
        };

        % 正常范围
        \node[reference] at (7,\basepos) {\normalsize\range};

        % 进度条相关
        \pgfmathsetmacro{\barypos}{\basepos-\valuebarspace+0.1}
        \def\barstart{10.75}

        % 进度条背景
        \fill[gray!10, rounded corners=2] (\barstart,\barypos)
            rectangle (\barstart+\barwidth,\barypos+\barheight);

        % 检测丰度值
        \node[value] at (11.5, {\basepos-\valuebarspace+0.6}) {\footnotesize\value};

        % 解析范围并计算进度条长度
        \def\parserange#1-#2\endparse{\def\minval{#1}\def\maxval{#2}}
        \expandafter\parserange\range\endparse

        % 计算进度条长度和颜色
        \pgfmathsetmacro{\progress}{min(\value/\maxval, 1.0)}
        \pgfmathparse{\value > \maxval ? "customred" : (\value < \minval ? "customred" : "green!50")}
        \let\barcolor=\pgfmathresult

        % 进度条显示
        \ifnum\pdfstrcmp{\status}{超标}=0
            \fill[customred, rounded corners=2] (\barstart,\barypos)
                rectangle (\barstart+\barwidth,\barypos+\barheight);
        \else
            \fill[\barcolor, rounded corners=2] (\barstart,\barypos)
                rectangle (\barstart+\barwidth*\progress,\barypos+\barheight);
        \fi

        % 结果评价
        \ifnum\pdfstrcmp{\status}{超标}=0
            \node[value, text=customRed] at (16,\basepos) {\normalsize\textbf{\status}};
        \else
            \node[value, text=customGreen] at (16,\basepos) {\normalsize\textbf{\status}};
        \fi


        % 添加卡片
        \pgfmathsetmacro{\cardypos}{\basepos-0.6}
        \begin{scope}[shift={(0,\cardypos)}]
            % 卡片背景
            \pgfmathsetmacro{\cardheight}{
                \ifnum\pdfstrcmp{\status}{超标}=0
                    1.0  % 两行内容时的高度
                \else
                    0.6  % 一行内容时的高度
                \fi
            }

            \fill[rounded corners=5pt, customTeal!5, draw=gray!5]
                (0.3,-\cardheight) rectangle (17.3,0);

            % 菌群简介图标和内容
            \node[anchor=west] at (0.5,-0.3) {
                \textbf{\color{black!80}\small \textcolor{customTeal}{\faInfoCircle}}
            };
            \node[anchor=west, text width=16cm] at (0.9,-0.3) {
                {\small\color{black!80}\small \intro}
            };

            % 异常解读标题和内容
            \ifnum\pdfstrcmp{\status}{超标}=0
                \node[anchor=west] at (0.55,-0.8) {
                    \textbf{\color{customRed}\footnotesize \textcolor{customRed}{\faBell}}
                };
                \node[anchor=west, text width=16cm] at (0.9,-0.8) {
                    {\small\color{gray}\footnotesize \suggestion}
                };
            \fi
        \end{scope}

        % 分割线
        \pgfmathsetmacro{\linepos}{
            \ifnum\pdfstrcmp{\status}{超标}=0
                \basepos-1.7  % 超标时的分割线位置
            \else
                \basepos-1.3  % 正常时的分割线位置
            \fi
        }
        \draw[gray!20] (0.2,\linepos) -- (\cardwidth-0.2,\linepos);

        % 根据当前行的状态计算下一行的位置增量
        \ifnum\pdfstrcmp{\status}{超标}=0
            \pgfmathsetmacro{\increment}{0.85}  % 超标行(两行内容)需要更大的增量
        \else
            \pgfmathsetmacro{\increment}{0.7}  % 正常行(一行内容)使用较小的增量
        \fi

        % 更新位置计数器
        \pgfmathsetmacro{\nextpos}{\currentpos+\increment}
        \xdef\currentpos{\nextpos}
    }

    % 最后一行的处理,消除多余的空白
    \pgfmathsetmacro{\lastincrement}{2}  % 最后一行的增量
    \pgfmathsetmacro{\nextpos}{\currentpos+\lastincrement}
    \xdef\currentpos{\nextpos}

\end{tikzpicture}
\end{center}

\begin{tcolorbox}[
enhanced,
colback=gray!3,
colframe=gray!3,
arc=3mm,
boxrule=0pt,
width=\textwidth,
top=8pt,
bottom=8pt
]
{\small{\textcolor{customRed}{\faBell}}\quad 有益菌水平需要关注:
\begin{itemize}
\item 有益菌(94,参考范围16-100):处于参考范围上限附近,表明肠道中有益菌群数量充足,这对维持肠道健康和免疫功能非常有利。
\end{itemize}

{\textcolor{green!85!orange}{\faBell}}\quad 其他指标情况:
\begin{itemize}
\item 有害菌(34,参考范围0-84):处于正常范围内的中低水平,显示肠道环境相对健康。
\item 革兰氏阴性菌(0,参考范围0-0.4):处于正常范围内的最低值,表明这类潜在致病菌得到了很好的控制。
\item 好氧菌(0,参考范围0-0.3):同样处于正常范围内的最低值,说明肠道氧化还原环境维持在理想状态。
\end{itemize}

总体来看,您的肠道菌群微环境状况良好,特别是有益菌含量充足,而其他可能造成不良影响的菌群都维持在较低水平,这是非常理想的状态。
}
\end{tcolorbox}

\newpage

\begin{tcolorbox}[
enhanced,
colback=white,
colframe=customTeal,
arc=2mm,
boxrule=1pt,
left=20pt,
right=20pt,
top=12pt,
bottom=12pt,
width=\textwidth,
fontupper=\sffamily,
overlay={
\draw[customTeal, line width=2pt]
([xshift=15pt]frame.south west) -- ([xshift=-15pt]frame.south east);
}
]
{\Large\bfseries\textcolor{customTeal}{\Huge 肠道菌群检测}}
\end{tcolorbox}

\begin{tcolorbox}[
    enhanced,
    colback=customTealBg,
    colframe=customTealBg,
    arc=3mm,
    boxrule=0pt,
    width=\textwidth,
    top=8pt,
    bottom=8pt
]
{\small{\color{customTeal}\faInfoCircle} 肠道菌群可以大分类成细菌,病毒,真菌,原生生物和寄生虫5个类别,下列是对各项组成部分的介绍及他们在您肠道中的分布比例图。
\begin{itemize}
    \item \textbf{细菌}:占肠道菌群的98\%以上。细菌是肠道微生物的主要组成部分,包含: - 益生菌(如乳酸杆菌、双歧杆菌):占肠道细菌的20\% - 30\%;对消化、营养吸收和免疫功能具有正面影响。 - 病原菌:数量较少(通常不超过1\%),如某些致病性大肠杆菌和沙门氏菌,但其存在可能导致肠道感染。 - 中性菌:占大约70\% - 80\%,在肠道中有助于维持微生物平衡和生态稳定。
    \item \textbf{病毒}:大约占肠道微生物的1\% - 5\%。病毒,尤其是噬菌体,在调节细菌的种类和数量方面扮演重要角色。它们既可以抑制有害细菌的生长,也可以推动微生物群落的多样性。
    \item \textbf{真菌}:占肠道微生物的0.1\% - 1\%。主要包括酵母菌(如白色念珠菌),在数量较少时是正常的一部分,但当免疫系统受损时,可能会因其快速繁殖而导致感染或其它问题。
    \item \textbf{原生生物}:通常占0.1\%以下。虽然数量较少,但原生生物如变形虫和细胞虫有时会影响肠道环境,参与营养循环和分解。
    \item \textbf{寄生虫}:在健康个体中,寄生虫的数量极少,通常小于0.1\%。常见的肠道寄生虫如蛔虫和鞭虫,虽然数量较少,但在某些情况下可能导致消化不良和腹痛等健康问题。

\end{itemize}

}
\end{tcolorbox}

\begin{center}
\begin{tikzpicture}[
    boxLeft/.style={
        rounded corners=15pt,  % 增加圆角
        fill=#1!6,  % 降低填充色深度
        inner sep=14pt,  % 增加内边距
        drop shadow={
            shadow xshift=1.5mm,
            shadow yshift=-1.5mm,
            fill=gray!20,
            opacity=0.2
        },
        line width=0.4pt,  % 添加细边框
        draw=#1!30  % 边框颜色
    },
    boxRight/.style={
        rounded corners=15pt,
        fill=#1!6,
        inner sep=14pt,
        drop shadow={
            shadow xshift=1.5mm,
            shadow yshift=-1.5mm,
            fill=gray!20,
            opacity=0.2
        },
        line width=0.4pt,
        draw=#1!30
    },
    title/.style={
        font=\bfseries,
        text=#1!95  % 增加文字颜色深度
    },
    percentage/.style={
        font=\bfseries,
        text=#1!95
    }
]
    \path (-0.5\textwidth, 0) -- (0.5\textwidth, 0);

    % 背景装饰(可选)
%    \fill[blue!2] (-0.65\textwidth,-0.5) rectangle (0.65\textwidth,9);

    % 左侧细菌区域
    \begin{scope}[local bounding box=bacteria]
        % 添加光晕效果
        \node[boxLeft=blue!80, minimum width=0.59\textwidth, minimum height=9.5cm, opacity=0.1] at (-0.3\textwidth,4.05) {};
        \node[boxLeft=blue, minimum width=0.59\textwidth, minimum height=9.5cm] at (-0.3\textwidth,4) {};

        % 标题和比例
        \node[title=blue, anchor=west] at (-0.57\textwidth,7.7) {\Huge\sffamily 细菌};  % 使用无衬线字体
        \node[percentage=blue, anchor=east] at (-0.03\textwidth,7.7) {\Huge\sffamily 90\%};
    \end{scope}

    % 右侧区域
    \begin{scope}[xshift=0.2\textwidth]
        % 病毒区域
        \node[boxRight=purple!90, minimum width=0.40\textwidth, minimum height=2.8cm, opacity=0.1] at (0,7.35) {};
        \node[boxRight=purple, minimum width=0.40\textwidth, minimum height=2.8cm] at (0,7.3) {};
        \node[title=purple, anchor=west] at (-0.18\textwidth,7.8) {\huge\sffamily 病毒};
        \node[percentage=purple, anchor=east] at (0.18\textwidth,7.8) {\huge\sffamily 5\%};

        % 真菌区域
        \node[boxRight=green!90, minimum width=0.4\textwidth, minimum height=2.6cm, opacity=0.1] at (0,4.55) {};
        \node[boxRight=green, minimum width=0.4\textwidth, minimum height=2.6cm] at (0,4.5) {};
        \node[title=green, anchor=west] at (-0.18\textwidth,4.7) {\Large\sffamily 真菌};
        \node[percentage=green, anchor=east] at (0.18\textwidth,4.7) {\Large\sffamily 3\%};

        % 真核生物区域
        \node[boxRight=orange!90, minimum width=0.40\textwidth, minimum height=1.8cm, opacity=0.1] at (0,2.15) {};
        \node[boxRight=orange, minimum width=0.40\textwidth, minimum height=1.8cm] at (0,2.1) {};
        \node[title=orange, anchor=west] at (-0.18\textwidth,2.1) {\large\sffamily 原生生物};
        \node[percentage=orange, anchor=east] at (0.18\textwidth,2.1) {\large\sffamily 1\%};

        % 寄生虫区域
        \node[boxRight=red!90, minimum width=0.40\textwidth, minimum height=1.8cm, opacity=0.1] at (0,0.25) {};
        \node[boxRight=red, minimum width=0.40\textwidth, minimum height=1.8cm] at (0,0.2) {};
        \node[title=red, anchor=west] at (-0.18\textwidth,0.2) {\large\sffamily 寄生虫};
        \node[percentage=red, anchor=east] at (0.18\textwidth,0.2) {\large\sffamily 1\%};
    \end{scope}

    % 添加装饰性元素
    \foreach \i in {1,...,5} {
        \fill[blue!10, opacity=0.1] (-0.65\textwidth + \i*0.2\textwidth, -0.5)
            circle (0.5mm);
    }

\end{tikzpicture}
\end{center}

\newpage

\begin{tcolorbox}[
    enhanced,
    colback=white,
    colframe=white,
    arc=2mm,
    boxrule=0pt,
    width=\textwidth,
    left=15pt,
    right=15pt,
    top=10pt,
    bottom=10pt,
    drop shadow={
        opacity=0.2,
        color=customTeal
    },
    borderline west={5pt}{0pt}{customTeal}
]
\textcolor{customTeal}{\Large\textbf{肠道细菌}}
\end{tcolorbox}

\begin{tcolorbox}[
    enhanced,
    colback=customTealBg,
    colframe=customTealBg,
    arc=3mm,
    boxrule=0pt,
    width=\textwidth,
    top=8pt,
    bottom=8pt
]
{\small{\color{customTeal}\faInfoCircle} 肠道微生物以细菌为主要组成部分,在人体肠道中约有数千种不同的细菌物种,总数量高达数十万亿个。健康成年人肠道中的细菌主要由厚壁菌门(Firmicutes)、拟杆菌门(Bacteroidetes)、放线菌门(Actinobacteria)和变形菌门(Proteobacteria)等构成,其中厚壁菌门和拟杆菌门的数量占比最高,可达90\%以上。我们可以从两个重要维度来评估肠道菌群中的细菌:丰度和致病性,这两个维度的组合可以帮助我们更全面地理解不同菌属在肠道微生态中的重要性和分布特征。
}
\end{tcolorbox}

\definecolor{titlecolor}{RGB}{75, 0, 130}
\definecolor{boxcolor1}{RGB}{245, 248, 255}
\definecolor{boxcolor2}{RGB}{255, 248, 245}
\definecolor{cardcolor1}{RGB}{255, 255, 255}

% 使用垂直空间调整整体位置
%\vspace*{2cm}

\begin{figure}[htbp]
\centering
\begin{tikzpicture}[
% 定义样式
mainbox/.style={
    rounded corners=12pt,
    minimum width=0.48\textwidth,
    minimum height=160mm,
    fill=#1,
    drop shadow={shadow xshift=0.7mm, shadow yshift=-0.7mm, fill=gray!30, opacity=0.3}
},
title/.style={
    font=\LARGE\bfseries\sffamily,
    text=customTeal
},
subtitle/.style={
    font=\Large\bfseries\sffamily,
    text=customTeal!80
},
smallcard/.style={   % 右侧小卡片样式
    rounded corners=8pt,
    minimum width=0.44\textwidth,
    minimum height=28mm,
    drop shadow={shadow xshift=0.5mm, shadow yshift=-0.5mm, fill=gray!30, opacity=0.3}
},
leftcard/.style={   % 左侧小卡片样式
    rounded corners=8pt,
    minimum width=0.44\textwidth,
    minimum height=60mm,
    drop shadow={shadow xshift=0.5mm, shadow yshift=-0.5mm, fill=gray!30, opacity=0.3}
}
]

% 定义小卡片的颜色
\definecolor{beneficialColor}{RGB}{200, 230, 201}
\definecolor{harmfulColor}{RGB}{239, 154, 154}     % 深粉红色 - 有害菌
\definecolor{opportunisticColor}{RGB}{255, 224, 178}
\definecolor{pathogenicColor}{RGB}{255, 205, 210}
\definecolor{coreColor}{RGB}{187, 222, 251}
\definecolor{nonCoreColor}{RGB}{225, 190, 231}

% 主标题
\node[title] at (0,9) {肠道细菌分类体系};

% 两个主要分类框
\path (0,0) coordinate (center);
\node[mainbox=boxcolor1] at (-0.25\textwidth,0) {};
\node[mainbox=boxcolor2] at (0.25\textwidth,0) {};

% 左右卡片标题
\node[subtitle] at (-0.25\textwidth,6) {按菌群丰度分};
\node[subtitle] at (0.25\textwidth,6) {按菌群致病性分};

% 左侧两个小卡片
% 核心菌属卡片
\node[leftcard, fill=coreColor] at (-0.25\textwidth,2) {};
\node[font=\Large\bfseries] at (-0.23\textwidth - 0.16\textwidth,4.4) {核心菌属};
\node[text width=0.38\textwidth, font=\small, align=left]
    at (-0.25\textwidth, 3) {\small 在绝大多数健康人群肠道中普遍存在的优势菌属,包括厚壁菌属、拟杆菌属等,占据总菌群的主要组成部分。};

% 非核心重要菌属卡片
\node[leftcard, fill=nonCoreColor] at (-0.25\textwidth,-4.5) {};
\node[font=\Large\bfseries] at (-0.19\textwidth - 0.15\textwidth,-2.2) {非核心重要菌属};
\node[text width=0.38\textwidth, font=\small, align=left]
    at (-0.25\textwidth,-3.5) {虽然丰度相对较低,但对维持肠道健康具有重要作用的菌属,如乳酸菌属、双歧杆菌属等。};

% 右侧三个小卡片
% 有益菌卡片
\node[smallcard, fill=beneficialColor] at (0.25\textwidth,3.6) {};
\node[font=\Large\bfseries] at (0.25\textwidth - 0.16\textwidth,4.4) {有益菌};
\node[text width=0.38\textwidth, font=\small, align=left]
    at (0.25\textwidth,3.0) {对人体健康有益的菌群,如双歧杆菌、乳酸菌等。它们参与营养物质的消化吸收,产生维生素,增强免疫力。};

% 有害菌卡片
\node[smallcard, fill=harmfulColor] at (0.25\textwidth,0.4) {};
\node[font=\Large\bfseries] at (0.25\textwidth - 0.16\textwidth,1.1) {有害菌};
\node[text width=0.38\textwidth, font=\small, align=left]
    at (0.25\textwidth,0.0) {有害菌通常是指那些在肠道中可能对宿主产生负面影响的细菌,这些细菌可能会干扰肠道平衡,导致消化问题,或促进肠道炎症。};

% 机会致病菌卡片
\node[smallcard, fill=opportunisticColor] at (0.25\textwidth,-2.9) {};
\node[font=\Large\bfseries] at (0.25\textwidth - 0.13\textwidth,-2.3) {机会致病菌};
\node[text width=0.38\textwidth, font=\small, align=left]
    at (0.25\textwidth,-3.5) {通常情况下与人体和平共处,但在特定条件下(如免疫力下降时)可能导致疾病的菌群。};

% 致病菌卡片
\node[smallcard, fill=pathogenicColor] at (0.25\textwidth,-6.1) {};
\node[font=\Large\bfseries] at (0.25\textwidth - 0.16\textwidth,-5.4) {致病菌};
\node[text width=0.38\textwidth, font=\small, align=left]
    at (0.25\textwidth,-6.5) {能够直接导致疾病的有害菌群,它们会破坏肠道环境,引起感染和炎症等症状。};

\end{tikzpicture}
\end{figure}















\newpage

\begin{tcolorbox}[
    enhanced,
    colback=white,
    colframe=white,
    arc=2mm,
    boxrule=0pt,
    width=\textwidth,
    left=15pt,
    right=15pt,
    top=10pt,
    bottom=10pt,
    drop shadow={
        opacity=0.2,
        color=customTeal
    },
    borderline west={5pt}{0pt}{customTeal}
]
\textcolor{customTeal}{\large\textbf{核心菌属}}
\end{tcolorbox}
\vspace{0.05cm}
\begin{tcolorbox}[
    enhanced,
    colback=customTealBg,
    colframe=customTealBg,
    arc=3mm,
    boxrule=0pt,
    width=\textwidth,
    top=8pt,
    bottom=8pt
]
{\small{\color{customTeal}\faInfoCircle} 核心菌属是指在肠道菌群中丰度(相对数量)较高、具有重要生理功能的优势菌群,它们在维持肠道健康、参与营养物质代谢、调节免疫系统等方面发挥着主导作用,是构成肠道微生态系统的基石。主要包括拟杆菌属(Bacteroides)、普雷沃氏菌属(Prevotella)和瘤胃球菌属(Ruminococcus)等,这些菌群在正常情况下能帮助消化纤维、产生短链脂肪酸、维持肠道屏障功能。如果这些核心菌属丰度异常(过高或过低),可能导致肠道菌群失衡,引发炎症反应、免疫功能紊乱,甚至增加肥胖、糖尿病等代谢性疾病的风险。
\begin{itemize}
    \item 在本报告中,核心菌属指的是在90\%人群中可被检测出,且人群平均丰度在1\%以上的菌属。
\end{itemize}
}
\end{tcolorbox}
\vspace{-0.5cm}
\begin{center}
\begin{tikzpicture}[
    font=\small,
    title/.style={font=\small\bfseries\color{white}},
    value/.style={font=\small},
    reference/.style={font=\small},
    cell/.style={anchor=west, text width=4.2cm},
    note/.style={anchor=west, text width=4.5cm, align=left}
]
    \def\cardwidth{\textwidth}
    \def\cardheight{15.75}
    \def\barheight{0.25}
    \def\barwidth{1.5}
    \def\valuebarspace{0.4}

    % 容器和标题栏背景
    \draw[rounded corners=5, fill=white, draw=gray!20]
        (0,0) rectangle (\cardwidth,-\cardheight);
    \path[fill=customTeal]
        (0,0) [rounded corners=5] -- (\cardwidth,0) --
        (\cardwidth,0.8) -- (0,0.8) -- cycle;

    % 表头
    \node[title, anchor=west] at (0.5,0.4) {\textbf{菌种名称}};
    \node[title] at (4.5,0.4) {\textbf{正常范围}};
    \node[title] at (7.5,0.4) {\textbf{检测丰度}};
    \node[title] at (10.5,0.4) {\textbf{结果评价}};
    \node[title] at (13,0.4) {\textbf{超过\%的人}};
    \node[title] at (16,0.4) {\textbf{有\%的正常人检出}};

    % 初始化位置计数器
    \def\currentpos{0.25}

    % 数据行和卡片
    \foreach \item/\enitem/\value/\range/\percentile/\detection/\status/\intro/\suggestion/\index in {
        {梭菌属}/{Clostridium}/1.90370/{0-4.464}/67\%/99.52\%/正常/{厌氧菌,能产生丁酸盐,参与胆汁酸代谢和色氨酸代谢,对维持肠道屏障功能和免疫系统调节具有重要作用}/{}/\currentpos,
        {普雷沃氏菌属}/{Prevotella}/0.04290/{0-67.8009}/50\%/99.52\%/正常/{革兰氏阴性厌氧菌,专门降解植物多糖和黏蛋白的菌群,产生琥珀酸和乙酸,与植物性饮食密切相关}/{}/\currentpos,
        {瘤胃球菌属}/{Ruminococcus}/9.84521/{0.0544-19.7985}/44\%/99.52\%/正常/{专性厌氧菌,是肠道主要的纤维素降解菌,能产生乙酸和丁酸,对维持结肠上皮细胞健康至关重要}/{}/\currentpos,
        {拟杆菌属}/{Bacteroides}/0.08974/{1.0578-47.3225}/2\%/99.04\%/偏低/{革兰氏阴性厌氧菌,能降解复杂碳水化合物,产生丙酸盐,参与胆固醇代谢,调节宿主免疫系统}/{建议增加全谷物、豆类等膳食纤维摄入}/\currentpos,
        {真杆菌属}/{Eubacterium}/5.75186/{0.1146-9.4883}/95\%/98.56\%/正常/{专性厌氧菌,主要产生丁酸盐,具有抗炎作用,参与胆固醇转化和胆汁酸代谢,维持肠道屏障完整性}/{}/\currentpos,
        {乳酸杆菌属}/{Lactobacillus}/0.00660/{0-0.4302}/8\%/91.83\%/正常/{革兰氏阳性兼性厌氧菌,产生乳酸和抗菌物质,增强肠道屏障功能,调节免疫系统,抑制有害菌生长}/{}/\currentpos,
        {芽孢杆菌属}/{Bacillus}/ND/{0.0001-0.5535}/27\%/77.88\%/未检出/{革兰氏阳性需氧菌,能形成芽孢,产生多种水解酶和抗菌肽,增强肠道免疫功能,改善肠道微生态平衡}/{建议适当补充含芽孢杆菌的活性益生菌制剂}/\currentpos,
        {Lachnoclostridium}/{Lachnoclostridium}/0.17516/{0-0.2086}/8\%/98.56\%/正常/{厌氧产丁酸菌,能降解复杂碳水化合物,产生短链脂肪酸,参与结肠上皮细胞能量代谢,维持肠道健康}/{}/\currentpos
    }
    {
        % 计算当前行的基础位置
        \pgfmathsetmacro{\basepos}{-2.8*\currentpos}

        % 菌种名称
        \node[cell, align=left] at (0.5,\basepos) {
            \small\textbf{\item}\\[-0.2em]
            {\color{gray}\small\enitem}
        };

        % 正常范围
        \node[reference] at (4.5,\basepos) {\footnotesize\range};

        % 进度条相关
        \pgfmathsetmacro{\barypos}{\basepos-\valuebarspace+0.1}
        \def\barstart{6.75}

        % 进度条背景
        \fill[gray!10, rounded corners=2] (\barstart,\barypos)
            rectangle (\barstart+\barwidth,\barypos+\barheight);

        % 检测丰度值
        \node[value] at (7.5,{\basepos-\valuebarspace+0.6}) {\footnotesize\value};

        % 解析范围并计算进度条长度
        \def\parserange#1-#2\endparse{\def\minval{#1}\def\maxval{#2}}
        \expandafter\parserange\range\endparse

        % 计算进度条长度和颜色
        \pgfmathsetmacro{\progress}{min(\value/\maxval, 1.0)}
        \pgfmathparse{\value > \maxval ? "customred" : (\value < \minval ? "customred" : "green!50")}
        \let\barcolor=\pgfmathresult

        % 进度条显示
        \ifnum\pdfstrcmp{\status}{超标}=0
            \fill[customred, rounded corners=2] (\barstart,\barypos)
                rectangle (\barstart+\barwidth,\barypos+\barheight);
        \else
            \fill[\barcolor, rounded corners=2] (\barstart,\barypos)
                rectangle (\barstart+\barwidth*\progress,\barypos+\barheight);
        \fi

        % 结果评价
        \ifnum\pdfstrcmp{\status}{超标}=0
            \node[value, text=customred] at (10.5,\basepos) {\footnotesize\textbf{\status}};
        \else
            \node[value, text=customGreen] at (10.5,\basepos) {\footnotesize\textbf{\status}};
        \fi

        % 丰度超过%人
        \node[value] at (13,\basepos) {\footnotesize\percentile};

        % %正常人检出
        \node[value] at (16,\basepos) {\footnotesize\detection};

        % 添加卡片
        \pgfmathsetmacro{\cardypos}{\basepos-0.5}
        \begin{scope}[shift={(0,\cardypos)}]
            % 卡片背景
            \pgfmathsetmacro{\cardheight}{
                \ifnum\pdfstrcmp{\status}{超标}=0
                    1.0  % 两行内容时的高度
                \else
                    0.6  % 一行内容时的高度
                \fi
            }

            \fill[rounded corners=5pt, customTeal!5, draw=gray!5]
                (0.3,-\cardheight) rectangle (17.3,0);

            % 菌群简介图标和内容
            \node[anchor=west] at (0.5,-0.3) {
                \textbf{\color{gray!90}\footnotesize \textcolor{customTeal}{\faInfoCircle}}
            };
            \node[anchor=west, text width=16cm] at (0.9,-0.3) {
                {\small\color{black}\footnotesize \intro}
            };

            % 异常解读标题和内容
            \ifnum\pdfstrcmp{\status}{超标}=0
                \node[anchor=west] at (0.55,-0.8) {
                    \textbf{\color{customRed}\footnotesize \textcolor{customRed}{\faBell}}
                };
                \node[anchor=west, text width=16cm] at (0.9,-0.8) {
                    {\small\color{gray}\footnotesize \suggestion}
                };
            \fi
        \end{scope}

        % 分割线
        \pgfmathsetmacro{\linepos}{
            \ifnum\pdfstrcmp{\status}{超标}=0
                \basepos-1.7  % 超标时的分割线位置
            \else
                \basepos-1.3  % 正常时的分割线位置
            \fi
        }
        \draw[gray!20] (0.2,\linepos) -- (\cardwidth-0.2,\linepos);

        % 根据当前行的状态计算下一行的位置增量
        \ifnum\pdfstrcmp{\status}{超标}=0
            \pgfmathsetmacro{\increment}{0.85}  % 超标行(两行内容)需要更大的增量
        \else
            \pgfmathsetmacro{\increment}{0.7}  % 正常行(一行内容)使用较小的增量
        \fi

        % 更新位置计数器
        \pgfmathsetmacro{\nextpos}{\currentpos+\increment}
        \xdef\currentpos{\nextpos}
    }

    % 最后一行的处理,消除多余的空白
    \pgfmathsetmacro{\lastincrement}{2}  % 最后一行的增量
    \pgfmathsetmacro{\nextpos}{\currentpos+\lastincrement}
    \xdef\currentpos{\nextpos}

\end{tikzpicture}
\end{center}

\newpage

\begin{center}
\begin{tikzpicture}[
    font=\small,
    title/.style={font=\small\bfseries\color{white}},
    value/.style={font=\small},
    reference/.style={font=\small},
    cell/.style={anchor=west, text width=4.2cm},
    note/.style={anchor=west, text width=4.5cm, align=left}
]
    \def\cardwidth{\textwidth}
    \def\cardheight{21}
    \def\barheight{0.25}
    \def\barwidth{1.5}
    \def\valuebarspace{0.4}

    % 容器和标题栏背景
    \draw[rounded corners=5, fill=white, draw=gray!20]
        (0,0) rectangle (\cardwidth,-\cardheight);
    \path[fill=customTeal]
        (0,0) [rounded corners=5] -- (\cardwidth,0) --
        (\cardwidth,0.8) -- (0,0.8) -- cycle;

    % 表头
    \node[title, anchor=west] at (0.5,0.4) {\textbf{菌种名称}};
    \node[title] at (4.5,0.4) {\textbf{正常范围}};
    \node[title] at (7.5,0.4) {\textbf{检测丰度}};
    \node[title] at (10.5,0.4) {\textbf{结果评价}};
    \node[title] at (13,0.4) {\textbf{超过\%的人}};
    \node[title] at (16,0.4) {\textbf{有\%的正常人检出}};

    % 初始化位置计数器
    \def\currentpos{0.25}

    % 数据行和卡片
    \foreach \item/\enitem/\value/\range/\percentile/\detection/\status/\intro/\suggestion/\index in {
        {粪杆菌属}/{Faecalibacterium}/28.86665/{1.9351-17.7942}/99\%/98.08\%/超标/{专性厌氧菌,是人体肠道中最主要的丁酸盐产生菌之一,具有显著的抗炎作用,通过代谢膳食纤维产生短链脂肪酸,维持肠道屏障功能}/{建议:适当减少膳食纤维摄入,可补充乳酸菌调节菌群平衡;避免过量摄入全谷物}/\currentpos,
        {经黏液真杆菌属}/{Blautia}/10.17279/{0.0846-6.9056}/75\%/97.60\%/超标/{专性厌氧菌,主要产生乙酸盐,参与碳水化合物发酵和氢气代谢,与肠道代谢稳态密切相关}/{建议:控制碳水化合物摄入,减少精制糖和淀粉类食物;可适量补充益生元}/\currentpos,
        {戴阿利斯特杆菌属}/{Dialister}/2.74325/{0-3.7365}/79\%/96.15\%/正常/{革兰氏阴性厌氧菌,参与丙酸盐代谢,与宿主免疫系统调节和炎症反应相关}/{}/\currentpos,
        {罗氏菌属}/{Roseburia}/6.33667/{0.583-16.3581}/10\%/96.15\%/正常/{专性厌氧产丁酸菌,通过降解膳食纤维产生丁酸盐,维持肠道屏障功能,具有显著抗炎作用}/{}/\currentpos,
        {直肠真杆菌属}/{Agathobacter}/0.00038/{0-8.6328}/27\%/95.67\%/正常/{专性厌氧菌,主要降解抗性淀粉,产生乙酸盐和丙酸盐,参与碳水化合物代谢}/{}/\currentpos,
        {吉米菌属}/{Gemmiger}/0.18163/{0.0038-0.973}/9\%/95.67\%/正常/{厌氧菌,参与复杂碳水化合物降解,与维生素代谢和神经递质产生相关}/{}/\currentpos,
        {副拟杆菌属}/{Parabacteroides}/0.00675/{0.1095-8.7303}/9\%/95.19\%/偏低/{革兰氏阴性厌氧菌,参与胆汁酸代谢和脂质代谢,产生琥珀酸和丙酸盐}/{建议:增加抗性淀粉摄入(绿色香蕉、冷却米饭);适量补充含铁食物和发酵食品}/\currentpos,
        {另枝菌属}/{Alistipes}/0.00099/{0.0807-18.1199}/15\%/91.83\%/偏低/{耐胆汁厌氧菌,参与氨基酸代谢,产生吲哚类物质和短链脂肪酸}/{建议:适量增加优质蛋白质摄入;添加深色蔬菜;可搭配咖啡酸类化合物}/\currentpos,
        {毛螺菌属}/{Lachnospira}/5.02851/{0.0347-8.6596}/70\%/88.94\%/正常/{厌氧菌,专门降解果胶的重要菌群,产生乙酸盐和丁酸盐,参与碳水化合物代谢}/{}/\currentpos,
        {考拉杆菌属}/{Phascolarctobacterium}/0.00081/{0-3.7492}/15\%/88.94\%/正常/{专性厌氧菌,利用琥珀酸产生丙酸盐,参与肠道代谢物质转化,与肠道稳态维持相关}/{}/\currentpos
    }
    {
        % 计算当前行的基础位置
        \pgfmathsetmacro{\basepos}{-2.8*\currentpos}

        % 菌种名称
        \node[cell, align=left] at (0.5,\basepos) {
            \small\textbf{\item}\\[-0.2em]
            {\color{lightgray}\small\enitem}
        };

        % 正常范围
        \node[reference] at (4.5,\basepos) {\footnotesize\range};

        % 进度条相关
        \pgfmathsetmacro{\barypos}{\basepos-\valuebarspace+0.1}
        \def\barstart{6.75}

        % 进度条背景
        \fill[gray!10, rounded corners=2] (\barstart,\barypos)
            rectangle (\barstart+\barwidth,\barypos+\barheight);

        % 检测丰度值
        \node[value] at (7.5,{\basepos-\valuebarspace+0.6}) {\footnotesize\value};

        % 解析范围并计算进度条长度
        \def\parserange#1-#2\endparse{\def\minval{#1}\def\maxval{#2}}
        \expandafter\parserange\range\endparse

        % 计算进度条长度和颜色
        \pgfmathsetmacro{\progress}{min(\value/\maxval, 1.0)}
        \pgfmathparse{\value > \maxval ? "customred" : (\value < \minval ? "customred" : "green!50")}
        \let\barcolor=\pgfmathresult

        % 进度条显示
        \ifnum\pdfstrcmp{\status}{超标}=0
            \fill[customred, rounded corners=2] (\barstart,\barypos)
                rectangle (\barstart+\barwidth,\barypos+\barheight);
        \else
            \fill[\barcolor, rounded corners=2] (\barstart,\barypos)
                rectangle (\barstart+\barwidth*\progress,\barypos+\barheight);
        \fi

        % 结果评价
        \ifnum\pdfstrcmp{\status}{超标}=0
            \node[value, text=customred] at (10.5,\basepos) {\footnotesize\textbf{\status}};
        \else
            \node[value, text=customGreen] at (10.5,\basepos) {\footnotesize\textbf{\status}};
        \fi

        % 丰度超过%人
        \node[value] at (13,\basepos) {\footnotesize\percentile};

        % %正常人检出
        \node[value] at (16,\basepos) {\footnotesize\detection};

        % 添加卡片
        \pgfmathsetmacro{\cardypos}{\basepos-0.5}
        \begin{scope}[shift={(0,\cardypos)}]
            % 卡片背景
            \pgfmathsetmacro{\cardheight}{
                \ifnum\pdfstrcmp{\status}{超标}=0
                    1.0  % 两行内容时的高度
                \else
                    0.6  % 一行内容时的高度
                \fi
            }

            \fill[rounded corners=5pt, customTeal!5, draw=gray!5]
                (0.3,-\cardheight) rectangle (17.3,0);

            % 菌群简介图标和内容
            \node[anchor=west] at (0.5,-0.3) {
                \textbf{\color{gray!90}\footnotesize \textcolor{customTeal}{\faInfoCircle}}
            };
            \node[anchor=west, text width=16cm] at (0.9,-0.3) {
                {\small\color{gray}\footnotesize \intro}
            };

            % 异常解读标题和内容
            \ifnum\pdfstrcmp{\status}{超标}=0
                \node[anchor=west] at (0.55,-0.8) {
                    \textbf{\color{customRed}\footnotesize \textcolor{customRed}{\faBell}}
                };
                \node[anchor=west, text width=16cm] at (0.9,-0.8) {
                    {\small\color{gray}\footnotesize \suggestion}
                };
            \fi
        \end{scope}

        % 分割线
        \pgfmathsetmacro{\linepos}{
            \ifnum\pdfstrcmp{\status}{超标}=0
                \basepos-1.7  % 超标时的分割线位置
            \else
                \basepos-1.3  % 正常时的分割线位置
            \fi
        }
        \draw[gray!20] (0.2,\linepos) -- (\cardwidth-0.2,\linepos);

        % 根据当前行的状态计算下一行的位置增量
        \ifnum\pdfstrcmp{\status}{超标}=0
            \pgfmathsetmacro{\increment}{0.85}  % 超标行(两行内容)需要更大的增量
        \else
            \pgfmathsetmacro{\increment}{0.7}  % 正常行(一行内容)使用较小的增量
        \fi

        % 更新位置计数器
        \pgfmathsetmacro{\nextpos}{\currentpos+\increment}
        \xdef\currentpos{\nextpos}
    }

    % 最后一行的处理,消除多余的空白
    \pgfmathsetmacro{\lastincrement}{2}  % 最后一行的增量
    \pgfmathsetmacro{\nextpos}{\currentpos+\lastincrement}
    \xdef\currentpos{\nextpos}

\end{tikzpicture}
\end{center}

\newpage

\begin{center}
\begin{tikzpicture}[
    font=\small,
    title/.style={font=\small\bfseries\color{white}},
    value/.style={font=\small},
    reference/.style={font=\small},
    cell/.style={anchor=west, text width=4.2cm},
    note/.style={anchor=west, text width=4.5cm, align=left}
]
    \def\cardwidth{\textwidth}
    \def\cardheight{15.75}
    \def\barheight{0.25}
    \def\barwidth{1.5}
    \def\valuebarspace{0.4}

    % 容器和标题栏背景
    \draw[rounded corners=5, fill=white, draw=gray!20]
        (0,0) rectangle (\cardwidth,-\cardheight);
    \path[fill=customTeal]
        (0,0) [rounded corners=5] -- (\cardwidth,0) --
        (\cardwidth,0.8) -- (0,0.8) -- cycle;

    % 表头
    \node[title, anchor=west] at (0.5,0.4) {\textbf{菌种名称}};
    \node[title] at (4.5,0.4) {\textbf{正常范围}};
    \node[title] at (7.5,0.4) {\textbf{检测丰度}};
    \node[title] at (10.5,0.4) {\textbf{结果评价}};
    \node[title] at (13,0.4) {\textbf{超过\%的人}};
    \node[title] at (16,0.4) {\textbf{有\%的正常人检出}};

    % 初始化位置计数器
    \def\currentpos{0.25}

    % 数据行和卡片
    \foreach \item/\enitem/\value/\range/\percentile/\detection/\status/\intro/\suggestion/\index in {
        {巨单胞菌属}/{Megamonas}/0.00049/{0-0.6916}/58\%/87.50\%/正常/{厌氧菌,专门降解复杂碳水化合物,产生乙酸盐、丙酸盐和乳酸盐,参与肠道能量代谢,与神经系统功能和免疫调节相关}/{}/\currentpos,
        {粪球菌属}/{Coprococcus}/0.05595/{0.0826-9.4195}/64\%/86.06\%/偏低/{专性厌氧产丁酸菌,通过代谢膳食纤维产生丁酸盐,具有免疫调节功能,参与色氨酸代谢,与神经递质合成相关}/{建议:补充水溶性膳食纤维(菊粉、燕麦β-葡聚糖);适量添加全谷物;可搭配益生元}/\currentpos,
        {双歧杆菌属}/{Bifidobacterium}/0.03296/{1.7546-35.5006}/24\%/97.12\%/偏低/{革兰氏阳性厌氧菌,核心益生菌,产生乙酸盐和乳酸盐,合成B族维生素和GABA,增强肠道屏障功能,调节免疫系统}/{建议:补充低聚果糖(FOS)和菊粉;添加发酵乳制品;可选用含双歧杆菌的益生菌制剂;减少精制糖摄入}/\currentpos,
        {乳杆菌属}/{Lactobacillus}/0.00660/{0-0.4302}/8\%/91.83\%/正常/{革兰氏阳性兼性厌氧菌,产生乳酸和抗菌物质,参与碳水化合物发酵,合成维生素B族和维生素K,增强肠道屏障功能}/{}/\currentpos,
        {芽孢杆菌属}/{Bacillus}/ND/{0.0001-0.5535}/27\%/77.88\%/未检出/{革兰氏阳性需氧菌,产生多种水解酶和抗菌肽,增强肠道免疫功能,参与碳水化合物和蛋白质代谢}/{建议:补充含枯草芽孢杆菌的益生菌制剂;增加发酵食品摄入;适量添加膳食纤维}/\currentpos
    }
    {
        % 计算当前行的基础位置
        \pgfmathsetmacro{\basepos}{-2.8*\currentpos}

        % 菌种名称
        \node[cell, align=left] at (0.5,\basepos) {
            \small\textbf{\item}\\[-0.2em]
            {\color{lightgray}\small\enitem}
        };

        % 正常范围
        \node[reference] at (4.5,\basepos) {\footnotesize\range};

        % 进度条相关
        \pgfmathsetmacro{\barypos}{\basepos-\valuebarspace+0.1}
        \def\barstart{6.75}

        % 进度条背景
        \fill[gray!10, rounded corners=2] (\barstart,\barypos)
            rectangle (\barstart+\barwidth,\barypos+\barheight);

        % 检测丰度值
        \node[value] at (7.5,{\basepos-\valuebarspace+0.6}) {\footnotesize\value};

        % 解析范围并计算进度条长度
        \def\parserange#1-#2\endparse{\def\minval{#1}\def\maxval{#2}}
        \expandafter\parserange\range\endparse

        % 计算进度条长度和颜色
        \pgfmathsetmacro{\progress}{min(\value/\maxval, 1.0)}
        \pgfmathparse{\value > \maxval ? "customred" : (\value < \minval ? "customred" : "green!50")}
        \let\barcolor=\pgfmathresult

        % 进度条显示
        \ifnum\pdfstrcmp{\status}{超标}=0
            \fill[customred, rounded corners=2] (\barstart,\barypos)
                rectangle (\barstart+\barwidth,\barypos+\barheight);
        \else
            \fill[\barcolor, rounded corners=2] (\barstart,\barypos)
                rectangle (\barstart+\barwidth*\progress,\barypos+\barheight);
        \fi

        % 结果评价
        \ifnum\pdfstrcmp{\status}{超标}=0
            \node[value, text=customRed] at (10.5,\basepos) {\footnotesize\textbf{\status}};
        \else
            \node[value, text=customGreen] at (10.5,\basepos) {\footnotesize\textbf{\status}};
        \fi

        % 丰度超过%人
        \node[value] at (13,\basepos) {\footnotesize\percentile};

        % %正常人检出
        \node[value] at (16,\basepos) {\footnotesize\detection};

        % 添加卡片
        \pgfmathsetmacro{\cardypos}{\basepos-0.5}
        \begin{scope}[shift={(0,\cardypos)}]
            % 卡片背景
            \pgfmathsetmacro{\cardheight}{
                \ifnum\pdfstrcmp{\status}{超标}=0
                    1.0  % 两行内容时的高度
                \else
                    0.6  % 一行内容时的高度
                \fi
            }

            \fill[rounded corners=5pt, customTeal!5, draw=gray!5]
                (0.3,-\cardheight) rectangle (17.3,0);

            % 菌群简介图标和内容
            \node[anchor=west] at (0.5,-0.3) {
                \textbf{\color{gray!90}\footnotesize \textcolor{customTeal}{\faInfoCircle}}
            };
            \node[anchor=west, text width=16cm] at (0.9,-0.3) {
                {\small\color{gray}\footnotesize \intro}
            };

            % 异常解读标题和内容
            \ifnum\pdfstrcmp{\status}{超标}=0
                \node[anchor=west] at (0.55,-0.8) {
                    \textbf{\color{customRed}\footnotesize \textcolor{customRed}{\faBell}}
                };
                \node[anchor=west, text width=16cm] at (0.9,-0.8) {
                    {\small\color{gray}\footnotesize \suggestion}
                };
            \fi
        \end{scope}

        % 分割线
        \pgfmathsetmacro{\linepos}{
            \ifnum\pdfstrcmp{\status}{超标}=0
                \basepos-1.7  % 超标时的分割线位置
            \else
                \basepos-1.3  % 正常时的分割线位置
            \fi
        }
        \draw[gray!20] (0.2,\linepos) -- (\cardwidth-0.2,\linepos);

        % 根据当前行的状态计算下一行的位置增量
        \ifnum\pdfstrcmp{\status}{超标}=0
            \pgfmathsetmacro{\increment}{0.85}  % 超标行(两行内容)需要更大的增量
        \else
            \pgfmathsetmacro{\increment}{0.7}  % 正常行(一行内容)使用较小的增量
        \fi

        % 更新位置计数器
        \pgfmathsetmacro{\nextpos}{\currentpos+\increment}
        \xdef\currentpos{\nextpos}
    }

    % 最后一行的处理,消除多余的空白
    \pgfmathsetmacro{\lastincrement}{2}  % 最后一行的增量
    \pgfmathsetmacro{\nextpos}{\currentpos+\lastincrement}
    \xdef\currentpos{\nextpos}

\end{tikzpicture}
\end{center}

\newpage

\begin{tcolorbox}[
    enhanced,
    colback=white,
    colframe=white,
    arc=2mm,
    boxrule=0pt,
    width=\textwidth,
    left=15pt,
    right=15pt,
    top=10pt,
    bottom=10pt,
    drop shadow={
        opacity=0.2,
        color=customTeal
    },
    borderline west={5pt}{0pt}{customTeal}
]
\textcolor{customTeal}{\large\textbf{其他非核心重要菌属}}
\end{tcolorbox}
\vspace{0.05cm}
\begin{tcolorbox}[
    enhanced,
    colback=customTealBg,
    colframe=customTealBg,
    arc=3mm,
    boxrule=0pt,
    width=\textwidth,
    top=8pt,
    bottom=8pt
]
{\small{\color{customTeal}\faInfoCircle} 其他非核心重要菌属指的是在人群中丰度不高,但对肠道健康有重要影响的菌属。
}
\end{tcolorbox}

\begin{tcolorbox}[
    enhanced,
    colback=lightpurple!5, % 卡片底色
    colframe=white,  % 边框颜色
    arc=3mm,
    boxrule=0.5pt,
    width=\textwidth,
    top=8pt,
    bottom=8pt
]
{\small{\color{lightpurple}\faQuestionCircle}\quad \textbf{如何阅读这个表格中的各项指标?}\\
{\color{orange!50}\faComments}\quad 下列为各项指标的解读方式:
\begin{itemize}
    \item 菌种名称:分别提供了各个其他非核心重要菌属的中文学术名和拉丁名。
    \item 正常范围:表示该菌种在健康人群中的丰度范围。
    \item 检测丰度:实际检测到的该菌种的数量,其中ND代表该菌属的丰度过低未检测到。
    \item 结果评价:根据检测丰度与正常范围的比较,给出的健康状态评估。
    \item 超过\%的人:表示您肠道检测的该菌属的丰度值比人群中\%的人要高。(-\%代表该菌属的丰度超标过高)
    \item 有\%的正常人检出:表示在正常人群中,有\%的健康人能够检测到该菌属。
\end{itemize}
}
\end{tcolorbox}
\vspace{-0.5cm}
\begin{center}
\begin{tikzpicture}[
    font=\small,
    title/.style={font=\small\bfseries\color{white}},
    value/.style={font=\small},
    reference/.style={font=\small},
    cell/.style={anchor=west, text width=4.2cm},
    note/.style={anchor=west, text width=4.5cm, align=left}
]
    \def\cardwidth{\textwidth}
    \def\cardheight{13}
    \def\barheight{0.25}
    \def\barwidth{1.5}
    \def\valuebarspace{0.4}

    % 容器和标题栏背景
    \draw[rounded corners=5, fill=white, draw=gray!20]
        (0,0) rectangle (\cardwidth,-\cardheight);
    \path[fill=customTeal]
        (0,0) [rounded corners=5] -- (\cardwidth,0) --
        (\cardwidth,0.8) -- (0,0.8) -- cycle;

    % 表头
    \node[title, anchor=west] at (0.5,0.4) {\textbf{菌种名称}};
    \node[title] at (4.5,0.4) {\textbf{正常范围}};
    \node[title] at (7.5,0.4) {\textbf{检测丰度}};
    \node[title] at (10.5,0.4) {\textbf{结果评价}};
    \node[title] at (13,0.4) {\textbf{超过\%的人}};
    \node[title] at (16,0.4) {\textbf{有\%的正常人检出}};

    % 初始化位置计数器
    \def\currentpos{0.25}

    % 数据行和卡片
    \foreach \item/\enitem/\value/\range/\percentile/\detection/\status/\intro/\suggestion/\index in {
        {柯林斯氏菌属}/{Collinsella}/0.01763/{0-9.1529}/82\%/90.38\%/正常/{柯林斯氏菌是一类革兰氏阳性细菌,其水平与肥胖、糖尿病和心血管疾病等代谢病风险相关。}/{}/\currentpos,
        {嗜酸菌属}/{Bilophila}/0.73111/{0-0.1545}/- \%/66.83\%/超标/{嗜酸菌属是一类革兰氏阴性细菌,在肠道中参与氨基酸的代谢,其丰度可能与宿主的营养状态和肠道健康相关。}/{为改善嗜酸菌属超标,建议增加膳食纤维与发酵食品的摄入,减少糖和精制碳水化合物,同时限制红肉和加工肉制品。}/\currentpos,
        {黄酮还原菌属}/{Flavonifractor}/2.44087/{0-0.7285}/57\%/91.83\%/超标/{黄酮还原菌属通过发酵黄酮等植物化合物来获取能量,参与植物性膳食成分的降解。}/{减少黄酮含量高的食物:限制富含黄酮的食物如水果、蔬菜和茶类。}/\currentpos,
        {多雷氏菌属}/{Dorea}/0.43735/{0.0581-5.212}/20\%/89.42\%/正常/{多雷氏菌属是一种厌氧菌,主要通过发酵作用代谢碳水化合物,尤其是膳食纤维和复杂的多糖,分解膳食纤维,产生短链脂肪酸。}/{}/\currentpos,
        {洪氏菌属}/{Hungatella}/0.14743/{0-0.027}/18\%/89.42\%/超标/{洪氏菌属是一种厌氧菌,主要通过发酵代谢有机物,尤其是耐消化的碳水化合物,如纤维和多糖,有助于分解食物残渣,产生短链脂肪酸。}/{减少高糖和高脂肪食品,增加膳食纤维,摄入发酵食品,保持水分。}/\currentpos,
        {颤螺菌属}/{Oscillospira}/ND/{0.0033-0.5346}/36\%/87.50\%/未检出/{颤螺菌属是一种厌氧细菌,参与分解和发酵食物成分,能够代谢一些膳食纤维和其他复杂的碳水化合物,产生短链脂肪酸,促进肠道健康。}/{建议适当补充益生菌}/\currentpos
    }
    {
        % 计算当前行的基础位置
        \pgfmathsetmacro{\basepos}{-2.8*\currentpos}

        % 菌种名称
        \node[cell, align=left] at (0.5,\basepos) {
            \small\textbf{\item}\\[-0.2em]
            {\color{lightgray}\small\enitem}
        };

        % 正常范围
        \node[reference] at (4.5,\basepos) {\footnotesize\range};

        % 进度条相关
        \pgfmathsetmacro{\barypos}{\basepos-\valuebarspace+0.1}
        \def\barstart{6.75}

        % 进度条背景
        \fill[gray!10, rounded corners=2] (\barstart,\barypos)
            rectangle (\barstart+\barwidth,\barypos+\barheight);

        % 检测丰度值
        \node[value] at (7.5,{\basepos-\valuebarspace+0.6}) {\footnotesize\value};

        % 解析范围并计算进度条长度
        \def\parserange#1-#2\endparse{\def\minval{#1}\def\maxval{#2}}
        \expandafter\parserange\range\endparse

        % 计算进度条长度和颜色
        \pgfmathsetmacro{\progress}{min(\value/\maxval, 1.0)}
        \pgfmathparse{\value > \maxval ? "customred" : (\value < \minval ? "customred" : "green!50")}
        \let\barcolor=\pgfmathresult

        % 进度条显示
        \ifnum\pdfstrcmp{\status}{超标}=0
            \fill[customred, rounded corners=2] (\barstart,\barypos)
                rectangle (\barstart+\barwidth,\barypos+\barheight);
        \else
            \fill[\barcolor, rounded corners=2] (\barstart,\barypos)
                rectangle (\barstart+\barwidth*\progress,\barypos+\barheight);
        \fi

        % 结果评价
        \ifnum\pdfstrcmp{\status}{超标}=0
            \node[value, text=customRed] at (10.5,\basepos) {\footnotesize\textbf{\status}};
        \else
            \node[value, text=customGreen] at (10.5,\basepos) {\footnotesize\textbf{\status}};
        \fi

        % 丰度超过%人
        \node[value] at (13,\basepos) {\footnotesize\percentile};

        % %正常人检出
        \node[value] at (16,\basepos) {\footnotesize\detection};

        % 添加卡片
        \pgfmathsetmacro{\cardypos}{\basepos-0.5}
        \begin{scope}[shift={(0,\cardypos)}]
            % 卡片背景
            \pgfmathsetmacro{\cardheight}{
                \ifnum\pdfstrcmp{\status}{超标}=0
                    1.0  % 两行内容时的高度
                \else
                    0.6  % 一行内容时的高度
                \fi
            }

            \fill[rounded corners=5pt, customTeal!5, draw=gray!5]
                (0.3,-\cardheight) rectangle (17.3,0);

            % 菌群简介图标和内容
            \node[anchor=west] at (0.5,-0.3) {
                \textbf{\color{gray!90}\footnotesize \textcolor{customTeal}{\faInfoCircle}}
            };
            \node[anchor=west, text width=16cm] at (0.9,-0.3) {
                {\small\color{gray}\footnotesize \intro}
            };

            % 异常解读标题和内容
            \ifnum\pdfstrcmp{\status}{超标}=0
                \node[anchor=west] at (0.55,-0.8) {
                    \textbf{\color{customRed}\footnotesize \textcolor{customRed}{\faBell}}
                };
                \node[anchor=west, text width=16cm] at (0.9,-0.8) {
                    {\small\color{gray}\footnotesize \suggestion}
                };
            \fi
        \end{scope}

        % 分割线
        \pgfmathsetmacro{\linepos}{
            \ifnum\pdfstrcmp{\status}{超标}=0
                \basepos-1.7  % 超标时的分割线位置
            \else
                \basepos-1.3  % 正常时的分割线位置
            \fi
        }
        \draw[gray!20] (0.2,\linepos) -- (\cardwidth-0.2,\linepos);

        % 根据当前行的状态计算下一行的位置增量
        \ifnum\pdfstrcmp{\status}{超标}=0
            \pgfmathsetmacro{\increment}{0.85}  % 超标行(两行内容)需要更大的增量
        \else
            \pgfmathsetmacro{\increment}{0.7}  % 正常行(一行内容)使用较小的增量
        \fi

        % 更新位置计数器
        \pgfmathsetmacro{\nextpos}{\currentpos+\increment}
        \xdef\currentpos{\nextpos}
    }

    % 最后一行的处理,消除多余的空白
    \pgfmathsetmacro{\lastincrement}{2}  % 最后一行的增量
    \pgfmathsetmacro{\nextpos}{\currentpos+\lastincrement}
    \xdef\currentpos{\nextpos}

\end{tikzpicture}
\end{center}

\newpage

\begin{center}
\begin{tikzpicture}[
    font=\small,
    title/.style={font=\small\bfseries\color{white}},
    value/.style={font=\small},
    reference/.style={font=\small},
    cell/.style={anchor=west, text width=4.2cm},
    note/.style={anchor=west, text width=4.5cm, align=left}
]
    \def\cardwidth{\textwidth}
    \def\cardheight{6}
    \def\barheight{0.25}
    \def\barwidth{1.5}
    \def\valuebarspace{0.4}

    % 容器和标题栏背景
    \draw[rounded corners=5, fill=white, draw=gray!20]
        (0,0) rectangle (\cardwidth,-\cardheight);
    \path[fill=customTeal]
        (0,0) [rounded corners=5] -- (\cardwidth,0) --
        (\cardwidth,0.8) -- (0,0.8) -- cycle;

    % 表头
    \node[title, anchor=west] at (0.5,0.4) {\textbf{菌种名称}};
    \node[title] at (4.5,0.4) {\textbf{正常范围}};
    \node[title] at (7.5,0.4) {\textbf{检测丰度}};
    \node[title] at (10.5,0.4) {\textbf{结果评价}};
    \node[title] at (13,0.4) {\textbf{超过\%的人}};
    \node[title] at (16,0.4) {\textbf{有\%的正常人检出}};

    % 初始化位置计数器
    \def\currentpos{0.25}

    % 数据行和卡片
    \foreach \item/\enitem/\value/\range/\percentile/\detection/\status/\intro/\suggestion/\index in {
        {图里希菌属}/{Turicibacter}/0.00252/{0-0.1234}/-\%/49.52\%/正常/{革兰氏阳性厌氧菌,主要产生乳酸盐,参与色氨酸代谢,与肌肉能量代谢和神经递质合成密切相关,在运动相关的肠-肌轴中发挥重要作用}/{}/\currentpos,
        {丁酸弧菌属}/{Butyrivibrio}/0.02421/{0-3.2594}/98\%/35.58\%/正常/{专性厌氧菌,主要的丁酸盐产生菌之一,能高效降解复杂碳水化合物(纤维素、半纤维素、果胶),产生丁酸盐和共轭亚油酸,参与肠道屏障功能维持}/{}/\currentpos,
        {克里斯滕森菌属}/{Christensenella}/0.00192/{0-0.0054}/83\%/5.77\%/正常/{专性厌氧革兰氏阴性菌,产生乙酸盐和丁酸盐,具有显著的遗传相关性,参与脂质代谢和葡萄糖稳态,与宿主代谢健康密切相关}/{}/\currentpos
    }
    {
        % 计算当前行的基础位置
        \pgfmathsetmacro{\basepos}{-2.8*\currentpos}

        % 菌种名称
        \node[cell, align=left] at (0.5,\basepos) {
            \small\textbf{\item}\\[-0.2em]
            {\color{gray}\small\enitem}
        };

        % 正常范围
        \node[reference] at (4.5,\basepos) {\footnotesize\range};

        % 进度条相关
        \pgfmathsetmacro{\barypos}{\basepos-\valuebarspace+0.1}
        \def\barstart{6.75}

        % 进度条背景
        \fill[gray!10, rounded corners=2] (\barstart,\barypos)
            rectangle (\barstart+\barwidth,\barypos+\barheight);

        % 检测丰度值
        \node[value] at (7.5,{\basepos-\valuebarspace+0.6}) {\footnotesize\value};

        % 解析范围并计算进度条长度
        \def\parserange#1-#2\endparse{\def\minval{#1}\def\maxval{#2}}
        \expandafter\parserange\range\endparse

        % 计算进度条长度和颜色
        \pgfmathsetmacro{\progress}{min(\value/\maxval, 1.0)}
        \pgfmathparse{\value > \maxval ? "customred" : (\value < \minval ? "customred" : "green!50")}
        \let\barcolor=\pgfmathresult

        % 进度条显示
        \ifnum\pdfstrcmp{\status}{超标}=0
            \fill[customred, rounded corners=2] (\barstart,\barypos)
                rectangle (\barstart+\barwidth,\barypos+\barheight);
        \else
            \fill[\barcolor, rounded corners=2] (\barstart,\barypos)
                rectangle (\barstart+\barwidth*\progress,\barypos+\barheight);
        \fi

        % 结果评价
        \ifnum\pdfstrcmp{\status}{超标}=0
            \node[value, text=customRed] at (10.5,\basepos) {\footnotesize\textbf{\status}};
        \else
            \node[value, text=customGreen] at (10.5,\basepos) {\footnotesize\textbf{\status}};
        \fi

        % 丰度超过%人
        \node[value] at (13,\basepos) {\footnotesize\percentile};

        % %正常人检出
        \node[value] at (16,\basepos) {\footnotesize\detection};

        % 添加卡片
        \pgfmathsetmacro{\cardypos}{\basepos-0.5}
        \begin{scope}[shift={(0,\cardypos)}]
            % 卡片背景
            \pgfmathsetmacro{\cardheight}{
                \ifnum\pdfstrcmp{\status}{超标}=0
                    1.0  % 两行内容时的高度
                \else
                    0.6  % 一行内容时的高度
                \fi
            }

            \fill[rounded corners=5pt, customTeal!5, draw=gray!5]
                (0.3,-\cardheight) rectangle (17.3,0);

            % 菌群简介图标和内容
            \node[anchor=west] at (0.5,-0.3) {
                \textbf{\color{gray!90}\footnotesize \textcolor{customTeal}{\faInfoCircle}}
            };
            \node[anchor=west, text width=16cm] at (0.9,-0.3) {
                {\small\color{gray}\footnotesize \intro}
            };

            % 异常解读标题和内容
            \ifnum\pdfstrcmp{\status}{超标}=0
                \node[anchor=west] at (0.55,-0.8) {
                    \textbf{\color{customRed}\footnotesize \textcolor{customRed}{\faBell}}
                };
                \node[anchor=west, text width=16cm] at (0.9,-0.8) {
                    {\small\color{gray}\footnotesize \suggestion}
                };
            \fi
        \end{scope}

        % 分割线
        \pgfmathsetmacro{\linepos}{
            \ifnum\pdfstrcmp{\status}{超标}=0
                \basepos-1.7  % 超标时的分割线位置
            \else
                \basepos-1.3  % 正常时的分割线位置
            \fi
        }
        \draw[gray!20] (0.2,\linepos) -- (\cardwidth-0.2,\linepos);

        % 根据当前行的状态计算下一行的位置增量
        \ifnum\pdfstrcmp{\status}{超标}=0
            \pgfmathsetmacro{\increment}{0.85}  % 超标行(两行内容)需要更大的增量
        \else
            \pgfmathsetmacro{\increment}{0.7}  % 正常行(一行内容)使用较小的增量
        \fi

        % 更新位置计数器
        \pgfmathsetmacro{\nextpos}{\currentpos+\increment}
        \xdef\currentpos{\nextpos}
    }

    % 最后一行的处理,消除多余的空白
    \pgfmathsetmacro{\lastincrement}{2}  % 最后一行的增量
    \pgfmathsetmacro{\nextpos}{\currentpos+\lastincrement}
    \xdef\currentpos{\nextpos}

\end{tikzpicture}
\end{center}

\newpage

\begin{tcolorbox}[
    enhanced,
    colback=white,
    colframe=white,
    arc=2mm,
    boxrule=0pt,
    width=\textwidth,
    left=15pt,
    right=15pt,
    top=10pt,
    bottom=10pt,
    drop shadow={
        opacity=0.2,
        color=customTeal
    },
    borderline west={5pt}{0pt}{customTeal}
]
\textcolor{customTeal}{\large\textbf{(常见)有益菌}}
\end{tcolorbox}
\vspace{0.05cm}
\begin{tcolorbox}[
    enhanced,
    colback=customTealBg,
    colframe=customTealBg,
    arc=3mm,
    boxrule=0pt,
    width=\textwidth,
    top=8pt,
    bottom=8pt
]
{\small{\color{customTeal}\faInfoCircle} 本报告列出了54种人体常见的肠道有益菌,有益菌能够调节肠道内生态平衡,促进人体健康。
}
\end{tcolorbox}

\begin{tcolorbox}[
    enhanced,
    colback=lightpurple!10, % 卡片底色
    colframe=lightpurple!10,  % 边框颜色
    arc=3mm,
    boxrule=0.5pt,
    width=\textwidth,
    top=8pt,
    bottom=8pt
]
{\small{\color{lightpurple}\faQuestionCircle}\quad \textbf{什么是肠道有益菌?}\\
{\color{orange!50}\faComments}\quad 有益菌是指那些对宿主健康有积极作用的微生物,它们通过与宿主和其他微生物的相互作用,在维持机体健康中发挥重要作用。从功能角度来看,有益菌主要通过产生营养物质、调节免疫系统、维持肠道屏障完整性、抑制致病菌生长等多种途径发挥作用,对预防疾病、增强免疫力和改善肠道健康具有重要意义。典型的有益菌包括:乳酸杆菌属(Lactobacillus)和双歧杆菌属(Bifidobacterium),它们能产生乳酸等有机酸,抑制有害菌生长;布特菌属(Butyrivibrio)能产生丁酸等短链脂肪酸;粪肠球菌属(Faecalibacterium)则具有显著的抗炎特性。
}
\end{tcolorbox}

\begin{tcolorbox}[
    enhanced,
    colback=lightpurple!10, % 卡片底色
    colframe=lightpurple!10,  % 边框颜色
    arc=3mm,
    boxrule=0.5pt,
    width=\textwidth,
    top=8pt,
    bottom=8pt
]
{\small{\color{lightpurple}\faQuestionCircle}\quad \textbf{肠道有益菌越多越好吗?}\\
{\color{orange!50}\faComments}\quad 肠道有益菌对健康至关重要,适度的数量和多样性能够增强免疫功能、改善消化和维护整体健康。然而,肠道微生物群是复杂的,过度生长或某一特定菌种的过量并不一定是有益的。维护肠道健康的关键是实现菌群的多样性与平衡,以及配合健康的饮食和生活方式。
}
\end{tcolorbox}

\begin{center}
\begin{tikzpicture}[
    font=\small,
    title/.style={font=\small\bfseries\color{white}},
    value/.style={font=\small},
    reference/.style={font=\small},
    cell/.style={anchor=west, text width=4.2cm},
    note/.style={anchor=west, text width=4.5cm, align=left}
]
    \def\cardwidth{\textwidth}
    \def\cardheight{12}
    \def\barheight{0.25}
    \def\barwidth{1.5}
    \def\valuebarspace{0.4}

    % 容器和标题栏背景
    \draw[rounded corners=5, fill=white, draw=gray!20]
        (0,0) rectangle (\cardwidth,-\cardheight);
    \path[fill=customTeal]
        (0,0) [rounded corners=5] -- (\cardwidth,0) --
        (\cardwidth,0.8) -- (0,0.8) -- cycle;

    % 表头
    \node[title, anchor=west] at (0.5,0.4) {\textbf{菌种名称}};
    \node[title] at (4.5,0.4) {\textbf{正常范围}};
    \node[title] at (7.5,0.4) {\textbf{检测丰度}};
    \node[title] at (10.5,0.4) {\textbf{结果评价}};
    \node[title] at (13,0.4) {\textbf{超过\%的人}};
    \node[title] at (16,0.4) {\textbf{有\%的正常人检出}};

    % 初始化位置计数器
    \def\currentpos{0.25}

    % 数据行和卡片
    \foreach \item/\enitem/\value/\range/\percentile/\detection/\status/\intro/\suggestion/\index in {
        {嗜粘蛋白-艾克曼菌}/{A. muciniphila}/0.00394/{0-6.6395}/27\%/95.67\%/丰度较低/{嗜粘蛋白-艾克曼菌是一种革兰氏阴性细菌,能够分解肠道黏液,有助于维持肠道屏障功能和调节代谢健康。}/{}/\currentpos,
        {凝结芽孢杆菌}/{B. coagulans}/0.00027/{0-0.08}/15\%/88.94\%/丰度较低/{凝结芽孢杆菌是一种耐热的革兰氏阳性芽孢杆菌,常用于作为益生菌,具有促进肠道健康和改善消化的潜在益处。}/{}/\currentpos,
        {枯草芽孢杆菌}/{B. subtilis}/0.00185/{0-0.08}/58\%/87.50\%/丰度较低/{枯草芽孢杆菌是一种广泛存在的革兰氏阳性芽孢杆菌,常用于发酵食品,有助于改善肠道微生态和促进植物生长。}/{}/\currentpos,
        {多形拟杆菌}/{B. thetaiotaomicron}/0.00020/{0-2.7856}/64\%/86.06\%/丰度较低/{多形拟杆菌是一种革兰氏阴性厌氧细菌,能够有效分解复杂碳水化合物,并在维持肠道微生态平衡和健康方面发挥重要作用。}/{}/\currentpos,
        {青春双歧杆菌}/{B. adolescentis}/ND/{0-14.286}/24\%/97.12\%/未检出/{青春双歧杆菌是一种革兰氏阳性益生菌,参与消化过程并有助于维持肠道健康,常被作为益生菌补充剂的一部分。}/{建议补充益生菌制剂}/\currentpos,
        {角双歧杆菌}/{B. angulatum}/ND/{0-0.15257}/8\%/91.83\%/未检出/{角双歧杆菌是一种革兰氏阳性益生菌,有助于维持肠道微生物平衡,并可能对宿主的免疫系统发挥积极作用。}/{建议补充益生菌制剂}/\currentpos
    }
    {
        % 计算当前行的基础位置
        \pgfmathsetmacro{\basepos}{-2.8*\currentpos}

        % 菌种名称
        \node[cell, align=left] at (0.5,\basepos) {
            \small\textbf{\item}\\[-0.2em]
            {\color{lightgray}\small\enitem}
        };

        % 正常范围
        \node[reference] at (4.5,\basepos) {\footnotesize\range};

        % 进度条相关
        \pgfmathsetmacro{\barypos}{\basepos-\valuebarspace+0.1}
        \def\barstart{6.75}

        % 进度条背景
        \fill[gray!10, rounded corners=2] (\barstart,\barypos)
            rectangle (\barstart+\barwidth,\barypos+\barheight);

        % 检测丰度值
        \node[value] at (7.5,{\basepos-\valuebarspace+0.6}) {\footnotesize\value};

        % 解析范围并计算进度条长度
        \def\parserange#1-#2\endparse{\def\minval{#1}\def\maxval{#2}}
        \expandafter\parserange\range\endparse

        % 计算进度条长度和颜色
        \pgfmathsetmacro{\progress}{min(\value/\maxval, 1.0)}
        \pgfmathparse{\value > \maxval ? "customred" : (\value < \minval ? "customred" : "green!50")}
        \let\barcolor=\pgfmathresult

        % 进度条显示
        \ifnum\pdfstrcmp{\status}{超标}=0
            \fill[customred, rounded corners=2] (\barstart,\barypos)
                rectangle (\barstart+\barwidth,\barypos+\barheight);
        \else
            \fill[\barcolor, rounded corners=2] (\barstart,\barypos)
                rectangle (\barstart+\barwidth*\progress,\barypos+\barheight);
        \fi

        % 结果评价
        \ifnum\pdfstrcmp{\status}{超标}=0
            \node[value, text=customred] at (10.5,\basepos) {\footnotesize\textbf{\status}};
        \else
            \node[value, text=customGreen] at (10.5,\basepos) {\footnotesize\textbf{\status}};
        \fi

        % 丰度超过%人
        \node[value] at (13,\basepos) {\footnotesize\percentile};

        % %正常人检出
        \node[value] at (16,\basepos) {\footnotesize\detection};

        % 添加卡片
        \pgfmathsetmacro{\cardypos}{\basepos-0.5}
        \begin{scope}[shift={(0,\cardypos)}]
            % 卡片背景
            \pgfmathsetmacro{\cardheight}{
                \ifnum\pdfstrcmp{\status}{超标}=0
                    1.0  % 两行内容时的高度
                \else
                    0.6  % 一行内容时的高度
                \fi
            }

            \fill[rounded corners=5pt, customTeal!5, draw=gray!5]
                (0.3,-\cardheight) rectangle (17.3,0);

            % 菌群简介图标和内容
            \node[anchor=west] at (0.5,-0.3) {
                \textbf{\color{gray!90}\footnotesize \textcolor{customTeal}{\faInfoCircle}}
            };
            \node[anchor=west, text width=16cm] at (0.9,-0.3) {
                {\small\color{gray}\footnotesize \intro}
            };

            % 异常解读标题和内容
            \ifnum\pdfstrcmp{\status}{超标}=0
                \node[anchor=west] at (0.55,-0.8) {
                    \textbf{\color{customRed}\footnotesize \textcolor{customRed}{\faBell}}
                };
                \node[anchor=west, text width=16cm] at (0.9,-0.8) {
                    {\small\color{gray}\footnotesize \suggestion}
                };
            \fi
        \end{scope}

        % 分割线
        \pgfmathsetmacro{\linepos}{
            \ifnum\pdfstrcmp{\status}{超标}=0
                \basepos-1.7  % 超标时的分割线位置
            \else
                \basepos-1.3  % 正常时的分割线位置
            \fi
        }
        \draw[gray!20] (0.2,\linepos) -- (\cardwidth-0.2,\linepos);

        % 根据当前行的状态计算下一行的位置增量
        \ifnum\pdfstrcmp{\status}{超标}=0
            \pgfmathsetmacro{\increment}{0.85}  % 超标行(两行内容)需要更大的增量
        \else
            \pgfmathsetmacro{\increment}{0.7}  % 正常行(一行内容)使用较小的增量
        \fi

        % 更新位置计数器
        \pgfmathsetmacro{\nextpos}{\currentpos+\increment}
        \xdef\currentpos{\nextpos}
    }

    % 最后一行的处理,消除多余的空白
    \pgfmathsetmacro{\lastincrement}{2}  % 最后一行的增量
    \pgfmathsetmacro{\nextpos}{\currentpos+\lastincrement}
    \xdef\currentpos{\nextpos}

\end{tikzpicture}
\end{center}

\newpage

\begin{center}
\begin{tikzpicture}[
    font=\small,
    title/.style={font=\small\bfseries\color{white}},
    value/.style={font=\small},
    reference/.style={font=\small},
    cell/.style={anchor=west, text width=4.2cm},
    note/.style={anchor=west, text width=4.5cm, align=left}
]
    \def\cardwidth{\textwidth}
    \def\cardheight{22}
    \def\barheight{0.25}
    \def\barwidth{1.5}
    \def\valuebarspace{0.4}

    % 容器和标题栏背景
    \draw[rounded corners=5, fill=white, draw=gray!20]
        (0,0) rectangle (\cardwidth,-\cardheight);
    \path[fill=customTeal]
        (0,0) [rounded corners=5] -- (\cardwidth,0) --
        (\cardwidth,0.8) -- (0,0.8) -- cycle;

    % 表头
    \node[title, anchor=west] at (0.5,0.4) {\textbf{菌种名称}};
    \node[title] at (4.5,0.4) {\textbf{正常范围}};
    \node[title] at (7.5,0.4) {\textbf{检测丰度}};
    \node[title] at (10.5,0.4) {\textbf{结果评价}};
    \node[title] at (13,0.4) {\textbf{超过\%的人}};
    \node[title] at (16,0.4) {\textbf{有\%的正常人检出}};

    % 初始化位置计数器
    \def\currentpos{0.25}

    % 数据行和卡片
    \foreach \item/\enitem/\value/\range/\percentile/\detection/\status/\intro/\suggestion/\index in {
        {动物双歧杆菌}/{B. animalis}/0.00113/{0-0.18833}/27\%/77.88\%/丰度较低/{动物双歧杆菌是一种革兰氏阳性益生菌,具有促进消化健康、增强免疫功能以及改善肠道微生态的潜在益处。}/{}/\currentpos,
        {两歧双歧杆菌}/{B. bifidum}/ND/{0-2.1225}/8\%/98.56\%/未检出/{两歧双歧杆菌是一种革兰氏阳性益生菌,能够帮助分解食物,促进营养吸收,有助于维护肠道健康和增强免疫系统功能。}/{建议补充益生菌制剂}/\currentpos,
        {短双歧杆菌}/{B. breve}/0.00158/{0-0.08}/67\%/99.52\%/丰度较低/{短双歧杆菌是一种革兰氏阳性益生菌,它在消化过程中发挥重要作用,能够帮助分解乳糖和其他碳水化合物,促进肠道健康。}/{}/\currentpos,
        {链状双歧杆菌}/{B. catenulatum}/ND/{0-0.50838}/50\%/99.52\%/未检出/{链状双歧杆菌是一种革兰氏阳性益生菌,它有助于改善肠道微生物群的平衡,促进消化健康,可能在发酵过程中可能产生有益的代谢物。}/{建议补充益生菌制剂}/\currentpos,
        {乳双歧杆菌}/{B. crudilactis}/ND/{0-0.08}/44\%/99.52\%/未检出/{乳双歧杆菌是一种革兰氏阳性益生菌,它在乳糖代谢方面发挥作用,能够帮助消化乳制品,维持肠道健康和增强免疫系统。}/{建议补充益生菌制剂}/\currentpos,
        {婴儿双歧杆菌}/{B. infantis}/ND/{0-0.08}/2\%/99.04\%/未检出/{婴儿双歧杆菌是一种革兰氏阳性益生菌,主要存在于新生儿和婴儿的肠道中。这种细菌对婴儿的消化系统发育至关重要。}/{建议补充益生菌制剂}/\currentpos,
        {长双歧杆菌}/{B. longum}/0.01861/{0-6.854}/95\%/98.56\%/丰度较低/{长双歧杆菌是一种革兰氏阳性益生菌,它可以促进消化、增强免疫系统、帮助维持肠道微生物平衡、减少肠道炎症等。}/{}/\currentpos,
        {胃痛双歧杆菌}/{B. merycicum}/ND/{0-0.08}/8\%/91.83\%/未检出/{胃痛双歧杆菌是一种相对较少研究的双歧杆菌,它具有促进肠道健康、改善消化和增强免疫反应的潜力。}/{建议补充益生菌制剂}/\currentpos,
        {穆卡拉巴双歧杆菌}/{B. moukalabense}/0.00021/{0-0.08}/27\%/77.88\%/丰度较低/{穆卡拉巴双歧杆菌是一种相对较新的双歧杆菌菌株,对肠道健康有积极影响。}/{}/\currentpos,
        {假长双歧杆菌}/{B. pseudolongum}/0.00142/{0-0.08}/83\%/5.77\%/丰度较低/{假长双歧杆菌是一种革兰氏阳性益生菌,具有促进肠道健康和改善消化的能力。}/{}/\currentpos,
        {分支双歧杆菌}/{B. ramosum}/ND/{0-0.08}/98\%/35.58\%/未检出/{分支双歧杆菌是一种革兰氏阳性益生菌,它主要存在于人类的肠道中,尤其是在婴儿期和幼儿期。}/{建议补充益生菌制剂}/\currentpos
    }
    {
        % 计算当前行的基础位置
        \pgfmathsetmacro{\basepos}{-2.8*\currentpos}

        % 菌种名称
        \node[cell, align=left] at (0.5,\basepos) {
            \small\textbf{\item}\\[-0.2em]
            {\color{lightgray}\small\enitem}
        };

        % 正常范围
        \node[reference] at (4.5,\basepos) {\footnotesize\range};

        % 进度条相关
        \pgfmathsetmacro{\barypos}{\basepos-\valuebarspace+0.1}
        \def\barstart{6.75}

        % 进度条背景
        \fill[gray!10, rounded corners=2] (\barstart,\barypos)
            rectangle (\barstart+\barwidth,\barypos+\barheight);

        % 检测丰度值
        \node[value] at (7.5,{\basepos-\valuebarspace+0.6}) {\footnotesize\value};

        % 解析范围并计算进度条长度
        \def\parserange#1-#2\endparse{\def\minval{#1}\def\maxval{#2}}
        \expandafter\parserange\range\endparse

        % 计算进度条长度和颜色
        \pgfmathsetmacro{\progress}{min(\value/\maxval, 1.0)}
        \pgfmathparse{\value > \maxval ? "customred" : (\value < \minval ? "customred" : "green!50")}
        \let\barcolor=\pgfmathresult

        % 进度条显示
        \ifnum\pdfstrcmp{\status}{超标}=0
            \fill[customred, rounded corners=2] (\barstart,\barypos)
                rectangle (\barstart+\barwidth,\barypos+\barheight);
        \else
            \fill[\barcolor, rounded corners=2] (\barstart,\barypos)
                rectangle (\barstart+\barwidth*\progress,\barypos+\barheight);
        \fi

        % 结果评价
        \ifnum\pdfstrcmp{\status}{超标}=0
            \node[value, text=customred] at (10.5,\basepos) {\footnotesize\textbf{\status}};
        \else
            \node[value, text=customGreen] at (10.5,\basepos) {\footnotesize\textbf{\status}};
        \fi

        % 丰度超过%人
        \node[value] at (13,\basepos) {\footnotesize\percentile};

        % %正常人检出
        \node[value] at (16,\basepos) {\footnotesize\detection};

        % 添加卡片
        \pgfmathsetmacro{\cardypos}{\basepos-0.5}
        \begin{scope}[shift={(0,\cardypos)}]
            % 卡片背景
            \pgfmathsetmacro{\cardheight}{
                \ifnum\pdfstrcmp{\status}{超标}=0
                    1.0  % 两行内容时的高度
                \else
                    0.6  % 一行内容时的高度
                \fi
            }

            \fill[rounded corners=5pt, customTeal!5, draw=gray!5]
                (0.3,-\cardheight) rectangle (17.3,0);

            % 菌群简介图标和内容
            \node[anchor=west] at (0.5,-0.3) {
                \textbf{\color{gray!90}\footnotesize \textcolor{customTeal}{\faInfoCircle}}
            };
            \node[anchor=west, text width=16cm] at (0.9,-0.3) {
                {\small\color{gray}\footnotesize \intro}
            };

            % 异常解读标题和内容
            \ifnum\pdfstrcmp{\status}{超标}=0
                \node[anchor=west] at (0.55,-0.8) {
                    \textbf{\color{customRed}\footnotesize \textcolor{customRed}{\faBell}}
                };
                \node[anchor=west, text width=16cm] at (0.9,-0.8) {
                    {\small\color{gray}\footnotesize \suggestion}
                };
            \fi
        \end{scope}

        % 分割线
        \pgfmathsetmacro{\linepos}{
            \ifnum\pdfstrcmp{\status}{超标}=0
                \basepos-1.7  % 超标时的分割线位置
            \else
                \basepos-1.3  % 正常时的分割线位置
            \fi
        }
        \draw[gray!20] (0.2,\linepos) -- (\cardwidth-0.2,\linepos);

        % 根据当前行的状态计算下一行的位置增量
        \ifnum\pdfstrcmp{\status}{超标}=0
            \pgfmathsetmacro{\increment}{0.85}  % 超标行(两行内容)需要更大的增量
        \else
            \pgfmathsetmacro{\increment}{0.7}  % 正常行(一行内容)使用较小的增量
        \fi

        % 更新位置计数器
        \pgfmathsetmacro{\nextpos}{\currentpos+\increment}
        \xdef\currentpos{\nextpos}
    }

    % 最后一行的处理,消除多余的空白
    \pgfmathsetmacro{\lastincrement}{2}  % 最后一行的增量
    \pgfmathsetmacro{\nextpos}{\currentpos+\lastincrement}
    \xdef\currentpos{\nextpos}

\end{tikzpicture}
\end{center}

\newpage

\begin{center}
\begin{tikzpicture}[
    font=\small,
    title/.style={font=\small\bfseries\color{white}},
    value/.style={font=\small},
    reference/.style={font=\small},
    cell/.style={anchor=west, text width=4.2cm},
    note/.style={anchor=west, text width=4.5cm, align=left}
]
    \def\cardwidth{\textwidth}
    \def\cardheight{22}
    \def\barheight{0.25}
    \def\barwidth{1.5}
    \def\valuebarspace{0.4}

    % 容器和标题栏背景
    \draw[rounded corners=5, fill=white, draw=gray!20]
        (0,0) rectangle (\cardwidth,-\cardheight);
    \path[fill=customTeal]
        (0,0) [rounded corners=5] -- (\cardwidth,0) --
        (\cardwidth,0.8) -- (0,0.8) -- cycle;

    % 表头
    \node[title, anchor=west] at (0.5,0.4) {\textbf{菌种名称}};
    \node[title] at (4.5,0.4) {\textbf{正常范围}};
    \node[title] at (7.5,0.4) {\textbf{检测丰度}};
    \node[title] at (10.5,0.4) {\textbf{结果评价}};
    \node[title] at (13,0.4) {\textbf{超过\%的人}};
    \node[title] at (16,0.4) {\textbf{有\%的正常人检出}};

    % 初始化位置计数器
    \def\currentpos{0.25}

    % 数据行和卡片
    \foreach \item/\enitem/\value/\range/\percentile/\detection/\status/\intro/\suggestion/\index in {
        {罗伊氏双歧杆菌}/{B. reuteri}/ND/{0-0.08}/83\%/5.77\%/未检出/{罗伊氏双歧杆菌是一种特有的益生菌,以其能在肠道中产生抗菌物质并改善消化健康而著称,有助于缓解腹泻和增强免疫功能。}/{建议补充益生菌制剂}/\currentpos,
        {短柄双歧杆菌}/{B. stellenboschense}/ND/{0-0.08}/27\%/77.88\%/未检出/{短柄双歧杆菌是一种益生菌,能够促进肠道健康并可能改善消化功能,常见于人类和动物的肠道微生物群中。}/{建议补充益生菌制剂}/\currentpos,
        {粪双歧杆菌}/{B. stercoris}/ND/{0-0.08}/8\%/91.83\%/未检出/{粪双歧杆菌是一种益生菌,主要存在于人类和动物的肠道中,以其助于维持肠道微生物平衡和促进消化健康而受到关注。}/{建议补充益生菌制剂}/\currentpos,
        {布劳特氏菌属氢营养菌}/{B. hydrogenotropica}/0.00262/{0-0.08}/27\%/77.88\%/丰度较低/{布劳特氏菌属氢营养菌是一种厌氧细菌,以其能够利用氢气作为电子供体并产生短链脂肪酸而著称。}/{}/\currentpos,
        {穗状丁酸弧菌}/{B. crossotus}/0.00239/{0-2.9334}/8\%/91.83\%/丰度较低/{穗状丁酸弧菌是一种厌氧细菌,以其发酵纤维素产生丁酸的能力而闻名,有助于促进肠道健康和维持消化功能。}/{}/\currentpos,
        {丁酸梭菌}/{C. butyricum}/0.01350/{0-0.08}/27\%/77.88\%/丰度较低/{丁酸梭菌是一种厌氧性细菌,能够产生丁酸,对于肠道健康和食品发酵具有重要作用。}/{}/\currentpos,
        {霍氏真杆菌}/{E. hallii}/0.02035/{0-3.5506}/8\%/91.83\%/丰度较低/{霍氏真杆菌是一种厌氧细菌,以其能够发酵膳食纤维并产生丁酸而著称,促进肠道健康。}/{}/\currentpos,
        {直肠真杆菌}/{E. rectale}/2.14287/{0-13.029}/75\%/95.67\%/检出/{直肠真杆菌是一种厌氧细菌,以其能发酵膳食纤维并产生短链脂肪酸,尤其是丁酸,促进肠道健康。}/{}/\currentpos,
        {普拉梭菌}/{F. prausnitzii}/26.23325/{0.18043-14.003}/98\%/99.52\%/超标/{普拉梭菌是一种益生菌,以其产生丁酸和抗炎作用而著称,对维持肠道健康和整体免疫功能有重要的积极影响。}/{可以通过增加纤维摄入、食用发酵食品以及减少糖和加工食品的摄入来改善普拉梭菌超标情况。}/\currentpos,
        {嗜酸乳杆菌}/{L. acidophilus}/ND/{0-0.08}/15\%/88.94\%/未检出/{嗜酸乳杆菌是一种常见的益生菌,以其能够发酵乳糖并生成乳酸而著称,具有助于消化、增强免疫力以及维持肠道微生物平衡的潜力。}/{建议补充益生菌制剂}/\currentpos,
        {短乳杆菌}/{L. brevis}/0.00028/{0-0.08}/27\%/77.88\%/丰度较低/{短乳杆菌是一种乳酸菌,以其发酵能力和产生乳酸而著称,具有促进消化、增强免疫力和抗菌特性。}/{}/\currentpos
    }
    {
        % 计算当前行的基础位置
        \pgfmathsetmacro{\basepos}{-2.8*\currentpos}

        % 菌种名称
        \node[cell, align=left] at (0.5,\basepos) {
            \small\textbf{\item}\\[-0.2em]
            {\color{lightgray}\small\enitem}
        };

        % 正常范围
        \node[reference] at (4.5,\basepos) {\footnotesize\range};

        % 进度条相关
        \pgfmathsetmacro{\barypos}{\basepos-\valuebarspace+0.1}
        \def\barstart{6.75}

        % 进度条背景
        \fill[gray!10, rounded corners=2] (\barstart,\barypos)
            rectangle (\barstart+\barwidth,\barypos+\barheight);

        % 检测丰度值
        \node[value] at (7.5,{\basepos-\valuebarspace+0.6}) {\footnotesize\value};

        % 解析范围并计算进度条长度
        \def\parserange#1-#2\endparse{\def\minval{#1}\def\maxval{#2}}
        \expandafter\parserange\range\endparse

        % 计算进度条长度和颜色
        \pgfmathsetmacro{\progress}{min(\value/\maxval, 1.0)}
        \pgfmathparse{\value > \maxval ? "customred" : (\value < \minval ? "customred" : "green!50")}
        \let\barcolor=\pgfmathresult

        % 进度条显示
        \ifnum\pdfstrcmp{\status}{超标}=0
            \fill[customred, rounded corners=2] (\barstart,\barypos)
                rectangle (\barstart+\barwidth,\barypos+\barheight);
        \else
            \fill[\barcolor, rounded corners=2] (\barstart,\barypos)
                rectangle (\barstart+\barwidth*\progress,\barypos+\barheight);
        \fi

        % 结果评价
        \ifnum\pdfstrcmp{\status}{超标}=0
            \node[value, text=customred] at (10.5,\basepos) {\footnotesize\textbf{\status}};
        \else
            \node[value, text=customGreen] at (10.5,\basepos) {\footnotesize\textbf{\status}};
        \fi

        % 丰度超过%人
        \node[value] at (13,\basepos) {\footnotesize\percentile};

        % %正常人检出
        \node[value] at (16,\basepos) {\footnotesize\detection};

        % 添加卡片
        \pgfmathsetmacro{\cardypos}{\basepos-0.5}
        \begin{scope}[shift={(0,\cardypos)}]
            % 卡片背景
            \pgfmathsetmacro{\cardheight}{
                \ifnum\pdfstrcmp{\status}{超标}=0
                    1.0  % 两行内容时的高度
                \else
                    0.6  % 一行内容时的高度
                \fi
            }

            \fill[rounded corners=5pt, customTeal!5, draw=gray!5]
                (0.3,-\cardheight) rectangle (17.3,0);

            % 菌群简介图标和内容
            \node[anchor=west] at (0.5,-0.3) {
                \textbf{\color{gray!90}\footnotesize \textcolor{customTeal}{\faInfoCircle}}
            };
            \node[anchor=west, text width=16cm] at (0.9,-0.3) {
                {\small\color{gray}\footnotesize \intro}
            };

            % 异常解读标题和内容
            \ifnum\pdfstrcmp{\status}{超标}=0
                \node[anchor=west] at (0.55,-0.8) {
                    \textbf{\color{customRed}\footnotesize \textcolor{customRed}{\faBell}}
                };
                \node[anchor=west, text width=16cm] at (0.9,-0.8) {
                    {\small\color{gray}\footnotesize \suggestion}
                };
            \fi
        \end{scope}

        % 分割线
        \pgfmathsetmacro{\linepos}{
            \ifnum\pdfstrcmp{\status}{超标}=0
                \basepos-1.7  % 超标时的分割线位置
            \else
                \basepos-1.3  % 正常时的分割线位置
            \fi
        }
        \draw[gray!20] (0.2,\linepos) -- (\cardwidth-0.2,\linepos);

        % 根据当前行的状态计算下一行的位置增量
        \ifnum\pdfstrcmp{\status}{超标}=0
            \pgfmathsetmacro{\increment}{0.85}  % 超标行(两行内容)需要更大的增量
        \else
            \pgfmathsetmacro{\increment}{0.7}  % 正常行(一行内容)使用较小的增量
        \fi

        % 更新位置计数器
        \pgfmathsetmacro{\nextpos}{\currentpos+\increment}
        \xdef\currentpos{\nextpos}
    }

    % 最后一行的处理,消除多余的空白
    \pgfmathsetmacro{\lastincrement}{2}  % 最后一行的增量
    \pgfmathsetmacro{\nextpos}{\currentpos+\lastincrement}
    \xdef\currentpos{\nextpos}

\end{tikzpicture}
\end{center}

\newpage

\begin{center}
\begin{tikzpicture}[
    font=\small,
    title/.style={font=\small\bfseries\color{white}},
    value/.style={font=\small},
    reference/.style={font=\small},
    cell/.style={anchor=west, text width=4.2cm},
    note/.style={anchor=west, text width=4.5cm, align=left}
]
    \def\cardwidth{\textwidth}
    \def\cardheight{12}
    \def\barheight{0.25}
    \def\barwidth{1.5}
    \def\valuebarspace{0.4}

    % 容器和标题栏背景
    \draw[rounded corners=5, fill=white, draw=gray!20]
        (0,0) rectangle (\cardwidth,-\cardheight);
    \path[fill=customTeal]
        (0,0) [rounded corners=5] -- (\cardwidth,0) --
        (\cardwidth,0.8) -- (0,0.8) -- cycle;

    % 表头
    \node[title, anchor=west] at (0.5,0.4) {\textbf{菌种名称}};
    \node[title] at (4.5,0.4) {\textbf{正常范围}};
    \node[title] at (7.5,0.4) {\textbf{检测丰度}};
    \node[title] at (10.5,0.4) {\textbf{结果评价}};
    \node[title] at (13,0.4) {\textbf{超过\%的人}};
    \node[title] at (16,0.4) {\textbf{有\%的正常人检出}};

    % 初始化位置计数器
    \def\currentpos{0.25}

    % 数据行和卡片
    \foreach \item/\enitem/\value/\range/\percentile/\detection/\status/\intro/\suggestion/\index in {
        {干酪乳杆菌}/{L. casei}/ND/{0-0.08}/8\%/91.83\%/未检出/{干酪乳杆菌是一种乳酸菌,它以其良好的发酵特性和能够耐受酸性环境而著称,具有促进消化、增强免疫力、改善肠道菌群平衡等益处。}/{建议补充益生菌制剂}/\currentpos,
        {卷曲乳杆菌}/{L. crispatus}/0.00489/{0-0.08}/67\%/99.52\%/丰度较低/{卷曲乳杆菌是一种重要的乳酸菌,它以其对维持阴道健康和抑制病原体生长的能力而闻名,同时也能促进消化系统的健康。}/{}/\currentpos,
        {德氏乳杆菌}/{L. delbrueckii}/ND/{0-0.08}/44\%/99.52\%/未检出/{德氏乳杆菌是一种乳酸菌,它能够在酸性环境中生存,具有良好的发酵能力,并能产生多种代谢产物,能够促进消化和增强免疫功能。}/{建议补充益生菌制剂}/\currentpos,
        {发酵乳杆菌}/{L. fermentum}/0.00013/{0-0.08}/27\%/77.88\%/丰度较低/{发酵乳杆菌是一种常见的乳酸菌,具有促进消化、增强免疫功能和抑制有害病原体生长的潜力,对肠道健康有益。}/{}/\currentpos,
        {格氏乳杆菌}/{L. gasseri}/ND/{0-0.08}/8\%/91.83\%/未检出/{格氏乳杆菌是一种重要的乳酸菌,具有促进消化、增强免疫功能和维持阴道健康的潜力,对于肠道和生殖道的健康有益。}/{建议补充益生菌制剂}/\currentpos,
        {瑞士乳杆菌}/{L. helveticus}/ND/{0-0.08}/44\%/99.52\%/未检出/{瑞士乳杆菌是一种重要的乳酸菌,有助于分解乳蛋白,释放出有益的氨基酸和生物活性肽,从而促进消化、增强免疫功能等。}/{建议补充益生菌制剂}/\currentpos
    }
    {
        % 计算当前行的基础位置
        \pgfmathsetmacro{\basepos}{-2.8*\currentpos}

        % 菌种名称
        \node[cell, align=left] at (0.5,\basepos) {
            \small\textbf{\item}\\[-0.2em]
            {\color{lightgray}\small\enitem}
        };

        % 正常范围
        \node[reference] at (4.5,\basepos) {\footnotesize\range};

        % 进度条相关
        \pgfmathsetmacro{\barypos}{\basepos-\valuebarspace+0.1}
        \def\barstart{6.75}

        % 进度条背景
        \fill[gray!10, rounded corners=2] (\barstart,\barypos)
            rectangle (\barstart+\barwidth,\barypos+\barheight);

        % 检测丰度值
        \node[value] at (7.5,{\basepos-\valuebarspace+0.6}) {\footnotesize\value};

        % 解析范围并计算进度条长度
        \def\parserange#1-#2\endparse{\def\minval{#1}\def\maxval{#2}}
        \expandafter\parserange\range\endparse

        % 计算进度条长度和颜色
        \pgfmathsetmacro{\progress}{min(\value/\maxval, 1.0)}
        \pgfmathparse{\value > \maxval ? "customred" : (\value < \minval ? "customred" : "green!50")}
        \let\barcolor=\pgfmathresult

        % 进度条显示
        \ifnum\pdfstrcmp{\status}{超标}=0
            \fill[customred, rounded corners=2] (\barstart,\barypos)
                rectangle (\barstart+\barwidth,\barypos+\barheight);
        \else
            \fill[\barcolor, rounded corners=2] (\barstart,\barypos)
                rectangle (\barstart+\barwidth*\progress,\barypos+\barheight);
        \fi

        % 结果评价
        \ifnum\pdfstrcmp{\status}{超标}=0
            \node[value, text=customred] at (10.5,\basepos) {\footnotesize\textbf{\status}};
        \else
            \node[value, text=customGreen] at (10.5,\basepos) {\footnotesize\textbf{\status}};
        \fi

        % 丰度超过%人
        \node[value] at (13,\basepos) {\footnotesize\percentile};

        % %正常人检出
        \node[value] at (16,\basepos) {\footnotesize\detection};

        % 添加卡片
        \pgfmathsetmacro{\cardypos}{\basepos-0.5}
        \begin{scope}[shift={(0,\cardypos)}]
            % 卡片背景
            \pgfmathsetmacro{\cardheight}{
                \ifnum\pdfstrcmp{\status}{超标}=0
                    1.0  % 两行内容时的高度
                \else
                    0.6  % 一行内容时的高度
                \fi
            }

            \fill[rounded corners=5pt, customTeal!5, draw=gray!5]
                (0.3,-\cardheight) rectangle (17.3,0);

            % 菌群简介图标和内容
            \node[anchor=west] at (0.5,-0.3) {
                \textbf{\color{gray!90}\footnotesize \textcolor{customTeal}{\faInfoCircle}}
            };
            \node[anchor=west, text width=16cm] at (0.9,-0.3) {
                {\small\color{gray}\footnotesize \intro}
            };

            % 异常解读标题和内容
            \ifnum\pdfstrcmp{\status}{超标}=0
                \node[anchor=west] at (0.55,-0.8) {
                    \textbf{\color{customRed}\footnotesize \textcolor{customRed}{\faBell}}
                };
                \node[anchor=west, text width=16cm] at (0.9,-0.8) {
                    {\small\color{gray}\footnotesize \suggestion}
                };
            \fi
        \end{scope}

        % 分割线
        \pgfmathsetmacro{\linepos}{
            \ifnum\pdfstrcmp{\status}{超标}=0
                \basepos-1.7  % 超标时的分割线位置
            \else
                \basepos-1.3  % 正常时的分割线位置
            \fi
        }
        \draw[gray!20] (0.2,\linepos) -- (\cardwidth-0.2,\linepos);

        % 根据当前行的状态计算下一行的位置增量
        \ifnum\pdfstrcmp{\status}{超标}=0
            \pgfmathsetmacro{\increment}{0.85}  % 超标行(两行内容)需要更大的增量
        \else
            \pgfmathsetmacro{\increment}{0.7}  % 正常行(一行内容)使用较小的增量
        \fi

        % 更新位置计数器
        \pgfmathsetmacro{\nextpos}{\currentpos+\increment}
        \xdef\currentpos{\nextpos}
    }

    % 最后一行的处理,消除多余的空白
    \pgfmathsetmacro{\lastincrement}{2}  % 最后一行的增量
    \pgfmathsetmacro{\nextpos}{\currentpos+\lastincrement}
    \xdef\currentpos{\nextpos}

\end{tikzpicture}
\end{center}

\begin{tcolorbox}[
    enhanced,
    colback=gray!3,
    colframe=gray!3,
    arc=3mm,
    boxrule=0pt,
    width=\textwidth,
    top=8pt,
    bottom=8pt
    ]
{\small{\textcolor{customRed}{\faBell}}\quad 异常的有益菌群需要关注:
\begin{itemize}
\item 普拉梭菌(26.23325,超出参考值1.87倍):虽然是重要的抗炎益生菌,但当前水平明显升高,提示肠道菌群结构可能失衡。
\item 食葡糖罗斯拜氏菌(5.12041,超出参考值1.85倍):该菌参与碳水化合物代谢,显著升高可能提示肠道代谢功能异常。
\item 经典益生菌普遍缺乏:多种重要乳杆菌(如植物乳杆菌、嗜酸乳杆菌等)未检出,提示肠道保护功能可能减弱。
\end{itemize}

{\textcolor{green!85!black}{\faLightbulb}}\quad 改善建议:
\begin{itemize}
\item 饮食调整:
    - 适量控制精制碳水化合物的摄入
    - 每天食用发酵乳制品,补充活性乳酸菌
    - 增加膳食纤维摄入,如全谷物、蔬菜水果等
\item 生活调节:
    - 保持规律作息,避免熬夜
    - 适度运动,每天30分钟有氧运动
    - 控制精神压力,保持心情愉悦
\item 益生菌补充:建议在医生指导下选择含有多菌株的复合益生菌制剂,特别是补充缺乏的乳酸菌类。
\end{itemize}

需要注意的是,有益菌群的异常往往反映了肠道微生态的整体失衡。建议通过综合调理来改善菌群结构,必要时可以咨询消化科医生进行专业评估和指导。同时,在调理过程中要循序渐进,避免激进干预可能带来的不适。
}
\end{tcolorbox}

\newpage

\begin{tcolorbox}[
    enhanced,
    colback=white,
    colframe=white,
    arc=2mm,
    boxrule=0pt,
    width=\textwidth,
    left=15pt,
    right=15pt,
    top=10pt,
    bottom=10pt,
    drop shadow={
        opacity=0.2,
        color=customTeal
    },
    borderline west={5pt}{0pt}{customTeal}
]
\textcolor{customTeal}{\large\textbf{(常见)有害菌}}
\end{tcolorbox}

\begin{tcolorbox}[
    enhanced,
    colback=customTealBg,
    colframe=customTealBg,
    arc=3mm,
    boxrule=0pt,
    width=\textwidth,
    top=8pt,
    bottom=8pt
]
{\small{\color{customTeal}\faInfoCircle} 本报告列出了17种人体常见的肠道有害菌,有害菌能够破坏肠道内生态平衡,引起肠道功能紊乱等,损害人体健康。
}
\end{tcolorbox}

\begin{tcolorbox}[
    enhanced,
    colback=lightpurple!10, % 卡片底色
    colframe=lightpurple!10,  % 边框颜色
    arc=3mm,
    boxrule=0.5pt,
    width=\textwidth,
    top=8pt,
    bottom=8pt
]
{\small{\color{lightpurple}\faQuestionCircle}\quad \textbf{什么是肠道有害菌?}\\
{\color{orange!50}\faComments}\quad 肠道有害菌是指那些对宿主健康产生负面影响的微生物,它们在肠道过度生长时可能导致各种健康问题。肠道有害菌通过产生毒素、引发炎症、破坏肠道屏障等方式,可能干扰正常的消化吸收过程,诱发肠道疾病和全身性疾病。典型的有害菌包括:沙门氏菌属(Salmonella)、大肠杆菌(Escherichia coli)和金黄色葡萄球菌(Staphylococcus aureus),它们可能导致食物中毒、肠炎及其他感染性疾病。因此,维护肠道微生物的平衡、抑制有害菌的生长是保护整体健康的重要环节。
}
\end{tcolorbox}

\begin{tcolorbox}[
    enhanced,
    colback=lightpurple!10, % 卡片底色
    colframe=lightpurple!10,  % 边框颜色
    arc=3mm,
    boxrule=0.5pt,
    width=\textwidth,
    top=8pt,
    bottom=8pt
]
{\small{\color{lightpurple}\faQuestionCircle}\quad \textbf{肠道有害菌越少越好吗?}\\
{\color{orange!50}\faComments}\quad 肠道有害菌的存在并不一定是完全不利的,适量的有害菌可以在特定条件下发挥作用。然而,当有害菌的数量过多时,可能会引发健康问题。维护肠道健康的关键在于保持微生物群的平衡与多样性,适当控制有害菌的生长,同时配合健康的饮食和生活方式,以促进整体健康。
}
\end{tcolorbox}

\begin{center}
\begin{tikzpicture}[
    font=\small,
    title/.style={font=\small\bfseries\color{white}},
    value/.style={font=\small},
    reference/.style={font=\small},
    cell/.style={anchor=west, text width=4.2cm},
    note/.style={anchor=west, text width=4.5cm, align=left}
]
    \def\cardwidth{\textwidth}
    \def\cardheight{10}
    \def\barheight{0.25}
    \def\barwidth{1.5}
    \def\valuebarspace{0.4}

    % 容器和标题栏背景
    \draw[rounded corners=5, fill=white, draw=gray!20]
        (0,0) rectangle (\cardwidth,-\cardheight);
    \path[fill=customTeal]
        (0,0) [rounded corners=5] -- (\cardwidth,0) --
        (\cardwidth,0.8) -- (0,0.8) -- cycle;

    % 表头
    \node[title, anchor=west] at (0.5,0.4) {\textbf{菌种名称}};
    \node[title] at (4.5,0.4) {\textbf{正常范围}};
    \node[title] at (7.5,0.4) {\textbf{检测丰度}};
    \node[title] at (10.5,0.4) {\textbf{结果评价}};
    \node[title] at (13,0.4) {\textbf{超过\%的人}};
    \node[title] at (16,0.4) {\textbf{有\%的正常人检出}};

    % 初始化位置计数器
    \def\currentpos{0.25}

    % 数据行和卡片
    \foreach \item/\enitem/\value/\range/\percentile/\detection/\status/\intro/\suggestion/\index in {
        {埃希氏菌属}/{Escherichia}/0.14296/{0-3.83}/6\%/99.52\%/丰度较低/{正常菌属,条件致病菌。过多致病,导致腹泻并失衡。}/{}/\currentpos,
        {链球菌属}/{Streptococcus}/0.31690/{0-0.3496}/13\%/99.52\%/丰度较低/{可引起化脓性炎症,个别菌为益生菌,常见皮肤、口腔、咽喉、婴幼儿常见菌。}/{}/\currentpos,
        {韦荣菌属}/{Veillonella}/0.38326/{0-0.0086}/7\%/98.56\%/超标/{韦荣菌属可能在特定条件下引发健康问题,其过度生长可能导致肠道炎症,加剧消化系统相关疾病的风险。}/{建议调整饮食结构,注意口腔卫生}/\currentpos,
        {泛菌属}/{Pantoea}/0.00225/{0-0.05}/-\%/0.02\%/丰度较低/{泛菌属可在特定条件下引发感染,特别是在免疫系统较弱的个体中,干扰正常的肠道功能,导致肠道不适、炎症或其他健康问题。}/{}/\currentpos,
        {梭杆菌属}/{Fusobacterium}/0.00976/{0-0.05}/38\%/80.29\%/丰度较低/{聚合梭杆菌可诱发促进宫颈癌,产生菌多糖,代谢生成素酶,引发炎症和肝炎症,胃癌到肺癌患者富集。}/{}/\currentpos
%        {志贺氏菌属}/{Shigella}/0.04732/{0-0.05}/-\%/62.98\%/丰度较低/{致病菌,引发腹泻等疾病,有不同的血清型,人类肠道菌唯一病毒,离口途径传播,通常与卫生条件差和食品安全有关。}/{}/\currentpos,
%        {弯曲杆菌属}/{Campylobacter}/0.04799/{0-0.05}/-\%/55.29\%/丰度较低/{人和动物兼性致病菌,导致肠道疾病,常见肠胃炎和菌群紊乱,空肠弯曲杆菌会引起腹泻发烧,食物家禽宠物都可能携带弯曲杆菌。}/{}/\currentpos,
%        {克雷伯氏菌属}/{Klebsiella}/0.09652/{0-0.05}/-\%/89.06\%/超标/{条件致病菌,人群检出率较高,过高导致肠道菌群紊乱,在口腔和肠道均有分布,易引发感染炎症,耐药,易感因素包括营养不良,抗生素,开放性伤口。}/{建议调整饮食,加强口腔卫生}/\currentpos,
%        {脱硫弧菌属}/{Desulfovibrio}/0.02325/{0-0.05}/-\%/33.65\%/丰度较低/{属于变形菌门,产生硫化氢,导致炎症、腹泻,对肠道上皮具严生毒性,导致肠际通透性,约50\%人的口腔和肠道携带,过多通常肥胖,硫转移酶易聚合过重重,铅含含,系统性硬化症患者富集,与普拉梭菌共存利于丁酸生产。}/{}/\currentpos,
%        {螺杆菌属}/{Helicobacter}/0.00373/{0-0.05}/-\%/22.12\%/丰度较低/{革兰氏阴性菌,微需氧菌,该菌属一些菌种被发现在人类上消化道内部,部分种为致病菌,与消化性溃疡、慢性胃炎、十二指肠炎、胃癌有关,包括幽门螺旋杆菌。}/{}/\currentpos,
%        {弓形菌属}/{Arcobacter}/0.00209/{0-0.05}/-\%/7.69\%/丰度较低/{好氧菌,肠际致病菌,与河流海洋污染相关的潜在病原菌,人畜共患菌。}/{}/\currentpos
    }
    {
        % 计算当前行的基础位置
        \pgfmathsetmacro{\basepos}{-2.8*\currentpos}

        % 菌种名称
        \node[cell, align=left] at (0.5,\basepos) {
            \small\textbf{\item}\\[-0.2em]
            {\color{lightgray}\small\enitem}
        };

        % 正常范围
        \node[reference] at (4.5,\basepos) {\footnotesize\range};

        % 进度条相关
        \pgfmathsetmacro{\barypos}{\basepos-\valuebarspace+0.1}
        \def\barstart{6.75}

        % 进度条背景
        \fill[gray!10, rounded corners=2] (\barstart,\barypos)
            rectangle (\barstart+\barwidth,\barypos+\barheight);

        % 检测丰度值
        \node[value] at (7.5,{\basepos-\valuebarspace+0.6}) {\footnotesize\value};

        % 解析范围并计算进度条长度
        \def\parserange#1-#2\endparse{\def\minval{#1}\def\maxval{#2}}
        \expandafter\parserange\range\endparse

        % 计算进度条长度和颜色
        \pgfmathsetmacro{\progress}{min(\value/\maxval, 1.0)}
        \pgfmathparse{\value > \maxval ? "customred" : (\value < \minval ? "customred" : "green!50")}
        \let\barcolor=\pgfmathresult

        % 进度条显示
        \ifnum\pdfstrcmp{\status}{超标}=0
            \fill[customred, rounded corners=2] (\barstart,\barypos)
                rectangle (\barstart+\barwidth,\barypos+\barheight);
        \else
            \fill[\barcolor, rounded corners=2] (\barstart,\barypos)
                rectangle (\barstart+\barwidth*\progress,\barypos+\barheight);
        \fi

        % 结果评价
        \ifnum\pdfstrcmp{\status}{超标}=0
            \node[value, text=customred] at (10.5,\basepos) {\footnotesize\textbf{\status}};
        \else
            \node[value, text=customGreen] at (10.5,\basepos) {\footnotesize\textbf{\status}};
        \fi

        % 丰度超过%人
        \node[value] at (13,\basepos) {\footnotesize\percentile};

        % %正常人检出
        \node[value] at (16,\basepos) {\footnotesize\detection};

        % 添加卡片
        \pgfmathsetmacro{\cardypos}{\basepos-0.5}
        \begin{scope}[shift={(0,\cardypos)}]
            % 卡片背景
            \pgfmathsetmacro{\cardheight}{
                \ifnum\pdfstrcmp{\status}{超标}=0
                    1.0  % 两行内容时的高度
                \else
                    0.6  % 一行内容时的高度
                \fi
            }

            \fill[rounded corners=5pt, customTeal!5, draw=gray!5]
                (0.3,-\cardheight) rectangle (17.3,0);

            % 菌群简介图标和内容
            \node[anchor=west] at (0.5,-0.3) {
                \textbf{\color{gray!90}\footnotesize \textcolor{customTeal}{\faInfoCircle}}
            };
            \node[anchor=west, text width=16cm] at (0.9,-0.3) {
                {\small\color{gray}\footnotesize \intro}
            };

            % 异常解读标题和内容
            \ifnum\pdfstrcmp{\status}{超标}=0
                \node[anchor=west] at (0.55,-0.8) {
                    \textbf{\color{customRed}\footnotesize \textcolor{customRed}{\faBell}}
                };
                \node[anchor=west, text width=16cm] at (0.9,-0.8) {
                    {\small\color{gray}\footnotesize \suggestion}
                };
            \fi
        \end{scope}

        % 分割线
        \pgfmathsetmacro{\linepos}{
            \ifnum\pdfstrcmp{\status}{超标}=0
                \basepos-1.7  % 超标时的分割线位置
            \else
                \basepos-1.3  % 正常时的分割线位置
            \fi
        }
        \draw[gray!20] (0.2,\linepos) -- (\cardwidth-0.2,\linepos);

        % 根据当前行的状态计算下一行的位置增量
        \ifnum\pdfstrcmp{\status}{超标}=0
            \pgfmathsetmacro{\increment}{0.85}  % 超标行(两行内容)需要更大的增量
        \else
            \pgfmathsetmacro{\increment}{0.7}  % 正常行(一行内容)使用较小的增量
        \fi

        % 更新位置计数器
        \pgfmathsetmacro{\nextpos}{\currentpos+\increment}
        \xdef\currentpos{\nextpos}
    }

    % 最后一行的处理,消除多余的空白
    \pgfmathsetmacro{\lastincrement}{2}  % 最后一行的增量
    \pgfmathsetmacro{\nextpos}{\currentpos+\lastincrement}
    \xdef\currentpos{\nextpos}

\end{tikzpicture}
\end{center}

\newpage

\begin{center}
\begin{tikzpicture}[
    font=\small,
    title/.style={font=\small\bfseries\color{white}},
    value/.style={font=\small},
    reference/.style={font=\small},
    cell/.style={anchor=west, text width=4.2cm},
    note/.style={anchor=west, text width=4.5cm, align=left}
]
    \def\cardwidth{\textwidth}
    \def\cardheight{22}
    \def\barheight{0.25}
    \def\barwidth{1.5}
    \def\valuebarspace{0.4}

    % 容器和标题栏背景
    \draw[rounded corners=5, fill=white, draw=gray!20]
        (0,0) rectangle (\cardwidth,-\cardheight);
    \path[fill=customTeal]
        (0,0) [rounded corners=5] -- (\cardwidth,0) --
        (\cardwidth,0.8) -- (0,0.8) -- cycle;

    % 表头
    \node[title, anchor=west] at (0.5,0.4) {\textbf{菌种名称}};
    \node[title] at (4.5,0.4) {\textbf{正常范围}};
    \node[title] at (7.5,0.4) {\textbf{检测丰度}};
    \node[title] at (10.5,0.4) {\textbf{结果评价}};
    \node[title] at (13,0.4) {\textbf{超过\%的人}};
    \node[title] at (16,0.4) {\textbf{有\%的正常人检出}};

    % 初始化位置计数器
    \def\currentpos{0.25}

    % 数据行和卡片
    \foreach \item/\enitem/\value/\range/\percentile/\detection/\status/\intro/\suggestion/\index in {
        {埃希氏菌属}/{Escherichia}/0.14296/{0-3.83}/6\%/99.52\%/丰度较低/{革兰氏阴性兼性厌氧菌,正常菌群成员,参与维生素K和B族维生素合成,但部分菌株可致病。过度生长可导致肠道屏障功能受损}/{建议:加强肠道屏障,补充益生菌;注意饮食卫生;可适量补充乳酸菌制剂}/\currentpos,
        {链球菌属}/{Streptococcus}/0.31690/{0-0.3496}/13\%/99.52\%/丰度较低/{革兰氏阳性兼性厌氧菌,产生乳酸和短链脂肪酸,部分菌株具有益生作用,但也可能导致机会性感染}/{建议:注意口腔卫生;补充益生菌;增加膳食纤维摄入}/\currentpos,
        {韦荣菌属}/{Veillonella}/0.38326/{0-0.0086}/7\%/98.56\%/超标/{革兰氏阴性厌氧菌,利用乳酸产生丙酸盐,参与口腔和肠道微生态平衡,与运动耐力相关}/{建议:控制乳制品摄入;加强口腔卫生;可补充益生菌调节菌群平衡}/\currentpos,
        {泛菌属}/{Pantoea}/0.00225/{0-0.05}/-\%/0.02\%/丰度较低/{革兰氏阴性兼性厌氧菌,环境中常见,机会性致病菌,与免疫功能密切相关}/{建议:加强免疫力;注意饮食卫生}/\currentpos,
        {梭杆菌属}/{Fusobacterium}/0.00976/{0-0.05}/38\%/80.29\%/丰度较低/{革兰氏阴性厌氧菌,产生短链脂肪酸和致炎因子,参与黏膜免疫调节,过度生长可能导致炎症反应}/{建议:增加膳食纤维;补充抗炎益生菌}/\currentpos,
        {志贺氏菌属}/{Shigella}/0.04732/{0-0.05}/-\%/62.98\%/丰度较低/{革兰氏阴性兼性厌氧菌,侵袭性肠道病原菌,与饮食卫生密切相关}/{建议:注意饮食卫生;避免生食;加强肠道屏障功能}/\currentpos,
        {弯曲杆菌属}/{Campylobacter}/0.04799/{0-0.05}/-\%/55.29\%/丰度较低/{革兰氏阴性微需氧菌,人畜共患病原菌,与食品安全密切相关}/{建议:注意食品卫生;充分烹饪肉类;避免交叉污染}/\currentpos,
        {克雷伯氏菌属}/{Klebsiella}/0.09652/{0-0.05}/-\%/89.06\%/超标/{革兰氏阴性兼性厌氧菌,条件致病菌,参与碳水化合物代谢,过度生长可能导致免疫失衡}/{建议:调节肠道菌群;补充益生菌;加强口腔卫生;控制碳水化合物摄入}/\currentpos,
        {脱硫弧菌属}/{Desulfovibrio}/0.02325/{0-0.05}/-\%/33.65\%/丰度较低/{革兰氏阴性厌氧菌,硫酸盐还原菌,产生硫化氢,参与硫代谢,过度生长可能影响肠道屏障}/{建议:减少含硫氨基酸食物摄入;增加膳食纤维;补充益生菌}/\currentpos,
        {螺杆菌属}/{Helicobacter}/0.00373/{0-0.05}/-\%/22.12\%/丰度较低/{革兰氏阴性微需氧菌,定植于胃肠道黏膜,部分菌种可能影响胃肠道健康}/{建议:注意饮食卫生;避免刺激性食物;保持规律作息}/\currentpos,
        {弓形菌属}/{Arcobacter}/0.00209/{0-0.05}/-\%/7.69\%/丰度较低/{革兰氏阴性微需氧菌,环境中常见的潜在致病菌,与水源污染相关}/{建议:注意饮用水卫生;避免生食海产品;加强食品卫生}/\currentpos
    }
    {
        % 计算当前行的基础位置
        \pgfmathsetmacro{\basepos}{-2.8*\currentpos}

        % 菌种名称
        \node[cell, align=left] at (0.5,\basepos) {
            \small\textbf{\item}\\[-0.2em]
            {\color{lightgray}\small\enitem}
        };

        % 正常范围
        \node[reference] at (4.5,\basepos) {\footnotesize\range};

        % 进度条相关
        \pgfmathsetmacro{\barypos}{\basepos-\valuebarspace+0.1}
        \def\barstart{6.75}

        % 进度条背景
        \fill[gray!10, rounded corners=2] (\barstart,\barypos)
            rectangle (\barstart+\barwidth,\barypos+\barheight);

        % 检测丰度值
        \node[value] at (7.5,{\basepos-\valuebarspace+0.6}) {\footnotesize\value};

        % 解析范围并计算进度条长度
        \def\parserange#1-#2\endparse{\def\minval{#1}\def\maxval{#2}}
        \expandafter\parserange\range\endparse

        % 计算进度条长度和颜色
        \pgfmathsetmacro{\progress}{min(\value/\maxval, 1.0)}
        \pgfmathparse{\value > \maxval ? "customred" : (\value < \minval ? "customred" : "green!50")}
        \let\barcolor=\pgfmathresult

        % 进度条显示
        \ifnum\pdfstrcmp{\status}{超标}=0
            \fill[customred, rounded corners=2] (\barstart,\barypos)
                rectangle (\barstart+\barwidth,\barypos+\barheight);
        \else
            \fill[\barcolor, rounded corners=2] (\barstart,\barypos)
                rectangle (\barstart+\barwidth*\progress,\barypos+\barheight);
        \fi

        % 结果评价
        \ifnum\pdfstrcmp{\status}{超标}=0
            \node[value, text=customRed] at (10.5,\basepos) {\footnotesize\textbf{\status}};
        \else
            \node[value, text=customGreen] at (10.5,\basepos) {\footnotesize\textbf{\status}};
        \fi

        % 丰度超过%人
        \node[value] at (13,\basepos) {\footnotesize\percentile};

        % %正常人检出
        \node[value] at (16,\basepos) {\footnotesize\detection};

        % 添加卡片
        \pgfmathsetmacro{\cardypos}{\basepos-0.5}
        \begin{scope}[shift={(0,\cardypos)}]
            % 卡片背景
            \pgfmathsetmacro{\cardheight}{
                \ifnum\pdfstrcmp{\status}{超标}=0
                    1.0  % 两行内容时的高度
                \else
                    0.6  % 一行内容时的高度
                \fi
            }

            \fill[rounded corners=5pt, customTeal!5, draw=gray!5]
                (0.3,-\cardheight) rectangle (17.3,0);

            % 菌群简介图标和内容
            \node[anchor=west] at (0.5,-0.3) {
                \textbf{\color{gray!90}\footnotesize \textcolor{customTeal}{\faInfoCircle}}
            };
            \node[anchor=west, text width=16cm] at (0.9,-0.3) {
                {\small\color{gray}\footnotesize \intro}
            };

            % 异常解读标题和内容
            \ifnum\pdfstrcmp{\status}{超标}=0
                \node[anchor=west] at (0.55,-0.8) {
                    \textbf{\color{customRed}\footnotesize \textcolor{customRed}{\faBell}}
                };
                \node[anchor=west, text width=16cm] at (0.9,-0.8) {
                    {\small\color{gray}\footnotesize \suggestion}
                };
            \fi
        \end{scope}

        % 分割线
        \pgfmathsetmacro{\linepos}{
            \ifnum\pdfstrcmp{\status}{超标}=0
                \basepos-1.7  % 超标时的分割线位置
            \else
                \basepos-1.3  % 正常时的分割线位置
            \fi
        }
        \draw[gray!20] (0.2,\linepos) -- (\cardwidth-0.2,\linepos);

        % 根据当前行的状态计算下一行的位置增量
        \ifnum\pdfstrcmp{\status}{超标}=0
            \pgfmathsetmacro{\increment}{0.85}  % 超标行(两行内容)需要更大的增量
        \else
            \pgfmathsetmacro{\increment}{0.7}  % 正常行(一行内容)使用较小的增量
        \fi

        % 更新位置计数器
        \pgfmathsetmacro{\nextpos}{\currentpos+\increment}
        \xdef\currentpos{\nextpos}
    }

    % 最后一行的处理,消除多余的空白
    \pgfmathsetmacro{\lastincrement}{2}  % 最后一行的增量
    \pgfmathsetmacro{\nextpos}{\currentpos+\lastincrement}
    \xdef\currentpos{\nextpos}

\end{tikzpicture}
\end{center}

\newpage

\begin{center}
\begin{tikzpicture}[
    font=\small,
    title/.style={font=\small\bfseries\color{white}},
    value/.style={font=\small},
    reference/.style={font=\small},
    cell/.style={anchor=west, text width=4.2cm},
    note/.style={anchor=west, text width=4.5cm, align=left}
]
    \def\cardwidth{\textwidth}
    \def\cardheight{22}
    \def\barheight{0.25}
    \def\barwidth{1.5}
    \def\valuebarspace{0.4}

    % 容器和标题栏背景
    \draw[rounded corners=5, fill=white, draw=gray!20]
        (0,0) rectangle (\cardwidth,-\cardheight);
    \path[fill=customTeal]
        (0,0) [rounded corners=5] -- (\cardwidth,0) --
        (\cardwidth,0.8) -- (0,0.8) -- cycle;

    % 表头
    \node[title, anchor=west] at (0.5,0.4) {\textbf{菌种名称}};
    \node[title] at (4.5,0.4) {\textbf{正常范围}};
    \node[title] at (7.5,0.4) {\textbf{检测丰度}};
    \node[title] at (10.5,0.4) {\textbf{结果评价}};
    \node[title] at (13,0.4) {\textbf{超过\%的人}};
    \node[title] at (16,0.4) {\textbf{有\%的正常人检出}};

    % 初始化位置计数器
    \def\currentpos{0.25}

    % 数据行和卡片
    \foreach \item/\enitem/\value/\range/\percentile/\detection/\status/\intro/\suggestion/\index in {
        {厌氧螺菌属}/{Anaerobiospirillum}/0.00044/{0-0.05}/-\%/0.00\%/丰度较低/{革兰氏阴性专性厌氧菌,主要分布在肠道,具有极强的运动性,参与碳水化合物代谢,过度生长可能影响肠道稳态}/{建议:注意饮食卫生;避免生食;补充益生菌调节菌群}/\currentpos,
        {肠球菌属}/{Enterococcus}/0.12259/{0-0.05}/52\%/81.25\%/超标/{革兰氏阳性兼性厌氧菌,产生细菌素,参与肠道微生态平衡,具有一定益生作用,但过度生长可能影响菌群稳态}/{建议:适当补充益生菌;控制碳水化合物摄入;增加膳食纤维;注意饮食卫生}/\currentpos,
        {奈瑟氏菌属}/{Neisseria}/0.00957/{0-0.05}/-\%/46.15\%/丰度较低/{革兰氏阴性需氧菌,主要定植于黏膜表面,参与口腔微生态平衡,产生脂多糖,与局部免疫功能相关}/{建议:加强口腔卫生;增强免疫力;保持环境通风}/\currentpos,
        {不动杆菌属}/{Acinetobacter}/0.11947/{0-0.05}/83\%/45.19\%/超标/{革兰氏阴性需氧菌,环境适应力强,参与有机物降解,过度生长可能影响免疫平衡}/{建议:加强个人卫生;避免环境污染;提高免疫力;可补充益生菌}/\currentpos,
        {假单胞菌属}/{Pseudomonas}/0.07830/{0-0.05}/69\%/45.19\%/超标/{革兰氏阴性需氧菌,具有多样代谢能力,产生多种次级代谢产物,与环境适应性相关,过度生长可能影响微生态平衡}/{建议:注意环境卫生;加强饮食卫生;补充有益菌群;增强免疫力}/\currentpos,
        {霍尔德曼氏菌属}/{Holdemania}/0.10581/{0-0.0003}/98\%/29.33\%/超标/{革兰氏阳性厌氧菌,参与蛋白质代谢,产生短链脂肪酸,与肠道代谢功能相关}/{建议:调节饮食结构;适量补充益生菌;增加膳食纤维摄入}/\currentpos
    }
    {
        % 计算当前行的基础位置
        \pgfmathsetmacro{\basepos}{-2.8*\currentpos}

        % 菌种名称
        \node[cell, align=left] at (0.5,\basepos) {
            \small\textbf{\item}\\[-0.2em]
            {\color{lightgray}\small\enitem}
        };

        % 正常范围
        \node[reference] at (4.5,\basepos) {\footnotesize\range};

        % 进度条相关
        \pgfmathsetmacro{\barypos}{\basepos-\valuebarspace+0.1}
        \def\barstart{6.75}

        % 进度条背景
        \fill[gray!10, rounded corners=2] (\barstart,\barypos)
            rectangle (\barstart+\barwidth,\barypos+\barheight);

        % 检测丰度值
        \node[value] at (7.5,{\basepos-\valuebarspace+0.6}) {\footnotesize\value};

        % 解析范围并计算进度条长度
        \def\parserange#1-#2\endparse{\def\minval{#1}\def\maxval{#2}}
        \expandafter\parserange\range\endparse

        % 计算进度条长度和颜色
        \pgfmathsetmacro{\progress}{min(\value/\maxval, 1.0)}
        \pgfmathparse{\value > \maxval ? "customred" : (\value < \minval ? "customred" : "green!50")}
        \let\barcolor=\pgfmathresult

        % 进度条显示
        \ifnum\pdfstrcmp{\status}{超标}=0
            \fill[customred, rounded corners=2] (\barstart,\barypos)
                rectangle (\barstart+\barwidth,\barypos+\barheight);
        \else
            \fill[\barcolor, rounded corners=2] (\barstart,\barypos)
                rectangle (\barstart+\barwidth*\progress,\barypos+\barheight);
        \fi

        % 结果评价
        \ifnum\pdfstrcmp{\status}{超标}=0
            \node[value, text=customRed] at (10.5,\basepos) {\footnotesize\textbf{\status}};
        \else
            \node[value, text=customGreen] at (10.5,\basepos) {\footnotesize\textbf{\status}};
        \fi

        % 丰度超过%人
        \node[value] at (13,\basepos) {\footnotesize\percentile};

        % %正常人检出
        \node[value] at (16,\basepos) {\footnotesize\detection};

        % 添加卡片
        \pgfmathsetmacro{\cardypos}{\basepos-0.5}
        \begin{scope}[shift={(0,\cardypos)}]
            % 卡片背景
            \pgfmathsetmacro{\cardheight}{
                \ifnum\pdfstrcmp{\status}{超标}=0
                    1.0  % 两行内容时的高度
                \else
                    0.6  % 一行内容时的高度
                \fi
            }

            \fill[rounded corners=5pt, customTeal!5, draw=gray!5]
                (0.3,-\cardheight) rectangle (17.3,0);

            % 菌群简介图标和内容
            \node[anchor=west] at (0.5,-0.3) {
                \textbf{\color{gray!90}\footnotesize \textcolor{customTeal}{\faInfoCircle}}
            };
            \node[anchor=west, text width=16cm] at (0.9,-0.3) {
                {\small\color{gray}\footnotesize \intro}
            };

            % 异常解读标题和内容
            \ifnum\pdfstrcmp{\status}{超标}=0
                \node[anchor=west] at (0.55,-0.8) {
                    \textbf{\color{customRed}\footnotesize \textcolor{customRed}{\faBell}}
                };
                \node[anchor=west, text width=16cm] at (0.9,-0.8) {
                    {\small\color{gray}\footnotesize \suggestion}
                };
            \fi
        \end{scope}

        % 分割线
        \pgfmathsetmacro{\linepos}{
            \ifnum\pdfstrcmp{\status}{超标}=0
                \basepos-1.7  % 超标时的分割线位置
            \else
                \basepos-1.3  % 正常时的分割线位置
            \fi
        }
        \draw[gray!20] (0.2,\linepos) -- (\cardwidth-0.2,\linepos);

        % 根据当前行的状态计算下一行的位置增量
        \ifnum\pdfstrcmp{\status}{超标}=0
            \pgfmathsetmacro{\increment}{0.85}  % 超标行(两行内容)需要更大的增量
        \else
            \pgfmathsetmacro{\increment}{0.7}  % 正常行(一行内容)使用较小的增量
        \fi

        % 更新位置计数器
        \pgfmathsetmacro{\nextpos}{\currentpos+\increment}
        \xdef\currentpos{\nextpos}
    }

    % 最后一行的处理,消除多余的空白
    \pgfmathsetmacro{\lastincrement}{2}  % 最后一行的增量
    \pgfmathsetmacro{\nextpos}{\currentpos+\lastincrement}
    \xdef\currentpos{\nextpos}

\end{tikzpicture}
\end{center}

\newpage

\begin{tcolorbox}[
    enhanced,
    colback=white,
    colframe=white,
    arc=2mm,
    boxrule=0pt,
    width=\textwidth,
    left=15pt,
    right=15pt,
    top=10pt,
    bottom=10pt,
    drop shadow={
        opacity=0.2,
        color=customTeal
    },
    borderline west={5pt}{0pt}{customTeal}
]
\textcolor{customTeal}{\large\textbf{(常见)机会致病菌}}
\end{tcolorbox}

\begin{tcolorbox}[
    enhanced,
    colback=customTealBg,
    colframe=customTealBg,
    arc=3mm,
    boxrule=0pt,
    width=\textwidth,
    top=8pt,
    bottom=8pt
]
{\small{\color{customTeal}\faInfoCircle} 机会致病菌是指在正常情况下与人体和平共处的微生物,但当宿主免疫力下降或菌群失衡时可能导致感染的细菌。它们通常以低水平存在于人体中,在特定条件下才表现出致病性。常见的机会致病菌包括:粪肠球菌(Enterococcus faecalis)在免疫力低下时可能引起尿路感染或心内膜炎;阴沟肠杆菌(Enterobacter cloacae)可能导致呼吸道和泌尿系统感染;鲍曼不动杆菌(Acinetobacter baumannii)在医院环境中可引起院内感染,特别是在危重患者中可导致肺炎。这类细菌的检出不一定需要立即治疗,但需要关注宿主状态和菌群平衡,必要时在医生指导下进行干预。
}
\end{tcolorbox}

\begin{center}
\begin{tikzpicture}[
    font=\small,
    title/.style={font=\small\bfseries\color{white}},
    value/.style={font=\small},
    reference/.style={font=\small},
    cell/.style={anchor=west, text width=4.2cm},
    note/.style={anchor=west, text width=4.5cm, align=left}
]
    \def\cardwidth{\textwidth}
    \def\cardheight{22}
    \def\barheight{0.25}
    \def\barwidth{1.5}
    \def\valuebarspace{0.4}

    % 容器和标题栏背景
    \draw[rounded corners=5, fill=white, draw=gray!20]
        (0,0) rectangle (\cardwidth,-\cardheight);
    \path[fill=customTeal]
        (0,0) [rounded corners=5] -- (\cardwidth,0) --
        (\cardwidth,0.8) -- (0,0.8) -- cycle;

    % 表头
    \node[title, anchor=west] at (0.5,0.4) {\textbf{菌种名称}};
    \node[title] at (4.5,0.4) {\textbf{正常范围}};
    \node[title] at (7.5,0.4) {\textbf{检测丰度}};
    \node[title] at (10.5,0.4) {\textbf{结果评价}};
    \node[title] at (13,0.4) {\textbf{超过\%的人}};
    \node[title] at (16,0.4) {\textbf{有\%的正常人检出}};

    % 初始化位置计数器
    \def\currentpos{0.25}

    % 数据行和卡片
    \foreach \item/\enitem/\value/\range/\percentile/\detection/\status/\intro/\suggestion/\index in {
        {鲍曼不动杆菌}/{A. baumannii}/0.11308/{0-0.05}/83\%/45.19\%/超标/{医院感染常见病原菌,正常存在于呼吸道。免疫力低下时可致肺炎,具有多重耐药性。}/{建议加强消毒防护,避免医院感染。}/\currentpos,
        {醋酸钙不动杆菌}/{A. calcoaceticus}/0.00023/{0-0.01}/15\%/88.94\%/正常/{常见于人体皮肤和黏膜。条件致病菌,可引起伤口感染。}/{}/\currentpos,
        {约氏不动杆菌}/{A. johnsonii}/0.00145/{0-0.02}/27\%/77.88\%/正常/{人体正常菌群成员。免疫力降低时可引起尿路感染。}/{}/\currentpos,
        {琼氏不动杆菌}/{A. junii}/0.00567/{0-0.03}/38\%/80.29\%/正常/{存在于人体皮肤表面的常见菌。可能引起软组织感染。}/{}/\currentpos,
        {鲁氏不动杆菌}/{A. lwoffii}/0.02145/{0-0.01}/52\%/81.25\%/超标/{皮肤正常菌群成员。医院感染的重要条件致病菌。}/{建议注意个人卫生,预防院内感染}/\currentpos,
        {蜡样芽孢杆菌}/{B. cereus}/0.00089/{0-0.02}/13\%/99.52\%/正常/{土壤中常见细菌。过量可导致食物中毒。}/{}/\currentpos,
        {脆弱拟杆菌}/{B. fragilis}/0.00145/{0-1.8852}/7\%/98.56\%/丰度较低/{重要肠道共生菌。参与维生素K的合成,维持肠道健康。}/{}/\currentpos,
        {普通拟杆菌}/{B. vulgatus}/ND/{0-20.066}/8\%/91.83\%/未检出/{肠道重要益生菌。维持肠道菌群平衡,促进免疫系统发育。}/{建议补充益生菌}/\currentpos,
        {难辨梭菌}/{C. difficile}/0.08647/{0-0.05}/98\%/99.52\%/超标/{正常菌群成员,使用抗生素后易过度生长。可引起腹泻等症状。}/{建议谨慎使用抗生素,必要时进行菌群干预}/\currentpos,
        {阴沟肠杆菌}/{E. cloacae}/0.00692/{0-0.05}/27\%/77.88\%/丰度较低/{肠道常驻菌群。免疫力低下时可致感染。}/{}/\currentpos,
        {粪肠球菌}/{E. faecalis}/0.01415/{0-0.05}/38\%/80.29\%/丰度较低/{肠道正常菌群成员。过量可致多器官感染。}/{}/\currentpos
    }
    {
        % 计算当前行的基础位置
        \pgfmathsetmacro{\basepos}{-2.8*\currentpos}

        % 菌种名称
        \node[cell, align=left] at (0.5,\basepos) {
            \small\textbf{\item}\\[-0.2em]
            {\color{lightgray}\small\enitem}
        };

        % 正常范围
        \node[reference] at (4.5,\basepos) {\footnotesize\range};

        % 进度条相关
        \pgfmathsetmacro{\barypos}{\basepos-\valuebarspace+0.1}
        \def\barstart{6.75}

        % 进度条背景
        \fill[gray!10, rounded corners=2] (\barstart,\barypos)
            rectangle (\barstart+\barwidth,\barypos+\barheight);

        % 检测丰度值
        \node[value] at (7.5,{\basepos-\valuebarspace+0.6}) {\footnotesize\value};

        % 解析范围并计算进度条长度
        \def\parserange#1-#2\endparse{\def\minval{#1}\def\maxval{#2}}
        \expandafter\parserange\range\endparse

        % 计算进度条长度和颜色
        \pgfmathsetmacro{\progress}{min(\value/\maxval, 1.0)}
        \pgfmathparse{\value > \maxval ? "customred" : (\value < \minval ? "customred" : "green!50")}
        \let\barcolor=\pgfmathresult

        % 进度条显示
        \ifnum\pdfstrcmp{\status}{超标}=0
            \fill[customred, rounded corners=2] (\barstart,\barypos)
                rectangle (\barstart+\barwidth,\barypos+\barheight);
        \else
            \fill[\barcolor, rounded corners=2] (\barstart,\barypos)
                rectangle (\barstart+\barwidth*\progress,\barypos+\barheight);
        \fi

        % 结果评价
        \ifnum\pdfstrcmp{\status}{超标}=0
            \node[value, text=customRed] at (10.5,\basepos) {\footnotesize\textbf{\status}};
        \else
            \node[value, text=customGreen] at (10.5,\basepos) {\footnotesize\textbf{\status}};
        \fi

        % 丰度超过%人
        \node[value] at (13,\basepos) {\footnotesize\percentile};

        % %正常人检出
        \node[value] at (16,\basepos) {\footnotesize\detection};

        % 添加卡片
        \pgfmathsetmacro{\cardypos}{\basepos-0.5}
        \begin{scope}[shift={(0,\cardypos)}]
            % 卡片背景
            \pgfmathsetmacro{\cardheight}{
                \ifnum\pdfstrcmp{\status}{超标}=0
                    1.0  % 两行内容时的高度
                \else
                    0.6  % 一行内容时的高度
                \fi
            }

            \fill[rounded corners=5pt, customTeal!5, draw=gray!5]
                (0.3,-\cardheight) rectangle (17.3,0);

            % 菌群简介图标和内容
            \node[anchor=west] at (0.5,-0.3) {
                \textbf{\color{gray!90}\footnotesize \textcolor{customTeal}{\faInfoCircle}}
            };
            \node[anchor=west, text width=16cm] at (0.9,-0.3) {
                {\small\color{gray}\footnotesize \intro}
            };

            % 异常解读标题和内容
            \ifnum\pdfstrcmp{\status}{超标}=0
                \node[anchor=west] at (0.55,-0.8) {
                    \textbf{\color{customRed}\footnotesize \textcolor{customRed}{\faBell}}
                };
                \node[anchor=west, text width=16cm] at (0.9,-0.8) {
                    {\small\color{gray}\footnotesize \suggestion}
                };
            \fi
        \end{scope}

        % 分割线
        \pgfmathsetmacro{\linepos}{
            \ifnum\pdfstrcmp{\status}{超标}=0
                \basepos-1.7  % 超标时的分割线位置
            \else
                \basepos-1.3  % 正常时的分割线位置
            \fi
        }
        \draw[gray!20] (0.2,\linepos) -- (\cardwidth-0.2,\linepos);

        % 根据当前行的状态计算下一行的位置增量
        \ifnum\pdfstrcmp{\status}{超标}=0
            \pgfmathsetmacro{\increment}{0.85}  % 超标行(两行内容)需要更大的增量
        \else
            \pgfmathsetmacro{\increment}{0.7}  % 正常行(一行内容)使用较小的增量
        \fi

        % 更新位置计数器
        \pgfmathsetmacro{\nextpos}{\currentpos+\increment}
        \xdef\currentpos{\nextpos}
    }

    % 最后一行的处理,消除多余的空白
    \pgfmathsetmacro{\lastincrement}{2}  % 最后一行的增量
    \pgfmathsetmacro{\nextpos}{\currentpos+\lastincrement}
    \xdef\currentpos{\nextpos}

\end{tikzpicture}
\end{center}

\newpage

\begin{center}
\begin{tikzpicture}[
    font=\small,
    title/.style={font=\small\bfseries\color{white}},
    value/.style={font=\small},
    reference/.style={font=\small},
    cell/.style={anchor=west, text width=4.2cm},
    note/.style={anchor=west, text width=4.5cm, align=left}
]
    \def\cardwidth{\textwidth}
    \def\cardheight{22}
    \def\barheight{0.25}
    \def\barwidth{1.5}
    \def\valuebarspace{0.4}

    % 容器和标题栏背景
    \draw[rounded corners=5, fill=white, draw=gray!20]
        (0,0) rectangle (\cardwidth,-\cardheight);
    \path[fill=customTeal]
        (0,0) [rounded corners=5] -- (\cardwidth,0) --
        (\cardwidth,0.8) -- (0,0.8) -- cycle;

    % 表头
    \node[title, anchor=west] at (0.5,0.4) {\textbf{菌种名称}};
    \node[title] at (4.5,0.4) {\textbf{正常范围}};
    \node[title] at (7.5,0.4) {\textbf{检测丰度}};
    \node[title] at (10.5,0.4) {\textbf{结果评价}};
    \node[title] at (13,0.4) {\textbf{超过\%的人}};
    \node[title] at (16,0.4) {\textbf{有\%的正常人检出}};

    % 初始化位置计数器
    \def\currentpos{0.25}

    % 数据行和卡片
    \foreach \item/\enitem/\value/\range/\percentile/\detection/\status/\intro/\suggestion/\index in {
        {屎肠球菌}/{E. faecium}/0.10070/{0-0.05}/83\%/45.19\%/超标/{肠道共生菌,具有较强耐药性。过量可致感染。}/{建议调节肠道菌群平衡}/\currentpos,
        {大肠埃希氏菌}/{E. coli}/0.14497/{0-3.8364}/6\%/99.52\%/正常/{肠道优势菌,维持肠道稳态。过量可致腹泻发热。}/{}/\currentpos,
        {具核梭杆菌}/{F. nucleatum}/0.00459/{0-0.05}/38\%/80.29\%/丰度较低/{口腔常见菌。可引起牙周感染。}/{}/\currentpos,
        {肺炎克雷伯菌}/{K. pneumoniae}/0.09579/{0-0.24019}/52\%/81.25\%/正常/{呼吸道定植菌。免疫力低下时可致肺炎。}/{}/\currentpos,
        {铜绿假单胞菌}/{P. aeruginosa}/0.04152/{0-0.05}/69\%/45.19\%/丰度较低/{环境常见菌,具有适应性强特点。可引起呼吸道感染。}/{}/\currentpos,
        {脱氮假单胞杆菌}/{P. denitrificans}/ND/{0-0.05}/8\%/91.83\%/未检出/{土壤中的分解菌。参与环境氮循环。}/{}/\currentpos,
        {荧光假单胞菌}/{P. fluorescens}/0.00093/{0-0.05}/27\%/77.88\%/丰度较低/{水体常见菌。可引起局部感染。}/{}/\currentpos,
        {黄褐假单胞菌}/{P. fulva}/ND/{0-0.05}/15\%/88.94\%/未检出/{环境菌群成员。很少引起感染。}/{}/\currentpos,
        {门多萨假单胞菌}/{P. mendocina}/ND/{0-0.05}/8\%/91.83\%/未检出/{土壤常见菌。极少引起感染。}/{}/\currentpos,
        {恶臭假单胞菌}/{P. putida}/0.00065/{0-0.05}/27\%/77.88\%/丰度较低/{环境中的益生菌。促进植物生长。}/{}/\currentpos,
        {斯氏假单胞菌}/{P. stutzeri}/0.00116/{0-0.05}/38\%/80.29\%/丰度较低/{环境脱氮菌。参与环境净化。}/{}/\currentpos
    }
    {
        % 计算当前行的基础位置
        \pgfmathsetmacro{\basepos}{-2.8*\currentpos}

        % 菌种名称
        \node[cell, align=left] at (0.5,\basepos) {
            \small\textbf{\item}\\[-0.2em]
            {\color{lightgray}\small\enitem}
        };

        % 正常范围
        \node[reference] at (4.5,\basepos) {\footnotesize\range};

        % 进度条相关
        \pgfmathsetmacro{\barypos}{\basepos-\valuebarspace+0.1}
        \def\barstart{6.75}

        % 进度条背景
        \fill[gray!10, rounded corners=2] (\barstart,\barypos)
            rectangle (\barstart+\barwidth,\barypos+\barheight);

        % 检测丰度值
        \node[value] at (7.5,{\basepos-\valuebarspace+0.6}) {\footnotesize\value};

        % 解析范围并计算进度条长度
        \def\parserange#1-#2\endparse{\def\minval{#1}\def\maxval{#2}}
        \expandafter\parserange\range\endparse

        % 计算进度条长度和颜色
        \pgfmathsetmacro{\progress}{min(\value/\maxval, 1.0)}
        \pgfmathparse{\value > \maxval ? "customred" : (\value < \minval ? "customred" : "green!50")}
        \let\barcolor=\pgfmathresult

        % 进度条显示
        \ifnum\pdfstrcmp{\status}{超标}=0
            \fill[customred, rounded corners=2] (\barstart,\barypos)
                rectangle (\barstart+\barwidth,\barypos+\barheight);
        \else
            \fill[\barcolor, rounded corners=2] (\barstart,\barypos)
                rectangle (\barstart+\barwidth*\progress,\barypos+\barheight);
        \fi

        % 结果评价
        \ifnum\pdfstrcmp{\status}{超标}=0
            \node[value, text=customRed] at (10.5,\basepos) {\footnotesize\textbf{\status}};
        \else
            \node[value, text=customGreen] at (10.5,\basepos) {\footnotesize\textbf{\status}};
        \fi

        % 丰度超过%人
        \node[value] at (13,\basepos) {\footnotesize\percentile};

        % %正常人检出
        \node[value] at (16,\basepos) {\footnotesize\detection};

        % 添加卡片
        \pgfmathsetmacro{\cardypos}{\basepos-0.5}
        \begin{scope}[shift={(0,\cardypos)}]
            % 卡片背景
            \pgfmathsetmacro{\cardheight}{
                \ifnum\pdfstrcmp{\status}{超标}=0
                    1.0  % 两行内容时的高度
                \else
                    0.6  % 一行内容时的高度
                \fi
            }

            \fill[rounded corners=5pt, customTeal!5, draw=gray!5]
                (0.3,-\cardheight) rectangle (17.3,0);

            % 菌群简介图标和内容
            \node[anchor=west] at (0.5,-0.3) {
                \textbf{\color{gray!90}\footnotesize \textcolor{customTeal}{\faInfoCircle}}
            };
            \node[anchor=west, text width=16cm] at (0.9,-0.3) {
                {\small\color{gray}\footnotesize \intro}
            };

            % 异常解读标题和内容
            \ifnum\pdfstrcmp{\status}{超标}=0
                \node[anchor=west] at (0.55,-0.8) {
                    \textbf{\color{customRed}\footnotesize \textcolor{customRed}{\faBell}}
                };
                \node[anchor=west, text width=16cm] at (0.9,-0.8) {
                    {\small\color{gray}\footnotesize \suggestion}
                };
            \fi
        \end{scope}

        % 分割线
        \pgfmathsetmacro{\linepos}{
            \ifnum\pdfstrcmp{\status}{超标}=0
                \basepos-1.7  % 超标时的分割线位置
            \else
                \basepos-1.3  % 正常时的分割线位置
            \fi
        }
        \draw[gray!20] (0.2,\linepos) -- (\cardwidth-0.2,\linepos);

        % 根据当前行的状态计算下一行的位置增量
        \ifnum\pdfstrcmp{\status}{超标}=0
            \pgfmathsetmacro{\increment}{0.85}  % 超标行(两行内容)需要更大的增量
        \else
            \pgfmathsetmacro{\increment}{0.7}  % 正常行(一行内容)使用较小的增量
        \fi

        % 更新位置计数器
        \pgfmathsetmacro{\nextpos}{\currentpos+\increment}
        \xdef\currentpos{\nextpos}
    }

    % 最后一行的处理,消除多余的空白
    \pgfmathsetmacro{\lastincrement}{2}  % 最后一行的增量
    \pgfmathsetmacro{\nextpos}{\currentpos+\lastincrement}
    \xdef\currentpos{\nextpos}

\end{tikzpicture}
\end{center}

\newpage

\begin{center}
\begin{tikzpicture}[
    font=\small,
    title/.style={font=\small\bfseries\color{white}},
    value/.style={font=\small},
    reference/.style={font=\small},
    cell/.style={anchor=west, text width=4.2cm},
    note/.style={anchor=west, text width=4.5cm, align=left}
]
    \def\cardwidth{\textwidth}
    \def\cardheight{6}
    \def\barheight{0.25}
    \def\barwidth{1.5}
    \def\valuebarspace{0.4}

    % 容器和标题栏背景
    \draw[rounded corners=5, fill=white, draw=gray!20]
        (0,0) rectangle (\cardwidth,-\cardheight);
    \path[fill=customTeal]
        (0,0) [rounded corners=5] -- (\cardwidth,0) --
        (\cardwidth,0.8) -- (0,0.8) -- cycle;

    % 表头
    \node[title, anchor=west] at (0.5,0.4) {\textbf{菌种名称}};
    \node[title] at (4.5,0.4) {\textbf{正常范围}};
    \node[title] at (7.5,0.4) {\textbf{检测丰度}};
    \node[title] at (10.5,0.4) {\textbf{结果评价}};
    \node[title] at (13,0.4) {\textbf{超过\%的人}};
    \node[title] at (16,0.4) {\textbf{有\%的正常人检出}};

    % 初始化位置计数器
    \def\currentpos{0.25}

    % 数据行和卡片
    \foreach \item/\enitem/\value/\range/\percentile/\detection/\status/\intro/\suggestion/\index in {
        {咽峡链球菌}/{S. anginosus}/0.00280/{0-0.05}/52\%/81.25\%/丰度较低/{口咽部定植菌。可引起口腔感染。}/{}/\currentpos,
        {肺炎链球菌}/{S. pneumoniae}/0.03111/{0-0.05}/69\%/45.19\%/丰度较低/{上呼吸道定植菌。可导致肺部感染。}/{}/\currentpos,
        {小韦荣球菌}/{V. parvula}/0.07774/{0-0.13203}/83\%/45.19\%/检出/{口腔常驻菌。参与乳酸代谢,维持口腔健康。}/{}/\currentpos
    }
    {
        % 计算当前行的基础位置
        \pgfmathsetmacro{\basepos}{-2.8*\currentpos}

        % 菌种名称
        \node[cell, align=left] at (0.5,\basepos) {
            \small\textbf{\item}\\[-0.2em]
            {\color{lightgray}\small\enitem}
        };

        % 正常范围
        \node[reference] at (4.5,\basepos) {\footnotesize\range};

        % 进度条相关
        \pgfmathsetmacro{\barypos}{\basepos-\valuebarspace+0.1}
        \def\barstart{6.75}

        % 进度条背景
        \fill[gray!10, rounded corners=2] (\barstart,\barypos)
            rectangle (\barstart+\barwidth,\barypos+\barheight);

        % 检测丰度值
        \node[value] at (7.5,{\basepos-\valuebarspace+0.6}) {\footnotesize\value};

        % 解析范围并计算进度条长度
        \def\parserange#1-#2\endparse{\def\minval{#1}\def\maxval{#2}}
        \expandafter\parserange\range\endparse

        % 计算进度条长度和颜色
        \pgfmathsetmacro{\progress}{min(\value/\maxval, 1.0)}
        \pgfmathparse{\value > \maxval ? "customred" : (\value < \minval ? "customred" : "green!50")}
        \let\barcolor=\pgfmathresult

        % 进度条显示
        \ifnum\pdfstrcmp{\status}{超标}=0
            \fill[customred, rounded corners=2] (\barstart,\barypos)
                rectangle (\barstart+\barwidth,\barypos+\barheight);
        \else
            \fill[\barcolor, rounded corners=2] (\barstart,\barypos)
                rectangle (\barstart+\barwidth*\progress,\barypos+\barheight);
        \fi

        % 结果评价
        \ifnum\pdfstrcmp{\status}{超标}=0
            \node[value, text=customRed] at (10.5,\basepos) {\footnotesize\textbf{\status}};
        \else
            \node[value, text=customGreen] at (10.5,\basepos) {\footnotesize\textbf{\status}};
        \fi

        % 丰度超过%人
        \node[value] at (13,\basepos) {\footnotesize\percentile};

        % %正常人检出
        \node[value] at (16,\basepos) {\footnotesize\detection};

        % 添加卡片
        \pgfmathsetmacro{\cardypos}{\basepos-0.5}
        \begin{scope}[shift={(0,\cardypos)}]
            % 卡片背景
            \pgfmathsetmacro{\cardheight}{
                \ifnum\pdfstrcmp{\status}{超标}=0
                    1.0  % 两行内容时的高度
                \else
                    0.6  % 一行内容时的高度
                \fi
            }

            \fill[rounded corners=5pt, customTeal!5, draw=gray!5]
                (0.3,-\cardheight) rectangle (17.3,0);

            % 菌群简介图标和内容
            \node[anchor=west] at (0.5,-0.3) {
                \textbf{\color{gray!90}\footnotesize \textcolor{customTeal}{\faInfoCircle}}
            };
            \node[anchor=west, text width=16cm] at (0.9,-0.3) {
                {\small\color{gray}\footnotesize \intro}
            };

            % 异常解读标题和内容
            \ifnum\pdfstrcmp{\status}{超标}=0
                \node[anchor=west] at (0.55,-0.8) {
                    \textbf{\color{customRed}\footnotesize \textcolor{customRed}{\faBell}}
                };
                \node[anchor=west, text width=16cm] at (0.9,-0.8) {
                    {\small\color{gray}\footnotesize \suggestion}
                };
            \fi
        \end{scope}

        % 分割线
        \pgfmathsetmacro{\linepos}{
            \ifnum\pdfstrcmp{\status}{超标}=0
                \basepos-1.7  % 超标时的分割线位置
            \else
                \basepos-1.3  % 正常时的分割线位置
            \fi
        }
        \draw[gray!20] (0.2,\linepos) -- (\cardwidth-0.2,\linepos);

        % 根据当前行的状态计算下一行的位置增量
        \ifnum\pdfstrcmp{\status}{超标}=0
            \pgfmathsetmacro{\increment}{0.85}  % 超标行(两行内容)需要更大的增量
        \else
            \pgfmathsetmacro{\increment}{0.7}  % 正常行(一行内容)使用较小的增量
        \fi

        % 更新位置计数器
        \pgfmathsetmacro{\nextpos}{\currentpos+\increment}
        \xdef\currentpos{\nextpos}
    }

    % 最后一行的处理,消除多余的空白
    \pgfmathsetmacro{\lastincrement}{2}  % 最后一行的增量
    \pgfmathsetmacro{\nextpos}{\currentpos+\lastincrement}
    \xdef\currentpos{\nextpos}

\end{tikzpicture}
\end{center}

\begin{tcolorbox}[
    enhanced,
    colback=gray!3,
    colframe=gray!3,
    arc=3mm,
    boxrule=0pt,
    width=\textwidth,
    top=8pt,
    bottom=8pt
    ]
{\small{\textcolor{yellow!85!orange}{\faLightbulb}}\quad 超标的机会致病菌群需要关注:
\begin{itemize}
\item 鲍曼不动杆菌(0.11308,超出参考值2.26倍):该菌正常存在于呼吸道,但当前水平明显升高,需要注意预防呼吸道感染风险。
\item 难辨梭菌(0.08647,超出参考值1.73倍):水平升高可能与抗生素使用史有关,需警惕腹泻风险。
\item 屎肠球菌(0.10070,超出参考值2.01倍):作为耐药性较强的条件致病菌,当前水平升高需要关注。
\end{itemize}

{\textcolor{green!85!black}{\faLightbulb}}\quad 改善建议:
\begin{itemize}
\item 加强免疫力:保证充足睡眠,适量运动,均衡饮食。
\item 谨慎用药:避免不必要的抗生素使用,必要时在医生指导下用药。
\item 饮食调整:增加益生菌食品摄入,如酸奶、泡菜等发酵食品。
\item 生活习惯:保持良好的个人卫生习惯,勤洗手,避免交叉感染。
\end{itemize}

需要注意的是,这些机会致病菌的升高往往与身体免疫状态密切相关。建议在日常生活中注意调节身体状态,必要时可以咨询专业医生进行进一步评估。
}
\end{tcolorbox}

\newpage

\begin{tcolorbox}[
    enhanced,
    colback=white,
    colframe=white,
    arc=2mm,
    boxrule=0pt,
    width=\textwidth,
    left=15pt,
    right=15pt,
    top=10pt,
    bottom=10pt,
    drop shadow={
        opacity=0.2,
        color=customTeal
    },
    borderline west={5pt}{0pt}{customTeal}
]
\textcolor{customTeal}{\large\textbf{致病菌}}
\end{tcolorbox}

\begin{tcolorbox}[
    enhanced,
    colback=customTealBg,
    colframe=customTealBg,
    arc=3mm,
    boxrule=0pt,
    width=\textwidth,
    top=8pt,
    bottom=8pt
]
{\small{\color{customTeal}\faInfoCircle}\quad 肠道致病菌是指能直接引起疾病的肠道病原微生物,它们具有明确的致病性,一旦在肠道中检出就可能对健康造成威胁。这类细菌通常具有特定的致病机制,如产生毒素、侵犯宿主细胞、触发炎症反应等。\\
\faCommentAlt
常见的肠道致病菌包括:
\begin{itemize}
    \item 沙门氏菌属(Salmonella):能引起食物中毒和肠道感染。
    \item 志贺氏菌属(Shigella):可导致细菌性痢疾。
    \item 产气荚膜梭菌(Clostridium perfringens):能产生多种毒素,引起食物中毒和肠道感染。
    \item 幽门螺杆菌(Helicobacter pylori):与胃炎、消化性溃疡等胃部疾病密切相关。
\end{itemize}
一旦检测到这些致病菌,通常需要及时进行针对性治疗以防止疾病的发生和发展。
}
\end{tcolorbox}

\newpage

\begin{center}
\begin{tikzpicture}[
    font=\small,
    title/.style={font=\small\bfseries\color{white}},
    value/.style={font=\small},
    reference/.style={font=\small},
    cell/.style={anchor=west, text width=4.2cm},
    note/.style={anchor=west, text width=4.5cm, align=left}
]
    \def\cardwidth{\textwidth}
    \def\cardheight{22}
    \def\barheight{0.25}
    \def\barwidth{1.5}
    \def\valuebarspace{0.4}

    % 容器和标题栏背景
    \draw[rounded corners=5, fill=white, draw=gray!20]
        (0,0) rectangle (\cardwidth,-\cardheight);
    \path[fill=customTeal]
        (0,0) [rounded corners=5] -- (\cardwidth,0) --
        (\cardwidth,0.8) -- (0,0.8) -- cycle;

    % 表头
    \node[title, anchor=west] at (0.5,0.4) {\textbf{菌种名称}};
    \node[title] at (4.5,0.4) {\textbf{正常范围}};
    \node[title] at (7.5,0.4) {\textbf{检测丰度}};
    \node[title] at (10.5,0.4) {\textbf{结果评价}};
    \node[title] at (13,0.4) {\textbf{超过\%的人}};
    \node[title] at (16,0.4) {\textbf{有\%的正常人检出}};

    % 初始化位置计数器
    \def\currentpos{0.25}

    % 数据行和卡片
    \foreach \item/\enitem/\value/\range/\percentile/\detection/\status/\intro/\suggestion/\index in {
        {豚鼠气单胞菌}/{Aeromonas caviae}/ND/{0-0.05}/15\%/88.94\%/未检出/{水环境常见病原菌。可引起胃肠道感染,常见腹泻症状。}/{注意饮水卫生,避免生食水产品}/\currentpos,
        {嗜水气单胞菌}/{Aeromonas hydrophila}/ND/{0-0.05}/8\%/91.83\%/未检出/{广泛分布于水环境中。可引起皮肤软组织感染和腹泻。}/{避免接触污染水体,及时处理伤口}/\currentpos,
        {温和气单胞菌}/{Aeromonas sobria}/0.00037/{0-0.05}/27\%/77.88\%/丰度较低/{水生环境常见菌。可引起急性胃肠炎和腹泻。}/{注意饮食卫生,避免食用生食}/\currentpos,
        {牛布鲁氏杆菌}/{Brucella abortus}/ND/{0-0.05}/8\%/91.83\%/未检出/{重要人畜共患病原体。通过接触感染牛或食用污染乳制品传播。}/{避免接触感染动物,注意饮食卫生}/\currentpos,
        {犬布鲁氏杆菌}/{Brucella canis}/ND/{0-0.05}/15\%/88.94\%/未检出/{犬类布鲁氏菌病病原体。可引起慢性感染和发热。}/{注意防护,避免与患病犬类密切接触}/\currentpos,
        {羊布鲁氏杆菌}/{Brucella melitensis}/0.00142/{0-0.05}/38\%/80.29\%/丰度较低/{最主要的人畜共患布鲁氏菌。可引起多系统感染。}/{避免食用生羊奶制品,加强职业防护}/\currentpos,
        {绵羊布鲁氏杆菌}/{Brucella ovis}/ND/{0-0.05}/8\%/91.83\%/未检出/{主要感染绵羊。人类感染风险较低。}/{做好职业防护}/\currentpos,
        {猪布鲁氏杆菌}/{Brucella suis}/ND/{0-0.05}/15\%/88.94\%/未检出/{猪群常见布鲁氏菌。可引起关节炎和全身症状。}/{注意职业防护,避免接触感染动物}/\currentpos,
        {唐菖蒲伯克霍尔德氏菌}/{Burkholderia gladioli}/0.00049/{0-0.05}/27\%/77.88\%/丰度较低/{主要与植物相关的环境菌。免疫力低下者易感。}/{加强个人防护,注意环境卫生}/\currentpos,
        {大肠弯曲菌}/{Campylobacter coli}/0.00283/{0-0.05}/38\%/80.29\%/丰度较低/{常见食源性致病菌。可引起急性肠炎和腹痛。}/{注意食品卫生,避免生食}/\currentpos,
        {胎儿弯曲菌}/{Campylobacter fetus}/0.00034/{0-0.05}/15\%/88.94\%/丰度较低/{生殖系统感染菌。可引起流产和全身感染。}/{加强孕期防护,注意个人卫生}/\currentpos
    }
    {
        % 计算当前行的基础位置
        \pgfmathsetmacro{\basepos}{-2.8*\currentpos}

        % 菌种名称
        \node[cell, align=left] at (0.5,\basepos) {
            \small\textbf{\item}\\[-0.2em]
            {\color{lightgray}\small\enitem}
        };

        % 正常范围
        \node[reference] at (4.5,\basepos) {\footnotesize\range};

        % 进度条相关
        \pgfmathsetmacro{\barypos}{\basepos-\valuebarspace+0.1}
        \def\barstart{6.75}

        % 进度条背景
        \fill[gray!10, rounded corners=2] (\barstart,\barypos)
            rectangle (\barstart+\barwidth,\barypos+\barheight);

        % 检测丰度值
        \node[value] at (7.5,{\basepos-\valuebarspace+0.6}) {\footnotesize\value};

        % 解析范围并计算进度条长度
        \def\parserange#1-#2\endparse{\def\minval{#1}\def\maxval{#2}}
        \expandafter\parserange\range\endparse

        % 计算进度条长度和颜色
        \pgfmathsetmacro{\progress}{min(\value/\maxval, 1.0)}
        \pgfmathparse{\value > \maxval ? "customred" : (\value < \minval ? "customred" : "green!50")}
        \let\barcolor=\pgfmathresult

        % 进度条显示
        \ifnum\pdfstrcmp{\status}{超标}=0
            \fill[customred, rounded corners=2] (\barstart,\barypos)
                rectangle (\barstart+\barwidth,\barypos+\barheight);
        \else
            \fill[\barcolor, rounded corners=2] (\barstart,\barypos)
                rectangle (\barstart+\barwidth*\progress,\barypos+\barheight);
        \fi

        % 结果评价
        \ifnum\pdfstrcmp{\status}{超标}=0
            \node[value, text=customred] at (10.5,\basepos) {\footnotesize\textbf{\status}};
        \else
            \node[value, text=customGreen] at (10.5,\basepos) {\footnotesize\textbf{\status}};
        \fi

        % 丰度超过%人
        \node[value] at (13,\basepos) {\footnotesize\percentile};

        % %正常人检出
        \node[value] at (16,\basepos) {\footnotesize\detection};

        % 添加卡片
        \pgfmathsetmacro{\cardypos}{\basepos-0.5}
        \begin{scope}[shift={(0,\cardypos)}]
            % 卡片背景
            \pgfmathsetmacro{\cardheight}{
                \ifnum\pdfstrcmp{\status}{超标}=0
                    1.0  % 两行内容时的高度
                \else
                    0.6  % 一行内容时的高度
                \fi
            }

            \fill[rounded corners=5pt, customTeal!5, draw=gray!5]
                (0.3,-\cardheight) rectangle (17.3,0);

            % 菌群简介图标和内容
            \node[anchor=west] at (0.5,-0.3) {
                \textbf{\color{gray!90}\footnotesize \textcolor{customTeal}{\faInfoCircle}}
            };
            \node[anchor=west, text width=16cm] at (0.9,-0.3) {
                {\small\color{gray}\footnotesize \intro}
            };

            % 异常解读标题和内容
            \ifnum\pdfstrcmp{\status}{超标}=0
                \node[anchor=west] at (0.55,-0.8) {
                    \textbf{\color{customRed}\footnotesize \textcolor{customRed}{\faBell}}
                };
                \node[anchor=west, text width=16cm] at (0.9,-0.8) {
                    {\small\color{gray}\footnotesize \suggestion}
                };
            \fi
        \end{scope}

        % 分割线
        \pgfmathsetmacro{\linepos}{
            \ifnum\pdfstrcmp{\status}{超标}=0
                \basepos-1.7  % 超标时的分割线位置
            \else
                \basepos-1.3  % 正常时的分割线位置
            \fi
        }
        \draw[gray!20] (0.2,\linepos) -- (\cardwidth-0.2,\linepos);

        % 根据当前行的状态计算下一行的位置增量
        \ifnum\pdfstrcmp{\status}{超标}=0
            \pgfmathsetmacro{\increment}{0.85}  % 超标行(两行内容)需要更大的增量
        \else
            \pgfmathsetmacro{\increment}{0.7}  % 正常行(一行内容)使用较小的增量
        \fi

        % 更新位置计数器
        \pgfmathsetmacro{\nextpos}{\currentpos+\increment}
        \xdef\currentpos{\nextpos}
    }

    % 最后一行的处理,消除多余的空白
    \pgfmathsetmacro{\lastincrement}{2}  % 最后一行的增量
    \pgfmathsetmacro{\nextpos}{\currentpos+\lastincrement}
    \xdef\currentpos{\nextpos}

\end{tikzpicture}
\end{center}

\newpage

\begin{center}
\begin{tikzpicture}[
    font=\small,
    title/.style={font=\small\bfseries\color{white}},
    value/.style={font=\small},
    reference/.style={font=\small},
    cell/.style={anchor=west, text width=4.2cm},
    note/.style={anchor=west, text width=4.5cm, align=left}
]
    \def\cardwidth{\textwidth}
    \def\cardheight{22}
    \def\barheight{0.25}
    \def\barwidth{1.5}
    \def\valuebarspace{0.4}

    % 容器和标题栏背景
    \draw[rounded corners=5, fill=white, draw=gray!20]
        (0,0) rectangle (\cardwidth,-\cardheight);
    \path[fill=customTeal]
        (0,0) [rounded corners=5] -- (\cardwidth,0) --
        (\cardwidth,0.8) -- (0,0.8) -- cycle;

    % 表头
    \node[title, anchor=west] at (0.5,0.4) {\textbf{菌种名称}};
    \node[title] at (4.5,0.4) {\textbf{正常范围}};
    \node[title] at (7.5,0.4) {\textbf{检测丰度}};
    \node[title] at (10.5,0.4) {\textbf{结果评价}};
    \node[title] at (13,0.4) {\textbf{超过\%的人}};
    \node[title] at (16,0.4) {\textbf{有\%的正常人检出}};

    % 初始化位置计数器
    \def\currentpos{0.25}

    % 数据行和卡片
    \foreach \item/\enitem/\value/\range/\percentile/\detection/\status/\intro/\suggestion/\index in {
        {空肠弯曲菌}/{Campylobacter jejuni}/0.03914/{0-0.05}/52\%/81.25\%/丰度较低/{最常见的弯曲菌属致病菌。可引起急性肠炎。}/{注意饮食卫生,避免食用未熟食品}/\currentpos,
        {肉毒梭菌}/{Clostridium botulinum}/0.00372/{0-0.05}/38\%/80.29\%/丰度较低/{产生致命神经毒素的土壤菌。可引起重症肉毒中毒。}/{注意食品储存,避免食用可疑罐头}/\currentpos,
        {产气荚膜梭菌}/{Clostridium perfringens}/0.00419/{0-0.05}/27\%/77.88\%/丰度较低/{常见食源性致病菌。可引起急性食物中毒。}/{确保食品充分加热,注意储存温度}/\currentpos,
        {阪崎肠杆菌}/{Cronobacter sakazakii}/0.00482/{0-0.05}/38\%/80.29\%/丰度较低/{环境条件致病菌。新生儿易感,可致严重感染。}/{注意婴儿配方奶粉卫生,谨慎储存}/\currentpos,
        {猪红斑丹毒丝菌}/{Erysipelothrix rhusiopathiae}/0.00020/{0-0.05}/15\%/88.94\%/丰度较低/{动物源性致病菌。主要感染猪类。可引起人类皮肤蜂窝织炎。}/{注意职业防护,避免直接接触感染动物}/\currentpos,
        {大肠埃希菌O157:H7}/{Escherichia coli O157:H7}/ND/{0-0.05}/8\%/91.83\%/未检出/{重要食源性致病菌。可引起出血性腹泻和肾功能损害。}/{注意饮食卫生,避免生食或未煮熟食品}/\currentpos,
        {幽门螺杆菌}/{Helicobacter pylori}/0.00204/{0-0.05}/38\%/80.29\%/丰度较低/{胃部定植菌。可引起慢性胃炎、消化性溃疡和胃癌。}/{注意个人卫生,必要时就医检查}/\currentpos,
        {博杰曼军团菌}/{Legionella bozemanae}/ND/{0-0.05}/15\%/88.94\%/未检出/{水环境中的机会性致病菌。可引起军团菌病。}/{注意环境卫生,定期清洁水系统}/\currentpos,
        {杜莫夫军团菌}/{Legionella dumoffii}/ND/{0-0.05}/8\%/91.83\%/未检出/{常见于空调冷却系统。可引起呼吸道感染。}/{定期维护空调系统,保持环境卫生}/\currentpos,
        {长滩军团菌}/{Legionella longbeachae}/ND/{0-0.05}/15\%/88.94\%/未检出/{土壤环境常见菌。可引起肺部感染。}/{接触土壤时注意防护}/\currentpos,
        {嗜肺军团菌}/{Legionella pneumophila}/0.00115/{0-0.05}/27\%/77.88\%/丰度较低/{水系统常见致病菌。可引起严重肺炎。}/{定期消毒水系统,注意环境卫生}/\currentpos
    }
    {
        % 计算当前行的基础位置
        \pgfmathsetmacro{\basepos}{-2.8*\currentpos}

        % 菌种名称
        \node[cell, align=left] at (0.5,\basepos) {
            \small\textbf{\item}\\[-0.2em]
            {\color{lightgray}\small\enitem}
        };

        % 正常范围
        \node[reference] at (4.5,\basepos) {\footnotesize\range};

        % 进度条相关
        \pgfmathsetmacro{\barypos}{\basepos-\valuebarspace+0.1}
        \def\barstart{6.75}

        % 进度条背景
        \fill[gray!10, rounded corners=2] (\barstart,\barypos)
            rectangle (\barstart+\barwidth,\barypos+\barheight);

        % 检测丰度值
        \node[value] at (7.5,{\basepos-\valuebarspace+0.6}) {\footnotesize\value};

        % 解析范围并计算进度条长度
        \def\parserange#1-#2\endparse{\def\minval{#1}\def\maxval{#2}}
        \expandafter\parserange\range\endparse

        % 计算进度条长度和颜色
        \pgfmathsetmacro{\progress}{min(\value/\maxval, 1.0)}
        \pgfmathparse{\value > \maxval ? "customred" : (\value < \minval ? "customred" : "green!50")}
        \let\barcolor=\pgfmathresult

        % 进度条显示
        \ifnum\pdfstrcmp{\status}{超标}=0
            \fill[customred, rounded corners=2] (\barstart,\barypos)
                rectangle (\barstart+\barwidth,\barypos+\barheight);
        \else
            \fill[\barcolor, rounded corners=2] (\barstart,\barypos)
                rectangle (\barstart+\barwidth*\progress,\barypos+\barheight);
        \fi

        % 结果评价
        \ifnum\pdfstrcmp{\status}{超标}=0
            \node[value, text=customred] at (10.5,\basepos) {\footnotesize\textbf{\status}};
        \else
            \node[value, text=customGreen] at (10.5,\basepos) {\footnotesize\textbf{\status}};
        \fi

        % 丰度超过%人
        \node[value] at (13,\basepos) {\footnotesize\percentile};

        % %正常人检出
        \node[value] at (16,\basepos) {\footnotesize\detection};

        % 添加卡片
        \pgfmathsetmacro{\cardypos}{\basepos-0.5}
        \begin{scope}[shift={(0,\cardypos)}]
            % 卡片背景
            \pgfmathsetmacro{\cardheight}{
                \ifnum\pdfstrcmp{\status}{超标}=0
                    1.0  % 两行内容时的高度
                \else
                    0.6  % 一行内容时的高度
                \fi
            }

            \fill[rounded corners=5pt, customTeal!5, draw=gray!5]
                (0.3,-\cardheight) rectangle (17.3,0);

            % 菌群简介图标和内容
            \node[anchor=west] at (0.5,-0.3) {
                \textbf{\color{gray!90}\footnotesize \textcolor{customTeal}{\faInfoCircle}}
            };
            \node[anchor=west, text width=16cm] at (0.9,-0.3) {
                {\small\color{gray}\footnotesize \intro}
            };

            % 异常解读标题和内容
            \ifnum\pdfstrcmp{\status}{超标}=0
                \node[anchor=west] at (0.55,-0.8) {
                    \textbf{\color{customRed}\footnotesize \textcolor{customRed}{\faBell}}
                };
                \node[anchor=west, text width=16cm] at (0.9,-0.8) {
                    {\small\color{gray}\footnotesize \suggestion}
                };
            \fi
        \end{scope}

        % 分割线
        \pgfmathsetmacro{\linepos}{
            \ifnum\pdfstrcmp{\status}{超标}=0
                \basepos-1.7  % 超标时的分割线位置
            \else
                \basepos-1.3  % 正常时的分割线位置
            \fi
        }
        \draw[gray!20] (0.2,\linepos) -- (\cardwidth-0.2,\linepos);

        % 根据当前行的状态计算下一行的位置增量
        \ifnum\pdfstrcmp{\status}{超标}=0
            \pgfmathsetmacro{\increment}{0.85}  % 超标行(两行内容)需要更大的增量
        \else
            \pgfmathsetmacro{\increment}{0.7}  % 正常行(一行内容)使用较小的增量
        \fi

        % 更新位置计数器
        \pgfmathsetmacro{\nextpos}{\currentpos+\increment}
        \xdef\currentpos{\nextpos}
    }

    % 最后一行的处理,消除多余的空白
    \pgfmathsetmacro{\lastincrement}{2}  % 最后一行的增量
    \pgfmathsetmacro{\nextpos}{\currentpos+\lastincrement}
    \xdef\currentpos{\nextpos}

\end{tikzpicture}
\end{center}

\newpage

\begin{center}
\begin{tikzpicture}[
    font=\small,
    title/.style={font=\small\bfseries\color{white}},
    value/.style={font=\small},
    reference/.style={font=\small},
    cell/.style={anchor=west, text width=4.2cm},
    note/.style={anchor=west, text width=4.5cm, align=left}
]
    \def\cardwidth{\textwidth}
    \def\cardheight{22}
    \def\barheight{0.25}
    \def\barwidth{1.5}
    \def\valuebarspace{0.4}

    % 容器和标题栏背景
    \draw[rounded corners=5, fill=white, draw=gray!20]
        (0,0) rectangle (\cardwidth,-\cardheight);
    \path[fill=customTeal]
        (0,0) [rounded corners=5] -- (\cardwidth,0) --
        (\cardwidth,0.8) -- (0,0.8) -- cycle;

    % 表头
    \node[title, anchor=west] at (0.5,0.4) {\textbf{菌种名称}};
    \node[title] at (4.5,0.4) {\textbf{正常范围}};
    \node[title] at (7.5,0.4) {\textbf{检测丰度}};
    \node[title] at (10.5,0.4) {\textbf{结果评价}};
    \node[title] at (13,0.4) {\textbf{超过\%的人}};
    \node[title] at (16,0.4) {\textbf{有\%的正常人检出}};

    % 初始化位置计数器
    \def\currentpos{0.25}

    % 数据行和卡片
    \foreach \item/\enitem/\value/\range/\percentile/\detection/\status/\intro/\suggestion/\index in {
        {单核细胞增生李斯特菌}/{L. monocytogenes}/0.02299/{0-0.05}/52\%/81.25\%/丰度较低/{环境和食品中常见。可引起脑膜炎,孕妇需特别注意。}/{避免食用未经充分加热的食品}/\currentpos,
        {鸟分枝杆菌复合群}/{M. avium complex sp.}/ND/{0-0.05}/8\%/91.83\%/未检出/{环境常见机会性致病菌。免疫力低下者易感。}/{加强个人防护,注意环境卫生}/\currentpos,
        {龟分枝杆菌}/{M. chelonae}/ND/{0-0.05}/15\%/88.94\%/未检出/{水环境常见菌。可引起皮肤和软组织感染。}/{避免接触污染水体,及时处理伤口}/\currentpos,
        {偶发分枝杆菌}/{M. fortuitum}/0.00024/{0-0.05}/27\%/77.88\%/丰度较低/{广泛分布于环境中。可引起皮肤和软组织感染。}/{注意个人卫生,避免环境污染}/\currentpos,
        {堪萨斯分枝杆菌}/{M. kansasii}/0.00026/{0-0.05}/38\%/80.29\%/丰度较低/{自来水系统常见。可引起类结核样肺部感染。}/{注意饮用水卫生,定期维护水系统}/\currentpos,
        {海洋分枝杆菌}/{M. marinum}/0.00031/{0-0.05}/27\%/77.88\%/丰度较低/{水生环境常见。可引起游泳池肉芽肿。}/{接触水体时注意防护,及时处理伤口}/\currentpos,
        {瘰疬分枝杆菌}/{M. scrofulaceum}/0.00020/{0-0.05}/15\%/88.94\%/丰度较低/{环境中广泛分布。可引起颈部淋巴结炎。}/{注意个人卫生,加强环境防护}/\currentpos,
        {结核分枝杆菌}/{M. tuberculosis}/0.13505/{0-0.05}/95\%/35.19\%/超标/{重要呼吸道传染病病原体。可引起肺结核等。需高度重视。}/{及时就医检查,进行规范治疗}/\currentpos,
        {溃疡分枝杆菌}/{M. ulcerans}/ND/{0-0.05}/8\%/91.83\%/未检出/{热带地区特有菌种。可引起伯鲁利溃疡。}/{注意皮肤防护,及时处理伤口}/\currentpos,
        {类志贺邻单胞菌}/{P. shigelloides}/0.00020/{0-0.05}/27\%/77.88\%/丰度较低/{水生环境常见。可引起胃肠道感染。}/{注意饮食卫生,避免生食水产品}/\currentpos,
        {猪霍乱沙门氏菌}/{S. enterica Choleraesuis}/ND/{0-0.05}/15\%/88.94\%/未检出/{猪群常见致病菌。可引起人类严重感染。}/{注意食品卫生,避免食用未熟食品}/\currentpos
    }
    {
        % 计算当前行的基础位置
        \pgfmathsetmacro{\basepos}{-2.8*\currentpos}

        % 菌种名称
        \node[cell, align=left] at (0.5,\basepos) {
            \small\textbf{\item}\\[-0.2em]
            {\color{lightgray}\small\enitem}
        };

        % 正常范围
        \node[reference] at (4.5,\basepos) {\footnotesize\range};

        % 进度条相关
        \pgfmathsetmacro{\barypos}{\basepos-\valuebarspace+0.1}
        \def\barstart{6.75}

        % 进度条背景
        \fill[gray!10, rounded corners=2] (\barstart,\barypos)
            rectangle (\barstart+\barwidth,\barypos+\barheight);

        % 检测丰度值
        \node[value] at (7.5,{\basepos-\valuebarspace+0.6}) {\footnotesize\value};

        % 解析范围并计算进度条长度
        \def\parserange#1-#2\endparse{\def\minval{#1}\def\maxval{#2}}
        \expandafter\parserange\range\endparse

        % 计算进度条长度和颜色
        \pgfmathsetmacro{\progress}{min(\value/\maxval, 1.0)}
        \pgfmathparse{\value > \maxval ? "customred" : (\value < \minval ? "customred" : "green!50")}
        \let\barcolor=\pgfmathresult

        % 进度条显示
        \ifnum\pdfstrcmp{\status}{超标}=0
            \fill[customred, rounded corners=2] (\barstart,\barypos)
                rectangle (\barstart+\barwidth,\barypos+\barheight);
        \else
            \fill[\barcolor, rounded corners=2] (\barstart,\barypos)
                rectangle (\barstart+\barwidth*\progress,\barypos+\barheight);
        \fi

        % 结果评价
        \ifnum\pdfstrcmp{\status}{超标}=0
            \node[value, text=customred] at (10.5,\basepos) {\footnotesize\textbf{\status}};
        \else
            \node[value, text=customGreen] at (10.5,\basepos) {\footnotesize\textbf{\status}};
        \fi

        % 丰度超过%人
        \node[value] at (13,\basepos) {\footnotesize\percentile};

        % %正常人检出
        \node[value] at (16,\basepos) {\footnotesize\detection};

        % 添加卡片
        \pgfmathsetmacro{\cardypos}{\basepos-0.5}
        \begin{scope}[shift={(0,\cardypos)}]
            % 卡片背景
            \pgfmathsetmacro{\cardheight}{
                \ifnum\pdfstrcmp{\status}{超标}=0
                    1.0  % 两行内容时的高度
                \else
                    0.6  % 一行内容时的高度
                \fi
            }

            \fill[rounded corners=5pt, customTeal!5, draw=gray!5]
                (0.3,-\cardheight) rectangle (17.3,0);

            % 菌群简介图标和内容
            \node[anchor=west] at (0.5,-0.3) {
                \textbf{\color{gray!90}\footnotesize \textcolor{customTeal}{\faInfoCircle}}
            };
            \node[anchor=west, text width=16cm] at (0.9,-0.3) {
                {\small\color{gray}\footnotesize \intro}
            };

            % 异常解读标题和内容
            \ifnum\pdfstrcmp{\status}{超标}=0
                \node[anchor=west] at (0.55,-0.8) {
                    \textbf{\color{customRed}\footnotesize \textcolor{customRed}{\faBell}}
                };
                \node[anchor=west, text width=16cm] at (0.9,-0.8) {
                    {\small\color{gray}\footnotesize \suggestion}
                };
            \fi
        \end{scope}

        % 分割线
        \pgfmathsetmacro{\linepos}{
            \ifnum\pdfstrcmp{\status}{超标}=0
                \basepos-1.7  % 超标时的分割线位置
            \else
                \basepos-1.3  % 正常时的分割线位置
            \fi
        }
        \draw[gray!20] (0.2,\linepos) -- (\cardwidth-0.2,\linepos);

        % 根据当前行的状态计算下一行的位置增量
        \ifnum\pdfstrcmp{\status}{超标}=0
            \pgfmathsetmacro{\increment}{0.85}  % 超标行(两行内容)需要更大的增量
        \else
            \pgfmathsetmacro{\increment}{0.7}  % 正常行(一行内容)使用较小的增量
        \fi

        % 更新位置计数器
        \pgfmathsetmacro{\nextpos}{\currentpos+\increment}
        \xdef\currentpos{\nextpos}
    }

    % 最后一行的处理,消除多余的空白
    \pgfmathsetmacro{\lastincrement}{2}  % 最后一行的增量
    \pgfmathsetmacro{\nextpos}{\currentpos+\lastincrement}
    \xdef\currentpos{\nextpos}

\end{tikzpicture}
\end{center}

\newpage

\begin{center}
\begin{tikzpicture}[
    font=\small,
    title/.style={font=\small\bfseries\color{white}},
    value/.style={font=\small},
    reference/.style={font=\small},
    cell/.style={anchor=west, text width=4.2cm},
    note/.style={anchor=west, text width=4.5cm, align=left}
]
    \def\cardwidth{\textwidth}
    \def\cardheight{22}
    \def\barheight{0.25}
    \def\barwidth{1.5}
    \def\valuebarspace{0.4}

    % 容器和标题栏背景
    \draw[rounded corners=5, fill=white, draw=gray!20]
        (0,0) rectangle (\cardwidth,-\cardheight);
    \path[fill=customTeal]
        (0,0) [rounded corners=5] -- (\cardwidth,0) --
        (\cardwidth,0.8) -- (0,0.8) -- cycle;

    % 表头
    \node[title, anchor=west] at (0.5,0.4) {\textbf{菌种名称}};
    \node[title] at (4.5,0.4) {\textbf{正常范围}};
    \node[title] at (7.5,0.4) {\textbf{检测丰度}};
    \node[title] at (10.5,0.4) {\textbf{结果评价}};
    \node[title] at (13,0.4) {\textbf{超过\%的人}};
    \node[title] at (16,0.4) {\textbf{有\%的正常人检出}};

    % 初始化位置计数器
    \def\currentpos{0.25}

    % 数据行和卡片
    \foreach \item/\enitem/\value/\range/\percentile/\detection/\status/\intro/\suggestion/\index in {
        {肠炎沙门氏菌}/{S. Enteritidis}/ND/{0-0.05}/15\%/88.94\%/未检出/{重要食源性致病菌。是沙门氏菌感染的主要原因之一。可引起急性胃肠炎。}/{注意食品卫生,避免食用未熟食品}/\currentpos,
        {副伤寒沙门氏菌-A}/{S. Paratyphi A}/ND/{0-0.05}/8\%/91.83\%/未检出/{人类特异性致病菌。可引起副伤寒,症状类似伤寒但较轻。}/{注意饮食卫生,预防粪-口传播}/\currentpos,
        {副伤寒沙门氏菌-B}/{S. Paratyphi B}/ND/{0-0.05}/15\%/88.94\%/未检出/{人类特异性病原体。引起B型副伤寒。发热和胃肠道症状显著。}/{加强个人卫生,注意饮食安全}/\currentpos,
        {副伤寒沙门氏菌-C}/{S. Paratyphi C}/ND/{0-0.05}/8\%/91.83\%/未检出/{致病性相对较弱的副伤寒菌。可引起C型副伤寒。}/{注意饮食卫生,避免交叉感染}/\currentpos,
        {伤寒沙门氏菌}/{S. Typhi}/ND/{0-0.05}/27\%/77.88\%/未检出/{严重的人类特异性病原体。可引起伤寒,威胁生命。}/{加强个人卫生,注意饮食和饮水安全}/\currentpos,
        {鼠伤寒沙门菌}/{S. Typhimurium}/ND/{0-0.05}/15\%/88.94\%/未检出/{广谱性致病菌。可感染多种宿主,引起急性胃肠炎。}/{注意食品安全,避免生食}/\currentpos,
        {鲍氏志贺氏菌}/{Shigella boydii}/ND/{0-0.05}/8\%/91.83\%/未检出/{肠道致病菌。可引起细菌性痢疾,主要在发展中国家流行。}/{注意个人卫生,预防粪-口传播}/\currentpos,
        {痢疾志贺氏菌}/{Shigella dysenteriae}/ND/{0-0.05}/15\%/88.94\%/未检出/{最严重的志贺氏菌。可引起出血性痢疾,产生致命毒素。}/{加强卫生防护,及时就医治疗}/\currentpos,
        {福氏志贺氏菌}/{Shigella flexneri}/0.03009/{0-0.05}/52\%/81.25\%/丰度较低/{痢疾常见病原体。可引起急性细菌性痢疾。}/{注意饮食卫生,预防交叉感染}/\currentpos,
        {宋内志贺氏菌}/{Shigella sonnei}/0.01473/{0-0.05}/38\%/80.29\%/丰度较低/{相对温和的痢疾菌。常引起轻症腹泻和肠道感染。}/{注意个人卫生,加强饮食卫生}/\currentpos,
        {金黄色葡萄球菌}/{S. aureus}/0.50389/{0-0.08}/95\%/35.19\%/超标/{重要致病菌。可引起皮肤感染、食物中毒等多种疾病。需警惕。}/{及时就医诊治,注意个人卫生}/\currentpos
    }
    {
        % 计算当前行的基础位置
        \pgfmathsetmacro{\basepos}{-2.8*\currentpos}

        % 菌种名称
        \node[cell, align=left] at (0.5,\basepos) {
            \small\textbf{\item}\\[-0.2em]
            {\color{lightgray}\small\enitem}
        };

        % 正常范围
        \node[reference] at (4.5,\basepos) {\footnotesize\range};

        % 进度条相关
        \pgfmathsetmacro{\barypos}{\basepos-\valuebarspace+0.1}
        \def\barstart{6.75}

        % 进度条背景
        \fill[gray!10, rounded corners=2] (\barstart,\barypos)
            rectangle (\barstart+\barwidth,\barypos+\barheight);

        % 检测丰度值
        \node[value] at (7.5,{\basepos-\valuebarspace+0.6}) {\footnotesize\value};

        % 解析范围并计算进度条长度
        \def\parserange#1-#2\endparse{\def\minval{#1}\def\maxval{#2}}
        \expandafter\parserange\range\endparse

        % 计算进度条长度和颜色
        \pgfmathsetmacro{\progress}{min(\value/\maxval, 1.0)}
        \pgfmathparse{\value > \maxval ? "customred" : (\value < \minval ? "customred" : "green!50")}
        \let\barcolor=\pgfmathresult

        % 进度条显示
        \ifnum\pdfstrcmp{\status}{超标}=0
            \fill[customred, rounded corners=2] (\barstart,\barypos)
                rectangle (\barstart+\barwidth,\barypos+\barheight);
        \else
            \fill[\barcolor, rounded corners=2] (\barstart,\barypos)
                rectangle (\barstart+\barwidth*\progress,\barypos+\barheight);
        \fi

        % 结果评价
        \ifnum\pdfstrcmp{\status}{超标}=0
            \node[value, text=customred] at (10.5,\basepos) {\footnotesize\textbf{\status}};
        \else
            \node[value, text=customGreen] at (10.5,\basepos) {\footnotesize\textbf{\status}};
        \fi

        % 丰度超过%人
        \node[value] at (13,\basepos) {\footnotesize\percentile};

        % %正常人检出
        \node[value] at (16,\basepos) {\footnotesize\detection};

        % 添加卡片
        \pgfmathsetmacro{\cardypos}{\basepos-0.5}
        \begin{scope}[shift={(0,\cardypos)}]
            % 卡片背景
            \pgfmathsetmacro{\cardheight}{
                \ifnum\pdfstrcmp{\status}{超标}=0
                    1.0  % 两行内容时的高度
                \else
                    0.6  % 一行内容时的高度
                \fi
            }

            \fill[rounded corners=5pt, customTeal!5, draw=gray!5]
                (0.3,-\cardheight) rectangle (17.3,0);

            % 菌群简介图标和内容
            \node[anchor=west] at (0.5,-0.3) {
                \textbf{\color{gray!90}\footnotesize \textcolor{customTeal}{\faInfoCircle}}
            };
            \node[anchor=west, text width=16cm] at (0.9,-0.3) {
                {\small\color{gray}\footnotesize \intro}
            };

            % 异常解读标题和内容
            \ifnum\pdfstrcmp{\status}{超标}=0
                \node[anchor=west] at (0.55,-0.8) {
                    \textbf{\color{customRed}\footnotesize \textcolor{customRed}{\faBell}}
                };
                \node[anchor=west, text width=16cm] at (0.9,-0.8) {
                    {\small\color{gray}\footnotesize \suggestion}
                };
            \fi
        \end{scope}

        % 分割线
        \pgfmathsetmacro{\linepos}{
            \ifnum\pdfstrcmp{\status}{超标}=0
                \basepos-1.7  % 超标时的分割线位置
            \else
                \basepos-1.3  % 正常时的分割线位置
            \fi
        }
        \draw[gray!20] (0.2,\linepos) -- (\cardwidth-0.2,\linepos);

        % 根据当前行的状态计算下一行的位置增量
        \ifnum\pdfstrcmp{\status}{超标}=0
            \pgfmathsetmacro{\increment}{0.85}  % 超标行(两行内容)需要更大的增量
        \else
            \pgfmathsetmacro{\increment}{0.7}  % 正常行(一行内容)使用较小的增量
        \fi

        % 更新位置计数器
        \pgfmathsetmacro{\nextpos}{\currentpos+\increment}
        \xdef\currentpos{\nextpos}
    }

    % 最后一行的处理,消除多余的空白
    \pgfmathsetmacro{\lastincrement}{2}  % 最后一行的增量
    \pgfmathsetmacro{\nextpos}{\currentpos+\lastincrement}
    \xdef\currentpos{\nextpos}

\end{tikzpicture}
\end{center}

\newpage

\begin{center}
\begin{tikzpicture}[
    font=\small,
    title/.style={font=\small\bfseries\color{white}},
    value/.style={font=\small},
    reference/.style={font=\small},
    cell/.style={anchor=west, text width=4.2cm},
    note/.style={anchor=west, text width=4.5cm, align=left}
]
    \def\cardwidth{\textwidth}
    \def\cardheight{22}
    \def\barheight{0.25}
    \def\barwidth{1.5}
    \def\valuebarspace{0.4}

    % 容器和标题栏背景
    \draw[rounded corners=5, fill=white, draw=gray!20]
        (0,0) rectangle (\cardwidth,-\cardheight);
    \path[fill=customTeal]
        (0,0) [rounded corners=5] -- (\cardwidth,0) --
        (\cardwidth,0.8) -- (0,0.8) -- cycle;

    % 表头
    \node[title, anchor=west] at (0.5,0.4) {\textbf{菌种名称}};
    \node[title] at (4.5,0.4) {\textbf{正常范围}};
    \node[title] at (7.5,0.4) {\textbf{检测丰度}};
    \node[title] at (10.5,0.4) {\textbf{结果评价}};
    \node[title] at (13,0.4) {\textbf{超过\%的人}};
    \node[title] at (16,0.4) {\textbf{有\%的正常人检出}};

    % 初始化位置计数器
    \def\currentpos{0.25}

    % 数据行和卡片
    \foreach \item/\enitem/\value/\range/\percentile/\detection/\status/\intro/\suggestion/\index in {
        {猪链球菌}/{S. suis}/0.00938/{0-0.05}/27\%/77.88\%/丰度较低/{重要人兽共患病原体。可引起脑膜炎等严重感染。}/{注意职业防护,避免接触感染动物}/\currentpos,
        {齿垢密螺旋体}/{T. denticola}/0.00071/{0-0.05}/15\%/88.94\%/丰度较低/{口腔常见菌。与牙周疾病发生相关。}/{加强口腔卫生,定期洁牙}/\currentpos,
        {梅毒密螺旋体}/{T. pallidum}/ND/{0-0.05}/8\%/91.83\%/未检出/{性传播疾病病原体。可引起梅毒,需及时治疗。}/{注意防护,必要时及时就医}/\currentpos,
        {文氏密螺旋体}/{T. vincentii}/ND/{0-0.05}/15\%/88.94\%/未检出/{口腔常见螺旋体。与急性坏死性龈炎有关。}/{保持口腔卫生,预防牙龈疾病}/\currentpos,
        {霍乱弧菌}/{V. cholerae}/0.00047/{0-0.05}/27\%/77.88\%/丰度较低/{重要肠道致病菌。可引起霍乱,造成严重腹泻。}/{注意饮食卫生,避免生食海产品}/\currentpos,
        {拟态弧菌}/{V. mimicus}/ND/{0-0.05}/8\%/91.83\%/未检出/{海洋环境常见菌。可引起胃肠炎和腹泻。}/{注意海产品卫生,避免生食}/\currentpos,
        {副溶血弧菌}/{V. parahaemolyticus}/0.00119/{0-0.05}/38\%/80.29\%/丰度较低/{海产品相关致病菌。可引起急性胃肠炎和食物中毒。}/{避免食用生海产品,注意加热充分}/\currentpos,
        {创伤弧菌}/{V. vulnificus}/0.00188/{0-0.05}/38\%/80.29\%/丰度较低/{海洋环境中的严重致病菌。可通过伤口感染或食用海产品引起重症感染,免疫力低下者风险更高。}/{避免生食海产品,伤口避免接触海水,及时处理伤口}/\currentpos,
        {小肠结肠炎耶尔森菌}/{Y. enterocolitica}/ND/{0-0.05}/15\%/88.94\%/未检出/{重要食源性致病菌。可引起急性肠炎,常通过污染的食品和水传播。}/{注意食品卫生,避免食用未充分加热的猪肉制品}/\currentpos,
        {鼠疫耶尔森菌}/{Yersinia pestis}/ND/{0-0.05}/8\%/91.83\%/未检出/{重要烈性传染病病原体。可引起鼠疫,主要通过啮齿类动物和跳蚤传播。需严格防控。}/{加强环境卫生,避免接触啮齿类动物}/\currentpos,
        {假结核耶尔森菌}/{Y. pseudotuberculosis}/ND/{0-0.05}/15\%/88.94\%/未检出/{人兽共患病病原体。可引起类似结核的症状,通过污染的食物和水传播。}/{注意饮食卫生,避免食用未经充分处理的食品}/\currentpos
    }
    {
        % 计算当前行的基础位置
        \pgfmathsetmacro{\basepos}{-2.8*\currentpos}

        % 菌种名称
        \node[cell, align=left] at (0.5,\basepos) {
            \small\textbf{\item}\\[-0.2em]
            {\color{lightgray}\small\enitem}
        };

        % 正常范围
        \node[reference] at (4.5,\basepos) {\footnotesize\range};

        % 进度条相关
        \pgfmathsetmacro{\barypos}{\basepos-\valuebarspace+0.1}
        \def\barstart{6.75}

        % 进度条背景
        \fill[gray!10, rounded corners=2] (\barstart,\barypos)
            rectangle (\barstart+\barwidth,\barypos+\barheight);

        % 检测丰度值
        \node[value] at (7.5,{\basepos-\valuebarspace+0.6}) {\footnotesize\value};

        % 解析范围并计算进度条长度
        \def\parserange#1-#2\endparse{\def\minval{#1}\def\maxval{#2}}
        \expandafter\parserange\range\endparse

        % 计算进度条长度和颜色
        \pgfmathsetmacro{\progress}{min(\value/\maxval, 1.0)}
        \pgfmathparse{\value > \maxval ? "customred" : (\value < \minval ? "customred" : "green!50")}
        \let\barcolor=\pgfmathresult

        % 进度条显示
        \ifnum\pdfstrcmp{\status}{超标}=0
            \fill[customred, rounded corners=2] (\barstart,\barypos)
                rectangle (\barstart+\barwidth,\barypos+\barheight);
        \else
            \fill[\barcolor, rounded corners=2] (\barstart,\barypos)
                rectangle (\barstart+\barwidth*\progress,\barypos+\barheight);
        \fi

        % 结果评价
        \ifnum\pdfstrcmp{\status}{超标}=0
            \node[value, text=customred] at (10.5,\basepos) {\footnotesize\textbf{\status}};
        \else
            \node[value, text=customGreen] at (10.5,\basepos) {\footnotesize\textbf{\status}};
        \fi

        % 丰度超过%人
        \node[value] at (13,\basepos) {\footnotesize\percentile};

        % %正常人检出
        \node[value] at (16,\basepos) {\footnotesize\detection};

        % 添加卡片
        \pgfmathsetmacro{\cardypos}{\basepos-0.5}
        \begin{scope}[shift={(0,\cardypos)}]
            % 卡片背景
            \pgfmathsetmacro{\cardheight}{
                \ifnum\pdfstrcmp{\status}{超标}=0
                    1.0  % 两行内容时的高度
                \else
                    0.6  % 一行内容时的高度
                \fi
            }

            \fill[rounded corners=5pt, customTeal!5, draw=gray!5]
                (0.3,-\cardheight) rectangle (17.3,0);

            % 菌群简介图标和内容
            \node[anchor=west] at (0.5,-0.3) {
                \textbf{\color{gray!90}\footnotesize \textcolor{customTeal}{\faInfoCircle}}
            };
            \node[anchor=west, text width=16cm] at (0.9,-0.3) {
                {\small\color{gray}\footnotesize \intro}
            };

            % 异常解读标题和内容
            \ifnum\pdfstrcmp{\status}{超标}=0
                \node[anchor=west] at (0.55,-0.8) {
                    \textbf{\color{customRed}\footnotesize \textcolor{customRed}{\faBell}}
                };
                \node[anchor=west, text width=16cm] at (0.9,-0.8) {
                    {\small\color{gray}\footnotesize \suggestion}
                };
            \fi
        \end{scope}

        % 分割线
        \pgfmathsetmacro{\linepos}{
            \ifnum\pdfstrcmp{\status}{超标}=0
                \basepos-1.7  % 超标时的分割线位置
            \else
                \basepos-1.3  % 正常时的分割线位置
            \fi
        }
        \draw[gray!20] (0.2,\linepos) -- (\cardwidth-0.2,\linepos);

        % 根据当前行的状态计算下一行的位置增量
        \ifnum\pdfstrcmp{\status}{超标}=0
            \pgfmathsetmacro{\increment}{0.85}  % 超标行(两行内容)需要更大的增量
        \else
            \pgfmathsetmacro{\increment}{0.7}  % 正常行(一行内容)使用较小的增量
        \fi

        % 更新位置计数器
        \pgfmathsetmacro{\nextpos}{\currentpos+\increment}
        \xdef\currentpos{\nextpos}
    }

    % 最后一行的处理,消除多余的空白
    \pgfmathsetmacro{\lastincrement}{2}  % 最后一行的增量
    \pgfmathsetmacro{\nextpos}{\currentpos+\lastincrement}
    \xdef\currentpos{\nextpos}

\end{tikzpicture}
\end{center}

\newpage

\begin{tcolorbox}[
    enhanced,
    colback=white,
    colframe=white,
    arc=2mm,
    boxrule=0pt,
    width=\textwidth,
    left=15pt,
    right=15pt,
    top=10pt,
    bottom=10pt,
    drop shadow={
        opacity=0.2,
        color=customTeal
    },
    borderline west={5pt}{0pt}{customTeal}
]
\textcolor{customTeal}{\Large\textbf{(常见)真菌}}
\end{tcolorbox}

\begin{tcolorbox}[
    enhanced,
    colback=customTealBg,
    colframe=customTealBg,
    arc=3mm,
    boxrule=0pt,
    width=\textwidth,
    top=8pt,
    bottom=8pt
]
{\small{\color{customTeal}\faInfoCircle} 真菌检测对于评估免疫状态和潜在感染风险非常重要。尽管一些真菌在正常情况下不会引起健康问题,但对于免疫系统较弱的人群来说,持续监测是预防真菌感染的关键。下列是一些常见肠道真菌在您肠道中的检测情况。
}
\end{tcolorbox}
\vspace{-0.5cm}
\begin{center}
\begin{tikzpicture}[
    font=\small,
    title/.style={font=\small\bfseries\color{white}},
    value/.style={font=\small},
    reference/.style={font=\small},
    cell/.style={anchor=west, text width=4.2cm},
    note/.style={anchor=west, text width=4.5cm, align=left}
]
    \def\cardwidth{\textwidth}
    \def\cardheight{18.75}
    \def\barheight{0.25}
    \def\barwidth{1.5}
    \def\valuebarspace{0.4}

    % 容器和标题栏背景
    \draw[rounded corners=5, fill=white, draw=gray!20]
        (0,0) rectangle (\cardwidth,-\cardheight);
    \path[fill=customTeal]
        (0,0) [rounded corners=5] -- (\cardwidth,0) --
        (\cardwidth,0.8) -- (0,0.8) -- cycle;

    % 表头
    \node[title, anchor=west] at (0.5,0.4) {\textbf{菌种名称}};
    \node[title] at (4.5,0.4) {\textbf{正常范围}};
    \node[title] at (7.5,0.4) {\textbf{检测丰度}};
    \node[title] at (10.5,0.4) {\textbf{结果评价}};
    \node[title] at (13,0.4) {\textbf{超过\%的人}};
    \node[title] at (16,0.4) {\textbf{有\%的正常人检出}};

    % 初始化位置计数器
    \def\currentpos{0.25}

    % 数据行和卡片
    \foreach \item/\enitem/\value/\range/\percentile/\detection/\status/\intro/\suggestion/\index in {
        {白色念珠菌}/{Candida albicans}/0.00034/{0-0.05}/27\%/77.88\%/丰度较低/{人体常见条件致病菌。正常菌群成员,免疫力下降时可过度生长。}/{注意个人卫生,增强免疫力}/\currentpos,
        {酿酒酵母}/{S. cerevisiae}/0.00022/{0-0.05}/15\%/88.94\%/丰度较低/{益生菌,广泛应用于发酵工业。参与营养代谢,促进肠道健康。}/{}/\currentpos,
        {真贝酵母}/{S. eubayanus}/ND/{0-0.05}/8\%/91.83\%/未检出/{低温发酵酵母。参与啤酒发酵过程。对人体无害。}/{}/\currentpos,
        {奇异酵母}/{S. paradoxus}/ND/{0-0.05}/15\%/88.94\%/未检出/{野生酵母菌种。与酿酒酵母亲缘关系密切。一般不致病。}/{}/\currentpos,
        {木糖发酵酵母}/{S. stipitis}/ND/{0-0.05}/8\%/91.83\%/未检出/{特殊碳源利用菌。可发酵木糖。工业应用菌种。}/{}/\currentpos,
        {黄曲霉}/{A. flavus}/ND/{0-0.05}/27\%/77.88\%/未检出/{常见真菌。可产生黄曲霉毒素。注意储存食品防霉。}/{加强食品储存管理}/\currentpos,
        {烟曲霉}/{A. fumigatus}/ND/{0-0.05}/38\%/80.29\%/未检出/{重要条件致病真菌。可引起肺部感染。免疫力低下者易感。}/{预防真菌感染}/\currentpos,
        {米曲霉}/{A. oryzae}/ND/{0-0.05}/15\%/88.94\%/未检出/{食品发酵工业常用菌种。参与酱油等发酵食品制作。益生菌。}/{}/\currentpos,
        {新型隐球菌}/{C. neoformans}/ND/{0-0.05}/8\%/91.83\%/未检出/{条件致病真菌。可引起中枢神经系统感染。需警惕。}/{加强防护,注意卫生}/\currentpos
    }
    {
        % 计算当前行的基础位置
        \pgfmathsetmacro{\basepos}{-2.8*\currentpos}

        % 菌种名称
        \node[cell, align=left] at (0.5,\basepos) {
            \small\textbf{\item}\\[-0.2em]
            {\color{lightgray}\small\enitem}
        };

        % 正常范围
        \node[reference] at (4.5,\basepos) {\footnotesize\range};

        % 进度条相关
        \pgfmathsetmacro{\barypos}{\basepos-\valuebarspace+0.1}
        \def\barstart{6.75}

        % 进度条背景
        \fill[gray!10, rounded corners=2] (\barstart,\barypos)
            rectangle (\barstart+\barwidth,\barypos+\barheight);

        % 检测丰度值
        \node[value] at (7.5,{\basepos-\valuebarspace+0.6}) {\footnotesize\value};

        % 解析范围并计算进度条长度
        \def\parserange#1-#2\endparse{\def\minval{#1}\def\maxval{#2}}
        \expandafter\parserange\range\endparse

        % 计算进度条长度和颜色
        \pgfmathsetmacro{\progress}{min(\value/\maxval, 1.0)}
        \pgfmathparse{\value > \maxval ? "customred" : (\value < \minval ? "customred" : "green!50")}
        \let\barcolor=\pgfmathresult

        % 进度条显示
        \ifnum\pdfstrcmp{\status}{超标}=0
            \fill[customred, rounded corners=2] (\barstart,\barypos)
                rectangle (\barstart+\barwidth,\barypos+\barheight);
        \else
            \fill[\barcolor, rounded corners=2] (\barstart,\barypos)
                rectangle (\barstart+\barwidth*\progress,\barypos+\barheight);
        \fi

        % 结果评价
        \ifnum\pdfstrcmp{\status}{超标}=0
            \node[value, text=customRed] at (10.5,\basepos) {\footnotesize\textbf{\status}};
        \else
            \node[value, text=customGreen] at (10.5,\basepos) {\footnotesize\textbf{\status}};
        \fi

        % 丰度超过%人
        \node[value] at (13,\basepos) {\footnotesize\percentile};

        % %正常人检出
        \node[value] at (16,\basepos) {\footnotesize\detection};

        % 添加卡片
        \pgfmathsetmacro{\cardypos}{\basepos-0.5}
        \begin{scope}[shift={(0,\cardypos)}]
            % 卡片背景
            \pgfmathsetmacro{\cardheight}{
                \ifnum\pdfstrcmp{\status}{超标}=0
                    1.0  % 两行内容时的高度
                \else
                    0.6  % 一行内容时的高度
                \fi
            }

            \fill[rounded corners=5pt, customTeal!5, draw=gray!5]
                (0.3,-\cardheight) rectangle (17.3,0);

            % 菌群简介图标和内容
            \node[anchor=west] at (0.5,-0.3) {
                \textbf{\color{gray!90}\footnotesize \textcolor{customTeal}{\faInfoCircle}}
            };
            \node[anchor=west, text width=16cm] at (0.9,-0.3) {
                {\small\color{gray}\footnotesize \intro}
            };

            % 异常解读标题和内容
            \ifnum\pdfstrcmp{\status}{超标}=0
                \node[anchor=west] at (0.55,-0.8) {
                    \textbf{\color{customRed}\footnotesize \textcolor{customRed}{\faBell}}
                };
                \node[anchor=west, text width=16cm] at (0.9,-0.8) {
                    {\small\color{gray}\footnotesize \suggestion}
                };
            \fi
        \end{scope}

        % 分割线
        \pgfmathsetmacro{\linepos}{
            \ifnum\pdfstrcmp{\status}{超标}=0
                \basepos-1.7  % 超标时的分割线位置
            \else
                \basepos-1.3  % 正常时的分割线位置
            \fi
        }
        \draw[gray!20] (0.2,\linepos) -- (\cardwidth-0.2,\linepos);

        % 根据当前行的状态计算下一行的位置增量
        \ifnum\pdfstrcmp{\status}{超标}=0
            \pgfmathsetmacro{\increment}{0.85}  % 超标行(两行内容)需要更大的增量
        \else
            \pgfmathsetmacro{\increment}{0.7}  % 正常行(一行内容)使用较小的增量
        \fi

        % 更新位置计数器
        \pgfmathsetmacro{\nextpos}{\currentpos+\increment}
        \xdef\currentpos{\nextpos}
    }

    % 最后一行的处理,消除多余的空白
    \pgfmathsetmacro{\lastincrement}{2}  % 最后一行的增量
    \pgfmathsetmacro{\nextpos}{\currentpos+\lastincrement}
    \xdef\currentpos{\nextpos}

\end{tikzpicture}
\end{center}

\newpage

\begin{center}
\begin{tikzpicture}[
    font=\small,
    title/.style={font=\small\bfseries\color{white}},
    value/.style={font=\small},
    reference/.style={font=\small},
    cell/.style={anchor=west, text width=4.2cm},
    note/.style={anchor=west, text width=4.5cm, align=left}
]
    \def\cardwidth{\textwidth}
    \def\cardheight{18.75}
    \def\barheight{0.25}
    \def\barwidth{1.5}
    \def\valuebarspace{0.4}

    % 容器和标题栏背景
    \draw[rounded corners=5, fill=white, draw=gray!20]
        (0,0) rectangle (\cardwidth,-\cardheight);
    \path[fill=customTeal]
        (0,0) [rounded corners=5] -- (\cardwidth,0) --
        (\cardwidth,0.8) -- (0,0.8) -- cycle;

    % 表头
    \node[title, anchor=west] at (0.5,0.4) {\textbf{菌种名称}};
    \node[title] at (4.5,0.4) {\textbf{正常范围}};
    \node[title] at (7.5,0.4) {\textbf{检测丰度}};
    \node[title] at (10.5,0.4) {\textbf{结果评价}};
    \node[title] at (13,0.4) {\textbf{超过\%的人}};
    \node[title] at (16,0.4) {\textbf{有\%的正常人检出}};

    % 初始化位置计数器
    \def\currentpos{0.25}

    % 数据行和卡片
    \foreach \item/\enitem/\value/\range/\percentile/\detection/\status/\intro/\suggestion/\index in {
        {汉氏德巴利氏酵母}/{D. hansenii}/0.00022/{0-0.05}/27\%/77.88\%/丰度较低/{耐盐酵母。参与发酵食品制作。对人体无害。}/{}/\currentpos,
        {产甘油假丝酵母}/{E. gossypii}/ND/{0-0.05}/15\%/88.94\%/未检出/{工业微生物。用于维生素B2生产。一般不致病。}/{}/\currentpos,
        {轮枝镰刀菌}/{F. verticillioides}/ND/{0-0.05}/8\%/91.83\%/未检出/{植物病原真菌。可产生真菌毒素。注意食品储存。}/{注意食品储存条件}/\currentpos,
        {尖孢镰刀菌}/{F. oxysporum}/0.00073/{0-0.05}/27\%/77.88\%/丰度较低/{常见土壤真菌。部分株系可致病。注意防护。}/{加强卫生防护}/\currentpos,
        {禾谷镰刀菌}/{F. graminearum}/ND/{0-0.05}/15\%/88.94\%/未检出/{作物病原真菌。可产生真菌毒素。影响粮食安全。}/{注意储存防霉}/\currentpos,
        {乳酸克鲁维酵母}/{K. lactis}/ND/{0-0.05}/8\%/91.83\%/未检出/{工业发酵菌种。参与乳制品发酵。益生作用。}/{}/\currentpos,
        {稻瘟病菌}/{P. grisea}/ND/{0-0.05}/15\%/88.94\%/未检出/{水稻重要病原真菌。一般不感染人类。环境真菌。}/{}/\currentpos,
        {玉米黑粉菌}/{U. maydis}/ND/{0-0.05}/8\%/91.83\%/未检出/{玉米特异性病原菌。食用菌品种。对人体无害。}/{}/\currentpos
    }
    {
        % 计算当前行的基础位置
        \pgfmathsetmacro{\basepos}{-2.8*\currentpos}

        % 菌种名称
        \node[cell, align=left] at (0.5,\basepos) {
            \small\textbf{\item}\\[-0.2em]
            {\color{lightgray}\small\enitem}
        };

        % 正常范围
        \node[reference] at (4.5,\basepos) {\footnotesize\range};

        % 进度条相关
        \pgfmathsetmacro{\barypos}{\basepos-\valuebarspace+0.1}
        \def\barstart{6.75}

        % 进度条背景
        \fill[gray!10, rounded corners=2] (\barstart,\barypos)
            rectangle (\barstart+\barwidth,\barypos+\barheight);

        % 检测丰度值
        \node[value] at (7.5,{\basepos-\valuebarspace+0.6}) {\footnotesize\value};

        % 解析范围并计算进度条长度
        \def\parserange#1-#2\endparse{\def\minval{#1}\def\maxval{#2}}
        \expandafter\parserange\range\endparse

        % 计算进度条长度和颜色
        \pgfmathsetmacro{\progress}{min(\value/\maxval, 1.0)}
        \pgfmathparse{\value > \maxval ? "customred" : (\value < \minval ? "customred" : "green!50")}
        \let\barcolor=\pgfmathresult

        % 进度条显示
        \ifnum\pdfstrcmp{\status}{超标}=0
            \fill[customred, rounded corners=2] (\barstart,\barypos)
                rectangle (\barstart+\barwidth,\barypos+\barheight);
        \else
            \fill[\barcolor, rounded corners=2] (\barstart,\barypos)
                rectangle (\barstart+\barwidth*\progress,\barypos+\barheight);
        \fi

        % 结果评价
        \ifnum\pdfstrcmp{\status}{超标}=0
            \node[value, text=customRed] at (10.5,\basepos) {\footnotesize\textbf{\status}};
        \else
            \node[value, text=customGreen] at (10.5,\basepos) {\footnotesize\textbf{\status}};
        \fi

        % 丰度超过%人
        \node[value] at (13,\basepos) {\footnotesize\percentile};

        % %正常人检出
        \node[value] at (16,\basepos) {\footnotesize\detection};

        % 添加卡片
        \pgfmathsetmacro{\cardypos}{\basepos-0.5}
        \begin{scope}[shift={(0,\cardypos)}]
            % 卡片背景
            \pgfmathsetmacro{\cardheight}{
                \ifnum\pdfstrcmp{\status}{超标}=0
                    1.0  % 两行内容时的高度
                \else
                    0.6  % 一行内容时的高度
                \fi
            }

            \fill[rounded corners=5pt, customTeal!5, draw=gray!5]
                (0.3,-\cardheight) rectangle (17.3,0);

            % 菌群简介图标和内容
            \node[anchor=west] at (0.5,-0.3) {
                \textbf{\color{gray!90}\footnotesize \textcolor{customTeal}{\faInfoCircle}}
            };
            \node[anchor=west, text width=16cm] at (0.9,-0.3) {
                {\small\color{gray}\footnotesize \intro}
            };

            % 异常解读标题和内容
            \ifnum\pdfstrcmp{\status}{超标}=0
                \node[anchor=west] at (0.55,-0.8) {
                    \textbf{\color{customRed}\footnotesize \textcolor{customRed}{\faBell}}
                };
                \node[anchor=west, text width=16cm] at (0.9,-0.8) {
                    {\small\color{gray}\footnotesize \suggestion}
                };
            \fi
        \end{scope}

        % 分割线
        \pgfmathsetmacro{\linepos}{
            \ifnum\pdfstrcmp{\status}{超标}=0
                \basepos-1.7  % 超标时的分割线位置
            \else
                \basepos-1.3  % 正常时的分割线位置
            \fi
        }
        \draw[gray!20] (0.2,\linepos) -- (\cardwidth-0.2,\linepos);

        % 根据当前行的状态计算下一行的位置增量
        \ifnum\pdfstrcmp{\status}{超标}=0
            \pgfmathsetmacro{\increment}{0.85}  % 超标行(两行内容)需要更大的增量
        \else
            \pgfmathsetmacro{\increment}{0.7}  % 正常行(一行内容)使用较小的增量
        \fi

        % 更新位置计数器
        \pgfmathsetmacro{\nextpos}{\currentpos+\increment}
        \xdef\currentpos{\nextpos}
    }

    % 最后一行的处理,消除多余的空白
    \pgfmathsetmacro{\lastincrement}{2}  % 最后一行的增量
    \pgfmathsetmacro{\nextpos}{\currentpos+\lastincrement}
    \xdef\currentpos{\nextpos}

\end{tikzpicture}
\end{center}

\newpage

\begin{tcolorbox}[
    enhanced,
    colback=white,
    colframe=white,
    arc=2mm,
    boxrule=0pt,
    width=\textwidth,
    left=15pt,
    right=15pt,
    top=10pt,
    bottom=10pt,
    drop shadow={
        opacity=0.2,
        color=customTeal
    },
    borderline west={5pt}{0pt}{customTeal}
]
\textcolor{customTeal}{\Large\textbf{(常见)肠道寄生虫}}
\end{tcolorbox}
\vspace{0.05cm}
\begin{tcolorbox}[
    enhanced,
    colback=customTealBg,
    colframe=customTealBg,
    arc=3mm,
    boxrule=0pt,
    width=\textwidth,
    top=8pt,
    bottom=8pt
]
{\small{\color{customTeal}\faInfoCircle} 肠道常见寄生虫检测对于评估肠道健康和潜在感染风险具有重要意义。虽然部分寄生虫在低水平时可能不会造成明显症状,但持续监测对预防寄生虫相关疾病至关重要。
}
\end{tcolorbox}
\vspace{-0.5cm}
\begin{center}
\begin{tikzpicture}[
    font=\small,
    title/.style={font=\small\bfseries\color{white}},
    value/.style={font=\small},
    reference/.style={font=\small},
    cell/.style={anchor=west, text width=4.2cm},
    note/.style={anchor=west, text width=4.5cm, align=left}
]
    \def\cardwidth{\textwidth}
    \def\cardheight{18.75}
    \def\barheight{0.25}
    \def\barwidth{1.5}
    \def\valuebarspace{0.4}

    % 容器和标题栏背景
    \draw[rounded corners=5, fill=white, draw=gray!20]
        (0,0) rectangle (\cardwidth,-\cardheight);
    \path[fill=customTeal]
        (0,0) [rounded corners=5] -- (\cardwidth,0) --
        (\cardwidth,0.8) -- (0,0.8) -- cycle;

    % 表头
    \node[title, anchor=west] at (0.5,0.4) {\textbf{菌种名称}};
    \node[title] at (4.5,0.4) {\textbf{正常范围}};
    \node[title] at (7.5,0.4) {\textbf{检测丰度}};
    \node[title] at (10.5,0.4) {\textbf{结果评价}};
    \node[title] at (13,0.4) {\textbf{超过\%的人}};
    \node[title] at (16,0.4) {\textbf{有\%的正常人检出}};

    % 初始化位置计数器
    \def\currentpos{0.25}

    % 数据行和卡片
    \foreach \item/\enitem/\value/\range/\percentile/\detection/\status/\intro/\suggestion/\index in {
        {人隐孢子虫}/{C. hominis}/ND/{0-0.05}/8\%/91.83\%/未检出/{寄生虫病原体,可引起腹泻。主要通过被污染的水传播。}/{建议注意饮水卫生}/\currentpos,
        {布氏麦片吸虫}/{F. buski}/0.00207/{0-0.05}/27\%/77.88\%/丰度极低/{消化道寄生虫。通过食用受污染的水生植物感染。}/{加强饮食卫生}/\currentpos,
        {脑炎微孢子虫}/{E. hellem}/ND/{0-0.05}/15\%/88.94\%/未检出/{机会性病原体。可引起免疫力低下者感染。}/{注意个人卫生防护}/\currentpos,
        {卫氏并殖吸虫}/{P. westermani}/0.00256/{0-0.05}/38\%/80.29\%/丰度极低/{肺部寄生虫。通过食用未煮熟的淡水蟹类感染。}/{注意食品安全}/\currentpos,
        {毛首鞭形线虫}/{T. trichiura}/0.00173/{0-0.05}/27\%/77.88\%/丰度极低/{肠道蠕虫。通过污染的土壤和食物传播。}/{保持个人卫生}/\currentpos,
        {疟原虫}/{P. ovale}/0.00389/{0-0.05}/8\%/91.83\%/丰度极低/{通过蚊虫叮咬传播。可引起间日疟。}/{做好防蚊措施}/\currentpos,
        {肉胞子虫}/{S. neurona}/ND/{0-0.05}/15\%/88.94\%/未检出/{寄生原虫。主要感染神经系统。}/{注意食品卫生}/\currentpos,
        {十二指肠钩口线虫}/{A. duodenale}/0.00136/{0-0.05}/27\%/77.88\%/丰度极低/{土源性线虫。可引起贫血。}/{注意个人卫生防护}/\currentpos,
        {蠕形住肠线虫}/{E. vermicularis}/0.00456/{0-0.05}/38\%/80.29\%/丰度极低/{常见蠕虫。引起肛门瘙痒。}/{保持个人卫生}/\currentpos
    }
    {
        % 计算当前行的基础位置
        \pgfmathsetmacro{\basepos}{-2.8*\currentpos}

        % 菌种名称
        \node[cell, align=left] at (0.5,\basepos) {
            \small\textbf{\item}\\[-0.2em]
            {\color{lightgray}\small\enitem}
        };

        % 正常范围
        \node[reference] at (4.5,\basepos) {\footnotesize\range};

        % 进度条相关
        \pgfmathsetmacro{\barypos}{\basepos-\valuebarspace+0.1}
        \def\barstart{6.75}

        % 进度条背景
        \fill[gray!10, rounded corners=2] (\barstart,\barypos)
            rectangle (\barstart+\barwidth,\barypos+\barheight);

        % 检测丰度值
        \node[value] at (7.5,{\basepos-\valuebarspace+0.6}) {\footnotesize\value};

        % 解析范围并计算进度条长度
        \def\parserange#1-#2\endparse{\def\minval{#1}\def\maxval{#2}}
        \expandafter\parserange\range\endparse

        % 计算进度条长度和颜色
        \pgfmathsetmacro{\progress}{min(\value/\maxval, 1.0)}
        \pgfmathparse{\value > \maxval ? "customred" : (\value < \minval ? "customred" : "green!50")}
        \let\barcolor=\pgfmathresult

        % 进度条显示
        \ifnum\pdfstrcmp{\status}{超标}=0
            \fill[customred, rounded corners=2] (\barstart,\barypos)
                rectangle (\barstart+\barwidth,\barypos+\barheight);
        \else
            \fill[\barcolor, rounded corners=2] (\barstart,\barypos)
                rectangle (\barstart+\barwidth*\progress,\barypos+\barheight);
        \fi

        % 结果评价
        \ifnum\pdfstrcmp{\status}{超标}=0
            \node[value, text=customRed] at (10.5,\basepos) {\footnotesize\textbf{\status}};
        \else
            \node[value, text=customGreen] at (10.5,\basepos) {\footnotesize\textbf{\status}};
        \fi

        % 丰度超过%人
        \node[value] at (13,\basepos) {\footnotesize\percentile};

        % %正常人检出
        \node[value] at (16,\basepos) {\footnotesize\detection};

        % 添加卡片
        \pgfmathsetmacro{\cardypos}{\basepos-0.5}
        \begin{scope}[shift={(0,\cardypos)}]
            % 卡片背景
            \pgfmathsetmacro{\cardheight}{
                \ifnum\pdfstrcmp{\status}{超标}=0
                    1.0  % 两行内容时的高度
                \else
                    0.6  % 一行内容时的高度
                \fi
            }

            \fill[rounded corners=5pt, customTeal!5, draw=gray!5]
                (0.3,-\cardheight) rectangle (17.3,0);

            % 菌群简介图标和内容
            \node[anchor=west] at (0.5,-0.3) {
                \textbf{\color{gray!90}\footnotesize \textcolor{customTeal}{\faInfoCircle}}
            };
            \node[anchor=west, text width=16cm] at (0.9,-0.3) {
                {\small\color{gray}\footnotesize \intro}
            };

            % 异常解读标题和内容
            \ifnum\pdfstrcmp{\status}{超标}=0
                \node[anchor=west] at (0.55,-0.8) {
                    \textbf{\color{customRed}\footnotesize \textcolor{customRed}{\faBell}}
                };
                \node[anchor=west, text width=16cm] at (0.9,-0.8) {
                    {\small\color{gray}\footnotesize \suggestion}
                };
            \fi
        \end{scope}

        % 分割线
        \pgfmathsetmacro{\linepos}{
            \ifnum\pdfstrcmp{\status}{超标}=0
                \basepos-1.7  % 超标时的分割线位置
            \else
                \basepos-1.3  % 正常时的分割线位置
            \fi
        }
        \draw[gray!20] (0.2,\linepos) -- (\cardwidth-0.2,\linepos);

        % 根据当前行的状态计算下一行的位置增量
        \ifnum\pdfstrcmp{\status}{超标}=0
            \pgfmathsetmacro{\increment}{0.85}  % 超标行(两行内容)需要更大的增量
        \else
            \pgfmathsetmacro{\increment}{0.7}  % 正常行(一行内容)使用较小的增量
        \fi

        % 更新位置计数器
        \pgfmathsetmacro{\nextpos}{\currentpos+\increment}
        \xdef\currentpos{\nextpos}
    }

    % 最后一行的处理,消除多余的空白
    \pgfmathsetmacro{\lastincrement}{2}  % 最后一行的增量
    \pgfmathsetmacro{\nextpos}{\currentpos+\lastincrement}
    \xdef\currentpos{\nextpos}

\end{tikzpicture}
\end{center}

\newpage

\begin{center}
\begin{tikzpicture}[
    font=\small,
    title/.style={font=\small\bfseries\color{white}},
    value/.style={font=\small},
    reference/.style={font=\small},
    cell/.style={anchor=west, text width=4.2cm},
    note/.style={anchor=west, text width=4.5cm, align=left}
]
    \def\cardwidth{\textwidth}
    \def\cardheight{18.75}
    \def\barheight{0.25}
    \def\barwidth{1.5}
    \def\valuebarspace{0.4}

    % 容器和标题栏背景
    \draw[rounded corners=5, fill=white, draw=gray!20]
        (0,0) rectangle (\cardwidth,-\cardheight);
    \path[fill=customTeal]
        (0,0) [rounded corners=5] -- (\cardwidth,0) --
        (\cardwidth,0.8) -- (0,0.8) -- cycle;

    % 表头
    \node[title, anchor=west] at (0.5,0.4) {\textbf{菌种名称}};
    \node[title] at (4.5,0.4) {\textbf{正常范围}};
    \node[title] at (7.5,0.4) {\textbf{检测丰度}};
    \node[title] at (10.5,0.4) {\textbf{结果评价}};
    \node[title] at (13,0.4) {\textbf{超过\%的人}};
    \node[title] at (16,0.4) {\textbf{有\%的正常人检出}};

    % 初始化位置计数器
    \def\currentpos{0.25}

    % 数据行和卡片
    \foreach \item/\enitem/\value/\range/\percentile/\detection/\status/\intro/\suggestion/\index in {
        {美洲钩口线虫}/{Necator americanus}/0.00252/{0-0.05}/27\%/77.88\%/丰度极低/{土源性线虫。可致贫血和营养不良。}/{做好个人防护}/\currentpos,
        {巴西利士曼原虫}/{L. braziliensis}/ND/{0-0.05}/8\%/91.83\%/未检出/{通过白蛉叮咬传播。可引起皮肤病变。}/{避免被虫叮咬}/\currentpos,
        {杜氏利什曼原虫}/{L. donovani}/ND/{0-0.05}/15\%/88.94\%/未检出/{内脏利什曼病病原体。影响免疫系统。}/{注意防虫措施}/\currentpos,
        {硕大利什曼虫}/{L. major}/ND/{0-0.05}/27\%/77.88\%/未检出/{皮肤利什曼病病原体。通过昆虫叮咬传播。}/{做好防护措施}/\currentpos,
        {婴儿利什曼虫}/{L. infantum}/ND/{0-0.05}/38\%/80.29\%/未检出/{可引起内脏利什曼病。儿童易感。}/{预防虫媒传播}/\currentpos,
        {布氏锥虫}/{T. brucei}/0.00028/{0-0.05}/8\%/91.83\%/丰度极低/{非洲锥虫病病原体。通过采采蝇传播。}/{注意防虫措施}/\currentpos,
        {柱氏锥虫}/{T. cruzi}/0.00859/{0-0.05}/38\%/80.29\%/丰度极低/{美洲锥虫病病原体。通过锥蝽叮咬传播。可引起心脏和消化道病变。}/{避免虫媒接触}/\currentpos,
        {阴道毛滴虫}/{T. vaginalis}/0.00693/{0-0.05}/27\%/77.88\%/丰度极低/{泌尿生殖道寄生虫。可引起感染性疾病。需及时诊治。}/{注意个人卫生}/\currentpos,
        {蓝氏贾第鞭毛虫}/{Giardia intestinalis}/ND/{0-0.05}/8\%/91.83\%/未检出/{肠道原虫。通过污染的水和食物传播。可致腹泻。}/{注意饮食卫生}/\currentpos,
        {溶组织内阿米巴}/{E. histolytica}/0.00062/{0-0.05}/15\%/88.94\%/丰度极低/{致病性原虫。可引起肠道和肝脏感染。需警惕。}/{加强饮食卫生}/\currentpos,
        {小隐孢子虫}/{C. parvum}/ND/{0-0.05}/27\%/77.88\%/未检出/{人畜共患寄生虫。通过污染的水传播。可致腹泻。}/{注意饮水卫生}/\currentpos
    }
    {
        % 计算当前行的基础位置
        \pgfmathsetmacro{\basepos}{-2.8*\currentpos}

        % 菌种名称
        \node[cell, align=left] at (0.5,\basepos) {
            \small\textbf{\item}\\[-0.2em]
            {\color{lightgray}\small\enitem}
        };

        % 正常范围
        \node[reference] at (4.5,\basepos) {\footnotesize\range};

        % 进度条相关
        \pgfmathsetmacro{\barypos}{\basepos-\valuebarspace+0.1}
        \def\barstart{6.75}

        % 进度条背景
        \fill[gray!10, rounded corners=2] (\barstart,\barypos)
            rectangle (\barstart+\barwidth,\barypos+\barheight);

        % 检测丰度值
        \node[value] at (7.5,{\basepos-\valuebarspace+0.6}) {\footnotesize\value};

        % 解析范围并计算进度条长度
        \def\parserange#1-#2\endparse{\def\minval{#1}\def\maxval{#2}}
        \expandafter\parserange\range\endparse

        % 计算进度条长度和颜色
        \pgfmathsetmacro{\progress}{min(\value/\maxval, 1.0)}
        \pgfmathparse{\value > \maxval ? "customred" : (\value < \minval ? "customred" : "green!50")}
        \let\barcolor=\pgfmathresult

        % 进度条显示
        \ifnum\pdfstrcmp{\status}{超标}=0
            \fill[customred, rounded corners=2] (\barstart,\barypos)
                rectangle (\barstart+\barwidth,\barypos+\barheight);
        \else
            \fill[\barcolor, rounded corners=2] (\barstart,\barypos)
                rectangle (\barstart+\barwidth*\progress,\barypos+\barheight);
        \fi

        % 结果评价
        \ifnum\pdfstrcmp{\status}{超标}=0
            \node[value, text=customRed] at (10.5,\basepos) {\footnotesize\textbf{\status}};
        \else
            \node[value, text=customGreen] at (10.5,\basepos) {\footnotesize\textbf{\status}};
        \fi

        % 丰度超过%人
        \node[value] at (13,\basepos) {\footnotesize\percentile};

        % %正常人检出
        \node[value] at (16,\basepos) {\footnotesize\detection};

        % 添加卡片
        \pgfmathsetmacro{\cardypos}{\basepos-0.5}
        \begin{scope}[shift={(0,\cardypos)}]
            % 卡片背景
            \pgfmathsetmacro{\cardheight}{
                \ifnum\pdfstrcmp{\status}{超标}=0
                    1.0  % 两行内容时的高度
                \else
                    0.6  % 一行内容时的高度
                \fi
            }

            \fill[rounded corners=5pt, customTeal!5, draw=gray!5]
                (0.3,-\cardheight) rectangle (17.3,0);

            % 菌群简介图标和内容
            \node[anchor=west] at (0.5,-0.3) {
                \textbf{\color{gray!90}\footnotesize \textcolor{customTeal}{\faInfoCircle}}
            };
            \node[anchor=west, text width=16cm] at (0.9,-0.3) {
                {\small\color{gray}\footnotesize \intro}
            };

            % 异常解读标题和内容
            \ifnum\pdfstrcmp{\status}{超标}=0
                \node[anchor=west] at (0.55,-0.8) {
                    \textbf{\color{customRed}\footnotesize \textcolor{customRed}{\faBell}}
                };
                \node[anchor=west, text width=16cm] at (0.9,-0.8) {
                    {\small\color{gray}\footnotesize \suggestion}
                };
            \fi
        \end{scope}

        % 分割线
        \pgfmathsetmacro{\linepos}{
            \ifnum\pdfstrcmp{\status}{超标}=0
                \basepos-1.7  % 超标时的分割线位置
            \else
                \basepos-1.3  % 正常时的分割线位置
            \fi
        }
        \draw[gray!20] (0.2,\linepos) -- (\cardwidth-0.2,\linepos);

        % 根据当前行的状态计算下一行的位置增量
        \ifnum\pdfstrcmp{\status}{超标}=0
            \pgfmathsetmacro{\increment}{0.85}  % 超标行(两行内容)需要更大的增量
        \else
            \pgfmathsetmacro{\increment}{0.7}  % 正常行(一行内容)使用较小的增量
        \fi

        % 更新位置计数器
        \pgfmathsetmacro{\nextpos}{\currentpos+\increment}
        \xdef\currentpos{\nextpos}
    }

    % 最后一行的处理,消除多余的空白
    \pgfmathsetmacro{\lastincrement}{2}  % 最后一行的增量
    \pgfmathsetmacro{\nextpos}{\currentpos+\lastincrement}
    \xdef\currentpos{\nextpos}

\end{tikzpicture}
\end{center}

\newpage

\begin{center}
\begin{tikzpicture}[
    font=\small,
    title/.style={font=\small\bfseries\color{white}},
    value/.style={font=\small},
    reference/.style={font=\small},
    cell/.style={anchor=west, text width=4.2cm},
    note/.style={anchor=west, text width=4.5cm, align=left}
]
    \def\cardwidth{\textwidth}
    \def\cardheight{18.75}
    \def\barheight{0.25}
    \def\barwidth{1.5}
    \def\valuebarspace{0.4}

    % 容器和标题栏背景
    \draw[rounded corners=5, fill=white, draw=gray!20]
        (0,0) rectangle (\cardwidth,-\cardheight);
    \path[fill=customTeal]
        (0,0) [rounded corners=5] -- (\cardwidth,0) --
        (\cardwidth,0.8) -- (0,0.8) -- cycle;

    % 表头
    \node[title, anchor=west] at (0.5,0.4) {\textbf{菌种名称}};
    \node[title] at (4.5,0.4) {\textbf{正常范围}};
    \node[title] at (7.5,0.4) {\textbf{检测丰度}};
    \node[title] at (10.5,0.4) {\textbf{结果评价}};
    \node[title] at (13,0.4) {\textbf{超过\%的人}};
    \node[title] at (16,0.4) {\textbf{有\%的正常人检出}};

    % 初始化位置计数器
    \def\currentpos{0.25}

    % 数据行和卡片
    \foreach \item/\enitem/\value/\range/\percentile/\detection/\status/\intro/\suggestion/\index in {
        {鼠隐孢子虫}/{C. muris}/0.00012/{0-0.05}/38\%/80.29\%/丰度极低/{主要感染啮齿类动物。偶见人类感染。可致消化道症状。}/{注意环境卫生}/\currentpos,
        {刚地弓形虫}/{T. gondii}/0.00767/{0-0.05}/52\%/81.25\%/丰度极低/{人畜共患虫病。通过生食传播。孕妇需特别注意。}/{避免食用生食}/\currentpos,
        {恶性疟原虫}/{P. falciparum}/0.00137/{0-0.05}/27\%/77.88\%/丰度极低/{最严重的疟疾病原体。通过按蚊传播。可致重症疟疾。}/{做好防蚊措施}/\currentpos,
        {间日疟原虫}/{P. vivax}/0.00318/{0-0.05}/38\%/80.29\%/丰度极低/{最常见的疟原虫。可引起间日疟。有潜伏期。}/{注意防蚊防护}/\currentpos,
        {牛环形泰勒虫}/{T. annulata}/ND/{0-0.05}/8\%/91.83\%/未检出/{主要感染牛。通过蜱虫传播。人类较少感染。}/{注意防虫}/\currentpos,
        {小泰勒虫}/{T. parva}/ND/{0-0.05}/15\%/88.94\%/未检出/{牛重要病原体。通过蜱虫传播。人类不易感染。}/{做好防护}/\currentpos,
        {肠脑炎微孢子虫}/{E. intestinalis}/ND/{0-0.05}/27\%/77.88\%/未检出/{机会性病原体。可感染免疫力低下者。影响消化道。}/{加强卫生防护}/\currentpos,
        {兔脑炎微孢子虫}/{E. cuniculi}/ND/{0-0.05}/38\%/80.29\%/未检出/{主要感染兔类。人类偶见感染。可致神经系统疾病。}/{注意接触防护}/\currentpos,
        {日本血吸虫}/{S. japonicum}/0.00343/{0-0.05}/52\%/81.25\%/丰度极低/{重要人体寄生虫。通过受染水体感染。可致肝纤维化。}/{避免接触疫水}/\currentpos,
        {曼氏血吸虫}/{S. mansoni}/0.00349/{0-0.05}/27\%/77.88\%/丰度极低/{人体寄生虫。通过淡水感染。影响肝脏和肠道。}/{预防水体接触}/\currentpos,
        {埃及血吸虫}/{S. haematobium}/0.00308/{0-0.05}/27\%/77.88\%/丰度极低/{泌尿生殖系统寄生虫。通过接触被污染水体感染。可引起尿路系统损害。}/{避免接触疫水,加强个人防护}/\currentpos
    }
    {
        % 计算当前行的基础位置
        \pgfmathsetmacro{\basepos}{-2.8*\currentpos}

        % 菌种名称
        \node[cell, align=left] at (0.5,\basepos) {
            \small\textbf{\item}\\[-0.2em]
            {\color{lightgray}\small\enitem}
        };

        % 正常范围
        \node[reference] at (4.5,\basepos) {\footnotesize\range};

        % 进度条相关
        \pgfmathsetmacro{\barypos}{\basepos-\valuebarspace+0.1}
        \def\barstart{6.75}

        % 进度条背景
        \fill[gray!10, rounded corners=2] (\barstart,\barypos)
            rectangle (\barstart+\barwidth,\barypos+\barheight);

        % 检测丰度值
        \node[value] at (7.5,{\basepos-\valuebarspace+0.6}) {\footnotesize\value};

        % 解析范围并计算进度条长度
        \def\parserange#1-#2\endparse{\def\minval{#1}\def\maxval{#2}}
        \expandafter\parserange\range\endparse

        % 计算进度条长度和颜色
        \pgfmathsetmacro{\progress}{min(\value/\maxval, 1.0)}
        \pgfmathparse{\value > \maxval ? "customred" : (\value < \minval ? "customred" : "green!50")}
        \let\barcolor=\pgfmathresult

        % 进度条显示
        \ifnum\pdfstrcmp{\status}{超标}=0
            \fill[customred, rounded corners=2] (\barstart,\barypos)
                rectangle (\barstart+\barwidth,\barypos+\barheight);
        \else
            \fill[\barcolor, rounded corners=2] (\barstart,\barypos)
                rectangle (\barstart+\barwidth*\progress,\barypos+\barheight);
        \fi

        % 结果评价
        \ifnum\pdfstrcmp{\status}{超标}=0
            \node[value, text=customRed] at (10.5,\basepos) {\footnotesize\textbf{\status}};
        \else
            \node[value, text=customGreen] at (10.5,\basepos) {\footnotesize\textbf{\status}};
        \fi

        % 丰度超过%人
        \node[value] at (13,\basepos) {\footnotesize\percentile};

        % %正常人检出
        \node[value] at (16,\basepos) {\footnotesize\detection};

        % 添加卡片
        \pgfmathsetmacro{\cardypos}{\basepos-0.5}
        \begin{scope}[shift={(0,\cardypos)}]
            % 卡片背景
            \pgfmathsetmacro{\cardheight}{
                \ifnum\pdfstrcmp{\status}{超标}=0
                    1.0  % 两行内容时的高度
                \else
                    0.6  % 一行内容时的高度
                \fi
            }

            \fill[rounded corners=5pt, customTeal!5, draw=gray!5]
                (0.3,-\cardheight) rectangle (17.3,0);

            % 菌群简介图标和内容
            \node[anchor=west] at (0.5,-0.3) {
                \textbf{\color{gray!90}\footnotesize \textcolor{customTeal}{\faInfoCircle}}
            };
            \node[anchor=west, text width=16cm] at (0.9,-0.3) {
                {\small\color{gray}\footnotesize \intro}
            };

            % 异常解读标题和内容
            \ifnum\pdfstrcmp{\status}{超标}=0
                \node[anchor=west] at (0.55,-0.8) {
                    \textbf{\color{customRed}\footnotesize \textcolor{customRed}{\faBell}}
                };
                \node[anchor=west, text width=16cm] at (0.9,-0.8) {
                    {\small\color{gray}\footnotesize \suggestion}
                };
            \fi
        \end{scope}

        % 分割线
        \pgfmathsetmacro{\linepos}{
            \ifnum\pdfstrcmp{\status}{超标}=0
                \basepos-1.7  % 超标时的分割线位置
            \else
                \basepos-1.3  % 正常时的分割线位置
            \fi
        }
        \draw[gray!20] (0.2,\linepos) -- (\cardwidth-0.2,\linepos);

        % 根据当前行的状态计算下一行的位置增量
        \ifnum\pdfstrcmp{\status}{超标}=0
            \pgfmathsetmacro{\increment}{0.85}  % 超标行(两行内容)需要更大的增量
        \else
            \pgfmathsetmacro{\increment}{0.7}  % 正常行(一行内容)使用较小的增量
        \fi

        % 更新位置计数器
        \pgfmathsetmacro{\nextpos}{\currentpos+\increment}
        \xdef\currentpos{\nextpos}
    }

    % 最后一行的处理,消除多余的空白
    \pgfmathsetmacro{\lastincrement}{2}  % 最后一行的增量
    \pgfmathsetmacro{\nextpos}{\currentpos+\lastincrement}
    \xdef\currentpos{\nextpos}

\end{tikzpicture}
\end{center}

\newpage

\begin{center}
\begin{tikzpicture}[
    font=\small,
    title/.style={font=\small\bfseries\color{white}},
    value/.style={font=\small},
    reference/.style={font=\small},
    cell/.style={anchor=west, text width=4.2cm},
    note/.style={anchor=west, text width=4.5cm, align=left}
]
    \def\cardwidth{\textwidth}
    \def\cardheight{18.75}
    \def\barheight{0.25}
    \def\barwidth{1.5}
    \def\valuebarspace{0.4}

    % 容器和标题栏背景
    \draw[rounded corners=5, fill=white, draw=gray!20]
        (0,0) rectangle (\cardwidth,-\cardheight);
    \path[fill=customTeal]
        (0,0) [rounded corners=5] -- (\cardwidth,0) --
        (\cardwidth,0.8) -- (0,0.8) -- cycle;

    % 表头
    \node[title, anchor=west] at (0.5,0.4) {\textbf{菌种名称}};
    \node[title] at (4.5,0.4) {\textbf{正常范围}};
    \node[title] at (7.5,0.4) {\textbf{检测丰度}};
    \node[title] at (10.5,0.4) {\textbf{结果评价}};
    \node[title] at (13,0.4) {\textbf{超过\%的人}};
    \node[title] at (16,0.4) {\textbf{有\%的正常人检出}};

    % 初始化位置计数器
    \def\currentpos{0.25}

    % 数据行和卡片
    \foreach \item/\enitem/\value/\range/\percentile/\detection/\status/\intro/\suggestion/\index in {
        {猪带绦虫}/{T. solium}/0.00044/{0-0.05}/15\%/88.94\%/丰度极低/{重要人兽共患寄生虫。通过食用感染猪肉传播。可引起囊虫病。}/{避免食用生猪肉,注意肉品烹饪}/\currentpos,
        {牛带绦虫}/{T. saginata}/0.00046/{0-0.05}/38\%/80.29\%/丰度极低/{人畜共患寄生虫。经感染牛肉传播。主要寄生于小肠。}/{确保牛肉充分煮熟,注意饮食卫生}/\currentpos,
        {蛔蚴蠕虫}/{A. lumbricoides}/0.00358/{0-0.05}/52\%/81.25\%/丰度极低/{最常见的人体蠕虫。通过被污染的土壤和食物传播。可致营养不良。}/{注意饮食卫生,勤洗手}/\currentpos,
        {旋毛形线虫}/{T. spiralis}/0.00099/{0-0.05}/27\%/77.88\%/丰度极低/{人兽共患寄生虫。通过食用感染肉类传播。可致肌肉疼痛。}/{避免食用生肉,确保肉类煮熟}/\currentpos,
        {华支睾吸虫}/{C. sinensis}/0.00137/{0-0.05}/38\%/80.29\%/丰度极低/{肝胆管寄生虫。通过食用生鱼传播。可引起肝胆疾病。}/{避免食用生鱼片,注意饮食卫生}/\currentpos,
        {环孢子虫}/{C. cayetanensis}/0.00635/{0-0.05}/27\%/77.88\%/丰度极低/{食源性原虫。通过污染的水果蔬菜传播。可致腹泻。}/{彻底清洗蔬果,注意饮食卫生}/\currentpos
    }
    {
        % 计算当前行的基础位置
        \pgfmathsetmacro{\basepos}{-2.8*\currentpos}

        % 菌种名称
        \node[cell, align=left] at (0.5,\basepos) {
            \small\textbf{\item}\\[-0.2em]
            {\color{lightgray}\small\enitem}
        };

        % 正常范围
        \node[reference] at (4.5,\basepos) {\footnotesize\range};

        % 进度条相关
        \pgfmathsetmacro{\barypos}{\basepos-\valuebarspace+0.1}
        \def\barstart{6.75}

        % 进度条背景
        \fill[gray!10, rounded corners=2] (\barstart,\barypos)
            rectangle (\barstart+\barwidth,\barypos+\barheight);

        % 检测丰度值
        \node[value] at (7.5,{\basepos-\valuebarspace+0.6}) {\footnotesize\value};

        % 解析范围并计算进度条长度
        \def\parserange#1-#2\endparse{\def\minval{#1}\def\maxval{#2}}
        \expandafter\parserange\range\endparse

        % 计算进度条长度和颜色
        \pgfmathsetmacro{\progress}{min(\value/\maxval, 1.0)}
        \pgfmathparse{\value > \maxval ? "customred" : (\value < \minval ? "customred" : "green!50")}
        \let\barcolor=\pgfmathresult

        % 进度条显示
        \ifnum\pdfstrcmp{\status}{超标}=0
            \fill[customred, rounded corners=2] (\barstart,\barypos)
                rectangle (\barstart+\barwidth,\barypos+\barheight);
        \else
            \fill[\barcolor, rounded corners=2] (\barstart,\barypos)
                rectangle (\barstart+\barwidth*\progress,\barypos+\barheight);
        \fi

        % 结果评价
        \ifnum\pdfstrcmp{\status}{超标}=0
            \node[value, text=customRed] at (10.5,\basepos) {\footnotesize\textbf{\status}};
        \else
            \node[value, text=customGreen] at (10.5,\basepos) {\footnotesize\textbf{\status}};
        \fi

        % 丰度超过%人
        \node[value] at (13,\basepos) {\footnotesize\percentile};

        % %正常人检出
        \node[value] at (16,\basepos) {\footnotesize\detection};

        % 添加卡片
        \pgfmathsetmacro{\cardypos}{\basepos-0.5}
        \begin{scope}[shift={(0,\cardypos)}]
            % 卡片背景
            \pgfmathsetmacro{\cardheight}{
                \ifnum\pdfstrcmp{\status}{超标}=0
                    1.0  % 两行内容时的高度
                \else
                    0.6  % 一行内容时的高度
                \fi
            }

            \fill[rounded corners=5pt, customTeal!5, draw=gray!5]
                (0.3,-\cardheight) rectangle (17.3,0);

            % 菌群简介图标和内容
            \node[anchor=west] at (0.5,-0.3) {
                \textbf{\color{gray!90}\footnotesize \textcolor{customTeal}{\faInfoCircle}}
            };
            \node[anchor=west, text width=16cm] at (0.9,-0.3) {
                {\small\color{gray}\footnotesize \intro}
            };

            % 异常解读标题和内容
            \ifnum\pdfstrcmp{\status}{超标}=0
                \node[anchor=west] at (0.55,-0.8) {
                    \textbf{\color{customRed}\footnotesize \textcolor{customRed}{\faBell}}
                };
                \node[anchor=west, text width=16cm] at (0.9,-0.8) {
                    {\small\color{gray}\footnotesize \suggestion}
                };
            \fi
        \end{scope}

        % 分割线
        \pgfmathsetmacro{\linepos}{
            \ifnum\pdfstrcmp{\status}{超标}=0
                \basepos-1.7  % 超标时的分割线位置
            \else
                \basepos-1.3  % 正常时的分割线位置
            \fi
        }
        \draw[gray!20] (0.2,\linepos) -- (\cardwidth-0.2,\linepos);

        % 根据当前行的状态计算下一行的位置增量
        \ifnum\pdfstrcmp{\status}{超标}=0
            \pgfmathsetmacro{\increment}{0.85}  % 超标行(两行内容)需要更大的增量
        \else
            \pgfmathsetmacro{\increment}{0.7}  % 正常行(一行内容)使用较小的增量
        \fi

        % 更新位置计数器
        \pgfmathsetmacro{\nextpos}{\currentpos+\increment}
        \xdef\currentpos{\nextpos}
    }

    % 最后一行的处理,消除多余的空白
    \pgfmathsetmacro{\lastincrement}{2}  % 最后一行的增量
    \pgfmathsetmacro{\nextpos}{\currentpos+\lastincrement}
    \xdef\currentpos{\nextpos}

\end{tikzpicture}
\end{center}

\newpage

\begin{tcolorbox}[
    enhanced,
    colback=white,
    colframe=white,
    arc=2mm,
    boxrule=0pt,
    width=\textwidth,
    left=15pt,
    right=15pt,
    top=10pt,
    bottom=10pt,
    drop shadow={
        opacity=0.2,
        color=customTeal
    },
    borderline west={5pt}{0pt}{customTeal}
]
%\textcolor{customTeal}{\Large\textbf{检出病毒}}
    \textcolor{customTeal}{\Large\CJKfontspec[UprightFont={* Semibold}]{PingFang SC}检出病毒}
\end{tcolorbox}

\begin{tcolorbox}[
    enhanced,
    colback=customTealBg,
    colframe=customTealBg,
    arc=3mm,
    boxrule=0pt,
    width=\textwidth,
    top=8pt,
    bottom=8pt
]
{\small{\color{customTeal}\faInfoCircle} 肠道微生物除了细菌外,还包含大量病毒。研究表明,健康人的肠道中约有数万种病毒,其中超过90\%是噬菌体。下表展示了在您肠道中检测到的病毒,按检出的病毒丰度从高到低排列了前15位(如无检出将没有显示)。
}
\end{tcolorbox}

\begin{tcolorbox}[
    enhanced,
    colback=lightpurple!10, % 卡片底色
    colframe=white,  % 边框颜色
    arc=3mm,
    boxrule=0.5pt,
    width=\textwidth,
    top=8pt,
    bottom=8pt
]
{\small{\color{lightpurple}\faQuestionCircle}\quad \textbf{有哪些常见的肠道菌群病毒分类?}\\
{\color{orange!50}\faComments}\quad 肠道菌群中的病毒可以分成噬菌体,植物性病毒,内源性病毒,人类肠道病毒和其它病毒等。
\begin{itemize}
    \item \textbf{噬菌体}
    \begin{itemize}
        \item 病毒简介:噬菌体是专门感染细菌的病毒,它们在肠道微生物群中扮演重要角色,通过控制细菌的种群动态,促进微生物的多样性。噬菌体可用于天然的抗菌剂,有助于对抗抗生素耐药菌。
        \item 感染途径:主要通过摄入被噬菌体污染的食物或水,或通过接触被感染细菌的表面传播。
        \item 潜在影响:噬菌体通过感染和裂解细菌来调节肠道内的细菌种群,增加微生物多样性,并可能用于对抗抗生素耐药菌。它们在维持肠道微生态平衡方面发挥着重要作用,并可能增强宿主的免疫反应。
    \end{itemize}
    \item \textbf{植物性病毒}
    \begin{itemize}
        \item 病毒简介:植物性病毒主要感染植物,但它们也可能通过植物和动物的食物链间接影响肠道微生物群。它们的存在可能会影响植物的营养成分,从而间接影响食物链中的微生物群落。
        \item 感染途径:主要通过食用受感染的植物或其衍生产品,尤其是生吃的蔬菜和水果。
        \item 潜在影响:虽然主要感染植物,植物性病毒通过改变植物的营养成分影响食物链,这可能间接影响人类肠道微生物的组成及功能。长时间接触受感染植物的食物可能影响宿主的健康和营养状况。
    \end{itemize}
    \item \textbf{内源性病毒}
    \begin{itemize}
        \item 病毒简介:内源性病毒是指病毒基因组的残余,这些基因组在进化过程中整合到宿主基因组中。它们可以在宿主基因组中长期存在,可能会影响基因表达和宿主的免疫反应。
        \item 感染途径:通常是体内已有的病毒,可能通过免疫系统的失调或肠道微生物失衡而激活。
        \item 潜在影响:内源性病毒的遗传物质在宿主基因组中长期存在,可能调节基因表达和免疫反应,影响宿主对疾病的易感性。它们的活动可能与某些疾病(如自身免疫疾病和癌症)相关。
    \end{itemize}

    \item \textbf{人类肠道病毒}
    \begin{itemize}
        \item 病毒简介:人类肠道病毒是通过口-粪传播的病毒,如某些腺病毒、柯萨奇病毒和诺如病毒等。这些病毒通常感染人类肠道,可能引起消化道疾病或其他系统感染,并影响微生物群落的组成。
        \item 感染途径:人类肠道病毒的感染途径主要是通过口-粪传播,但也可以通过直接接触、空气传播、食物和水源污染等方式传播。
        \item 潜在影响:人类肠道病毒(如腺病毒、柯萨奇病毒等)能引起消化道疾病,进而影响肠道功能和微生物群的平衡。这些病毒感染可能导致肠道炎症,增加对其他病原体的易感性,并对整体健康产生负面影响。
    \end{itemize}

    \item \textbf{其它病毒}
    \begin{itemize}
        \item 病毒简介:其它病毒包括多种不同的病毒类型,可能影响肠道微生物群,例如某些哺乳动物病毒以及其他非典型病毒。它们的作用和影响尚在研究中,可能在不同机制下参与微生物生态平衡和宿主健康。
        \item 感染途径:其他病毒的感染途径多样,常见方式包括但不限于直接接触,空气传播,食物和水等。
        \item 潜在影响:其他尚未具体分类的病毒可能通过改变肠道微生物群落及其多样性而影响宿主健康。
    \end{itemize}

\end{itemize}

}
\end{tcolorbox}

\newpage

\begin{center}
\begin{tikzpicture}[
    font=\small,
    title/.style={font=\small\bfseries\color{white}},
    value/.style={font=\small},
    reference/.style={font=\small},
    cell/.style={anchor=west, text width=5.0cm},
    note/.style={anchor=west, text width=4.5cm, align=left}
]
    \def\cardwidth{\textwidth}
    \def\cardheight{9.85}
    \def\barheight{0.25}
    \def\barwidth{1.5}
    \def\valuebarspace{0.4}

    % 容器和标题栏背景
    \draw[rounded corners=5, fill=white, draw=gray!20]
        (0,0) rectangle (\cardwidth,-\cardheight);
    \path[fill=customTeal]
        (0,0) [rounded corners=5] -- (\cardwidth,0) --
        (\cardwidth,0.8) -- (0,0.8) -- cycle;

    % 表头
    \node[title, anchor=west] at (0.5,0.4) {\textbf{菌种名称}};
    \node[title] at (7, 0.4) {\textbf{正常范围}};
    \node[title] at (10.5, 0.4) {\textbf{检测丰度}};
    \node[title] at (13.5, 0.4) {\textbf{结果评价}};
    \node[title] at (16, 0.4) {\textbf{分类}};

    % 初始化位置计数器
    \def\currentpos{0.25}

    % 数据行和卡片
    \foreach \item/\enitem/\value/\range/\percentile/\detection/\status/\intro/\suggestion/\index/\category in {
        {分枝杆菌噬菌体 Echild}/{Mycobacterium phage Echild}/0.00331/{0-0.05}/67\%/99.52\%/正常/{分枝杆菌噬菌体 Echild 可以被视为一种有益的肠道病毒,因为它专门感染和攻击有害的分枝杆菌,从而帮助对抗由这些细菌引起的感染。}/{}/\currentpos/{噬菌体},
        {Cbastvirus ST}/{Cbastvirus ST}/0.00274/{0-0.05}/2\%/99.04\%/正常/{Cbastvirus ST 是一种新发现的噬菌体,属于巨噬菌体(Myoviridae)家族,专门感染某些细菌,尤其是与细菌性疾病相关的病原体。}/{}/\currentpos/{噬菌体},
        {Shalavirus Shbh1}/{Shalavirus Shbh1}/0.00265/{0-0.05}/95\%/98.56\%/正常/{Shalavirus Shbh1 是一种噬菌体,属于双链DNA噬菌体的类别,主要感染特定细菌。}/{}/\currentpos/{噬菌体},
        {草履虫绿藻病毒 1 号}/{PbCV-1}/0.00318/{0-0.05}/50\%/99.52\%/正常/{草履虫绿藻病毒1号是一种专门感染特定种类的绿藻和草履虫的双链DNA病毒。}/{}/\currentpos/{植物性病毒},
        {人内源逆转录病毒 K}/{Human endogenous retrovirus K}/0.00308/{0-0.05}/44\%/99.52\%/正常/{人类内源性逆转录病毒 K是一个属于人类基因组的内源性逆转录病毒家族成员,它的表达与多种疾病相关,也可能影响免疫调节。}/{}/\currentpos/{内源性病毒}
    }
    {
        % 计算当前行的基础位置
        \pgfmathsetmacro{\basepos}{-2.8*\currentpos}

        % 菌种名称
        \node[cell, align=left] at (0.5,\basepos) {
            \small\textbf{\item}\\[-0.2em]
            {\color{lightgray}\small\enitem}
        };

        % 正常范围
        \node[reference] at (7,\basepos) {\footnotesize\range};

        % 进度条相关
        \pgfmathsetmacro{\barypos}{\basepos-\valuebarspace+0.1}
        \def\barstart{9.75}

        % 进度条背景
        \fill[gray!10, rounded corners=2] (\barstart,\barypos)
            rectangle (\barstart+\barwidth,\barypos+\barheight);

        % 检测丰度值
        \node[value] at (10.5, {\basepos-\valuebarspace+0.6}) {\footnotesize\value};

        % 解析范围并计算进度条长度
        \def\parserange#1-#2\endparse{\def\minval{#1}\def\maxval{#2}}
        \expandafter\parserange\range\endparse

        % 计算进度条长度和颜色
        \pgfmathsetmacro{\progress}{min(\value/\maxval, 1.0)}
        \pgfmathparse{\value > \maxval ? "customred" : (\value < \minval ? "customred" : "green!50")}
        \let\barcolor=\pgfmathresult

        % 进度条显示
        \ifnum\pdfstrcmp{\status}{超标}=0
            \fill[customred, rounded corners=2] (\barstart,\barypos)
                rectangle (\barstart+\barwidth,\barypos+\barheight);
        \else
            \fill[\barcolor, rounded corners=2] (\barstart,\barypos)
                rectangle (\barstart+\barwidth*\progress,\barypos+\barheight);
        \fi

        % 结果评价
        \ifnum\pdfstrcmp{\status}{超标}=0
            \node[value, text=customRed] at (13.5,\basepos) {\footnotesize\textbf{\status}};
        \else
            \node[value, text=customGreen] at (13.5,\basepos) {\footnotesize\textbf{\status}};
        \fi

        % 分类
        \node[value] at (16,\basepos) {\footnotesize\category};

        % 添加卡片
        \pgfmathsetmacro{\cardypos}{\basepos-0.5}
        \begin{scope}[shift={(0,\cardypos)}]
            % 卡片背景
            \pgfmathsetmacro{\cardheight}{
                \ifnum\pdfstrcmp{\status}{超标}=0
                    1.0  % 两行内容时的高度
                \else
                    0.6  % 一行内容时的高度
                \fi
            }

            \fill[rounded corners=5pt, customTeal!5, draw=gray!5]
                (0.3,-\cardheight) rectangle (17.3,0);

            % 菌群简介图标和内容
            \node[anchor=west] at (0.5,-0.3) {
                \textbf{\color{gray!90}\footnotesize \textcolor{customTeal}{\faInfoCircle}}
            };
            \node[anchor=west, text width=16cm] at (0.9,-0.3) {
                {\small\color{gray}\footnotesize \intro}
            };

            % 异常解读标题和内容
            \ifnum\pdfstrcmp{\status}{超标}=0
                \node[anchor=west] at (0.55,-0.8) {
                    \textbf{\color{customRed}\footnotesize \textcolor{customRed}{\faBell}}
                };
                \node[anchor=west, text width=16cm] at (0.9,-0.8) {
                    {\small\color{gray}\footnotesize \suggestion}
                };
            \fi
        \end{scope}

        % 分割线
        \pgfmathsetmacro{\linepos}{
            \ifnum\pdfstrcmp{\status}{超标}=0
                \basepos-1.7  % 超标时的分割线位置
            \else
                \basepos-1.3  % 正常时的分割线位置
            \fi
        }
        \draw[gray!20] (0.2,\linepos) -- (\cardwidth-0.2,\linepos);

        % 根据当前行的状态计算下一行的位置增量
        \ifnum\pdfstrcmp{\status}{超标}=0
            \pgfmathsetmacro{\increment}{0.85}  % 超标行(两行内容)需要更大的增量
        \else
            \pgfmathsetmacro{\increment}{0.7}  % 正常行(一行内容)使用较小的增量
        \fi

        % 更新位置计数器
        \pgfmathsetmacro{\nextpos}{\currentpos+\increment}
        \xdef\currentpos{\nextpos}
    }

    % 最后一行的处理,消除多余的空白
    \pgfmathsetmacro{\lastincrement}{2}  % 最后一行的增量
    \pgfmathsetmacro{\nextpos}{\currentpos+\lastincrement}
    \xdef\currentpos{\nextpos}

\end{tikzpicture}
\end{center}

\begin{tcolorbox}[
    enhanced,
    colback=lightgreen, % 卡片底色
    colframe=white,  % 边框颜色
    arc=3mm,
    boxrule=0.5pt,
    width=\textwidth,
    top=8pt,
    bottom=8pt
]
{\small{\color{customGreen}\faBell}\quad
检出的病毒结果均在正常参考范围内,表明肠道内的病毒群落相对健康。
}
\end{tcolorbox}

\newpage

\begin{tcolorbox}[
    enhanced,
    colback=white,
    colframe=customTeal,
    arc=2mm,
    boxrule=1pt,
    left=20pt,
    right=20pt,
    top=12pt,
    bottom=12pt,
    width=\textwidth,
    fontupper=\sffamily,
    overlay={
    \draw[customTeal, line width=2pt]
    ([xshift=15pt]frame.south west) -- ([xshift=-15pt]frame.south east);
    }
]
{\Large\bfseries\textcolor{customTeal}{\Huge 肠道菌群抗性与基因分布}}
\end{tcolorbox}

\begin{tcolorbox}[
    enhanced,
    colback=customTealBg,
    colframe=customTealBg,
    arc=3mm,
    boxrule=0pt,
    width=\textwidth,
    top=8pt,
    bottom=8pt
]
{\small{\color{customTeal}\faInfoCircle} 肠道菌群抗性与适应性基因是细菌适应环境压力的重要遗传元件,主要包括抗生素耐药基因、金属离子耐受基因、外排系统基因等多种类型。这些基因使细菌获得了对不同环境压力的适应能力,其中既包括自然选择过程中进化获得的基因,也包括通过水平基因转移获得的基因。
}
\end{tcolorbox}

% 定义现代配色
\definecolor{modernblue}{RGB}{37, 99, 235}
\definecolor{moderngreen}{RGB}{22, 163, 74}
\definecolor{modernpurple}{RGB}{147, 51, 234}
\definecolor{modernrose}{RGB}{225, 29, 72}

% 定义背景色
\definecolor{lightblue}{RGB}{239, 246, 255}
\definecolor{lightgreen}{RGB}{240, 253, 244}
%\definecolor{lightpurple}{RGB}{245, 243, 255}
\definecolor{lightrose}{RGB}{255, 241, 242}

% 页面背景
%\pagecolor{gray!5}



% 定义现代配色
\definecolor{modernblue}{RGB}{41, 128, 185}
\definecolor{moderngreen}{RGB}{46, 204, 113}
%\definecolor{modernpurple}{RGB}{142, 68, 173}
\definecolor{modernrose}{RGB}{231, 76, 60}

% 定义现代卡片样式
% 定义现代卡片样式
% 定义现代卡片样式
\newtcolorbox{moderncard}[3][]{
    enhanced,
    colback=white,
    colframe=#2!5,
    fonttitle=\bfseries\large,
    title={\textcolor{#2}{\Large\faIcon{#3}}~~\textcolor{#2}{#1}},  % 文字颜色使用主题色
    arc=8pt,
    boxrule=0.5pt,
    left=15pt,
    right=15pt,
    top=20pt,
    bottom=20pt,
    minimum height=12em,
    before title={\vspace{3pt}},
    after title={\vspace{5pt}},
    title style={
        top=10pt,
        bottom=10pt,
        colback=white,  % 标题背景改为白色
        colframe=white,
        rounded corners=8pt
    }
}

\begin{center}
% 定义现代化的颜色
\definecolor{modernblue}{RGB}{41, 128, 185}
\definecolor{moderngreen}{RGB}{39, 174, 96}
\definecolor{modernpurple}{RGB}{142, 68, 173}
\definecolor{modernrose}{RGB}{192, 57, 43}

\begin{tcolorbox}[
    enhanced,
    colback=lightpurple!10, % 卡片底色
    colframe=white,  % 边框颜色
    arc=3mm,
    boxrule=0.5pt,
    width=\textwidth,
    top=8pt,
    bottom=8pt
]
{\small{\color{lightpurple}\faQuestionCircle}\quad \textbf{什么是肠道菌群的耐药基因?}\\
{\color{orange!50}\faComments}\quad 肠道菌群的耐药基因是指那些使细菌对特定抗生素或其他抗微生物药物产生耐药性的基因。这些基因编码的蛋白质可以干扰药物的作用机制,使细菌能够在抗生素存在的环境中生存和繁殖。

}
\end{tcolorbox}

\begin{tcolorbox}[
    enhanced,
    colback=lightpurple!10, % 卡片底色
    colframe=white,  % 边框颜色
    arc=3mm,
    boxrule=0.5pt,
    width=\textwidth,
    top=8pt,
    bottom=8pt
]
{\small{\color{lightpurple}\faQuestionCircle}\quad \textbf{什么是肠道菌群的金属耐受基因?}\\
{\color{orange!50}\faComments}\quad 肠道菌群的金属耐受基因是指存在于肠道微生物中的基因,这些基因使细菌能够在高浓度金属(如铜、铅、镉等)环境中生存。耐受机制包括金属离子排出、金属结合、细胞壁改造等。

}
\end{tcolorbox}

\begin{tcolorbox}[
    enhanced,
    colback=lightpurple!10, % 卡片底色
    colframe=white,  % 边框颜色
    arc=3mm,
    boxrule=0.5pt,
    width=\textwidth,
    top=8pt,
    bottom=8pt
]
{\small{\color{lightpurple}\faQuestionCircle}\quad \textbf{什么是肠道菌群的外排系统基因?}\\
{\color{orange!50}\faComments}\quad 肠道菌群的外排系统基因是指一类编码外排泵或外排系统的基因,这些系统使细菌能够将有害物质(如抗生素,毒素,重金属和杀生物剂等)从细胞内主动排出,以抵御这些物质的毒性。
%\textbf{临床意义}:
%\begin{enumerate}
%    \item 感染机制:毒力基因揭示了细菌如何通过生物机制引发感染,从而帮助理解感染的病理过程。
%    \item 疾病诊断:检测肠道菌群中毒力基因的存在可以作为某些感染性疾病的诊断标志。
%    \item 临床预防:了解毒力基因的影响可为开发抗感染策略和疫苗提供依据,有助于预防相关疾病。
%    \item 抗生素使用策略:识别含有毒力基因的细菌,帮助医生制定更有效的抗生素治疗方案。
%\end{enumerate}
}
\end{tcolorbox}

\begin{tcolorbox}[
    enhanced,
    colback=lightpurple!10, % 卡片底色
    colframe=white,  % 边框颜色
    arc=3mm,
    boxrule=0.5pt,
    width=\textwidth,
    top=8pt,
    bottom=8pt
]
{\small{\color{lightpurple}\faQuestionCircle}\quad \textbf{什么是肠道菌群的毒力基因?}\\
{\color{orange!50}\faComments}\quad 肠道菌群的毒力基因是指那些使细菌能够造成宿主组织损伤或引发炎症反应的基因。这些基因通常与细菌的致病性有关,使其能在宿主内存活、繁殖并引起疾病。

}
\end{tcolorbox}

%\begin{tcolorbox}[
%    enhanced,
%    colback=lightpurple!10, % 卡片底色
%    colframe=white,  % 边框颜色
%    arc=3mm,
%    boxrule=0.5pt,
%    width=\textwidth,
%    top=8pt,
%    bottom=8pt
%]
%{\small{\color{lightpurple}\faQuestionCircle}\quad \textbf{如何阅读每个菌群基因的表格?}\\
%{\color{orange!50}\faComments}\quad 下列是对标的各个列的介绍:
%\begin{itemize}
%    \item 来源菌株:指检测到的基因来源于哪些特定的肠道菌群。
%    \item 耐药基因/耐受基因/外排基因/毒力基因:
%    \item 耐药类型/基因类型:
%    \item reads 数量:指在次世代测序(NGS)过程中,与特定耐药基因相关的原始测序读取数据的数量。
%    \item 耐药药物:
%\end{itemize}
%
%}
%\end{tcolorbox}

% 第一行卡片
\begin{minipage}{0.48\textwidth}
    \begin{tcolorbox}[
        enhanced,
        colback=white,
        colframe=modernblue!20,
        arc=3mm,
        boxrule=1pt,
        title={\textcolor{modernblue}{\faIcon{capsules}\hspace{0.5em}\textbf{抗生素耐药基因}}},
        fonttitle=\large,
        top=8pt,
        bottom=8pt
    ]
        \begin{itemize}[leftmargin=*,nosep]
            \item[\textcolor{modernblue}{\small\faIcon{circle}}] \textbf{β-内酰胺类}
            \item[\textcolor{modernblue}{\small\faIcon{circle}}] \textbf{氨基糖苷类}
            \item[\textcolor{modernblue}{\small\faIcon{circle}}] \textbf{四环素类}
        \end{itemize}
    \end{tcolorbox}
\end{minipage}
\hfill
\begin{minipage}{0.48\textwidth}
    \begin{tcolorbox}[
        enhanced,
        colback=white,
        colframe=moderngreen!20,
        arc=3mm,
        boxrule=1pt,
        title={\textcolor{moderngreen}{\faIcon{atom}\hspace{0.5em}\textbf{金属离子耐受基因}}},
        fonttitle=\large,
        top=8pt,
        bottom=8pt
    ]
        \begin{itemize}[leftmargin=*,nosep]
            \item[\textcolor{moderngreen}{\small\faIcon{circle}}] \textbf{铜离子耐受}
            \item[\textcolor{moderngreen}{\small\faIcon{circle}}] \textbf{银离子耐受}
            \item[\textcolor{moderngreen}{\small\faIcon{circle}}] \textbf{碲、砷离子耐受}
        \end{itemize}
    \end{tcolorbox}
\end{minipage}

\vspace{0.3cm}

% 第二行卡片
\begin{minipage}{0.48\textwidth}
    \begin{tcolorbox}[
        enhanced,
        colback=white,
        colframe=modernpurple!20,
        arc=3mm,
        boxrule=1pt,
        title={\textcolor{modernpurple}{\faIcon{exchange-alt}\hspace{0.5em}\textbf{外排系统基因}}},
        fonttitle=\large,
        top=8pt,
        bottom=8pt
    ]
        \begin{itemize}[leftmargin=*,nosep]
            \item[\textcolor{modernpurple}{\small\faIcon{circle}}] \textbf{杀生物剂外排}
        \end{itemize}
    \end{tcolorbox}
\end{minipage}
\hfill
\begin{minipage}{0.48\textwidth}
    \begin{tcolorbox}[
        enhanced,
        colback=white,
        colframe=modernrose!20,
        arc=3mm,
        boxrule=1pt,
        title={\textcolor{modernrose}{\faIcon{dna}\hspace{0.5em}\textbf{毒力基因}}},
        fonttitle=\large,
        top=8pt,
        bottom=8pt
    ]
        \begin{itemize}[leftmargin=*,nosep]
            \item[\textcolor{modernrose}{\small\faIcon{circle}}] \textbf{毒力基因}
        \end{itemize}
    \end{tcolorbox}
\end{minipage}
\end{center}

\newpage

\begin{tcolorbox}[
    enhanced,
    colback=white,
    colframe=white,
    arc=2mm,
    boxrule=0pt,
    width=\textwidth,
    left=15pt,
    right=15pt,
    top=10pt,
    bottom=10pt,
    drop shadow={
        opacity=0.2,
        color=customTeal
    },
    borderline west={5pt}{0pt}{customTeal}
]
\textcolor{customTeal}{\Large\textbf{抗生素耐药基因检测}}
\end{tcolorbox}

\begin{tcolorbox}[
    enhanced,
    colback=customTealBg,
    colframe=customTealBg,
    arc=3mm,
    boxrule=0pt,
    width=\textwidth,
    top=8pt,
    bottom=8pt
]
{\small{\color{customTeal}\faInfoCircle} 下列是在您肠道内检测到的肠道细菌抗生素耐药基因。
}
\end{tcolorbox}

% 定义新的颜色
\definecolor{notebg}{RGB}{242, 249, 249}  % 非常浅的青绿色
\definecolor{noteframe}{RGB}{74, 184, 184} % 与标题相同的青绿色

% 定义颜色
\definecolor{boxbg}{RGB}{255, 255, 255}
\definecolor{green}{RGB}{34, 197, 94}
\definecolor{blue}{RGB}{37, 99, 235}
\definecolor{orange}{RGB}{249, 115, 22}
\definecolor{titlegray}{RGB}{75, 85, 99}


\begin{center}
\begin{minipage}{0.32\textwidth}
    \begin{tcolorbox}[
        width=\textwidth,
        height=4cm,
        colframe=white!80!gray,
        colback=white,
        boxrule=0.5pt,
        arc=15pt,
        enhanced,
        left=20pt,
        right=20pt,
        top=15pt,
        bottom=15pt,
        valign=center
    ]
        {\large\color{titlegray}\textbf{检测菌株}}

        \vspace{0.5em}
        {\Huge\textbf{21株}}

        \vspace{0.5em}
        {\large\color{green}\textbf{100\% 完成检测}}
    \end{tcolorbox}
\end{minipage}
\hspace{0.12em}
\begin{minipage}{0.32\textwidth}
    \begin{tcolorbox}[
        width=\textwidth,
        height=4cm,
        colframe=white!80!gray,
        colback=white,
        boxrule=0.5pt,
        arc=15pt,
        enhanced,
        left=20pt,
        right=20pt,
        top=15pt,
        bottom=15pt,
        valign=center
    ]
        {\large\color{titlegray}\textbf{耐药基因种类}}

        \vspace{0.5em}
        {\Huge\textbf{12种}}

        \vspace{0.5em}
        {\large\color{blue}\textbf{覆盖3大类别}}
    \end{tcolorbox}
\end{minipage}
\hspace{0.12em}
\begin{minipage}{0.32\textwidth}
    \begin{tcolorbox}[
        width=\textwidth,
        height=4cm,
        colframe=white!80!gray,
        colback=white,
        boxrule=0.5pt,
        arc=15pt,
        enhanced,
        left=20pt,
        right=20pt,
        top=15pt,
        bottom=15pt,
        valign=center
    ]
        {\large\color{titlegray}\textbf{最高reads数}}

        \vspace{0.5em}
        {\Huge\textbf{321,552}}

        \vspace{0.5em}
        {\large\color{orange}\textbf{四环素类耐药}}
    \end{tcolorbox}
\end{minipage}
\end{center}

\begin{tcolorbox}[
    enhanced,
    colback=lightpurple!10, % 卡片底色
    colframe=white,  % 边框颜色
    arc=3mm,
    boxrule=0.5pt,
    width=\textwidth,
    top=8pt,
    bottom=8pt
]
{\small{\color{lightpurple}\faQuestionCircle}\quad \textbf{肠道菌群的抗生素耐药基因有什么临床意义?}\\
{\color{orange!50}\faComments}\quad 肠道菌群的抗生素耐药基因具有重要的临床意义,主要体现在以下几个方面:
\begin{itemize}
    \item \textbf{感染控制}:抗生素耐药基因的存在使得某些细菌对常用抗生素产生耐药性,这会导致感染治疗的困难,增加病原体传播的风险,并可能引起严重的临床后果。
    \item \textbf{耐药性传播}:肠道菌群中耐药基因可以通过基因转移的方式在细菌之间传播。这种耐药性可能会传播到肠道外的病原体中,使得原本可控的感染变得难以治疗。
    \item \textbf{临床决策}:了解肠道菌群的抗生素耐药性状况可以帮助医生在选择抗生素时作出更好的决策,以确保选用的抗生素对感染病原体有效。
    \item \textbf{公共卫生监测}:监测肠道菌群中的抗生素耐药基因有助于公共卫生机构评估耐药菌的流行情况,为制定抗生素使用政策和控制耐药性传播提供数据支持。
    \item \textbf{微生物组健康}:抗生素的使用不仅会影响致病菌,还会改变正常肠道菌群的结构,这可能导致肠道健康问题。因此,监测和研究抗生素耐药基因可以为维持肠道微生物组的平衡提供指导。
\end{itemize}
}
\end{tcolorbox}

\newpage

\begin{center}
\begin{tikzpicture}[
    font=\small,
    title/.style={font=\small\bfseries\color{white}},
    value/.style={font=\small},
    reference/.style={font=\small},
    cell/.style={anchor=west, text width=4.2cm},
    note/.style={anchor=west, text width=4.5cm, align=left}
]
    \def\cardwidth{\textwidth}
    \def\cardheight{16}
    \def\barheight{0.25}
    \def\barwidth{1.5}
    \def\valuebarspace{0.4}

    % 容器和标题栏背景
    \draw[rounded corners=5, fill=white, draw=gray!20]
        (0,0) rectangle (\cardwidth,-\cardheight);
    \path[fill=customTeal]
        (0,0) [rounded corners=5] -- (\cardwidth,0) --
        (\cardwidth,0.8) -- (0,0.8) -- cycle;

    % 表头
    \node[title, anchor=west] at (0.5, 0.4) {\textbf{来源菌株}};
    \node[title] at (5, 0.4) {\textbf{耐药基因}};
    \node[title] at (8.5, 0.4) {\textbf{耐药类型}};
    \node[title] at (12, 0.4) {\textbf{reads数量}};
    \node[title] at (16, 0.4) {\textbf{耐药药物}};

    % 初始化位置计数器
    \def\currentpos{0.25}

    \foreach \zhname/\enname/\gene/\type/\reads/\drugs/\intro/\suggestion/\pos in {
        {细菌}/{Bacterium}/{aac(6')-E111}/{\text{氨基糖苷类}}/974/{阿米卡星\\ \footnotesize 卡那霉素}/{耐药基因可能导致对氨基糖苷类抗生素产生耐药性,影响治疗效果。}/{建议在使用氨基糖苷类抗生素时进行耐药性检测。}/\currentpos,
        {细菌}/{Bacterium}/{aph(3')-IIIa}/{\text{氨基糖苷类}}/974/{阿米卡星\\ \footnotesize 卡那霉素}/{该基因能使细菌对特定氨基糖苷类抗生素产生耐药性。}/{临床用药时需注意药物选择。}/\currentpos,
        {细菌}/{Bacterium}/{aac(6')-Ie2}/{\text{氨基糖苷类}}/974/{阿米卡星\\ \footnotesize 卡那霉素}/{编码氨基糖苷修饰酶,可使细菌对多种氨基糖苷类抗生素产生耐药性。}/{建议进行药敏试验指导用药。}/\currentpos,
        {罗伊氏乳杆菌}/{L. reuteri}/{aadE}/{\text{氨基糖苷类}}/76331/{链霉素}/{该基因主要与链霉素耐药相关,在乳酸菌中较为常见。}/{使用链霉素时需注意耐药情况。}/\currentpos,
        {罗伊氏乳杆菌}/{L. reuteri}/{aad9}/{\text{氨基糖苷类}}/76331/{氨基糖苷类}/{能够使细菌对多种氨基糖苷类抗生素产生耐药性。}/{建议选择替代药物进行治疗。}/\currentpos,
        {大肠杆菌}/{Escherichia coli}/{npmA}/{\text{氨基糖苷类}}/8649/{氨基糖苷类}/{该基因可导致大肠杆菌对氨基糖苷类抗生素广谱耐药。}/{需进行耐药性监测和合理用药。}/\currentpos,
        {粪肠球菌}/{E. faecium EnGen0025}/{aac(6')-Ie}/{\text{氨基糖苷类}}/79765/{阿米卡星\\ \footnotesize 卡那霉素}/{肠球菌中的耐药基因,可导致高水平氨基糖苷类耐药。}/{建议选择替代药物或联合用药。}/\currentpos,
        {毛螺菌科}/{Lachnospiraceae}/{aac(6')-Im}/{\text{氨基糖苷类}}/8649/{阿米卡星,卡那霉素\\ {\footnotesize 妥布霉素}}/{该基因可导致对多种氨基糖苷类药物产生耐药性。}/{建议进行耐药性检测,合理选择抗生素。}/\currentpos
    }
    {
        % 计算当前行的基础位置
        \pgfmathsetmacro{\basepos}{-2.8*\currentpos}

        % 来源菌株
        \node[cell, align=left] at (0.5,\basepos) {
            \small\textbf{\zhname}\\[-0.2em]
            {\color{lightgray}\small\textit{\enname}}
        };

        % 其余列
        \node[reference] at (5,\basepos) {\footnotesize\gene};
        \node[value] at (8.5,\basepos) {\footnotesize\type};
        \node[value] at (12,\basepos) {\footnotesize\reads};

        % 修改耐药药物列的显示方式
        \node[value, align=right] at (16,\basepos) {\footnotesize\drugs};

        % 添加卡片
        \pgfmathsetmacro{\cardypos}{\basepos-0.5}
        \begin{scope}[shift={(0,\cardypos)}]
            % 卡片背景
            \fill[rounded corners=5pt, customTeal!5, draw=gray!5]
                (0.3,-0.6) rectangle (17.3,0);

            % 基因说明图标和内容
            \node[anchor=west] at (0.5,-0.3) {
                \textbf{\color{gray!90}\footnotesize \textcolor{customTeal}{\faInfoCircle}}
            };
            \node[anchor=west, text width=16cm] at (0.9,-0.3) {
                {\small\color{gray}\footnotesize \intro}
            };
        \end{scope}

        % 分割线
        \draw[gray!20] (0.2,\basepos-1.3) -- (\cardwidth-0.2,\basepos-1.3);

        % 更新位置计数器
        \pgfmathsetmacro{\nextpos}{\currentpos+0.70}
        \xdef\currentpos{\nextpos}
    }

\end{tikzpicture}
\end{center}

\newpage

\begin{center}
\begin{tikzpicture}[
    % 基础样式定义
    font=\small,
    title/.style={font=\footnotesize\bfseries\color{white}},
    cell/.style={anchor=west, text width=4.2cm, font=\small\bfseries}, % 加粗
    gene/.style={anchor=east, text width=2.8cm, font=\small\bfseries, align=right}, % 右对齐且加粗
    type/.style={anchor=east, text width=2.5cm, align=right}, % 右对齐
    reads/.style={anchor=east, text width=2cm, align=right}, % 右对齐
    drug/.style={anchor=west, text width=3.5cm},
    category/.style={font=\small\bfseries\color{customTeal}}
]

% 颜色定义
\definecolor{customTeal}{RGB}{73, 175, 177}
\definecolor{categoryColor}{RGB}{230, 240, 240}
\definecolor{rowLineColor}{RGB}{230, 235, 235}
\definecolor{containerBorder}{RGB}{240, 240, 240}

% 基础尺寸
\def\cardwidth{\textwidth}
\def\cardheight{12}
\def\margin{0.3}
\def\rowspace{0.4} % 行间距参数

% 容器背景
\draw[rounded corners=8, fill=white, draw=containerBorder, line width=1pt]
    (0,0) rectangle (\cardwidth,-\cardheight);

% 标题栏
\fill[customTeal, rounded corners=5] (0,0) rectangle (\cardwidth,0.8);
\node[title, anchor=west] at (\margin,0.4) {\normalsize 来源菌株};
\node[title, anchor=east] at (7.6,0.4) {\normalsize 耐药基因};
\node[title, anchor=east] at (10,0.4) {\normalsize 耐药类型};
\node[title, anchor=east] at (12.5,0.4) {\normalsize reads数量};
\node[title, anchor=west] at (13.8,0.4) {\normalsize 具体耐药药物};

% 定义一个宏来绘制耐药分类
\newcommand{\drawResistanceCategory}[5]{
    % 绘制分类背景
    \fill[categoryColor, rounded corners=5] (\margin,#1) rectangle (\cardwidth-\margin-5,#2);
    \node[category, anchor=west] at (0.6,#3) {#4};

    % 数据行
    \foreach \bacteria/\gene/\type/\reads/\drug/\ypos in #5 {
        \node[cell] at (\margin,\ypos) {\bacteria};
        \node[gene] at (7.6,\ypos) {\gene};
        \node[type] at (10,\ypos) {\type};
        \node[reads] at (12.5,\ypos) {\reads};
        \node[drug] at (13.8,\ypos) {\drug};
        \draw[rowLineColor] (\margin,\ypos+\rowspace) -- (\cardwidth-\margin,\ypos+\rowspace);
    }
}

% 氨基糖苷类
\drawResistanceCategory{-0.1}{-0.7}{-0.45}{氨基糖苷类抗生素耐药基因}{
    {Bacterium}/{aac(6')-E111}/{\text{氨基糖苷类}}/974/{阿米卡星/卡那霉素}/-1.2,
    {Bacterium}/{aph(3')-IIIa}/{\text{氨基糖苷类}}/974/{阿米卡星/卡那霉素}/-2.0,
    {Bacterium}/{aac(6')-Ie2}/{\text{氨基糖苷类}}/974/{阿米卡星/卡那霉素}/-2.8,
    {Limosilactobacillus reuteri}/{aadE}/{\text{氨基糖苷类}}/76331/{链霉素}/-3.6,
    {Limosilactobacillus reuteri}/{aad9}/{\text{氨基糖苷类}}/76331/{氨基糖苷类}/-4.4,
    {Escherichia coli}/{npmA}/{\text{氨基糖苷类}}/8649/{氨基糖苷类}/-5.2,
    {E. faecium EnGen0025}/{aac(6')-Ie}/{\text{氨基糖苷类}}/79765/{阿米卡星/卡那霉素}/-6.0,
    {Lachnospiraceae}/{aac(6')-Im}/{\text{氨基糖苷类}}/8649/{阿米卡星/卡那霉素/妥布霉素}/-6.8
}

% β-内酰胺类
\drawResistanceCategory{-5.5}{-6}{-5.2}{β-内酰胺类抗生素耐药基因}{
    {Terrabacteria group}/{dfrF}/{\text{甲氧苄啶}}/28043/{甲氧苄啶}/-5.2,
    {S. sp. GMQS-10BH}/{cat414}/{\text{氯霉素}}/1446/{氯霉素}/-6.0
}

% 大环内酯类
\drawResistanceCategory{-6.4}{-7.1}{-6.75}{大环内酯类抗生素耐药基因}{
    {Firmicutes}/{mef(A)}/{\text{大环内酯}}/1476/{阿奇霉素/红霉素}/-7.5,
    {Firmicutes}/{lnu(C)}/{\text{林可霉素}}/26567/{林可霉素}/-8.3,
    {E. clostridioformis}/{vanS-D}/{\text{万古霉素}}/746/{万古霉素}/-9.1
}

\end{tikzpicture}
\end{center}

\newpage

\begin{tcolorbox}[
    enhanced,
    colback=white,
    colframe=white,
    arc=2mm,
    boxrule=0pt,
    width=\textwidth,
    left=15pt,
    right=15pt,
    top=10pt,
    bottom=10pt,
    drop shadow={
        opacity=0.2,
        color=customTeal
    },
    borderline west={5pt}{0pt}{customTeal}
]
\textcolor{customTeal}{\Large\textbf{肠道细菌金属耐受基因检测}}
\end{tcolorbox}

\begin{tcolorbox}[
    enhanced,
    colback=customTealBg,
    colframe=customTealBg,
    arc=3mm,
    boxrule=0pt,
    width=\textwidth,
    top=8pt,
    bottom=8pt
]
{\small{\color{customTeal}\faInfoCircle} 下列是在您肠道内检测到的肠道细菌金属耐受基因。表格按照检测到的金属耐受基因丰度值(reads 数量)从多到少依次排序展示。
}
\end{tcolorbox}

% 定义新的颜色
\definecolor{notebg}{RGB}{242, 249, 249}  % 非常浅的青绿色
\definecolor{noteframe}{RGB}{74, 184, 184} % 与标题相同的青绿色

% 定义颜色
\definecolor{boxbg}{RGB}{255, 255, 255}
\definecolor{green}{RGB}{34, 197, 94}
\definecolor{blue}{RGB}{37, 99, 235}
\definecolor{orange}{RGB}{249, 115, 22}
\definecolor{titlegray}{RGB}{75, 85, 99}


\begin{center}
\begin{minipage}{0.32\textwidth}
    \begin{tcolorbox}[
        width=\textwidth,
        height=4cm,
        colframe=white!80!gray,
        colback=white,
        boxrule=0.5pt,
        arc=15pt,
        enhanced,
        left=20pt,
        right=20pt,
        top=15pt,
        bottom=15pt,
        valign=center
    ]
        {\large\color{titlegray}\textbf{检测菌株}}

        \vspace{0.5em}
        {\Huge\textbf{5株}}

        \vspace{0.5em}
        {\large\color{green}\textbf{100\% 完成检测}}
    \end{tcolorbox}
\end{minipage}
\hspace{0.12em}
\begin{minipage}{0.32\textwidth}
    \begin{tcolorbox}[
        width=\textwidth,
        height=4cm,
        colframe=white!80!gray,
        colback=white,
        boxrule=0.5pt,
        arc=15pt,
        enhanced,
        left=20pt,
        right=20pt,
        top=15pt,
        bottom=15pt,
        valign=center
    ]
        {\large\color{titlegray}\textbf{耐药基因种类}}

        \vspace{0.5em}
        {\Huge\textbf{5种}}

        \vspace{0.5em}
        {\large\color{blue}\textbf{覆盖3大类别}}
    \end{tcolorbox}
\end{minipage}
\hspace{0.12em}
\begin{minipage}{0.32\textwidth}
    \begin{tcolorbox}[
        width=\textwidth,
        height=4cm,
        colframe=white!80!gray,
        colback=white,
        boxrule=0.5pt,
        arc=15pt,
        enhanced,
        left=20pt,
        right=20pt,
        top=15pt,
        bottom=15pt,
        valign=center
    ]
        {\large\color{titlegray}\textbf{最高reads数}}

        \vspace{0.5em}
        {\Huge\textbf{1,221}}

        \vspace{0.5em}
        {\large\color{orange}\textbf{铜离子耐受}}
    \end{tcolorbox}
\end{minipage}
\end{center}

\begin{tcolorbox}[
    enhanced,
    colback=lightpurple!10, % 卡片底色
    colframe=white,  % 边框颜色
    arc=3mm,
    boxrule=0.5pt,
    width=\textwidth,
    top=8pt,
    bottom=8pt
]
{\small{\color{lightpurple}\faQuestionCircle}\quad \textbf{肠道菌群的金属耐受基因检测有什么临床意义?}\\
{\color{orange!50}\faComments}\quad 肠道菌群的金属耐受基因具有重要的临床意义,主要体现在以下几个方面:
\begin{itemize}
    \item \textbf{感染管理}:检测金属耐受基因可以帮助识别对金属(如铜、铅或汞)耐受的细菌,为感染控制和治疗提供重要信息。
    \item \textbf{抗生素替代疗法}:了解菌群对重金属的耐受性,可以为重金属作为潜在抗生素替代品的使用提供依据,尤其是在抗生素耐药性日益严重的背景下。
    \item \textbf{评估微生物组健康}:金属耐受基因的表达可以反映肠道菌群的平衡及其生态适应性,帮助评估微生物组健康状况。
    \item \textbf{预防毒性反应}:检测金属耐受基因可以识别那些潜在会引起毒性反应的细菌,帮助医生在治疗中避免使用可能导致细菌生长的药物或治疗方法。
    \item \textbf{公共卫生监测}:通过监测肠道菌群的金属耐受基因,可以评估环境中重金属污染对人群健康的影响,从而为公共卫生政策制定提供科学依据。
\end{itemize}
}
\end{tcolorbox}

\newpage

\begin{center}
\begin{tikzpicture}[
    font=\small,
    title/.style={font=\small\bfseries\color{white}},
    value/.style={font=\small},
    reference/.style={font=\small},
    cell/.style={anchor=west, text width=4.2cm},
    note/.style={anchor=west, text width=4.5cm, align=left}
]
    \def\cardwidth{\textwidth}
    \def\cardheight{16}
    \def\barheight{0.25}
    \def\barwidth{1.5}
    \def\valuebarspace{0.4}

    % 容器和标题栏背景
    \draw[rounded corners=5, fill=white, draw=gray!20]
        (0,0) rectangle (\cardwidth,-\cardheight);
    \path[fill=customTeal]
        (0,0) [rounded corners=5] -- (\cardwidth,0) --
        (\cardwidth,0.8) -- (0,0.8) -- cycle;

    % 表头
    \node[title, anchor=west] at (0.5, 0.4) {\textbf{来源菌株}};
    \node[title] at (5, 0.4) {\textbf{耐受基因}};
    \node[title] at (8.5, 0.4) {\textbf{耐受类型}};
    \node[title] at (12, 0.4) {\textbf{reads数量}};
    \node[title] at (16, 0.4) {\textbf{耐受金属}};

    % 初始化位置计数器
    \def\currentpos{0.25}

    \foreach \zhname/\enname/\gene/\type/\reads/\drugs/\intro/\suggestion/\pos in {
        {肠杆菌科}/{Enterobacteriaceae}/{pcoD}/{金属}/1221/{铜离子}/{耐药基因可能导致对氨基糖苷类抗生素产生耐药性,影响治疗效果。}/{建议在使用氨基糖苷类抗生素时进行耐药性检测。}/\currentpos,
        {肠杆菌科}/{Enterobacteriaceae}/{pcoC}/{金属}/1221/{铜离子}/{耐药基因可能导致对氨基糖苷类抗生素产生耐药性,影响治疗效果。}/{建议在使用氨基糖苷类抗生素时进行耐药性检测。}/\currentpos,
        {肠杆菌科}/{Enterobacteriaceae}/{silS}/{金属}/939/{铜离子,银离子}/{该基因主要与链霉素耐药相关,在乳酸菌中较为常见。}/{使用链霉素时需注意耐药情况。}/\currentpos,
        {友好柠檬酸杆菌}/{Citrobacter freundii}/{silR }/{金属}/534/{铜离子,银离子}/{编码氨基糖苷修饰酶,可使细菌对多种氨基糖苷类抗生素产生耐药性。}/{建议进行药敏试验指导用药。}/\currentpos,
        {Gammaproteo}/{Gammaproteo}/{terZ}/{金属}/380/{碲离子}/{该基因可导致大肠杆菌对氨基糖苷类抗生素广谱耐药。}/{需进行耐药性监测和合理用药。}/\currentpos,
        {细菌}/{Bacteria}/{pcoC}/{金属}/214/{铜离子}/{能够使细菌对多种氨基糖苷类抗生素产生耐药性。}/{建议选择替代药物进行治疗。}/\currentpos,
        {肠杆菌目}/{Enterobacterales}/{arsD}/{金属}/0/{砷离子}/{肠球菌中的耐药基因,可导致高水平氨基糖苷类耐药。}/{建议选择替代药物或联合用药。}/\currentpos,
        {C. freundii complex}/{C. freundii complex}/{fieF}/{金属}/0/{-}/{该基因可导致对多种氨基糖苷类药物产生耐药性。}/{建议进行耐药性检测,合理选择抗生素。}/\currentpos
    }
    {
        % 计算当前行的基础位置
        \pgfmathsetmacro{\basepos}{-2.8*\currentpos}

        % 来源菌株
        \node[cell, align=left] at (0.5,\basepos) {
            \small\textbf{\zhname}\\[-0.2em]
            {\color{lightgray}\small\textit{\enname}}
        };

        % 其余列
        \node[reference] at (5,\basepos) {\footnotesize\gene};
        \node[value] at (8.5,\basepos) {\footnotesize\type};
        \node[value] at (12,\basepos) {\footnotesize\reads};

        % 修改耐药药物列的显示方式
        \node[value, align=right] at (16,\basepos) {\footnotesize\drugs};

        % 添加卡片
        \pgfmathsetmacro{\cardypos}{\basepos-0.5}
        \begin{scope}[shift={(0,\cardypos)}]
            % 卡片背景
            \fill[rounded corners=5pt, customTeal!5, draw=gray!5]
                (0.3,-0.6) rectangle (17.3,0);

            % 基因说明图标和内容
            \node[anchor=west] at (0.5,-0.3) {
                \textbf{\color{gray!90}\footnotesize \textcolor{customTeal}{\faInfoCircle}}
            };
            \node[anchor=west, text width=16cm] at (0.9,-0.3) {
                {\small\color{gray}\footnotesize \intro}
            };

        \end{scope}

        % 分割线
        \draw[gray!20] (0.2,\basepos-1.3) -- (\cardwidth-0.2,\basepos-1.3);

        % 更新位置计数器
        \pgfmathsetmacro{\nextpos}{\currentpos+0.7}
        \xdef\currentpos{\nextpos}
    }

\end{tikzpicture}
\end{center}

\newpage

\begin{tcolorbox}[
    enhanced,
    colback=white,
    colframe=white,
    arc=2mm,
    boxrule=0pt,
    width=\textwidth,
    left=15pt,
    right=15pt,
    top=10pt,
    bottom=10pt,
    drop shadow={
        opacity=0.2,
        color=customTeal
    },
    borderline west={5pt}{0pt}{customTeal}
]
\textcolor{customTeal}{\Large\textbf{肠道细菌外排系统基因检测}}
\end{tcolorbox}

\begin{center}
\begin{minipage}{0.32\textwidth}
    \begin{tcolorbox}[
        width=\textwidth,
        height=4cm,
        colframe=white!80!gray,
        colback=white,
        boxrule=0.5pt,
        arc=15pt,
        enhanced,
        left=20pt,
        right=20pt,
        top=15pt,
        bottom=15pt,
        valign=center
    ]
        {\large\color{titlegray}\textbf{检测菌株}}

        \vspace{0.5em}
        {\Huge\textbf{1株}}

        \vspace{0.5em}
        {\large\color{green}\textbf{100\% 完成检测}}
    \end{tcolorbox}
\end{minipage}
\hspace{0.12em}
\begin{minipage}{0.32\textwidth}
    \begin{tcolorbox}[
        width=\textwidth,
        height=4cm,
        colframe=white!80!gray,
        colback=white,
        boxrule=0.5pt,
        arc=15pt,
        enhanced,
        left=20pt,
        right=20pt,
        top=15pt,
        bottom=15pt,
        valign=center
    ]
        {\large\color{titlegray}\textbf{耐药基因种类}}

        \vspace{0.5em}
        {\Huge\textbf{2种}}

        \vspace{0.5em}
        {\large\color{blue}\textbf{覆盖3大类别}}
    \end{tcolorbox}
\end{minipage}
\hspace{0.12em}
\begin{minipage}{0.32\textwidth}
    \begin{tcolorbox}[
        width=\textwidth,
        height=4cm,
        colframe=white!80!gray,
        colback=white,
        boxrule=0.5pt,
        arc=15pt,
        enhanced,
        left=20pt,
        right=20pt,
        top=15pt,
        bottom=15pt,
        valign=center
    ]
        {\large\color{titlegray}\textbf{最高reads数}}

        \vspace{0.5em}
        {\Huge\textbf{234}}

        \vspace{0.5em}
        {\large\color{orange}\textbf{杀生物剂外排基因}}
    \end{tcolorbox}
\end{minipage}
\end{center}

\begin{center}
\begin{tikzpicture}[
    font=\small,
    title/.style={font=\small\bfseries\color{white}},
    value/.style={font=\small},
    reference/.style={font=\small},
    cell/.style={anchor=west, text width=4.2cm},
    note/.style={anchor=west, text width=4.5cm, align=left}
]
    \def\cardwidth{\textwidth}
    \def\cardheight{3}
    \def\barheight{0.25}
    \def\barwidth{1.5}
    \def\valuebarspace{0.4}

    % 容器和标题栏背景
    \draw[rounded corners=5, fill=white, draw=gray!20]
        (0,0) rectangle (\cardwidth,-\cardheight);
    \path[fill=customTeal]
        (0,0) [rounded corners=5] -- (\cardwidth,0) --
        (\cardwidth,0.8) -- (0,0.8) -- cycle;

    % 表头
    \node[title, anchor=west] at (0.5, 0.4) {\textbf{来源菌株}};
    \node[title] at (5, 0.4) {\textbf{外排基因}};
    \node[title] at (8.5, 0.4) {\textbf{基因类型}};
    \node[title] at (12, 0.4) {\textbf{reads数量}};
    \node[title] at (16, 0.4) {\textbf{外排药物}};

    % 初始化位置计数器
    \def\currentpos{0.25}

    \foreach \zhname/\enname/\gene/\type/\reads/\drugs/\intro/\suggestion/\pos in {
        {肠杆菌科}/{Enterobacteriaceae}/{SMR$_$efflux$_$emrE}/{外排}/234/{杀生物剂}/{耐药基因可能导致对氨基糖苷类抗生素产生耐药性,影响治疗效果。}/{建议在使用氨基糖苷类抗生素时进行耐药性检测。}/\currentpos
    }
    {
        % 计算当前行的基础位置
        \pgfmathsetmacro{\basepos}{-2.8*\currentpos}

        % 来源菌株
        \node[cell, align=left] at (0.5,\basepos) {
            \small\textbf{\zhname}\\[-0.2em]
            {\color{lightgray}\small\textit{\enname}}
        };

        % 其余列
        \node[reference] at (5,\basepos) {\footnotesize\gene};
        \node[value] at (8.5,\basepos) {\footnotesize\type};
        \node[value] at (12,\basepos) {\footnotesize\reads};

        % 修改耐药药物列的显示方式
        \node[value, align=right] at (16,\basepos) {\footnotesize\drugs};

        % 添加卡片
        \pgfmathsetmacro{\cardypos}{\basepos-0.5}
        \begin{scope}[shift={(0,\cardypos)}]
            % 卡片背景
            \fill[rounded corners=5pt, customTeal!5, draw=gray!5]
                (0.3,-0.6) rectangle (17.3,0);

            % 基因说明图标和内容
            \node[anchor=west] at (0.5,-0.3) {
                \textbf{\color{gray!90}\footnotesize \textcolor{customTeal}{\faInfoCircle}}
            };
            \node[anchor=west, text width=16cm] at (0.9,-0.3) {
                {\small\color{gray}\footnotesize \intro}
            };

        \end{scope}

        % 分割线
        \draw[gray!20] (0.2,\basepos-1.3) -- (\cardwidth-0.2,\basepos-1.3);

        % 更新位置计数器
        \pgfmathsetmacro{\nextpos}{\currentpos+0.7}
        \xdef\currentpos{\nextpos}
    }

\end{tikzpicture}
\end{center}

\newpage

\begin{tcolorbox}[
    enhanced,
    colback=white,
    colframe=white,
    arc=2mm,
    boxrule=0pt,
    width=\textwidth,
    left=15pt,
    right=15pt,
    top=10pt,
    bottom=10pt,
    drop shadow={
        opacity=0.2,
        color=customTeal
    },
    borderline west={5pt}{0pt}{customTeal}
]
\textcolor{customTeal}{\Large\textbf{肠道细菌毒力基因检测}}
\end{tcolorbox}

\begin{center}
\begin{minipage}{0.32\textwidth}
    \begin{tcolorbox}[
        width=\textwidth,
        height=4cm,
        colframe=white!80!gray,
        colback=white,
        boxrule=0.5pt,
        arc=15pt,
        enhanced,
        left=20pt,
        right=20pt,
        top=15pt,
        bottom=15pt,
        valign=center
    ]
        {\large\color{titlegray}\textbf{检测菌株}}

        \vspace{0.5em}
        {\Huge\textbf{8株}}

        \vspace{0.5em}
        {\large\color{green}\textbf{100\% 完成检测}}
    \end{tcolorbox}
\end{minipage}
\hspace{0.12em}
\begin{minipage}{0.32\textwidth}
    \begin{tcolorbox}[
        width=\textwidth,
        height=4cm,
        colframe=white!80!gray,
        colback=white,
        boxrule=0.5pt,
        arc=15pt,
        enhanced,
        left=20pt,
        right=20pt,
        top=15pt,
        bottom=15pt,
        valign=center
    ]
        {\large\color{titlegray}\textbf{毒力基因种类}}

        \vspace{0.5em}
        {\Huge\textbf{35种}}

        \vspace{0.5em}
        {\large\color{blue}\textbf{覆盖8大类别}}
    \end{tcolorbox}
\end{minipage}
\hspace{0.12em}
\begin{minipage}{0.32\textwidth}
    \begin{tcolorbox}[
        width=\textwidth,
        height=4cm,
        colframe=white!80!gray,
        colback=white,
        boxrule=0.5pt,
        arc=15pt,
        enhanced,
        left=20pt,
        right=20pt,
        top=15pt,
        bottom=15pt,
        valign=center
    ]
        {\large\color{titlegray}\textbf{最高reads数}}

        \vspace{0.5em}
        {\Huge\textbf{0}}

        \vspace{0.5em}
        {\large\color{orange}\textbf{未检测到毒力基因}}
    \end{tcolorbox}
\end{minipage}
\end{center}

\begin{center}
\begin{tikzpicture}[
    font=\small,
    title/.style={font=\small\bfseries\color{white}},
    value/.style={font=\small},
    reference/.style={font=\small},
    cell/.style={anchor=west, text width=4.2cm},
    note/.style={anchor=west, text width=4.5cm, align=left}
]
    \def\cardwidth{\textwidth}
    \def\cardheight{13}
    \def\barheight{0.25}
    \def\barwidth{1.5}
    \def\valuebarspace{0.4}

    % 容器和标题栏背景
    \draw[rounded corners=5, fill=white, draw=gray!20]
        (0,0) rectangle (\cardwidth,-\cardheight);
    \path[fill=customTeal]
        (0,0) [rounded corners=5] -- (\cardwidth,0) --
        (\cardwidth,0.8) -- (0,0.8) -- cycle;

    % 表头
    \node[title, anchor=west] at (0.5, 0.4) {\textbf{来源菌株}};
    \node[title] at (5, 0.4) {\textbf{外排基因}};
    \node[title] at (8.5, 0.4) {\textbf{基因类型}};
    \node[title] at (12, 0.4) {\textbf{reads数量}};
    \node[title] at (16, 0.4) {\textbf{检测结果}};

    % 初始化位置计数器
    \def\currentpos{0.25}

    \foreach \zhname/\enname/\gene/\type/\reads/\drugs/\intro/\suggestion/\pos in {
        {产志贺毒素大肠杆菌}/{STEC}/{stx1A}/{毒力}/-/{阴性}/{耐药基因可能导致对氨基糖苷类抗生素产生耐药性,影响治疗效果。}/{}/\currentpos,
        {产志贺毒素大肠杆菌}/{STEC}/{stx1B}/{毒力}/-/{阴性}/{耐药基因可能导致对氨基糖苷类抗生素产生耐药性,影响治疗效果。}/{}/\currentpos,
        {产志贺毒素大肠杆菌}/{STEC}/{stx2A}/{毒力}/-/{阴性}/{耐药基因可能导致对氨基糖苷类抗生素产生耐药性,影响治疗效果。}/{}/\currentpos,
        {产志贺毒素大肠杆菌}/{STEC}/{stx2B}/{毒力}/-/{阴性}/{耐药基因可能导致对氨基糖苷类抗生素产生耐药性,影响治疗效果。}/{}/\currentpos
    }
    {
        % 计算当前行的基础位置
        \pgfmathsetmacro{\basepos}{-2.8*\currentpos}

        % 来源菌株
        \node[cell, align=left] at (0.5,\basepos) {
            \small\textbf{\zhname}\\[-0.2em]
            {\color{lightgray}\small\textit{\enname}}
        };

        % 其余列
        \node[reference] at (5,\basepos) {\footnotesize\gene};
        \node[value] at (8.5,\basepos) {\footnotesize\type};
        \node[value] at (12,\basepos) {\footnotesize\reads};

        % 修改耐药药物列的显示方式
        \node[value, align=right] at (16,\basepos) {\footnotesize\drugs};

        % 添加卡片
        \pgfmathsetmacro{\cardypos}{\basepos-0.5}
        \begin{scope}[shift={(0,\cardypos)}]
            % 卡片背景
            \fill[rounded corners=5pt, customTeal!5, draw=gray!5]
                (0.3,-0.6) rectangle (17.3,0);

            % 基因说明图标和内容
            \node[anchor=west] at (0.5,-0.3) {
                \textbf{\color{gray!90}\footnotesize \textcolor{customTeal}{\faInfoCircle}}
            };
            \node[anchor=west, text width=16cm] at (0.9,-0.3) {
                {\small\color{gray}\footnotesize \intro}
            };


        \end{scope}

        % 分割线
        \draw[gray!20] (0.2,\basepos-1.3) -- (\cardwidth-0.2,\basepos-1.3);

        % 更新位置计数器
        \pgfmathsetmacro{\nextpos}{\currentpos+0.7}
        \xdef\currentpos{\nextpos}
    }

\end{tikzpicture}
\end{center}

\newpage

\begin{tcolorbox}[
    enhanced,
    colback=white,
    colframe=customTeal,
    arc=2mm,
    boxrule=1pt,
    left=20pt,
    right=20pt,
    top=12pt,
    bottom=12pt,
    width=\textwidth,
    fontupper=\sffamily,
    overlay={
    \draw[customTeal, line width=2pt]
    ([xshift=15pt]frame.south west) -- ([xshift=-15pt]frame.south east);
    }
]
{\Large\bfseries\textcolor{customTeal}{\Huge 菌群代谢物及神经递质评估}}
\end{tcolorbox}

\begin{tcolorbox}[
    enhanced,
    colback=customTealBg,
    colframe=customTealBg,
    arc=3mm,
    boxrule=0pt,
    width=\textwidth,
    top=8pt,
    bottom=8pt
]
{\small{\color{customTeal}\faInfoCircle} 菌群代谢物和神经递质是肠道微生物与宿主相互作用的重要媒介,已发现数百种不同的活性分子。这些代谢物在维持肠道健康和影响宿主生理功能方面发挥着关键作用。健康成年人肠道中的主要代谢产物包括短链脂肪酸(SCFAs)、胆汁酸、色氨酸代谢物和神经递质等。以下是对这些代谢物的详细介绍:

\begin{itemize}
    \item \textbf{有机酸类}:
    \begin{itemize}
        \item 肠道微生物主要产生短链脂肪酸(如乙酸盐、丙酸盐和丁酸盐),这些有机酸在肠道稳态维持中发挥核心作用,通常占所有代谢产物的60\%以上。
    \end{itemize}
    \item \textbf{神经活性物质}:
    \begin{itemize}
        \item 包括γ-氨基丁酸(GABA)、血清素(5-HT)和多巴胺等,这些神经递质在调节情绪、行为和睡眠等方面具有重要作用。
    \end{itemize}
    \item \textbf{含氮/含硫化合物}:
    \begin{itemize}
        \item 这些化合物在肠道内的代谢过程中产生,可能影响宿主的代谢和免疫反应。
    \end{itemize}
    \item \textbf{其他代谢物}:
    \begin{itemize}
        \item 包括胆汁酸和色氨酸代谢产物等,这些物质在肠道健康和全身代谢中也发挥着重要作用。
    \end{itemize}
\end{itemize}

{\color{orange}\faExclamationTriangle} \textbf{特别注意}:
\begin{itemize}
    \item 这些菌群代谢物指标及神经递质并非通过直接检测获得,而是通过肠道菌群数据进行评估推算,结果仅供参考!
\end{itemize}
}
\end{tcolorbox}


\begin{center}
\begin{minipage}{0.49\textwidth}
\begin{tcolorbox}[
    enhanced,
    width=\textwidth,
    height=5cm,  % 设置卡片总高度
    arc=5pt,
    colback=white,
    colframe=gray!30,
    title={\textbf{有机酸类}},
    title style={
        colback=gray!30,
        top=0.3cm,  % 标题上边距
        bottom=0.3cm,  % 标题下边距
        minipage
    }
]
本报告中评估的有机酸类菌群代谢物为下列4项
%\begin{itemize}
%    \item 丁酸盐
%    \item 丙酸盐
%    \item 乙酸盐
%    \item 异戊酸盐
%\end{itemize}
\end{tcolorbox}
\end{minipage}
\begin{minipage}{0.49\textwidth}
\begin{tcolorbox}[
    enhanced,
    width=\textwidth,
    height=5cm,  % 设置卡片总高度
    arc=5pt,
    colback=white,
    colframe=gray!30,
    title={\textbf{神经活性物质}},
    title style={
        colback=gray!30,
        top=0.3cm,  % 标题上边距
        bottom=0.3cm,  % 标题下边距
        minipage
    }
]
测试内容2
\end{tcolorbox}
\end{minipage}
\end{center}

\begin{center}
\begin{minipage}{0.49\textwidth}
\begin{tcolorbox}[
    enhanced,
    width=\textwidth,
    height=5cm,  % 设置卡片总高度
    arc=5pt,
    colback=white,
    colframe=gray!30,
    title={\textbf{含氮/含硫化合物}},
    title style={
        colback=gray!30,
        top=0.3cm,  % 标题上边距
        bottom=0.3cm,  % 标题下边距
        minipage
    }
]
测试内容1
\end{tcolorbox}
\end{minipage}
\begin{minipage}{0.49\textwidth}
\begin{tcolorbox}[
    enhanced,
    width=\textwidth,
    height=5cm,  % 设置卡片总高度
    arc=5pt,
    colback=white,
    colframe=gray!30,
    title={\textbf{其他代谢物}},
    title style={
        colback=gray!30,
        top=0.3cm,  % 标题上边距
        bottom=0.3cm,  % 标题下边距
        minipage
    }
]
测试内容2
\end{tcolorbox}
\end{minipage}
\end{center}

\newpage

\begin{tcolorbox}[
    enhanced,
    colback=white,
    colframe=white,
    arc=2mm,
    boxrule=0pt,
    width=\textwidth,
    left=15pt,
    right=15pt,
    top=10pt,
    bottom=10pt,
    drop shadow={
        opacity=0.2,
        color=customTeal
    },
    borderline west={5pt}{0pt}{customTeal}
]
\textcolor{customTeal}{\Large\textbf{有机酸类代谢物评估}}
\end{tcolorbox}

\vspace{0.05cm}

\begin{tcolorbox}[
    enhanced,
    colback=customTealBg,
    colframe=customTealBg,
    arc=3mm,
    boxrule=0pt,
    width=\textwidth,
    top=8pt,
    bottom=8pt
]
{\small{\color{customTeal}\faInfoCircle} 有机酸类代谢物是由肠道微生物发酵碳水化合物和蛋白质产生的重要中间产物。这些物质参与能量代谢、神经传导和免疫调节等多个生理过程。通过检测这些指标,可以评估肠道微生物的代谢活性和宿主健康状况。\\

{\color{orange}\faExclamationTriangle} \textbf{特别注意}:
}
\end{tcolorbox}

\begin{tcolorbox}[
    enhanced,
    colback=lightpurple!10, % 卡片底色
    colframe=white,  % 边框颜色
    arc=3mm,
    boxrule=0.5pt,
    width=\textwidth,
    top=8pt,
    bottom=8pt
]
{\small{\color{lightpurple}\faQuestionCircle}\quad \textbf{这些有机酸类代谢物指标有什么临床意义?}\\
{\color{orange!50}\faComments}\quad 有机酸类代谢物指标的临床意义体现在以下方面:
\begin{itemize}
    \item 肠道健康评估:通过短链脂肪酸水平判断肠道微生物代谢活性。
    \item 能量代谢评价:有机酸水平反映机体能量利用效率。
    \item 神经功能参考:氨基酸代谢物可作为评估神经系统状态的指标。
    \item 免疫功能指标:部分代谢物与免疫调节和炎症反应密切相关。
\end{itemize}
}
\end{tcolorbox}
\begin{center}
\begin{tikzpicture}[
    font=\small,
    title/.style={font=\small\bfseries\color{white}},
    value/.style={font=\small},
    reference/.style={font=\small},
    cell/.style={anchor=west, text width=4.2cm},
    note/.style={anchor=west, text width=4.5cm, align=left}
]
    \def\cardwidth{\textwidth}
    \def\cardheight{7.9}
    \def\barheight{0.25}
    \def\barwidth{1.5}
    \def\valuebarspace{0.4}

    % 容器和标题栏背景
    \draw[rounded corners=5, fill=white, draw=gray!20]
        (0,0) rectangle (\cardwidth,-\cardheight);
    \path[fill=customTeal]
        (0,0) [rounded corners=5] -- (\cardwidth,0) --
        (\cardwidth,0.8) -- (0,0.8) -- cycle;

    % 表头
    \node[title, anchor=west] at (0.5,0.4) {\textbf{菌种名称}};
    \node[title] at (6, 0.4) {\textbf{正常范围}};
    \node[title] at (11, 0.4) {\textbf{检测丰度}};
    \node[title] at (16, 0.4) {\textbf{结果评价}};

    % 初始化位置计数器
    \def\currentpos{0.25}

    % 数据行和卡片
    % 数据行和卡片
    \foreach \item/\enitem/\value/\range/\status/\intro/\suggestion/\index in {
        {丁酸盐}/{Butyrate}/24/{15-98}/正常/{重要的短链脂肪酸,为肠道细胞提供能量,具有抗炎和维护肠道屏障功能。}/{}/\currentpos,
        {丙酸盐}/{Propionate}/62/{15-98}/正常/{参与糖异生过程,调节食欲和能量代谢,具有抗炎作用。}/{}/\currentpos,
        {乙酸盐}/{Acetate}/7/{15-98}/缺乏/{最丰富的短链脂肪酸,参与脂质代谢,为周边组织提供能量。}/{}/\currentpos,
        {异戊酸盐}/{Isovaleric}/11/{15-98}/正常/{支链氨基酸代谢产物,反映蛋白质发酵状态。}/{}/\currentpos
%        {谷氨酸}/{Glutamate}/11/{15-98}/缺乏/{参与血液凝固过程,促进骨骼钙化,具有抗动脉粥样硬化的作用。}/{}/\currentpos,
%        {色氨}/{Tryptophan}/11/{15-98}/缺乏/{肠道菌群代谢芳香族氨基酸产生的代谢物,反映肠道菌群的代谢活性。}/{}/\currentpos,
%        {喹啉}/{Quinolinic}/11/{15-98}/缺乏/{参与细胞信号传导,对神经系统功能和胰岛素敏感性有重要影响。}/{}/\currentpos
    }
    {
        % 计算当前行的基础位置
        \pgfmathsetmacro{\basepos}{-2.8*\currentpos}

        % 菌种名称
        \node[cell, align=left] at (0.5,\basepos) {
            \small\textbf{\item}\\[-0.2em]
            {\color{lightgray}\small\enitem}
        };

        % 正常范围
        \node[reference] at (6,\basepos) {\footnotesize\range};

        % 进度条相关
        \pgfmathsetmacro{\barypos}{\basepos-\valuebarspace+0.1}
        \def\barstart{10.25}

        % 进度条背景
        \fill[gray!10, rounded corners=2] (\barstart,\barypos)
            rectangle (\barstart+\barwidth,\barypos+\barheight);

        % 检测丰度值
        \node[value] at (11, {\basepos-\valuebarspace+0.6}) {\footnotesize\value};

        % 解析范围并计算进度条长度
        \def\parserange#1-#2\endparse{\def\minval{#1}\def\maxval{#2}}
        \expandafter\parserange\range\endparse

        % 计算进度条长度和颜色
        \pgfmathsetmacro{\progress}{min(\value/\maxval, 1.0)}
        \pgfmathparse{\value > \maxval ? "customred" : (\value < \minval ? "customred" : "green!50")}
        \let\barcolor=\pgfmathresult

        % 进度条显示
        \ifnum\pdfstrcmp{\status}{超标}=0
            \fill[customred, rounded corners=2] (\barstart,\barypos)
                rectangle (\barstart+\barwidth,\barypos+\barheight);
        \else
            \fill[\barcolor, rounded corners=2] (\barstart,\barypos)
                rectangle (\barstart+\barwidth*\progress,\barypos+\barheight);
        \fi

        % 结果评价
        \ifnum\pdfstrcmp{\status}{超标}=0
            \node[value, text=customRed] at (16,\basepos) {\footnotesize\textbf{\status}};
        \else
            \node[value, text=customGreen] at (16,\basepos) {\footnotesize\textbf{\status}};
        \fi

        % 添加卡片
        \pgfmathsetmacro{\cardypos}{\basepos-0.5}
        \begin{scope}[shift={(0,\cardypos)}]
            % 卡片背景
            \pgfmathsetmacro{\cardheight}{
                \ifnum\pdfstrcmp{\status}{超标}=0
                    1.0  % 两行内容时的高度
                \else
                    0.6  % 一行内容时的高度
                \fi
            }

            \fill[rounded corners=5pt, customTeal!5, draw=gray!5]
                (0.3,-\cardheight) rectangle (17.3,0);

            % 菌群简介图标和内容
            \node[anchor=west] at (0.5,-0.3) {
                \textbf{\color{gray!90}\footnotesize \textcolor{customTeal}{\faInfoCircle}}
            };
            \node[anchor=west, text width=16cm] at (0.9,-0.3) {
                {\small\color{gray}\footnotesize \intro}
            };

            % 异常解读标题和内容
            \ifnum\pdfstrcmp{\status}{超标}=0
                \node[anchor=west] at (0.55,-0.8) {
                    \textbf{\color{customRed}\footnotesize \textcolor{customRed}{\faLightbulb}}
                };
                \node[anchor=west, text width=16cm] at (0.9,-0.8) {
                    {\small\color{gray}\footnotesize \suggestion}
                };
            \fi
        \end{scope}

        % 分割线
        \pgfmathsetmacro{\linepos}{
            \ifnum\pdfstrcmp{\status}{超标}=0
                \basepos-1.7  % 超标时的分割线位置
            \else
                \basepos-1.3  % 正常时的分割线位置
            \fi
        }
        \draw[gray!20] (0.2,\linepos) -- (\cardwidth-0.2,\linepos);

        % 根据当前行的状态计算下一行的位置增量
        \ifnum\pdfstrcmp{\status}{超标}=0
            \pgfmathsetmacro{\increment}{0.85}  % 超标行(两行内容)需要更大的增量
        \else
            \pgfmathsetmacro{\increment}{0.7}  % 正常行(一行内容)使用较小的增量
        \fi

        % 更新位置计数器
        \pgfmathsetmacro{\nextpos}{\currentpos+\increment}
        \xdef\currentpos{\nextpos}
    }

\end{tikzpicture}
\end{center}

\newpage

\begin{tcolorbox}[
    enhanced,
    colback=white,
    colframe=white,
    arc=2mm,
    boxrule=0pt,
    width=\textwidth,
    left=15pt,
    right=15pt,
    top=10pt,
    bottom=10pt,
    drop shadow={
        opacity=0.2,
        color=customTeal
    },
    borderline west={5pt}{0pt}{customTeal}
]
\textcolor{customTeal}{\Large\textbf{神经活性物质评估}}
\end{tcolorbox}

\vspace{0.05cm}

\begin{tcolorbox}[
    enhanced,
    colback=customTealBg,
    colframe=customTealBg,
    arc=3mm,
    boxrule=0pt,
    width=\textwidth,
    top=8pt,
    bottom=8pt
]
{\small{\color{customTeal}\faInfoCircle} 神经活性物质是一类在神经系统中发挥重要作用的化学物质,包括神经递质和激素。这些物质参与调节情绪、认知、行为和内分泌功能。通过检测这些指标,可以评估神经系统功能状态和神经内分泌平衡。\\

{\color{orange}\faExclamationTriangle} \textbf{特别注意}:
}
\end{tcolorbox}

\begin{tcolorbox}[
    enhanced,
    colback=lightpurple!10, % 卡片底色
    colframe=white,  % 边框颜色
    arc=3mm,
    boxrule=0.5pt,
    width=\textwidth,
    top=8pt,
    bottom=8pt
]
{\small{\color{lightpurple}\faQuestionCircle}\quad \textbf{这些其他代谢物指标有什么临床意义?}\\
{\color{orange!50}\faComments}\quad 其他代谢物指标的临床意义体现在以下方面:
\begin{itemize}
    \item 神经功能评估:通过神经递质水平判断神经系统活性。
    \item 内分泌状态:激素水平反映内分泌系统功能。
    \item 心理健康参考:神经递质平衡可作为评估情绪和行为的指标。
\end{itemize}
}
\end{tcolorbox}
\vspace{-0.5cm}
\begin{center}
\begin{tikzpicture}[
    font=\small,
    title/.style={font=\small\bfseries\color{white}},
    value/.style={font=\small},
    reference/.style={font=\small},
    cell/.style={anchor=west, text width=4.2cm},
    note/.style={anchor=west, text width=4.5cm, align=left}
]
    \def\cardwidth{\textwidth}
    \def\cardheight{7.9}
    \def\barheight{0.25}
    \def\barwidth{1.5}
    \def\valuebarspace{0.4}

    % 容器和标题栏背景
    \draw[rounded corners=5, fill=white, draw=gray!20]
        (0,0) rectangle (\cardwidth,-\cardheight);
    \path[fill=customTeal]
        (0,0) [rounded corners=5] -- (\cardwidth,0) --
        (\cardwidth,0.8) -- (0,0.8) -- cycle;

    % 表头
    \node[title, anchor=west] at (0.5,0.4) {\textbf{神经活性物质名称}};
    \node[title] at (6, 0.4) {\textbf{正常范围}};
    \node[title] at (11, 0.4) {\textbf{检测丰度}};
    \node[title] at (16, 0.4) {\textbf{结果评价}};

    % 初始化位置计数器
    \def\currentpos{0.25}

    % 数据行和卡片
    % 数据行和卡片
    \foreach \item/\enitem/\value/\range/\status/\intro/\suggestion/\index in {
        {$\gamma$-氨基丁酸}/{GABA}/24/{15-98}/正常/{主要的抑制性神经递质,调节神经元兴奋性,参与焦虑、睡眠等功能。}/{}/\currentpos,
        {血清素}/{5-HT}/99/{20-95}/超标/{调节情绪、睡眠和食欲的关键神经递质,影响心理健康状态。}/{}/\currentpos,
        {组胺}/{Histamine}/67/{5-95}/正常/{参与免疫应答和过敏反应,同时调节觉醒和食欲。}/{}/\currentpos,
        {DOPAC}/{DOPAC}/66/{10-95}/正常/{多巴胺的主要代谢产物,反映多巴胺能神经系统的活性。}/{}/\currentpos,
        {雌激素}/{Estrogen}/89/{10-95}/正常/{重要的性激素,影响生殖功能、骨密度和心血管健康。}/{}/\currentpos,
        {多巴胺}/{Dopamine}/87/{5-95}/正常/{与奖励、愉悦、运动功能相关的神经递质,影响动机和行为。}/{}/\currentpos
    }
    {
        % 计算当前行的基础位置
        \pgfmathsetmacro{\basepos}{-2.8*\currentpos}

        % 菌种名称
        \node[cell, align=left] at (0.5,\basepos) {
            \small\textbf{\item}\\[-0.2em]
            {\color{lightgray}\small\enitem}
        };

        % 正常范围
        \node[reference] at (6,\basepos) {\footnotesize\range};

        % 进度条相关
        \pgfmathsetmacro{\barypos}{\basepos-\valuebarspace+0.1}
        \def\barstart{10.25}

        % 进度条背景
        \fill[gray!10, rounded corners=2] (\barstart,\barypos)
            rectangle (\barstart+\barwidth,\barypos+\barheight);

        % 检测丰度值
        \node[value] at (11, {\basepos-\valuebarspace+0.6}) {\footnotesize\value};

        % 解析范围并计算进度条长度
        \def\parserange#1-#2\endparse{\def\minval{#1}\def\maxval{#2}}
        \expandafter\parserange\range\endparse

        % 计算进度条长度和颜色
        \pgfmathsetmacro{\progress}{min(\value/\maxval, 1.0)}
        \pgfmathparse{\value > \maxval ? "customred" : (\value < \minval ? "customred" : "green!50")}
        \let\barcolor=\pgfmathresult

        % 进度条显示
        \ifnum\pdfstrcmp{\status}{超标}=0
            \fill[customred, rounded corners=2] (\barstart,\barypos)
                rectangle (\barstart+\barwidth,\barypos+\barheight);
        \else
            \fill[\barcolor, rounded corners=2] (\barstart,\barypos)
                rectangle (\barstart+\barwidth*\progress,\barypos+\barheight);
        \fi

        % 结果评价
        \ifnum\pdfstrcmp{\status}{超标}=0
            \node[value, text=customRed] at (16,\basepos) {\footnotesize\textbf{\status}};
        \else
            \node[value, text=customGreen] at (16,\basepos) {\footnotesize\textbf{\status}};
        \fi

        % 添加卡片
        \pgfmathsetmacro{\cardypos}{\basepos-0.5}
        \begin{scope}[shift={(0,\cardypos)}]
            % 卡片背景
            \pgfmathsetmacro{\cardheight}{
                \ifnum\pdfstrcmp{\status}{超标}=0
                    1.0  % 两行内容时的高度
                \else
                    0.6  % 一行内容时的高度
                \fi
            }

            \fill[rounded corners=5pt, customTeal!5, draw=gray!5]
                (0.3,-\cardheight) rectangle (17.3,0);

            % 菌群简介图标和内容
            \node[anchor=west] at (0.5,-0.3) {
                \textbf{\color{gray!90}\footnotesize \textcolor{customTeal}{\faInfoCircle}}
            };
            \node[anchor=west, text width=16cm] at (0.9,-0.3) {
                {\small\color{gray}\footnotesize \intro}
            };

            % 异常解读标题和内容
            \ifnum\pdfstrcmp{\status}{超标}=0
                \node[anchor=west] at (0.55,-0.8) {
                    \textbf{\color{customRed}\footnotesize \textcolor{customRed}{\faLightbulb}}
                };
                \node[anchor=west, text width=16cm] at (0.9,-0.8) {
                    {\small\color{gray}\footnotesize \suggestion}
                };
            \fi
        \end{scope}

        % 分割线
        \pgfmathsetmacro{\linepos}{
            \ifnum\pdfstrcmp{\status}{超标}=0
                \basepos-1.7  % 超标时的分割线位置
            \else
                \basepos-1.3  % 正常时的分割线位置
            \fi
        }
        \draw[gray!20] (0.2,\linepos) -- (\cardwidth-0.2,\linepos);

        % 根据当前行的状态计算下一行的位置增量
        \ifnum\pdfstrcmp{\status}{超标}=0
            \pgfmathsetmacro{\increment}{0.85}  % 超标行(两行内容)需要更大的增量
        \else
            \pgfmathsetmacro{\increment}{0.7}  % 正常行(一行内容)使用较小的增量
        \fi

        % 更新位置计数器
        \pgfmathsetmacro{\nextpos}{\currentpos+\increment}
        \xdef\currentpos{\nextpos}
    }

\end{tikzpicture}
\end{center}

\newpage

\begin{tcolorbox}[
    enhanced,
    colback=white,
    colframe=white,
    arc=2mm,
    boxrule=0pt,
    width=\textwidth,
    left=15pt,
    right=15pt,
    top=10pt,
    bottom=10pt,
    drop shadow={
        opacity=0.2,
        color=customTeal
    },
    borderline west={5pt}{0pt}{customTeal}
]
\textcolor{customTeal}{\Large\textbf{含氮/含硫化合物评估}}
\end{tcolorbox}

\begin{tcolorbox}[
    enhanced,
    colback=customTealBg,
    colframe=customTealBg,
    arc=3mm,
    boxrule=0pt,
    width=\textwidth,
    top=8pt,
    bottom=8pt
]
{\small{\color{customTeal}\faInfoCircle} 体液免疫是机体重要的防御系统,通过产生多种免疫活性物质来识别和清除病原体。通过检测这些指标,可以评估机体的免疫防御能力和炎症状态。\\
{\color{orange}\faExclamationTriangle} \textbf{特别注意}:
}
\end{tcolorbox}

\begin{tcolorbox}[
    enhanced,
    colback=lightpurple!10, % 卡片底色
    colframe=white,  % 边框颜色
    arc=3mm,
    boxrule=0.5pt,
    width=\textwidth,
    top=8pt,
    bottom=8pt
]
{\small{\color{lightpurple}\faQuestionCircle}\quad \textbf{这些其他代谢物指标有什么临床意义?}\\
{\color{orange!50}\faComments}\quad 其他代谢物指标的临床意义体现在以下方面:
\begin{itemize}
    \item 免疫功能评估:通过IgA、IgD、IgM水平判断免疫系统状态。
    \item 营养状况评价:白蛋白水平反映机体营养状况。
    \item 肝功能参考:白蛋白可作为评估肝脏合成功能的指标。
\end{itemize}
}
\end{tcolorbox}
\vspace{-0.5cm}
\begin{center}
\begin{tikzpicture}[
    font=\small,
    title/.style={font=\small\bfseries\color{white}},
    value/.style={font=\small},
    reference/.style={font=\small},
    cell/.style={anchor=west, text width=4.2cm},
    note/.style={anchor=west, text width=4.5cm, align=left}
]
    \def\cardwidth{\textwidth}
    \def\cardheight{7.9}
    \def\barheight{0.25}
    \def\barwidth{1.5}
    \def\valuebarspace{0.4}

    % 容器和标题栏背景
    \draw[rounded corners=5, fill=white, draw=gray!20]
        (0,0) rectangle (\cardwidth,-\cardheight);
    \path[fill=customTeal]
        (0,0) [rounded corners=5] -- (\cardwidth,0) --
        (\cardwidth,0.8) -- (0,0.8) -- cycle;

    % 表头
    \node[title, anchor=west] at (0.5,0.4) {\textbf{菌种名称}};
    \node[title] at (6, 0.4) {\textbf{正常范围}};
    \node[title] at (11, 0.4) {\textbf{检测丰度}};
    \node[title] at (16, 0.4) {\textbf{结果评价}};

    % 初始化位置计数器
    \def\currentpos{0.25}

    % 数据行和卡片
    % 数据行和卡片
    \foreach \item/\enitem/\value/\range/\status/\intro/\suggestion/\index in {
        {丁酸盐}/{Butyrate}/24/{15-98}/正常/{参与血液凝固过程,促进骨骼钙化,具有抗动脉粥样硬化的作用。}/{}/\currentpos,
        {丙酸盐}/{Propionate}/62/{15-98}/正常/{肠道菌群代谢芳香族氨基酸产生的代谢物,反映肠道菌群的代谢活性。}/{}/\currentpos,
        {乙酸盐}/{Acetate}/7/{15-98}/缺乏/{参与细胞信号传导,对神经系统功能和胰岛素敏感性有重要影响。}/{}/\currentpos,
        {异戊酸盐}/{Isovaleric}/11/{15-98}/缺乏/{蛋白质代谢产物,其水平反映肠道菌群的蛋白质降解能力。}/{}/\currentpos,
        {谷氨酸}/{Glutamate}/11/{15-98}/缺乏/{参与血液凝固过程,促进骨骼钙化,具有抗动脉粥样硬化的作用。}/{}/\currentpos,
        {色氨}/{Tryptophan}/11/{15-98}/缺乏/{肠道菌群代谢芳香族氨基酸产生的代谢物,反映肠道菌群的代谢活性。}/{}/\currentpos,
        {喹啉}/{Quinolinic}/11/{15-98}/缺乏/{参与细胞信号传导,对神经系统功能和胰岛素敏感性有重要影响。}/{}/\currentpos
    }
    {
        % 计算当前行的基础位置
        \pgfmathsetmacro{\basepos}{-2.8*\currentpos}

        % 菌种名称
        \node[cell, align=left] at (0.5,\basepos) {
            \small\textbf{\item}\\[-0.2em]
            {\color{lightgray}\small\enitem}
        };

        % 正常范围
        \node[reference] at (6,\basepos) {\footnotesize\range};

        % 进度条相关
        \pgfmathsetmacro{\barypos}{\basepos-\valuebarspace+0.1}
        \def\barstart{10.25}

        % 进度条背景
        \fill[gray!10, rounded corners=2] (\barstart,\barypos)
            rectangle (\barstart+\barwidth,\barypos+\barheight);

        % 检测丰度值
        \node[value] at (11, {\basepos-\valuebarspace+0.6}) {\footnotesize\value};

        % 解析范围并计算进度条长度
        \def\parserange#1-#2\endparse{\def\minval{#1}\def\maxval{#2}}
        \expandafter\parserange\range\endparse

        % 计算进度条长度和颜色
        \pgfmathsetmacro{\progress}{min(\value/\maxval, 1.0)}
        \pgfmathparse{\value > \maxval ? "customred" : (\value < \minval ? "customred" : "green!50")}
        \let\barcolor=\pgfmathresult

        % 进度条显示
        \ifnum\pdfstrcmp{\status}{超标}=0
            \fill[customred, rounded corners=2] (\barstart,\barypos)
                rectangle (\barstart+\barwidth,\barypos+\barheight);
        \else
            \fill[\barcolor, rounded corners=2] (\barstart,\barypos)
                rectangle (\barstart+\barwidth*\progress,\barypos+\barheight);
        \fi

        % 结果评价
        \ifnum\pdfstrcmp{\status}{超标}=0
            \node[value, text=customRed] at (16,\basepos) {\footnotesize\textbf{\status}};
        \else
            \node[value, text=customGreen] at (16,\basepos) {\footnotesize\textbf{\status}};
        \fi

        % 添加卡片
        \pgfmathsetmacro{\cardypos}{\basepos-0.5}
        \begin{scope}[shift={(0,\cardypos)}]
            % 卡片背景
            \pgfmathsetmacro{\cardheight}{
                \ifnum\pdfstrcmp{\status}{超标}=0
                    1.0  % 两行内容时的高度
                \else
                    0.6  % 一行内容时的高度
                \fi
            }

            \fill[rounded corners=5pt, customTeal!5, draw=gray!5]
                (0.3,-\cardheight) rectangle (17.3,0);

            % 菌群简介图标和内容
            \node[anchor=west] at (0.5,-0.3) {
                \textbf{\color{gray!90}\footnotesize \textcolor{customTeal}{\faInfoCircle}}
            };
            \node[anchor=west, text width=16cm] at (0.9,-0.3) {
                {\small\color{gray}\footnotesize \intro}
            };

            % 异常解读标题和内容
            \ifnum\pdfstrcmp{\status}{超标}=0
                \node[anchor=west] at (0.55,-0.8) {
                    \textbf{\color{customRed}\footnotesize \textcolor{customRed}{\faLightbulb}}
                };
                \node[anchor=west, text width=16cm] at (0.9,-0.8) {
                    {\small\color{gray}\footnotesize \suggestion}
                };
            \fi
        \end{scope}

        % 分割线
        \pgfmathsetmacro{\linepos}{
            \ifnum\pdfstrcmp{\status}{超标}=0
                \basepos-1.7  % 超标时的分割线位置
            \else
                \basepos-1.3  % 正常时的分割线位置
            \fi
        }
        \draw[gray!20] (0.2,\linepos) -- (\cardwidth-0.2,\linepos);

        % 根据当前行的状态计算下一行的位置增量
        \ifnum\pdfstrcmp{\status}{超标}=0
            \pgfmathsetmacro{\increment}{0.85}  % 超标行(两行内容)需要更大的增量
        \else
            \pgfmathsetmacro{\increment}{0.7}  % 正常行(一行内容)使用较小的增量
        \fi

        % 更新位置计数器
        \pgfmathsetmacro{\nextpos}{\currentpos+\increment}
        \xdef\currentpos{\nextpos}
    }

\end{tikzpicture}
\end{center}

\newpage

\begin{tcolorbox}[
    enhanced,
    colback=white,
    colframe=white,
    arc=2mm,
    boxrule=0pt,
    width=\textwidth,
    left=15pt,
    right=15pt,
    top=10pt,
    bottom=10pt,
    drop shadow={
        opacity=0.2,
        color=customTeal
    },
    borderline west={5pt}{0pt}{customTeal}
]
\textcolor{customTeal}{\Large\textbf{其他代谢物评估}}
\end{tcolorbox}

\vspace{0.05cm}

\begin{tcolorbox}[
    enhanced,
    colback=customTealBg,
    colframe=customTealBg,
    arc=3mm,
    boxrule=0pt,
    width=\textwidth,
    top=8pt,
    bottom=8pt
]
{\small{\color{customTeal}\faInfoCircle} 其他代谢物包含了以下的这些指标。\\
{\color{orange}\faExclamationTriangle} \textbf{特别注意}:
}
\end{tcolorbox}

\begin{tcolorbox}[
    enhanced,
    colback=lightpurple!10, % 卡片底色
    colframe=white,  % 边框颜色
    arc=3mm,
    boxrule=0.5pt,
    width=\textwidth,
    top=8pt,
    bottom=8pt
]
{\small{\color{lightpurple}\faQuestionCircle}\quad \textbf{这些其他代谢物指标有什么临床意义?}\\
{\color{orange!50}\faComments}\quad 其他代谢物指标的临床意义体现在以下方面:
\begin{itemize}
    \item 微生物代谢评估:通过代谢物水平判断肠道菌群的代谢活性。
    \item 营养状况评价:维生素K2水平反映微生物源性维生素的合成能力。
    \item 代谢功能参考:代谢物组成可作为评估肠道健康的重要指标。
\end{itemize}
}
\end{tcolorbox}
\vspace{-0.5cm}
\begin{center}
\begin{tikzpicture}[
    font=\small,
    title/.style={font=\small\bfseries\color{white}},
    value/.style={font=\small},
    reference/.style={font=\small},
    cell/.style={anchor=west, text width=4.2cm},
    note/.style={anchor=west, text width=4.5cm, align=left}
]
    \def\cardwidth{\textwidth}
    \def\cardheight{7.9}
    \def\barheight{0.25}
    \def\barwidth{1.5}
    \def\valuebarspace{0.4}

    % 容器和标题栏背景
    \draw[rounded corners=5, fill=white, draw=gray!20]
        (0,0) rectangle (\cardwidth,-\cardheight);
    \path[fill=customTeal]
        (0,0) [rounded corners=5] -- (\cardwidth,0) --
        (\cardwidth,0.8) -- (0,0.8) -- cycle;

    % 表头
    \node[title, anchor=west] at (0.5,0.4) {\textbf{菌种名称}};
    \node[title] at (6, 0.4) {\textbf{正常范围}};
    \node[title] at (11, 0.4) {\textbf{检测丰度}};
    \node[title] at (16, 0.4) {\textbf{结果评价}};

    % 初始化位置计数器
    \def\currentpos{0.25}

    % 数据行和卡片
    % 数据行和卡片
    \foreach \item/\enitem/\value/\range/\status/\intro/\suggestion/\index in {
        {维生素K2}/{Vitamin K2}/9/{5-95}/正常/{参与血液凝固过程,促进骨骼钙化,具有抗动脉粥样硬化的作用。}/{}/\currentpos,
        {对甲酚}/{p-Cresol}/43/{0-83}/正常/{肠道菌群代谢芳香族氨基酸产生的代谢物,反映肠道菌群的代谢活性。}/{}/\currentpos,
        {肌醇}/{Inositol}/35/{5-95}/正常/{参与细胞信号传导,对神经系统功能和胰岛素敏感性有重要影响。}/{}/\currentpos,
        {苯酚}/{Phenol}/5/{5-85}/正常/{蛋白质代谢产物,其水平反映肠道菌群的蛋白质降解能力。}/{}/\currentpos
    }
    {
        % 计算当前行的基础位置
        \pgfmathsetmacro{\basepos}{-2.8*\currentpos}

        % 菌种名称
        \node[cell, align=left] at (0.5,\basepos) {
            \small\textbf{\item}\\[-0.2em]
            {\color{lightgray}\small\enitem}
        };

        % 正常范围
        \node[reference] at (6,\basepos) {\footnotesize\range};

        % 进度条相关
        \pgfmathsetmacro{\barypos}{\basepos-\valuebarspace+0.1}
        \def\barstart{10.25}

        % 进度条背景
        \fill[gray!10, rounded corners=2] (\barstart,\barypos)
            rectangle (\barstart+\barwidth,\barypos+\barheight);

        % 检测丰度值
        \node[value] at (11, {\basepos-\valuebarspace+0.6}) {\footnotesize\value};

        % 解析范围并计算进度条长度
        \def\parserange#1-#2\endparse{\def\minval{#1}\def\maxval{#2}}
        \expandafter\parserange\range\endparse

        % 计算进度条长度和颜色
        \pgfmathsetmacro{\progress}{min(\value/\maxval, 1.0)}
        \pgfmathparse{\value > \maxval ? "customred" : (\value < \minval ? "customred" : "green!50")}
        \let\barcolor=\pgfmathresult

        % 进度条显示
        \ifnum\pdfstrcmp{\status}{超标}=0
            \fill[customred, rounded corners=2] (\barstart,\barypos)
                rectangle (\barstart+\barwidth,\barypos+\barheight);
        \else
            \fill[\barcolor, rounded corners=2] (\barstart,\barypos)
                rectangle (\barstart+\barwidth*\progress,\barypos+\barheight);
        \fi

        % 结果评价
        \ifnum\pdfstrcmp{\status}{超标}=0
            \node[value, text=customRed] at (16,\basepos) {\footnotesize\textbf{\status}};
        \else
            \node[value, text=customGreen] at (16,\basepos) {\footnotesize\textbf{\status}};
        \fi

        % 添加卡片
        \pgfmathsetmacro{\cardypos}{\basepos-0.5}
        \begin{scope}[shift={(0,\cardypos)}]
            % 卡片背景
            \pgfmathsetmacro{\cardheight}{
                \ifnum\pdfstrcmp{\status}{超标}=0
                    1.0  % 两行内容时的高度
                \else
                    0.6  % 一行内容时的高度
                \fi
            }

            \fill[rounded corners=5pt, customTeal!5, draw=gray!5]
                (0.3,-\cardheight) rectangle (17.3,0);

            % 菌群简介图标和内容
            \node[anchor=west] at (0.5,-0.3) {
                \textbf{\color{gray!90}\footnotesize \textcolor{customTeal}{\faInfoCircle}}
            };
            \node[anchor=west, text width=16cm] at (0.9,-0.3) {
                {\small\color{gray}\footnotesize \intro}
            };

            % 异常解读标题和内容
            \ifnum\pdfstrcmp{\status}{超标}=0
                \node[anchor=west] at (0.55,-0.8) {
                    \textbf{\color{customRed}\footnotesize \textcolor{customRed}{\faLightbulb}}
                };
                \node[anchor=west, text width=16cm] at (0.9,-0.8) {
                    {\small\color{gray}\footnotesize \suggestion}
                };
            \fi
        \end{scope}

        % 分割线
        \pgfmathsetmacro{\linepos}{
            \ifnum\pdfstrcmp{\status}{超标}=0
                \basepos-1.7  % 超标时的分割线位置
            \else
                \basepos-1.3  % 正常时的分割线位置
            \fi
        }
        \draw[gray!20] (0.2,\linepos) -- (\cardwidth-0.2,\linepos);

        % 根据当前行的状态计算下一行的位置增量
        \ifnum\pdfstrcmp{\status}{超标}=0
            \pgfmathsetmacro{\increment}{0.85}  % 超标行(两行内容)需要更大的增量
        \else
            \pgfmathsetmacro{\increment}{0.7}  % 正常行(一行内容)使用较小的增量
        \fi

        % 更新位置计数器
        \pgfmathsetmacro{\nextpos}{\currentpos+\increment}
        \xdef\currentpos{\nextpos}
    }

\end{tikzpicture}
\end{center}

\begin{tcolorbox}[
    enhanced,
    colback=white,
    colframe=gray!3,
    arc=3mm,
    boxrule=0pt,
    width=\textwidth,
    top=8pt,
    bottom=8pt
]
{\small{\textcolor{yellow!85!orange}{\faLightbulb}}\quad 维生素K2和苯酚的水平偏低需要关注:
\begin{itemize}
    \item 维生素K2降低(9,参考范围5-95):这种物质就像身体的"钙质调节员",它低了可能影响骨骼健康和血液凝固功能,就像建筑工地缺少了重要的施工监理。
    \item 苯酚降低(5,参考范围5-85):处于临界值,反映肠道菌群的蛋白质代谢活性偏弱,就像工厂的生产效率下降了。
\end{itemize}

{\textcolor{green!85!orange}{\faLightbulb}}\quad 其他指标都在正常范围:
\begin{itemize}
    \item 对甲酚(43,参考范围0-83):处于适中水平,表明肠道菌群的代谢功能稳定。
    \item 肌醇(35,参考范围5-95):维持在正常水平,说明细胞信号传导和代谢调节功能正常。
\end{itemize}
}

\end{tcolorbox}

\newpage




\newpage

\begin{tcolorbox}[
enhanced,
colback=white,
colframe=customTeal,
arc=2mm,
boxrule=1pt,
left=20pt,
right=20pt,
top=12pt,
bottom=12pt,
width=\textwidth,
fontupper=\sffamily,
overlay={
\draw[customTeal, line width=2pt]
([xshift=15pt]frame.south west) -- ([xshift=-15pt]frame.south east);
}
]
{\Large\bfseries\textcolor{customTeal}{\Huge 免疫指标评估}}
\end{tcolorbox}

\begin{tcolorbox}[
    enhanced,
    colback=customTealBg,
    colframe=customTealBg,
    arc=3mm,
    boxrule=0pt,
    width=\textwidth,
    top=8pt,
    bottom=8pt
]
{\small{\color{customTeal}\faInfoCircle} 免疫指标是评估人体免疫系统功能状态的重要参数,主要包括体液免疫、细胞免疫、炎症因子和代谢物质等多个类别,总计涵盖数十种具体指标。健康成年人的免疫系统主要由免疫球蛋白(如IgG、IgA、IgM)、补体系统(C3、C4等)、细胞因子(IL-1β、IL-6、TNF-α等)和免疫细胞(T细胞、B细胞、NK细胞等)构成,其中免疫球蛋白和免疫细胞在免疫防御中发挥核心作用,占据免疫防御功能的80\%以上。我们可以从两个关键维度来评估免疫指标:活性和特异性,这两个维度的结合可以帮助我们更全面地理解不同免疫指标在人体免疫系统中的功能地位和分布特点。\\

{\color{orange}\faExclamationTriangle} \textbf{特别注意}:这些免疫指标并非通过直接检测获得,而是通过肠道菌群数据进行评估推算,结果仅供参考!
}
\end{tcolorbox}

% 定义主色调 - 降低饱和度和亮度的版本
\definecolor{modernblue}{RGB}{75, 134, 230}     % 更柔和的蓝色
\definecolor{modernred}{RGB}{225, 95, 95}       % 更柔和的红色
\definecolor{moderngreen}{RGB}{76, 175, 120}    % 更柔和的绿色
\definecolor{modernpurple}{RGB}{165, 110, 220}  % 更柔和的紫色
\definecolor{modernorange}{RGB}{230, 160, 50}   % 更柔和的橙色

% 定义浅色背景 - 降低饱和度的版本
\definecolor{lightblue}{RGB}{242, 247, 255}     % 更柔和的浅蓝色
\definecolor{lightred}{RGB}{255, 245, 245}      % 更柔和的浅红色
\definecolor{lightgreen}{RGB}{244, 252, 247}    % 更柔和的浅绿色
%\definecolor{lightpurple}{RGB}{252, 247, 255}   % 更柔和的浅紫色
\definecolor{lightorange}{RGB}{255, 249, 242}   % 更柔和的浅橙色

% 定义左侧标题框
\newtcolorbox{titlebox}[2][]{
    enhanced,
    colback=#2,
    colframe=#2,
    arc=10pt,
    boxrule=0pt,
    width=0.25\textwidth,
    height=3cm,
    valign=center,
    halign=left,
    left=1em,
    fontupper=\color{white}\Large\bfseries
}

% 定义右侧内容框
\newtcolorbox{contentbox}[2][]{
    enhanced,
    colback=white,
    colframe=#2!20,
    arc=10pt,
    boxrule=1pt,
    width=0.72\textwidth,
    height=3cm,
    valign=top,
    left=1em
}

%\begin{center}
%    \Huge\textbf{免疫指标分类体系}
%\end{center}

\vspace{0.1cm}

% 体液免疫
\noindent
\begin{minipage}{\textwidth}
\parbox{0.25\textwidth}{%
\begin{titlebox}{modernblue}
    \faIcon{vial}\quad 体液免疫
\end{titlebox}}%
\hfill
\parbox{0.72\textwidth}{%
\begin{contentbox}{modernblue}
    体液免疫是机体重要的防御系统,主要由免疫球蛋白(IgA、IgG、IgM等)和补体系统(C3、C4等)构成。其中,免疫球蛋白能特异性识别和中和病原体,而补体系统则通过级联反应增强免疫应答,共同构成体液免疫防御网络。
\end{contentbox}}
\end{minipage}

\vspace{0.1cm}

% 炎症指标
\noindent
\begin{minipage}{\textwidth}
\parbox{0.25\textwidth}{
\begin{titlebox}{modernred}
    \faIcon{fire} \quad 炎症感染
\end{titlebox}}
\hfill
\parbox{0.72\textwidth}{
\begin{contentbox}{modernred}
    炎症指标是反映机体炎症状态的重要生物标志物,主要包括急性时相蛋白(如CRP、SAA)、促炎因子(IL-1β、IL-6、TNF-α等)和抗炎因子(IL-10、TGF-β等)。这些指标能够及时反映机体的炎症水平和免疫状态,是疾病诊断和预后评估的重要依据。
\end{contentbox}}
\end{minipage}

\vspace{0.1cm}

% 代谢物质
\noindent\begin{minipage}{\textwidth}
\parbox{0.25\textwidth}{
\begin{titlebox}{moderngreen}
    \faIcon{flask} \quad 菌群代谢
\end{titlebox}}
\hfill
\parbox{0.72\textwidth}{
\begin{contentbox}{moderngreen}
    代谢物质是肠道菌群与宿主互作的关键媒介,主要包括短链脂肪酸(乙酸盐、丙酸盐、丁酸盐)、胆汁酸代谢物、色氨酸代谢物和神经递质(GABA、5-HT等)。这些代谢产物不仅参与能量代谢,还具有调节免疫、影响神经功能等多重生物学作用。
\end{contentbox}}
\end{minipage}

\vspace{0.1cm}

% 细胞因子
\noindent\begin{minipage}{\textwidth}
\parbox{0.25\textwidth}{
\begin{titlebox}{modernpurple}
    \faIcon{dna} \quad 细胞因子
\end{titlebox}}
\hfill
\parbox{0.72\textwidth}{
\begin{contentbox}{modernpurple}
    细胞因子是一类调节免疫和炎症反应的小分子蛋白,主要包括白介素家族(IL-1β、IL-6、IL-10等)、肿瘤坏死因子(TNF-α)和转化生长因子(TGF-β)等。它们作为免疫细胞间的信使分子,在免疫应答、炎症调控和组织修复中发挥关键作用。
\end{contentbox}}
\end{minipage}

\vspace{0.1cm}

% 免疫肿瘤
\noindent\begin{minipage}{\textwidth}
\parbox{0.25\textwidth}{
\begin{titlebox}{modernorange}
    \faIcon{microscope} \quad 免疫肿瘤
\end{titlebox}}
\hfill
\parbox{0.72\textwidth}{
\begin{contentbox}{modernorange}
    免疫系统是机体抵抗肿瘤的重要防线,通过免疫监视、免疫编辑和免疫清除等机制识别和消灭肿瘤细胞。肿瘤免疫涉及天然免疫和适应性免疫多个组分,包括NK细胞、T细胞、树突状细胞等免疫细胞以及细胞因子网络。
\end{contentbox}}
\end{minipage}

\newpage

\begin{tcolorbox}[
    enhanced,
    colback=white,
    colframe=white,
    arc=2mm,
    boxrule=0pt,
    width=\textwidth,
    left=15pt,
    right=15pt,
    top=10pt,
    bottom=10pt,
    drop shadow={
        opacity=0.2,
        color=customTeal
    },
    borderline west={5pt}{0pt}{customTeal}
]
\textcolor{customTeal}{\Large\textbf{体液免疫指标评估}}
\end{tcolorbox}

\vspace{0.05cm}

\begin{tcolorbox}[
    enhanced,
    colback=customTealBg,
    colframe=customTealBg,
    arc=3mm,
    boxrule=0pt,
    width=\textwidth,
    top=8pt,
    bottom=8pt
]
{\small{\color{customTeal}\faInfoCircle} 体液免疫是机体重要的防御系统,通过产生多种免疫活性物质来识别和清除病原体。通过检测这些指标,可以评估机体的免疫防御能力和炎症状态。\\
{\color{orange}\faExclamationTriangle} \textbf{特别注意}:
}
\end{tcolorbox}

\begin{tcolorbox}[
    enhanced,
    colback=lightpurple!10, % 卡片底色
    colframe=white,  % 边框颜色
    arc=3mm,
    boxrule=0.5pt,
    width=\textwidth,
    top=8pt,
    bottom=8pt
]
{\small{\color{lightpurple}\faQuestionCircle}\quad \textbf{这些炎症指标有什么临床意义?}\\
{\color{orange!50}\faComments}\quad 炎症指标的临床意义体现在以下方面:
\begin{itemize}
    \item 免疫功能评估:通过IgA、IgD、IgM水平判断免疫系统状态
    \item 营养状况评价:白蛋白水平反映机体营养状况
    \item 肝功能参考:白蛋白可作为评估肝脏合成功能的指标
\end{itemize}
}
\end{tcolorbox}

\vspace{-0.5cm}
\begin{center}
\begin{tikzpicture}[
    font=\small,
    title/.style={font=\small\bfseries\color{white}},
    value/.style={font=\small},
    reference/.style={font=\small},
    cell/.style={anchor=west, text width=4.2cm},
    note/.style={anchor=west, text width=4.5cm, align=left}
]
    \def\cardwidth{\textwidth}
    \def\cardheight{7.9}
    \def\barheight{0.25}
    \def\barwidth{1.5}
    \def\valuebarspace{0.4}

    % 容器和标题栏背景
    \draw[rounded corners=5, fill=white, draw=gray!20]
        (0,0) rectangle (\cardwidth,-\cardheight);
    \path[fill=customTeal]
        (0,0) [rounded corners=5] -- (\cardwidth,0) --
        (\cardwidth,0.8) -- (0,0.8) -- cycle;

    % 表头
    \node[title, anchor=west] at (0.5,0.4) {\textbf{菌种名称}};
    \node[title] at (6, 0.4) {\textbf{正常范围}};
    \node[title] at (11, 0.4) {\textbf{检测丰度}};
    \node[title] at (16, 0.4) {\textbf{结果评价}};

    % 初始化位置计数器
    \def\currentpos{0.25}

    % 数据行和卡片
    % 数据行和卡片
    \foreach \item/\enitem/\value/\range/\status/\intro/\suggestion/\index in {
        {免疫球蛋白A}/{IgA}/62/{15-98}/正常/{主要分布于呼吸道、消化道等黏膜表面,是抵抗病原体入侵的第一道防线。}/{}/\currentpos,
        {免疫球蛋白D}/{IgD}/62/{15-98}/正常/{存在于 B 淋巴细胞表面,在免疫系统的调节和抗原识别中发挥作用。}/{}/\currentpos,
        {免疫球蛋白M}/{IgM}/7/{15-98}/缺乏/{是机体产生的第一类抗体,在初次免疫应答中起关键作用。}/{}/\currentpos,
        {白蛋白}/{Albumin}/11/{15-98}/缺乏/{由肝脏合成的重要蛋白质,维持血浆渗透压,反映营养状况和肝功能。}/{}/\currentpos
    }
    {
        % 计算当前行的基础位置
        \pgfmathsetmacro{\basepos}{-2.8*\currentpos}

        % 菌种名称
        \node[cell, align=left] at (0.5,\basepos) {
            \small\textbf{\item}\\[-0.2em]
            {\color{gray}\small\enitem}
        };

        % 正常范围
        \node[reference] at (6,\basepos) {\footnotesize\range};

        % 进度条相关
        \pgfmathsetmacro{\barypos}{\basepos-\valuebarspace+0.1}
        \def\barstart{10.25}

        % 进度条背景
        \fill[gray!10, rounded corners=2] (\barstart,\barypos)
            rectangle (\barstart+\barwidth,\barypos+\barheight);

        % 检测丰度值
        \node[value] at (11, {\basepos-\valuebarspace+0.6}) {\footnotesize\value};

        % 解析范围并计算进度条长度
        \def\parserange#1-#2\endparse{\def\minval{#1}\def\maxval{#2}}
        \expandafter\parserange\range\endparse

        % 计算进度条长度和颜色
        \pgfmathsetmacro{\progress}{min(\value/\maxval, 1.0)}
        \pgfmathparse{\value > \maxval ? "customred" : (\value < \minval ? "customred" : "green!50")}
        \let\barcolor=\pgfmathresult

        % 进度条显示
        \ifnum\pdfstrcmp{\status}{超标}=0
            \fill[customred, rounded corners=2] (\barstart,\barypos)
                rectangle (\barstart+\barwidth,\barypos+\barheight);
        \else
            \fill[\barcolor, rounded corners=2] (\barstart,\barypos)
                rectangle (\barstart+\barwidth*\progress,\barypos+\barheight);
        \fi

        % 结果评价
        \ifnum\pdfstrcmp{\status}{超标}=0
            \node[value, text=customRed] at (16,\basepos) {\footnotesize\textbf{\status}};
        \else
            \node[value, text=customGreen] at (16,\basepos) {\footnotesize\textbf{\status}};
        \fi

        % 添加卡片
        \pgfmathsetmacro{\cardypos}{\basepos-0.5}
        \begin{scope}[shift={(0,\cardypos)}]
            % 卡片背景
            \pgfmathsetmacro{\cardheight}{
                \ifnum\pdfstrcmp{\status}{超标}=0
                    1.0  % 两行内容时的高度
                \else
                    0.6  % 一行内容时的高度
                \fi
            }

            \fill[rounded corners=5pt, customTeal!5, draw=gray!5]
                (0.3,-\cardheight) rectangle (17.3,0);

            % 菌群简介图标和内容
            \node[anchor=west] at (0.5,-0.3) {
                \textbf{\color{gray!90}\footnotesize \textcolor{customTeal}{\faInfoCircle}}
            };
            \node[anchor=west, text width=16cm] at (0.9,-0.3) {
                {\small\color{gray}\footnotesize \intro}
            };

            % 异常解读标题和内容
            \ifnum\pdfstrcmp{\status}{超标}=0
                \node[anchor=west] at (0.55,-0.8) {
                    \textbf{\color{customRed}\footnotesize \textcolor{customRed}{\faLightbulb}}
                };
                \node[anchor=west, text width=16cm] at (0.9,-0.8) {
                    {\small\color{gray}\footnotesize \suggestion}
                };
            \fi
        \end{scope}

        % 分割线
        \pgfmathsetmacro{\linepos}{
            \ifnum\pdfstrcmp{\status}{超标}=0
                \basepos-1.7  % 超标时的分割线位置
            \else
                \basepos-1.3  % 正常时的分割线位置
            \fi
        }
        \draw[gray!20] (0.2,\linepos) -- (\cardwidth-0.2,\linepos);

        % 根据当前行的状态计算下一行的位置增量
        \ifnum\pdfstrcmp{\status}{超标}=0
            \pgfmathsetmacro{\increment}{0.85}  % 超标行(两行内容)需要更大的增量
        \else
            \pgfmathsetmacro{\increment}{0.7}  % 正常行(一行内容)使用较小的增量
        \fi

        % 更新位置计数器
        \pgfmathsetmacro{\nextpos}{\currentpos+\increment}
        \xdef\currentpos{\nextpos}
    }

\end{tikzpicture}
\end{center}

\begin{tcolorbox}[
    enhanced,
    colback=gray!3,
    colframe=gray!3,
    arc=3mm,
    boxrule=0pt,
    width=\textwidth,
    top=8pt,
    bottom=8pt
]
{\small{\textcolor{yellow!85!orange}{\faLightbulb}}\quad 免疫球蛋白M(IgM)和白蛋白是人体免疫系统中重要的防御物质,它们的水平偏低需要关注:
\begin{itemize}
    \item IgM降低:这种物质就像身体的“快速反应部队”,它低了会让身体对新入侵者的第一道防线变弱,就像城市失去了预警系统,容易让敌人突破防线。
    \item 白蛋白降低:可以理解为身体的“营养储备库”减少了,不利于维持机体正常功能,就像仓库里的物资不足,影响了整体供应。
\end{itemize}
{\textcolor{green!85!orange}{\faLightbulb}}\quad 其他指标都在正常范围:IgA(黏膜防护者)和IgD(免疫调节者)都处于正常水平,说明目前身体的基础防御能力保持稳定。
}

\end{tcolorbox}

\newpage



\newpage

\begin{tcolorbox}[
    enhanced,
    colback=white,
    colframe=white,
    arc=2mm,
    boxrule=0pt,
    width=\textwidth,
    left=15pt,
    right=15pt,
    top=10pt,
    bottom=10pt,
    drop shadow={
        opacity=0.2,
        color=customTeal
    },
    borderline west={5pt}{0pt}{customTeal}
]
\textcolor{customTeal}{\Large\textbf{炎症指标评估}}
\end{tcolorbox}

\vspace{0.05cm}

\begin{tcolorbox}[
    enhanced,
    colback=customTealBg,
    colframe=customTealBg,
    arc=3mm,
    boxrule=0pt,
    width=\textwidth,
    top=8pt,
    bottom=8pt
]
{\small{\color{customTeal}\faInfoCircle} 炎症标志物是评估机体炎症状态的重要指标。通过检测不同的炎症标志物,可以了解系统的炎症程度。\\
{\color{orange}\faExclamationTriangle} \textbf{特别注意}:
}
\end{tcolorbox}

\begin{tcolorbox}[
    enhanced,
    colback=lightpurple!10, % 卡片底色
    colframe=white,  % 边框颜色
    arc=3mm,
    boxrule=0.5pt,
    width=\textwidth,
    top=8pt,
    bottom=8pt
]
{\small{\color{lightpurple}\faQuestionCircle}\quad \textbf{这些炎症指标有什么临床意义?}\\
{\color{orange!50}\faComments}\quad 炎症指标的临床意义体现在以下方面:
\begin{itemize}
    \item 感染评估:PCT和白细胞可帮助判断感染性疾病
    \item 心血管风险:hsCRP用于评估心脑血管疾病风险
    \item 肠道炎症:FC和EPX可评估肠道炎症的类型和程度
    \item 治疗监测:这些指标可用于评估治疗效果和预测疾病预后
\end{itemize}
}
\end{tcolorbox}

\vspace{-0.5cm}
\begin{center}
\begin{tikzpicture}[
    font=\small,
    title/.style={font=\small\bfseries\color{white}},
    value/.style={font=\small},
    reference/.style={font=\small},
    cell/.style={anchor=west, text width=4.2cm},
    note/.style={anchor=west, text width=4.5cm, align=left}
]
    \def\cardwidth{\textwidth}
    \def\cardheight{9.85}
    \def\barheight{0.25}
    \def\barwidth{1.5}
    \def\valuebarspace{0.4}

    % 容器和标题栏背景
    \draw[rounded corners=5, fill=white, draw=gray!20]
        (0,0) rectangle (\cardwidth,-\cardheight);
    \path[fill=customTeal]
        (0,0) [rounded corners=5] -- (\cardwidth,0) --
        (\cardwidth,0.8) -- (0,0.8) -- cycle;

    % 表头
    \node[title, anchor=west] at (0.5,0.4) {\textbf{菌种名称}};
    \node[title] at (6, 0.4) {\textbf{正常范围}};
    \node[title] at (11, 0.4) {\textbf{检测丰度}};
    \node[title] at (16, 0.4) {\textbf{结果评价}};

    % 初始化位置计数器
    \def\currentpos{0.25}

    % 数据行和卡片
    % 数据行和卡片
    \foreach \item/\enitem/\value/\range/\status/\intro/\suggestion/\index in {
        {白细胞总数及分类}/{IgA}/7.35/{3.5-9.5}/正常/{机体抵抗感染的重要免疫细胞,数量变化可反映炎症感染状态。}/{}/\currentpos,
        {降钙素原}/{PCT}/0/{0-0}/正常/{细菌感染的特异性标志物,对全身性感染和脓毒症具有重要诊断价值。}/{}/\currentpos,
        {高敏C反应蛋白}/{hsCRP}/4.48/{0.068-8.2}/正常/{心脑血管系统炎症的敏感指标,可预测心血管疾病风险。}/{}/\currentpos,
        {类便钙蛋白}/{FC}/1/{0-85}/正常/{来源于中性粒细胞,是炎症性肠病的特异性标志物。}/{}/\currentpos,
        {粪便嗜酸性粒细胞蛋白X}/{EPX}/1/{0-85}/正常/{反映肠道嗜酸性粒细胞介导的炎症反应,与食物过敏和肠道炎症相关。}/{}/\currentpos
    }
    {
        % 计算当前行的基础位置
        \pgfmathsetmacro{\basepos}{-2.8*\currentpos}

        % 菌种名称
        \node[cell, align=left] at (0.5,\basepos) {
            \small\textbf{\item}\\[-0.2em]
            {\color{gray}\small\enitem}
        };

        % 正常范围
        \node[reference] at (6,\basepos) {\footnotesize\range};

        % 进度条相关
        \pgfmathsetmacro{\barypos}{\basepos-\valuebarspace+0.1}
        \def\barstart{10.25}

        % 进度条背景
        \fill[gray!10, rounded corners=2] (\barstart,\barypos)
            rectangle (\barstart+\barwidth,\barypos+\barheight);

        % 检测丰度值
        \node[value] at (11, {\basepos-\valuebarspace+0.6}) {\footnotesize\value};

        % 解析范围并计算进度条长度
        \def\parserange#1-#2\endparse{\def\minval{#1}\def\maxval{#2}}
        \expandafter\parserange\range\endparse

        % 计算进度条长度和颜色
        \pgfmathsetmacro{\progress}{min(\value/\maxval, 1.0)}
        \pgfmathparse{\value > \maxval ? "customred" : (\value < \minval ? "customred" : "green!50")}
        \let\barcolor=\pgfmathresult

        % 进度条显示
        \ifnum\pdfstrcmp{\status}{超标}=0
            \fill[customred, rounded corners=2] (\barstart,\barypos)
                rectangle (\barstart+\barwidth,\barypos+\barheight);
        \else
            \fill[\barcolor, rounded corners=2] (\barstart,\barypos)
                rectangle (\barstart+\barwidth*\progress,\barypos+\barheight);
        \fi

        % 结果评价
        \ifnum\pdfstrcmp{\status}{超标}=0
            \node[value, text=customRed] at (16,\basepos) {\footnotesize\textbf{\status}};
        \else
            \node[value, text=customGreen] at (16,\basepos) {\footnotesize\textbf{\status}};
        \fi

        % 添加卡片
        \pgfmathsetmacro{\cardypos}{\basepos-0.5}
        \begin{scope}[shift={(0,\cardypos)}]
            % 卡片背景
            \pgfmathsetmacro{\cardheight}{
                \ifnum\pdfstrcmp{\status}{超标}=0
                    1.0  % 两行内容时的高度
                \else
                    0.6  % 一行内容时的高度
                \fi
            }

            \fill[rounded corners=5pt, customTeal!5, draw=gray!5]
                (0.3,-\cardheight) rectangle (17.3,0);

            % 菌群简介图标和内容
            \node[anchor=west] at (0.5,-0.3) {
                \textbf{\color{gray!90}\footnotesize \textcolor{customTeal}{\faInfoCircle}}
            };
            \node[anchor=west, text width=16cm] at (0.9,-0.3) {
                {\small\color{gray}\footnotesize \intro}
            };

            % 异常解读标题和内容
            \ifnum\pdfstrcmp{\status}{超标}=0
                \node[anchor=west] at (0.55,-0.8) {
                    \textbf{\color{customRed}\footnotesize \textcolor{customRed}{\faLightbulb}}
                };
                \node[anchor=west, text width=16cm] at (0.9,-0.8) {
                    {\small\color{gray}\footnotesize \suggestion}
                };
            \fi
        \end{scope}

        % 分割线
        \pgfmathsetmacro{\linepos}{
            \ifnum\pdfstrcmp{\status}{超标}=0
                \basepos-1.7  % 超标时的分割线位置
            \else
                \basepos-1.3  % 正常时的分割线位置
            \fi
        }
        \draw[gray!20] (0.2,\linepos) -- (\cardwidth-0.2,\linepos);

        % 根据当前行的状态计算下一行的位置增量
        \ifnum\pdfstrcmp{\status}{超标}=0
            \pgfmathsetmacro{\increment}{0.85}  % 超标行(两行内容)需要更大的增量
        \else
            \pgfmathsetmacro{\increment}{0.7}  % 正常行(一行内容)使用较小的增量
        \fi

        % 更新位置计数器
        \pgfmathsetmacro{\nextpos}{\currentpos+\increment}
        \xdef\currentpos{\nextpos}
    }

\end{tikzpicture}
\end{center}

\begin{tcolorbox}[
    enhanced,
    colback=gray!3,
    colframe=gray!3,
    arc=3mm,
    boxrule=0pt,
    width=\textwidth,
    top=8pt,
    bottom=8pt
]
{\small{\textcolor{green!85!orange}{\faBell}}\quad 指标的评估值都在正常范围:IgA(黏膜防护者)和IgD(免疫调节者)都处于正常水平,说明目前身体的基础防御能力保持稳定。
}

\end{tcolorbox}


\newpage

\begin{tcolorbox}[
    enhanced,
    colback=white,
    colframe=white,
    arc=2mm,
    boxrule=0pt,
    width=\textwidth,
    left=15pt,
    right=15pt,
    top=10pt,
    bottom=10pt,
    drop shadow={
        opacity=0.2,
        color=customTeal
    },
    borderline west={5pt}{0pt}{customTeal}
]
\textcolor{customTeal}{\Large\textbf{代谢物指标评估}}
\end{tcolorbox}

\vspace{0.05cm}

\begin{tcolorbox}[
    enhanced,
    colback=customTealBg,
    colframe=customTealBg,
    arc=3mm,
    boxrule=0pt,
    width=\textwidth,
    top=8pt,
    bottom=8pt
]
{\small{\color{customTeal}\faInfoCircle} 肠道菌群代谢物是评估肠道健康和免疫功能的重要指标。这些指标可以反应肠道微生态平衡状况和炎症反应水平。\\
{\color{orange}\faExclamationTriangle} \textbf{特别注意}:
}
\end{tcolorbox}

\begin{tcolorbox}[
    enhanced,
    colback=lightpurple!10, % 卡片底色
    colframe=white,  % 边框颜色
    arc=3mm,
    boxrule=0.5pt,
    width=\textwidth,
    top=8pt,
    bottom=8pt
]
{\small{\color{lightpurple}\faQuestionCircle}\quad \textbf{这些代谢物指标有什么临床意义?}\\
{\color{orange!50}\faComments}\quad 代谢物指标的临床意义体现在以下方面:
\begin{itemize}[]
    \item 炎症评估:通过LPS水平判断肠道炎症状态
    \item 代谢功能:胆汁酸和电解质反映代谢平衡情况
    \item 微生态健康:短链脂肪酸水平反映肠道菌群代谢活性
\end{itemize}
}
\end{tcolorbox}

\vspace{-0.5cm}

\begin{center}
\begin{tikzpicture}[
    font=\small,
    title/.style={font=\small\bfseries\color{white}},
    value/.style={font=\small},
    reference/.style={font=\small},
    cell/.style={anchor=west, text width=4.2cm},
    note/.style={anchor=west, text width=4.5cm, align=left}
]
    \def\cardwidth{\textwidth}
    \def\cardheight{9.85}
    \def\barheight{0.25}
    \def\barwidth{1.5}
    \def\valuebarspace{0.4}

    % 容器和标题栏背景
    \draw[rounded corners=5, fill=white, draw=gray!20]
        (0,0) rectangle (\cardwidth,-\cardheight);
    \path[fill=customTeal]
        (0,0) [rounded corners=5] -- (\cardwidth,0) --
        (\cardwidth,0.8) -- (0,0.8) -- cycle;

    % 表头
    \node[title, anchor=west] at (0.5,0.4) {\textbf{菌种名称}};
    \node[title] at (6, 0.4) {\textbf{正常范围}};
    \node[title] at (9.5, 0.4) {\textbf{检测丰度}};
    \node[title] at (13, 0.4) {\textbf{结果评价}};
    \node[title] at (16, 0.4) {\textbf{分类}};

    % 初始化位置计数器
    \def\currentpos{0.25}

    % 数据行和卡片
    % 数据行和卡片
    \foreach \item/\enitem/\value/\range/\status/\intro/\suggestion/\category/\index in {
        {脂多糖}/{LPS}/1/{0-85}/正常/{细菌细胞壁的重要组成部分,是评估肠道通透性和炎症反应的关键指标。}/{}/致炎代谢物/\currentpos,
        {胆汁酸}/{Bile acid}/44/{15-95}/正常/{参与脂质代谢和吸收的重要物质,同时具有调节肠道菌群的作用。}/{}/抗炎代谢物/\currentpos,
        {氢气}/{$H_{2}$}/16/{15-95}/正常/{由肠道菌群产生的重要代谢物,具有抗氧化、抗炎和调节免疫功能的作用。}/{}/抗炎代谢物/\currentpos,
        {丁酸盐}/{Butyrate}/13/{15-98}/缺乏/{重要的短链脂肪酸,为肠道细胞提供能量,具有抗炎和维护肠道屏障功能。}/{}/抗炎代谢物/\currentpos,
        {乙酸盐}/{Acetate}/13/{15-98}/缺乏/{最丰富的短链脂肪酸之一,参与能量代谢,具有抗炎作用。}/{}/抗炎代谢物/\currentpos
    }
    {
        % 计算当前行的基础位置
        \pgfmathsetmacro{\basepos}{-2.8*\currentpos}

        % 菌种名称
        \node[cell, align=left] at (0.5,\basepos) {
            \small\textbf{\item}\\[-0.2em]
            {\color{gray}\small\enitem}
        };

        % 正常范围
        \node[reference] at (6,\basepos) {\footnotesize\range};

        % 进度条相关
        \pgfmathsetmacro{\barypos}{\basepos-\valuebarspace+0.1}
        \def\barstart{8.75}

        % 进度条背景
        \fill[gray!10, rounded corners=2] (\barstart,\barypos)
            rectangle (\barstart+\barwidth,\barypos+\barheight);

        % 检测丰度值
        \node[value] at (9.5, {\basepos-\valuebarspace+0.6}) {\footnotesize\value};

        % 解析范围并计算进度条长度
        \def\parserange#1-#2\endparse{\def\minval{#1}\def\maxval{#2}}
        \expandafter\parserange\range\endparse

        % 计算进度条长度和颜色
        \pgfmathsetmacro{\progress}{min(\value/\maxval, 1.0)}
        \pgfmathparse{\value > \maxval ? "customred" : (\value < \minval ? "customred" : "green!50")}
        \let\barcolor=\pgfmathresult

        % 进度条显示
        \ifnum\pdfstrcmp{\status}{超标}=0
            \fill[customred, rounded corners=2] (\barstart,\barypos)
                rectangle (\barstart+\barwidth,\barypos+\barheight);
        \else
            \fill[\barcolor, rounded corners=2] (\barstart,\barypos)
                rectangle (\barstart+\barwidth*\progress,\barypos+\barheight);
        \fi

        % 结果评价
        \ifnum\pdfstrcmp{\status}{超标}=0
            \node[value, text=customRed] at (13,\basepos) {\footnotesize\textbf{\status}};
        \else
            \node[value, text=customGreen] at (13,\basepos) {\footnotesize\textbf{\status}};
        \fi

        % 分类列
        \node[category] at (16,\basepos) {\footnotesize\category};

        % 添加卡片
        \pgfmathsetmacro{\cardypos}{\basepos-0.5}
        \begin{scope}[shift={(0,\cardypos)}]
            % 卡片背景
            \pgfmathsetmacro{\cardheight}{
                \ifnum\pdfstrcmp{\status}{超标}=0
                    1.0  % 两行内容时的高度
                \else
                    0.6  % 一行内容时的高度
                \fi
            }

            \fill[rounded corners=5pt, customTeal!5, draw=gray!5]
                (0.3,-\cardheight) rectangle (17.3,0);

            % 菌群简介图标和内容
            \node[anchor=west] at (0.5,-0.3) {
                \textbf{\color{gray!90}\footnotesize \textcolor{customTeal}{\faInfoCircle}}
            };
            \node[anchor=west, text width=16cm] at (0.9,-0.3) {
                {\small\color{gray}\footnotesize \intro}
            };

            % 异常解读标题和内容
            \ifnum\pdfstrcmp{\status}{超标}=0
                \node[anchor=west] at (0.55,-0.8) {
                    \textbf{\color{customRed}\footnotesize \textcolor{customRed}{\faLightbulb}}
                };
                \node[anchor=west, text width=16cm] at (0.9,-0.8) {
                    {\small\color{gray}\footnotesize \suggestion}
                };
            \fi
        \end{scope}

        % 分割线
        \pgfmathsetmacro{\linepos}{
            \ifnum\pdfstrcmp{\status}{超标}=0
                \basepos-1.7  % 超标时的分割线位置
            \else
                \basepos-1.3  % 正常时的分割线位置
            \fi
        }
        \draw[gray!20] (0.2,\linepos) -- (\cardwidth-0.2,\linepos);

        % 根据当前行的状态计算下一行的位置增量
        \ifnum\pdfstrcmp{\status}{超标}=0
            \pgfmathsetmacro{\increment}{0.85}  % 超标行(两行内容)需要更大的增量
        \else
            \pgfmathsetmacro{\increment}{0.7}  % 正常行(一行内容)使用较小的增量
        \fi

        % 更新位置计数器
        \pgfmathsetmacro{\nextpos}{\currentpos+\increment}
        \xdef\currentpos{\nextpos}
    }

\end{tikzpicture}
\end{center}

\begin{tcolorbox}[
    enhanced,
    colback=gray!3,
    colframe=gray!3,
    arc=3mm,
    boxrule=0pt,
    width=\textwidth,
    top=8pt,
    bottom=8pt
]
{\small{\textcolor{yellow!85!orange}{\faLightbulb}}\quad 丁酸盐和乙酸盐都是肠道中重要的“好物质”,它们的水平偏低需要关注:
\begin{itemize}
    \item 丁酸盐降低:这种物质就像肠道的“保护伞”,它低了会让肠道细胞缺少营养,就像城墙失去了维护,容易让有害物质趁虚而入。
    \item 乙酸盐降低:可以理解为肠道里的“和平使者”减少了,不利于维持肠道内好菌和坏菌的平衡,就像少了调解纷争的外交官。
\end{itemize}
其他指标都在正常范围:脂多糖(坏菌产生的有害物质)、胆汁酸(帮助消化的物质)和氯离子(维持平衡的物质)都处于正常水平,说明目前肠道整体状况稳定。
}
\end{tcolorbox}





\newpage

\begin{tcolorbox}[
    enhanced,
    colback=white,
    colframe=white,
    arc=2mm,
    boxrule=0pt,
    width=\textwidth,
    left=15pt,
    right=15pt,
    top=10pt,
    bottom=10pt,
    drop shadow={
        opacity=0.2,
        color=customTeal
    },
    borderline west={5pt}{0pt}{customTeal}
]
\textcolor{customTeal}{\Large\textbf{细胞因子指标评估}}
\end{tcolorbox}

\vspace{0.05cm}

\begin{tcolorbox}[
    enhanced,
    colback=customTealBg,
    colframe=customTealBg,
    arc=3mm,
    boxrule=0pt,
    width=\textwidth,
    top=8pt,
    bottom=8pt
]
{\small{\color{customTeal}\faInfoCircle} 细胞因子是免疫系统中重要的信号分子,通过检测不同的细胞因子水平,可以了解机体的免疫状态和炎症程度。
}
\end{tcolorbox}

\begin{tcolorbox}[
    enhanced,
    colback=lightpurple!10, % 卡片底色
    colframe=white,  % 边框颜色
    arc=3mm,
    boxrule=0.5pt,
    width=\textwidth,
    top=8pt,
    bottom=8pt
]
{\small{\color{lightpurple}\faQuestionCircle}\quad \textbf{这些细胞因子指标有什么临床意义?}\\
{\color{orange!50}\faComments}\quad 免疫肿瘤指标的临床意义体现在以下方面:
\begin{itemize}
    \item 炎症评估:IL-6、TNF-$\alpha$和IL-17a反映炎症活性。
    \item 免疫调节:IL-10水平反映免疫抑制能力。
    \item 免疫状态:IFN-$\gamma$反映细胞免疫功能。
    \item 疾病监测:可用于自身免疫性疾病的活动度评估。
\end{itemize}
}
\end{tcolorbox}
\vspace{-0.5cm}
\begin{center}
\begin{tikzpicture}[
    font=\small,
    title/.style={font=\small\bfseries\color{white}},
    value/.style={font=\small},
    reference/.style={font=\small},
    cell/.style={anchor=west, text width=4.2cm},
    note/.style={anchor=west, text width=4.5cm, align=left}
]
    \def\cardwidth{\textwidth}
    \def\cardheight{9.85}
    \def\barheight{0.25}
    \def\barwidth{1.5}
    \def\valuebarspace{0.4}

    % 容器和标题栏背景
    \draw[rounded corners=5, fill=white, draw=gray!20]
        (0,0) rectangle (\cardwidth,-\cardheight);
    \path[fill=customTeal]
        (0,0) [rounded corners=5] -- (\cardwidth,0) --
        (\cardwidth,0.8) -- (0,0.8) -- cycle;

    % 表头
    \node[title, anchor=west] at (0.5,0.4) {\textbf{菌种名称}};
    \node[title] at (6, 0.4) {\textbf{正常范围}};
    \node[title] at (11, 0.4) {\textbf{检测丰度}};
    \node[title] at (16, 0.4) {\textbf{结果评价}};

    % 初始化位置计数器
    \def\currentpos{0.25}

    % 数据行和卡片
    % 数据行和卡片
    \foreach \item/\enitem/\value/\range/\status/\intro/\suggestion/\index in {
        {白介素6}/{IgA}/1/{1-85}/正常/{白介素A是重要的促炎性细胞因子参与急性期反应和炎症过程由 T 细胞和巨噬细胞等产生。}/{}/\currentpos,
        {肿瘤坏死因子-$\alpha$}/{TNF-$\alpha$}/1/{5-85}/缺乏/{典型的促炎性细胞因子在炎症早期发挥关键作用可诱导其他炎症介质的产生。}/{}/\currentpos,
        {II-17a}/{hsCRP}/64/{5-85}/正常/{由Th17 细胞产生的促炎性因子与自身免疫性疾病密切相关参与组织炎症反应。}/{}/\currentpos,
        {II-10}/{FC}/9/{15-95}/缺乏/{重要的抗炎性细胞因子具有免疫抑制作用可抑制促炎性因子的产生。}/{}/\currentpos,
        {干扰素-$\gamma$}/{IFN-$\gamma$}/83/{15-95}/正常/{具有双重免疫调节作用参与抗病毒免疫应答可激活巨噬细胞。}/{}/\currentpos
    }
    {
        % 计算当前行的基础位置
        \pgfmathsetmacro{\basepos}{-2.8*\currentpos}

        % 菌种名称
        \node[cell, align=left] at (0.5,\basepos) {
            \small\textbf{\item}\\[-0.2em]
            {\color{gray}\small\enitem}
        };

        % 正常范围
        \node[reference] at (6,\basepos) {\footnotesize\range};

        % 进度条相关
        \pgfmathsetmacro{\barypos}{\basepos-\valuebarspace+0.1}
        \def\barstart{10.25}

        % 进度条背景
        \fill[gray!10, rounded corners=2] (\barstart,\barypos)
            rectangle (\barstart+\barwidth,\barypos+\barheight);

        % 检测丰度值
        \node[value] at (11, {\basepos-\valuebarspace+0.6}) {\footnotesize\value};

        % 解析范围并计算进度条长度
        \def\parserange#1-#2\endparse{\def\minval{#1}\def\maxval{#2}}
        \expandafter\parserange\range\endparse

        % 计算进度条长度和颜色
        \pgfmathsetmacro{\progress}{min(\value/\maxval, 1.0)}
        \pgfmathparse{\value > \maxval ? "customred" : (\value < \minval ? "customred" : "green!50")}
        \let\barcolor=\pgfmathresult

        % 进度条显示
        \ifnum\pdfstrcmp{\status}{超标}=0
            \fill[customred, rounded corners=2] (\barstart,\barypos)
                rectangle (\barstart+\barwidth,\barypos+\barheight);
        \else
            \fill[\barcolor, rounded corners=2] (\barstart,\barypos)
                rectangle (\barstart+\barwidth*\progress,\barypos+\barheight);
        \fi

        % 结果评价
        \ifnum\pdfstrcmp{\status}{超标}=0
            \node[value, text=customRed] at (16,\basepos) {\footnotesize\textbf{\status}};
        \else
            \node[value, text=customGreen] at (16,\basepos) {\footnotesize\textbf{\status}};
        \fi

        % 添加卡片
        \pgfmathsetmacro{\cardypos}{\basepos-0.5}
        \begin{scope}[shift={(0,\cardypos)}]
            % 卡片背景
            \pgfmathsetmacro{\cardheight}{
                \ifnum\pdfstrcmp{\status}{超标}=0
                    1.0  % 两行内容时的高度
                \else
                    0.6  % 一行内容时的高度
                \fi
            }

            \fill[rounded corners=5pt, customTeal!5, draw=gray!5]
                (0.3,-\cardheight) rectangle (17.3,0);

            % 菌群简介图标和内容
            \node[anchor=west] at (0.5,-0.3) {
                \textbf{\color{gray!90}\footnotesize \textcolor{customTeal}{\faInfoCircle}}
            };
            \node[anchor=west, text width=16cm] at (0.9,-0.3) {
                {\small\color{gray}\footnotesize \intro}
            };

            % 异常解读标题和内容
            \ifnum\pdfstrcmp{\status}{超标}=0
                \node[anchor=west] at (0.55,-0.8) {
                    \textbf{\color{customRed}\footnotesize \textcolor{customRed}{\faLightbulb}}
                };
                \node[anchor=west, text width=16cm] at (0.9,-0.8) {
                    {\small\color{gray}\footnotesize \suggestion}
                };
            \fi
        \end{scope}

        % 分割线
        \pgfmathsetmacro{\linepos}{
            \ifnum\pdfstrcmp{\status}{超标}=0
                \basepos-1.7  % 超标时的分割线位置
            \else
                \basepos-1.3  % 正常时的分割线位置
            \fi
        }
        \draw[gray!20] (0.2,\linepos) -- (\cardwidth-0.2,\linepos);

        % 根据当前行的状态计算下一行的位置增量
        \ifnum\pdfstrcmp{\status}{超标}=0
            \pgfmathsetmacro{\increment}{0.85}  % 超标行(两行内容)需要更大的增量
        \else
            \pgfmathsetmacro{\increment}{0.7}  % 正常行(一行内容)使用较小的增量
        \fi

        % 更新位置计数器
        \pgfmathsetmacro{\nextpos}{\currentpos+\increment}
        \xdef\currentpos{\nextpos}
    }

\end{tikzpicture}
\end{center}



\begin{tcolorbox}[
    enhanced,
    colback=gray!3,
    colframe=gray!3,
    arc=3mm,
    boxrule=0pt,
    width=\textwidth,
    top=8pt,
    bottom=8pt
]
{\small{\textcolor{yellow!85!orange}{\faBell}}\quad 细胞因子都是免疫系统中重要的"调控物质",它们的水平异常需要关注:
\begin{itemize}
    \item IL-10降低:这种情况就像免疫系统的"刹车"失灵了,IL-10是重要的抗炎因子,它低了会让免疫系统缺少必要的抑制,就像汽车失去了刹车系统,容易导致炎症反应过度。
    \item TNF-$\alpha$ 降低:可以理解为免疫系统的"警报器"减弱了,不利于及时发现和应对潜在的感染威胁,就像警报系统灵敏度下降,可能会延迟对危险信号的响应。
\end{itemize}
\textcolor{customGreen}{\faBell}\quad 其他指标都在正常范围:IL-6(急性炎症反应物质)、IL-17a(自身免疫相关因子)和IFN-$\gamma$(免疫调节物质)都处于正常水平,说明目前整体的免疫炎症状态基本稳定。
}

\end{tcolorbox}

\newpage

\begin{tcolorbox}[
    enhanced,
    colback=white,
    colframe=white,
    arc=2mm,
    boxrule=0pt,
    width=\textwidth,
    left=15pt,
    right=15pt,
    top=10pt,
    bottom=10pt,
    drop shadow={
        opacity=0.2,
        color=customTeal
    },
    borderline west={5pt}{0pt}{customTeal}
]
\textcolor{customTeal}{\Large\textbf{免疫肿瘤指标评估}}
\end{tcolorbox}

\vspace{0.05cm}

\begin{tcolorbox}[
    enhanced,
    colback=customTealBg,
    colframe=customTealBg,
    arc=3mm,
    boxrule=0pt,
    width=\textwidth,
    top=8pt,
    bottom=8pt
]
{\small{\color{customTeal}\faInfoCircle} 肿瘤免疫指标是评估机体抗肿瘤免疫功能的重要标志物,通过检测这些指标可以了解机体的免疫监视和抗肿瘤能力。这些指标的变化可以反映肿瘤的发生发展,也可用于免疫治疗的疗效评估和预后判断。\\

{\color{orange}\faExclamationTriangle} \textbf{注意事项}
\begin{itemize}[]
\item 质子泵抑制剂等药物可能影响检测结果
\item 肾功能不全可导致假性升高
\item CgA只是众多肿瘤标志物中的一种,通常需要结合其他检查和临床症状来综合判断。
\end{itemize}
}
\end{tcolorbox}

\begin{tcolorbox}[
    enhanced,
    colback=lightpurple!10, % 卡片底色
    colframe=white,  % 边框颜色
    arc=3mm,
    boxrule=0.5pt,
    width=\textwidth,
    top=8pt,
    bottom=8pt
]
{\small{\color{lightpurple}\faQuestionCircle}\quad \textbf{这些免疫肿瘤指标有什么临床意义?}\\
{\color{orange!50}\faComments}\quad 免疫肿瘤指标的临床意义体现在以下方面:
\begin{itemize}
    \item 肿瘤诊断:可作为神经内分泌肿瘤的筛查和诊断指标。
    \item 病情监测:用于评估神经内分泌肿瘤的治疗效果和复发情况。
    \item 预后评估:CgA水平与神经内分泌肿瘤的预后相关。
    \item 鉴别诊断:有助于区分神经内分泌肿瘤与其他类型肿瘤。
\end{itemize}
}
\end{tcolorbox}

\begin{center}
\begin{tikzpicture}[
    font=\small,
    title/.style={font=\small\bfseries\color{white}},
    value/.style={font=\small},
    reference/.style={font=\small},
    cell/.style={anchor=west, text width=4.2cm},
    note/.style={anchor=west, text width=4.5cm, align=left}
]
    \def\cardwidth{\textwidth}
    \def\cardheight{2}
    \def\barheight{0.25}
    \def\barwidth{1.5}
    \def\valuebarspace{0.4}

    % 容器和标题栏背景
    \draw[rounded corners=5, fill=white, draw=gray!20]
        (0,0) rectangle (\cardwidth,-\cardheight);
    \path[fill=customTeal]
        (0,0) [rounded corners=5] -- (\cardwidth,0) --
        (\cardwidth,0.8) -- (0,0.8) -- cycle;

    % 表头
    \node[title, anchor=west] at (0.5,0.4) {\textbf{菌种名称}};
    \node[title] at (6, 0.4) {\textbf{正常范围}};
    \node[title] at (11, 0.4) {\textbf{检测丰度}};
    \node[title] at (16, 0.4) {\textbf{结果评价}};

    % 初始化位置计数器
    \def\currentpos{0.25}

    % 数据行和卡片
    \foreach \item/\enitem/\value/\range/\status/\intro/\suggestion/\index in {
        {嗜铬粒蛋白A}/{CgA}/64/{5-95}/正常/{嗜铬粒蛋白A是神经内分泌细胞分泌的一种酸性糖蛋白,是评估神经内分泌肿瘤的重要标志物。}/{}/\currentpos
    }
    {
        % 计算当前行的基础位置
        \pgfmathsetmacro{\basepos}{-2.8*\currentpos}

        % 菌种名称
        \node[cell, align=left] at (0.5,\basepos) {
            \small\textbf{\item}\\[-0.2em]
            {\color{gray}\small\enitem}
        };

        % 正常范围
        \node[reference] at (6,\basepos) {\footnotesize\range};

        % 进度条相关
        \pgfmathsetmacro{\barypos}{\basepos-\valuebarspace+0.1}
        \def\barstart{10.25}

        % 进度条背景
        \fill[gray!10, rounded corners=2] (\barstart,\barypos)
            rectangle (\barstart+\barwidth,\barypos+\barheight);

        % 检测丰度值
        \node[value] at (11, {\basepos-\valuebarspace+0.6}) {\footnotesize\value};

        % 解析范围并计算进度条长度
        \def\parserange#1-#2\endparse{\def\minval{#1}\def\maxval{#2}}
        \expandafter\parserange\range\endparse

        % 计算进度条长度和颜色
        \pgfmathsetmacro{\progress}{min(\value/\maxval, 1.0)}
        \pgfmathparse{\value > \maxval ? "customred" : (\value < \minval ? "customred" : "green!50")}
        \let\barcolor=\pgfmathresult

        % 进度条显示
        \ifnum\pdfstrcmp{\status}{超标}=0
            \fill[customred, rounded corners=2] (\barstart,\barypos)
                rectangle (\barstart+\barwidth,\barypos+\barheight);
        \else
            \fill[\barcolor, rounded corners=2] (\barstart,\barypos)
                rectangle (\barstart+\barwidth*\progress,\barypos+\barheight);
        \fi

        % 结果评价
        \ifnum\pdfstrcmp{\status}{超标}=0
            \node[value, text=customRed] at (16,\basepos) {\footnotesize\textbf{\status}};
        \else
            \node[value, text=customGreen] at (16,\basepos) {\footnotesize\textbf{\status}};
        \fi

        % 添加卡片
        \pgfmathsetmacro{\cardypos}{\basepos-0.5}
        \begin{scope}[shift={(0,\cardypos)}]
            % 卡片背景
            \pgfmathsetmacro{\cardheight}{
                \ifnum\pdfstrcmp{\status}{超标}=0
                    1.0  % 两行内容时的高度
                \else
                    0.6  % 一行内容时的高度
                \fi
            }

            \fill[rounded corners=5pt, customTeal!5, draw=gray!5]
                (0.3,-\cardheight) rectangle (17.3,0);

            % 菌群简介图标和内容
            \node[anchor=west] at (0.5,-0.3) {
                \textbf{\color{gray!90}\footnotesize \textcolor{customTeal}{\faInfoCircle}}
            };
            \node[anchor=west, text width=16cm] at (0.9,-0.3) {
                {\small\color{gray}\footnotesize \intro}
            };

            % 异常解读标题和内容
            \ifnum\pdfstrcmp{\status}{超标}=0
                \node[anchor=west] at (0.55,-0.8) {
                    \textbf{\color{customRed}\footnotesize \textcolor{customRed}{\faLightbulb}}
                };
                \node[anchor=west, text width=16cm] at (0.9,-0.8) {
                    {\small\color{gray}\footnotesize \suggestion}
                };
            \fi
        \end{scope}

        % 分割线
        \pgfmathsetmacro{\linepos}{
            \ifnum\pdfstrcmp{\status}{超标}=0
                \basepos-1.7  % 超标时的分割线位置
            \else
                \basepos-1.3  % 正常时的分割线位置
            \fi
        }
        \draw[gray!20] (0.2,\linepos) -- (\cardwidth-0.2,\linepos);

        % 根据当前行的状态计算下一行的位置增量
        \ifnum\pdfstrcmp{\status}{超标}=0
            \pgfmathsetmacro{\increment}{0.85}  % 超标行(两行内容)需要更大的增量
        \else
            \pgfmathsetmacro{\increment}{0.7}  % 正常行(一行内容)使用较小的增量
        \fi

        % 更新位置计数器
        \pgfmathsetmacro{\nextpos}{\currentpos+\increment}
        \xdef\currentpos{\nextpos}
    }

\end{tikzpicture}
\end{center}

\begin{tcolorbox}[
    enhanced,
    colback=gray!3,
    colframe=gray!3,
    arc=3mm,
    boxrule=0pt,
    width=\textwidth,
    top=8pt,
    bottom=8pt
]
{\small{\textcolor{green!85!orange}{\faBell}}\quad 该肿瘤免疫指标的评估值在正常范围:说明神经内分泌系统功能正常。
}

\end{tcolorbox}

\newpage

\begin{tcolorbox}[
    enhanced,
    colback=white,
    colframe=customTeal,
    arc=2mm,
    boxrule=1pt,
    left=20pt,
    right=20pt,
    top=12pt,
    bottom=12pt,
    width=\textwidth,
    fontupper=\sffamily,
    overlay={
    \draw[customTeal, line width=2pt]
    ([xshift=15pt]frame.south west) -- ([xshift=-15pt]frame.south east);
    }
]
{\Large\bfseries\textcolor{customTeal}{\Huge 营养饮食指标评估}}
\end{tcolorbox}

\begin{tcolorbox}[
    enhanced,
    colback=customTealBg,
    colframe=customTealBg,
    arc=3mm,
    boxrule=0pt,
    width=\textwidth,
    top=8pt,
    bottom=8pt
]
{\small{\color{customTeal}\faInfoCircle} 营养饮食评估模块旨在分析个体的饮食习惯与营养摄入情况,以评估其对健康的影响。该模块主要涵盖宏观营养素(如碳水化合物、蛋白质、脂肪)和微观营养素(如维生素、矿物质)的摄入量,以及膳食纤维和水分的摄入。通过对饮食数据的综合分析,我们可以了解个体的营养状态和潜在的营养缺乏或过量问题,从而为制定个性化的饮食干预方案提供科学依据。

营养饮食评估不仅关注食物的种类和数量,还考虑饮食的多样性和均衡性,这些因素对维持身体健康和预防慢性疾病至关重要。通过对饮食模式的评估,我们能够识别出可能影响健康的饮食习惯,并提出相应的改善建议,以促进整体健康水平的提升。\\

{\color{orange}\faExclamationTriangle} \textbf{特别注意}:营养饮食评估结果基于自我报告的饮食数据,可能受到个体记忆和报告偏差的影响,结果仅供参考!
}
\end{tcolorbox}


\begin{figure}[htbp]
\centering
\begin{tikzpicture}[scale=1.2]
    % 定义现代风格的颜色
    \definecolor{moderngreen}{HTML}{77DD77}  % 柔和绿
    \definecolor{modernred}{HTML}{FF6B6B}    % 柔和红
    \definecolor{modernorange}{HTML}{FFB347} % 柔和橙
    \definecolor{modernblue}{HTML}{79B8FF}   % 柔和蓝
    \definecolor{lightgray}{HTML}{F8F9FA}    % 背景灰
    \definecolor{textgray}{HTML}{666666}     % 文字灰

    % 左侧饼图部分 - 垂直位置调整到与右侧卡片中心对齐
    \begin{scope}[xshift=-3cm, yshift=1cm]  % 调整yshift使圆环垂直居中
        % 背景圆
        \fill[lightgray, opacity=0.3] (0,0) circle (3.2cm);

        % 主饼图
        \fill[moderngreen] (0,0) -- (0:3) arc (0:174.24:3) -- cycle;
        \fill[modernred] (0,0) -- (174.24:3) arc (174.24:278.64:3) -- cycle;
        \fill[modernorange] (0,0) -- (278.64:3) arc (278.64:336.6:3) -- cycle;
        \fill[modernblue] (0,0) -- (336.6:3) arc (336.6:360:3) -- cycle;

        % 中心白色圆
        \fill[white] (0,0) circle (1.8cm);

        % 中心文字
        \node[align=center] at (0,0) {
            \textbf{\large 营养指标}\\\small 总计31项
        };
    \end{scope}

    % 右侧图例和说明 - 增加每个卡片的高度
    \begin{scope}[xshift=4cm]
        % 绘制四个更高的方框
        \draw[rounded corners] (-2.5,4.5) rectangle (3,2.5);  % 第一个方框
        \draw[rounded corners] (-2.5,2.2) rectangle (3,-0.2); % 第二个方框
        \draw[rounded corners] (-2.5,-0.5) rectangle (3,-2.5); % 第三个方框
        \draw[rounded corners] (-2.5,-2.8) rectangle (3,-4.8); % 第四个方框

        % 正常
        \node[inner sep=0pt] at (-2.2,3.8) {\color{moderngreen}\rule{8pt}{8pt}};
        \node[anchor=west] at (-1.9,3.8) {\large 正常};
        \node[anchor=west,text=textgray] at (-1.9,3.0) {\small 15 项指标达标};
        \node[anchor=east] at (2.7,3.8) {\large 48.4\%};

        % 缺乏
        \node[inner sep=0pt] at (-2.2,1.5) {\color{modernred}\rule{8pt}{8pt}};
        \node[anchor=west] at (-1.9,1.5) {\large 缺乏};
        \node[anchor=west,text=textgray] at (-1.9,0.7) {\small 纤维、乳制品、乙酸、维生素 A/B1/E};
        \node[anchor=west,text=textgray] at (-1.9,0.0) {\small 叶酸、甘氨酸、色氨酸};
        \node[anchor=east] at (2.7,1.5) {\large 29.0\%};

        % 不足
        \node[inner sep=0pt] at (-2.2,-1.2) {\color{modernorange}\rule{8pt}{8pt}};
        \node[anchor=west] at (-1.9,-1.2) {\large 不足};
        \node[anchor=west,text=textgray] at (-1.9,-2.0) {\small 铁、锌、维生素 B6/C/K};
        \node[anchor=east] at (2.7,-1.2) {\large 16.1\%};

        % 过量
        \node[inner sep=0pt] at (-2.2,-3.5) {\color{modernblue}\rule{8pt}{8pt}};
        \node[anchor=west] at (-1.9,-3.5) {\large 过量};
        \node[anchor=west,text=textgray] at (-1.9,-4.3) {\small 糖、苏氨酸};
        \node[anchor=east] at (2.7,-3.5) {\large 6.5\%};
    \end{scope}

\end{tikzpicture}\label{fig:figure}
\end{figure}

\newpage

\begin{tcolorbox}[
    enhanced,
    colback=white,
    colframe=white,
    arc=2mm,
    boxrule=0pt,
    width=\textwidth,
    left=15pt,
    right=15pt,
    top=10pt,
    bottom=10pt,
    drop shadow={
        opacity=0.2,
        color=customTeal
    },
    borderline west={5pt}{0pt}{customTeal}
]
\textcolor{customTeal}{\Large\textbf{日常饮食评估}}
\end{tcolorbox}

\vspace{0.05cm}

\begin{tcolorbox}[
enhanced,
colback=customTealBg,
colframe=customTeal!5,
arc=3mm,
boxrule=0pt,
width=\textwidth,
top=8pt,
bottom=8pt
]
{\small{\color{customTeal}\faInfoCircle}\quad 糖、盐、膳食纤维和乳制品是人体日常饮食中重要的营养成分,对维持人体基本生理功能和健康发挥着关键作用。本报告将通过肠道菌群等检测指标,全面评估这些营养成分在人体中的代谢水平和整体状况。
}
\end{tcolorbox}
\vspace{-0.5cm}
\begin{center}
\begin{tikzpicture}[
    font=\small,
    title/.style={font=\small\bfseries\color{white}},
    value/.style={font=\small},
    reference/.style={font=\small},
    cell/.style={anchor=west, text width=4.2cm},
    note/.style={anchor=west, text width=4.5cm, align=left}
]
    \def\cardwidth{\textwidth}
    \def\cardheight{7.88}
    \def\barheight{0.25}
    \def\barwidth{1.5}
    \def\valuebarspace{0.4}

    % 容器和标题栏背景
    \draw[rounded corners=5, fill=white, draw=gray!20]
        (0,0) rectangle (\cardwidth,-\cardheight);
    \path[fill=customTeal]
        (0,0) [rounded corners=5] -- (\cardwidth,0) --
        (\cardwidth,0.8) -- (0,0.8) -- cycle;

    % 表头
    \node[title, anchor=west] at (0.5,0.4) {\textbf{评估指标}};
    \node[title] at (6, 0.4) {\textbf{正常范围}};
    \node[title] at (11, 0.4) {\textbf{评估数值}};
    \node[title] at (16, 0.4) {\textbf{评估结果}};

    % 初始化位置计数器
    \def\currentpos{0.25}

    % 数据行和卡片
    \foreach \item/\enitem/\value/\range/\status/\intro/\suggestion/\index in {
        {糖}/{Sugar}/63/{0-75}/正常/{碳水化合物的一种,提供快速能量,参与代谢过程。}/{}/\currentpos,
        {盐}/{Salt}/0.00023/{0-0.01}/正常/{一种矿物质,主要成分为氯化钠。它在体内维持水分平衡、调节血压,并参与神经和肌肉的正常功能。}/{}/\currentpos,
        {膳食纤维}/{Dietary Fiber}/0.00145/{0-0.02}/正常/{植物性食物中的一种不可消化的碳水化合物,有助于促进肠道健康、改善消化、控制血糖水平,并降低胆固醇。}/{}/\currentpos,
        {乳制品}/{Dairy Products}/0.00567/{0-0.03}/正常/{由牛奶或其他动物的乳汁制成的食品,如牛奶、酸奶和奶酪。它们是钙、蛋白质和维生素D的重要来源,有助于骨骼健康和肌肉生长。}/{}/\currentpos
    }
    {
        % 计算当前行的基础位置
        \pgfmathsetmacro{\basepos}{-2.8*\currentpos}

        % 菌种名称
        \node[cell, align=left] at (0.5,\basepos) {
            \small\textbf{\item}\\[-0.2em]
            {\color{lightgray}\small\enitem}
        };

        % 正常范围
        \node[reference] at (6,\basepos) {\footnotesize\range};

        % 进度条相关
        \pgfmathsetmacro{\barypos}{\basepos-\valuebarspace+0.1}
        \def\barstart{10.25}

        % 进度条背景
        \fill[gray!10, rounded corners=2] (\barstart,\barypos)
            rectangle (\barstart+\barwidth,\barypos+\barheight);

        % 检测丰度值
        \node[value] at (11, {\basepos-\valuebarspace+0.6}) {\footnotesize\value};

        % 解析范围并计算进度条长度
        \def\parserange#1-#2\endparse{\def\minval{#1}\def\maxval{#2}}
        \expandafter\parserange\range\endparse

        % 计算进度条长度和颜色
        \pgfmathsetmacro{\progress}{min(\value/\maxval, 1.0)}
        \pgfmathparse{\value > \maxval ? "customred" : (\value < \minval ? "customred" : "green!50")}
        \let\barcolor=\pgfmathresult

        % 进度条显示
        \ifnum\pdfstrcmp{\status}{超标}=0
            \fill[customred, rounded corners=2] (\barstart,\barypos)
                rectangle (\barstart+\barwidth,\barypos+\barheight);
        \else
            \fill[\barcolor, rounded corners=2] (\barstart,\barypos)
                rectangle (\barstart+\barwidth*\progress,\barypos+\barheight);
        \fi

        % 结果评价
        \ifnum\pdfstrcmp{\status}{超标}=0
            \node[value, text=customRed] at (16,\basepos) {\footnotesize\textbf{\status}};
        \else
            \node[value, text=customGreen] at (16,\basepos) {\footnotesize\textbf{\status}};
        \fi

        % 添加卡片
        \pgfmathsetmacro{\cardypos}{\basepos-0.5}
        \begin{scope}[shift={(0,\cardypos)}]
            % 卡片背景
            \pgfmathsetmacro{\cardheight}{
                \ifnum\pdfstrcmp{\status}{超标}=0
                    1.0  % 两行内容时的高度
                \else
                    0.6  % 一行内容时的高度
                \fi
            }

            \fill[rounded corners=5pt, customTeal!5, draw=gray!5]
                (0.3,-\cardheight) rectangle (17.3,0);

            % 菌群简介图标和内容
            \node[anchor=west] at (0.5,-0.3) {
                \textbf{\color{gray!90}\footnotesize \textcolor{customTeal}{\faInfoCircle}}
            };
            \node[anchor=west, text width=16cm] at (0.9,-0.3) {
                {\small\color{gray}\footnotesize \intro}
            };

            % 异常解读标题和内容
            \ifnum\pdfstrcmp{\status}{超标}=0
                \node[anchor=west] at (0.55,-0.8) {
                    \textbf{\color{customRed}\footnotesize \textcolor{customRed}{\faLightbulb}}
                };
                \node[anchor=west, text width=16cm] at (0.9,-0.8) {
                    {\small\color{gray}\footnotesize \suggestion}
                };
            \fi
        \end{scope}

        % 分割线
        \pgfmathsetmacro{\linepos}{
            \ifnum\pdfstrcmp{\status}{超标}=0
                \basepos-1.7  % 超标时的分割线位置
            \else
                \basepos-1.3  % 正常时的分割线位置
            \fi
        }
        \draw[gray!20] (0.2,\linepos) -- (\cardwidth-0.2,\linepos);

        % 根据当前行的状态计算下一行的位置增量
        \ifnum\pdfstrcmp{\status}{超标}=0
            \pgfmathsetmacro{\increment}{0.85}  % 超标行(两行内容)需要更大的增量
        \else
            \pgfmathsetmacro{\increment}{0.7}  % 正常行(一行内容)使用较小的增量
        \fi

        % 更新位置计数器
        \pgfmathsetmacro{\nextpos}{\currentpos+\increment}
        \xdef\currentpos{\nextpos}
    }

\end{tikzpicture}
\end{center}

\begin{tcolorbox}[
    enhanced,
    colback=gray!3,
    colframe=gray!3,
    arc=3mm,
    boxrule=0pt,
    width=\textwidth,
    top=8pt,
    bottom=8pt
]
{\small \textcolor{customGreen}{\faBell}\quad 整体来看,糖、盐、膳食纤维和乳制品的指标均在正常范围内,反映出良好的饮食习惯和营养状态,有助于维持身体健康。
}

\end{tcolorbox}

\newpage

\begin{tcolorbox}[
    enhanced,
    colback=white,
    colframe=white,
    arc=2mm,
    boxrule=0pt,
    width=\textwidth,
    left=15pt,
    right=15pt,
    top=10pt,
    bottom=10pt,
    drop shadow={
        opacity=0.2,
        color=customTeal
    },
    borderline west={5pt}{0pt}{customTeal}
]
\textcolor{customTeal}{\Large\textbf{矿物质微量元素和有机酸评估}}
\end{tcolorbox}

\vspace{0.05cm}

\begin{tcolorbox}[
    enhanced,
    colback=customTealBg,
    colframe=customTeal!5,
    arc=3mm,
    boxrule=0pt,
    width=\textwidth,
    top=8pt,
    bottom=8pt
]
{\small{\color{customTeal}\faInfoCircle}\quad 矿物质微量元素和有机酸是人体代谢过程中的重要组成部分,它们通过不同的机制参与人体生理功能的调节。本报告将通过肠道菌群等检测指标,全面评估其在人体中的代谢水平和整体状况。
}
\end{tcolorbox}
\vspace{-0.5cm}
\begin{center}
\begin{tikzpicture}[
    font=\small,
    title/.style={font=\small\bfseries\color{white}},
    value/.style={font=\small},
    reference/.style={font=\small},
    cell/.style={anchor=west, text width=4.2cm},
    note/.style={anchor=west, text width=4.5cm, align=left}
]
    \def\cardwidth{\textwidth}
    \def\cardheight{9.8}
    \def\barheight{0.25}
    \def\barwidth{1.5}
    \def\valuebarspace{0.4}

    % 容器和标题栏背景
    \draw[rounded corners=5, fill=white, draw=gray!20]
        (0,0) rectangle (\cardwidth,-\cardheight);
    \path[fill=customTeal]
        (0,0) [rounded corners=5] -- (\cardwidth,0) --
        (\cardwidth,0.8) -- (0,0.8) -- cycle;

    % 表头
    \node[title, anchor=west] at (0.5,0.4) {\textbf{评估指标}};
    \node[title] at (6, 0.4) {\textbf{正常范围}};
    \node[title] at (11, 0.4) {\textbf{评估数值}};
    \node[title] at (16, 0.4) {\textbf{评估结果}};

    % 初始化位置计数器
    \def\currentpos{0.25}

    % 数据行和卡片
    \foreach \item/\enitem/\value/\range/\status/\intro/\suggestion/\index in {
        {铁}/{Fe$^{3+}$}/63/{0-75}/正常/{作为微量元素的一种,提供快速能量,参与代谢过程。}/{}/\currentpos,
        {锌}/{Zn$^{2+}$}/0.00023/{0-0.01}/正常/{主要成分为氧化锌,在体内维持水分平衡,调节血压,并参与神经和肌肉的正常功能。}/{}/\currentpos,
        {乙酸}/{Dietary Fiber}/0.00145/{0-0.02}/正常/{植物性食物中的一种不可消化的碳水化合物,有助于促进肠道健康、改善消化、控制血糖水平,并降低胆固醇。}/{}/\currentpos,
        {丙酸}/{Dairy Products}/0.00567/{0-0.03}/正常/{由牛奶或其他动物的乳汁制成的食品,是钙、蛋白质和维生素D的重要来源,有助于骨骼健康和肌肉生长。}/{}/\currentpos,
        {丁酸}/{Dairy Products}/0.00567/{0-0.03}/正常/{同样源自乳制品,是钙、蛋白质和维生素D的重要来源,对骨骼健康和肌肉生长有重要作用。}/{}/\currentpos
    }
    {
        % 计算当前行的基础位置
        \pgfmathsetmacro{\basepos}{-2.8*\currentpos}

        % 菌种名称
        \node[cell, align=left] at (0.5,\basepos) {
            \small\textbf{\item}\\[-0.2em]
            {\color{lightgray}\small\enitem}
        };

        % 正常范围
        \node[reference] at (6,\basepos) {\footnotesize\range};

        % 进度条相关
        \pgfmathsetmacro{\barypos}{\basepos-\valuebarspace+0.1}
        \def\barstart{10.25}

        % 进度条背景
        \fill[gray!10, rounded corners=2] (\barstart,\barypos)
            rectangle (\barstart+\barwidth,\barypos+\barheight);

        % 检测丰度值
        \node[value] at (11, {\basepos-\valuebarspace+0.6}) {\footnotesize\value};

        % 解析范围并计算进度条长度
        \def\parserange#1-#2\endparse{\def\minval{#1}\def\maxval{#2}}
        \expandafter\parserange\range\endparse

        % 计算进度条长度和颜色
        \pgfmathsetmacro{\progress}{min(\value/\maxval, 1.0)}
        \pgfmathparse{\value > \maxval ? "customred" : (\value < \minval ? "customred" : "green!50")}
        \let\barcolor=\pgfmathresult

        % 进度条显示
        \ifnum\pdfstrcmp{\status}{超标}=0
            \fill[customred, rounded corners=2] (\barstart,\barypos)
                rectangle (\barstart+\barwidth,\barypos+\barheight);
        \else
            \fill[\barcolor, rounded corners=2] (\barstart,\barypos)
                rectangle (\barstart+\barwidth*\progress,\barypos+\barheight);
        \fi

        % 结果评价
        \ifnum\pdfstrcmp{\status}{超标}=0
            \node[value, text=customRed] at (16,\basepos) {\footnotesize\textbf{\status}};
        \else
            \node[value, text=customGreen] at (16,\basepos) {\footnotesize\textbf{\status}};
        \fi

        % 添加卡片
        \pgfmathsetmacro{\cardypos}{\basepos-0.5}
        \begin{scope}[shift={(0,\cardypos)}]
            % 卡片背景
            \pgfmathsetmacro{\cardheight}{
                \ifnum\pdfstrcmp{\status}{超标}=0
                    1.0  % 两行内容时的高度
                \else
                    0.6  % 一行内容时的高度
                \fi
            }

            \fill[rounded corners=5pt, customTeal!5, draw=gray!5]
                (0.3,-\cardheight) rectangle (17.3,0);

            % 菌群简介图标和内容
            \node[anchor=west] at (0.5,-0.3) {
                \textbf{\color{gray!90}\footnotesize \textcolor{customTeal}{\faInfoCircle}}
            };
            \node[anchor=west, text width=16cm] at (0.9,-0.3) {
                {\small\color{gray}\footnotesize \intro}
            };

            % 异常解读标题和内容
            \ifnum\pdfstrcmp{\status}{超标}=0
                \node[anchor=west] at (0.55,-0.8) {
                    \textbf{\color{customRed}\footnotesize \textcolor{customRed}{\faLightbulb}}
                };
                \node[anchor=west, text width=16cm] at (0.9,-0.8) {
                    {\small\color{gray}\footnotesize \suggestion}
                };
            \fi
        \end{scope}

        % 分割线
        \pgfmathsetmacro{\linepos}{
            \ifnum\pdfstrcmp{\status}{超标}=0
                \basepos-1.7  % 超标时的分割线位置
            \else
                \basepos-1.3  % 正常时的分割线位置
            \fi
        }
        \draw[gray!20] (0.2,\linepos) -- (\cardwidth-0.2,\linepos);

        % 根据当前行的状态计算下一行的位置增量
        \ifnum\pdfstrcmp{\status}{超标}=0
            \pgfmathsetmacro{\increment}{0.85}  % 超标行(两行内容)需要更大的增量
        \else
            \pgfmathsetmacro{\increment}{0.7}  % 正常行(一行内容)使用较小的增量
        \fi

        % 更新位置计数器
        \pgfmathsetmacro{\nextpos}{\currentpos+\increment}
        \xdef\currentpos{\nextpos}
    }

\end{tikzpicture}
\end{center}

\begin{tcolorbox}[
    enhanced,
    colback=gray!3,
    colframe=gray!3,
    arc=3mm,
    boxrule=0pt,
    width=\textwidth,
    top=8pt,
    bottom=8pt
]
{\small \textcolor{customGreen}{\faBell}\quad 评估结果显示所有指标均在正常范围内,表明样本的微量元素和有机酸水平处于健康状态,这对维持人体正常生理功能具有重要意义。
}

\end{tcolorbox}

\newpage

\begin{tcolorbox}[
    enhanced,
    colback=white,
    colframe=white,
    arc=2mm,
    boxrule=0pt,
    width=\textwidth,
    left=15pt,
    right=15pt,
    top=10pt,
    bottom=10pt,
    drop shadow={
        opacity=0.2,
        color=customTeal
    },
    borderline west={5pt}{0pt}{customTeal}
]
\textcolor{customTeal}{\Large\textbf{维生素评估}}
\end{tcolorbox}

\vspace{0.05cm}

\begin{tcolorbox}[
    enhanced,
    colback=customTealBg,
    colframe=customTealBg,
    arc=3mm,
    boxrule=0pt,
    width=\textwidth,
    top=8pt,
    bottom=8pt
]
{\small{\color{customTeal}\faInfoCircle} 维生素是人体必需的微量营养素,本报告通过肠道菌群等检测指标来评估下列各项维生素的水平。
}
\end{tcolorbox}
\vspace{-0.5cm}
\begin{center}
\begin{tikzpicture}[
    font=\small,
    title/.style={font=\small\bfseries\color{white}},
    value/.style={font=\small},
    reference/.style={font=\small},
    cell/.style={anchor=west, text width=4.2cm},
    note/.style={anchor=west, text width=4.5cm, align=left}
]
    \def\cardwidth{\textwidth}
    \def\cardheight{17.65}
    \def\barheight{0.25}
    \def\barwidth{1.5}
    \def\valuebarspace{0.4}

    % 容器和标题栏背景
    \draw[rounded corners=5, fill=white, draw=gray!20]
        (0,0) rectangle (\cardwidth,-\cardheight);
    \path[fill=customTeal]
        (0,0) [rounded corners=5] -- (\cardwidth,0) --
        (\cardwidth,0.8) -- (0,0.8) -- cycle;

    % 表头
    \node[title, anchor=west] at (0.5,0.4) {\textbf{检测项目}};
    \node[title] at (6, 0.4) {\textbf{正常范围}};
    \node[title] at (11, 0.4) {\textbf{检测丰度}};
    \node[title] at (16, 0.4) {\textbf{结果评价}};

    % 初始化位置计数器
    \def\currentpos{0.25}

    % 数据行和卡片
    \foreach \item/\enitem/\value/\range/\status/\intro/\suggestion/\index in {
        {维生素 A}/{Vitamin A}/7.35/{350-700}/正常/{维持视力,促进生长发育,增强免疫力,缺乏可能导致夜盲症和免疫力下降。}/{}/\currentpos,
        {维生素 B1}/{Vitamin B1}/0.48/{0.75-1.3}/正常/{参与糖代谢,维持神经系统功能,缺乏可导致脚气病,表现为神经和心血管问题。}/{}/\currentpos,
        {维生素 B2}/{Vitamin B2}/1.77/{0.6-1.2}/正常/{参与能量代谢,维持皮肤黏膜健康,缺乏可能导致口角炎和舌炎。}/{}/\currentpos,
        {维生素 B6}/{Vitamin B6}/1.02/{0.6-1.2}/正常/{参与氨基酸代谢,促进造血功能,缺乏可能导致贫血和抑郁。}/{}/\currentpos,
        {维生素 B12}/{Vitamin B12}/3.53/{1.2-2.4}/正常/{促进细胞生成,维持神经系统功能,缺乏可导致巨幼细胞性贫血和神经损伤。}/{}/\currentpos,
        {维生素 B9 (叶酸)}/{Vitamin B9}/160.90/{500-100}/正常/{抗氧化,促进胚胎蛋白合成,增强免疫力,缺乏可导致巨幼细胞性贫血和胎儿神经管缺陷。}/{}/\currentpos,
        {维生素 D}/{Vitamin D}/47.24/{20-30}/正常/{促进钙吸收,维持骨骼健康,调节免疫系统,缺乏可导致佝偻病和骨质疏松。}/{}/\currentpos,
        {维生素 E}/{Vitamin E}/3.96/{7-14}/正常/{抗氧化,保护细胞膜,延缓衰老,缺乏可能导致神经和肌肉问题。}/{}/\currentpos,
        {维生素 K}/{Vitamin K}/9/{7-15}/正常/{参与凝血过程,促进骨骼代谢,缺乏可导致出血倾向和骨质疏松。}/{}/\currentpos
    }
    {
        % 计算当前行的基础位置
        \pgfmathsetmacro{\basepos}{-2.8*\currentpos}

        % 检测项目名称
        \node[cell, align=left] at (0.5,\basepos) {
            \small\textbf{\item}\\[-0.2em]
            {\color{lightgray}\small\enitem}
        };

        % 正常范围
        \node[reference] at (6,\basepos) {\footnotesize\range};

        % 进度条相关
        \pgfmathsetmacro{\barypos}{\basepos-\valuebarspace+0.1}
        \def\barstart{10.25}

        % 进度条背景
        \fill[gray!10, rounded corners=2] (\barstart,\barypos)
            rectangle (\barstart+\barwidth,\barypos+\barheight);

        % 检测丰度值
        \node[value] at (11, {\basepos-\valuebarspace+0.6}) {\footnotesize\value};

        % 解析范围并计算进度条长度
        \def\parserange#1-#2\endparse{\def\minval{#1}\def\maxval{#2}}
        \expandafter\parserange\range\endparse

        % 计算进度条长度和颜色
        \pgfmathsetmacro{\progress}{min(\value/\maxval, 1.0)}
        \pgfmathparse{\value > \maxval ? "customred" : (\value < \minval ? "customred" : "green!50")}
        \let\barcolor=\pgfmathresult

        % 进度条显示
        \ifnum\pdfstrcmp{\status}{超标}=0
            \fill[customred, rounded corners=2] (\barstart,\barypos)
                rectangle (\barstart+\barwidth,\barypos+\barheight);
        \else
            \fill[\barcolor, rounded corners=2] (\barstart,\barypos)
                rectangle (\barstart+\barwidth*\progress,\barypos+\barheight);
        \fi

        % 结果评价
        \ifnum\pdfstrcmp{\status}{超标}=0
            \node[value, text=customRed] at (16,\basepos) {\footnotesize\textbf{\status}};
        \else
            \node[value, text=customGreen] at (16,\basepos) {\footnotesize\textbf{\status}};
        \fi

        % 添加卡片
        \pgfmathsetmacro{\cardypos}{\basepos-0.5}
        \begin{scope}[shift={(0,\cardypos)}]
            % 卡片背景
            \pgfmathsetmacro{\cardheight}{
                \ifnum\pdfstrcmp{\status}{超标}=0
                    1.0  % 两行内容时的高度
                \else
                    0.6  % 一行内容时的高度
                \fi
            }

            \fill[rounded corners=5pt, customTeal!5, draw=gray!5]
                (0.3,-\cardheight) rectangle (17.3,0);

            % 菌群简介图标和内容
            \node[anchor=west] at (0.5,-0.3) {
                \textbf{\color{gray!90}\footnotesize \textcolor{customTeal}{\faInfoCircle}}
            };
            \node[anchor=west, text width=16cm] at (0.9,-0.3) {
                {\small\color{gray}\footnotesize \intro}
            };

            % 异常解读标题和内容
            \ifnum\pdfstrcmp{\status}{超标}=0
                \node[anchor=west] at (0.55,-0.8) {
                    \textbf{\color{customRed}\footnotesize \textcolor{customRed}{\faLightbulb}}
                };
                \node[anchor=west, text width=16cm] at (0.9,-0.8) {
                    {\small\color{gray}\footnotesize \suggestion}
                };
            \fi
        \end{scope}

        % 分割线
        \pgfmathsetmacro{\linepos}{
            \ifnum\pdfstrcmp{\status}{超标}=0
                \basepos-1.7  % 超标时的分割线位置
            \else
                \basepos-1.3  % 正常时的分割线位置
            \fi
        }
        \draw[gray!20] (0.2,\linepos) -- (\cardwidth-0.2,\linepos);

        % 根据当前行的状态计算下一行的位置增量
        \ifnum\pdfstrcmp{\status}{超标}=0
            \pgfmathsetmacro{\increment}{0.85}  % 超标行(两行内容)需要更大的增量
        \else
            \pgfmathsetmacro{\increment}{0.7}  % 正常行(一行内容)使用较小的增量
        \fi

        % 更新位置计数器
        \pgfmathsetmacro{\nextpos}{\currentpos+\increment}
        \xdef\currentpos{\nextpos}
    }

\end{tikzpicture}
\end{center}

\newpage

\begin{tcolorbox}[
    enhanced,
    colback=white,
    colframe=white,
    arc=2mm,
    boxrule=0pt,
    width=\textwidth,
    left=15pt,
    right=15pt,
    top=10pt,
    bottom=10pt,
    drop shadow={
        opacity=0.2,
        color=customTeal
    },
    borderline west={5pt}{0pt}{customTeal}
]
\textcolor{customTeal}{\Large\textbf{氨基酸营养评估}}
\end{tcolorbox}

\vspace{0.05cm}

\begin{tcolorbox}[
enhanced,
colback=customTealBg,
colframe=customTealBg,
arc=3mm,
boxrule=0pt,
width=\textwidth,
top=8pt,
bottom=8pt
]
{\small{\color{customTeal}\faInfoCircle} 氨基酸是蛋白质的基本构建单位,在人体内参与多种重要的生理功能。人体所需的20种氨基酸中,9种必需氨基酸需要从食物中获取,其余可由人体自身合成。氨基酸评估可以反映蛋白质营养状况以及代谢功能等。
}
\end{tcolorbox}

\begin{tcolorbox}[
enhanced,
colback=blue!3,
colframe=gray!3,
arc=3mm,
boxrule=0pt,
width=\textwidth,
top=8pt,
bottom=8pt
]
{\large \textbf{食源性必需氨基酸}}

\vspace{0.1cm}
\small {\color{customTeal}\faInfoCircle} 食源性必需氨基酸是指人体无法自身合成、必须从食物中获取的氨基酸,共有9种。它们是构建人体蛋白质的重要原料,对生长发育、免疫功能和各种生理活动都至关重要。优质蛋白食物(如肉、蛋、奶)含量丰富,植物性食物(如豆类、谷物)通过合理搭配也能满足需求。
\end{tcolorbox}



\begin{center}
\begin{tikzpicture}[
    font=\small,
    title/.style={font=\small\bfseries\color{white}},
    value/.style={font=\small},
    reference/.style={font=\small},
    cell/.style={anchor=west, text width=4.2cm},
    note/.style={anchor=west, text width=4.5cm, align=left}
]
    \def\cardwidth{\textwidth}
    \def\cardheight{17.65}
    \def\barheight{0.25}
    \def\barwidth{1.5}
    \def\valuebarspace{0.4}

    % 容器和标题栏背景
    \draw[rounded corners=5, fill=white, draw=gray!20]
        (0,0) rectangle (\cardwidth,-\cardheight);
    \path[fill=customTeal]
        (0,0) [rounded corners=5] -- (\cardwidth,0) --
        (\cardwidth,0.8) -- (0,0.8) -- cycle;

    % 表头
    \node[title, anchor=west] at (0.5,0.4) {\textbf{检测项目}};
    \node[title] at (6, 0.4) {\textbf{正常范围}};
    \node[title] at (11, 0.4) {\textbf{检测丰度}};
    \node[title] at (16, 0.4) {\textbf{结果评价}};

    % 初始化位置计数器
    \def\currentpos{0.25}

    % 数据行和卡片
    \foreach \item/\enitem/\value/\range/\status/\intro/\suggestion/\index in {
        {苏氨酸}/{Threonine}/7.35/{350-700}/正常/{促进消化吸收,维持肠道功能。缺乏可能影响消化系统和免疫系统功能。}/{}/\currentpos,
        {异亮氨酸}/{Isoleucine}/0.48/{0.75-1.3}/缺乏/{促进肌肉生长,提供能量。缺乏可能导致肌肉无力和疲劳。}/{建议适当增加优质蛋白质的摄入,如瘦肉、鱼类、蛋类等。}/\currentpos,
        {亮氨酸}/{Leucine}/1.77/{0.6-1.2}/超标/{合成蛋白质,修复肌肉组织。过量可能影响其他氨基酸的吸收。}/{建议适当减少含亮氨酸较多的食物摄入,保持均衡饮食。}/\currentpos,
        {赖氨酸}/{Lysine}/1.77/{0.6-1.2}/超标/{促进生长发育,增强免疫力。过量可能影响钙的吸收。}/{建议调整饮食结构,适当减少高赖氨酸食物的摄入。}/\currentpos,
        {蛋氨酸}/{Methionine}/7.35/{350-700}/正常/{解毒护肝,促进毛发生长。对维持肝脏功能和毛发健康很重要。}/{}/\currentpos,
        {苯丙氨酸}/{Phenylalanine}/0.48/{0.75-1.3}/缺乏/{合成神经递质,调节情绪。缺乏可能影响精神状态。}/{建议适当增加含苯丙氨酸的食物摄入,如豆类、坚果等。}/\currentpos,
        {色氨酸}/{Tryptophan}/1.77/{0.6-1.2}/超标/{改善睡眠,稳定情绪。过量可能影响血清素水平。}/{建议适当控制含色氨酸较多的食物摄入,保持作息规律。}/\currentpos,
        {缬氨酸}/{Valine}/1.77/{0.6-1.2}/超标/{促进肌肉代谢,提供能量。过量可能影响其他支链氨基酸的平衡。}/{建议调整饮食结构,减少高缬氨酸食物的摄入。}/\currentpos,
        {组氨酸}/{Histidine}/1.77/{0.6-1.2}/超标/{参与血红蛋白合成,维持pH平衡,支持免疫系统。过量可能影响锌的吸收。}/{建议适当控制含组氨酸较多的食物摄入,注意补充锌。}/\currentpos
    }
    {
        % 计算当前行的基础位置
        \pgfmathsetmacro{\basepos}{-2.8*\currentpos}

        % 检测项目名称
        \node[cell, align=left] at (0.5,\basepos) {
            \small\textbf{\item}\\[-0.2em]
            {\color{gray}\small\enitem}
        };

        % 正常范围
        \node[reference] at (6,\basepos) {\footnotesize\range};

        % 进度条相关
        \pgfmathsetmacro{\barypos}{\basepos-\valuebarspace+0.1}
        \def\barstart{10.25}

        % 进度条背景
        \fill[gray!10, rounded corners=2] (\barstart,\barypos)
            rectangle (\barstart+\barwidth,\barypos+\barheight);

        % 检测丰度值
        \node[value] at (11, {\basepos-\valuebarspace+0.6}) {\footnotesize\value};

        % 解析范围并计算进度条长度
        \def\parserange#1-#2\endparse{\def\minval{#1}\def\maxval{#2}}
        \expandafter\parserange\range\endparse

        % 计算进度条长度和颜色
        \pgfmathsetmacro{\progress}{min(\value/\maxval, 1.0)}
        \pgfmathparse{\value > \maxval ? "customred" : (\value < \minval ? "customred" : "green!50")}
        \let\barcolor=\pgfmathresult

        % 进度条显示
        \ifnum\pdfstrcmp{\status}{超标}=0
            \fill[customred, rounded corners=2] (\barstart,\barypos)
                rectangle (\barstart+\barwidth,\barypos+\barheight);
        \else
            \fill[\barcolor, rounded corners=2] (\barstart,\barypos)
                rectangle (\barstart+\barwidth*\progress,\barypos+\barheight);
        \fi

        % 结果评价
        \ifnum\pdfstrcmp{\status}{超标}=0
            \node[value, text=customRed] at (16,\basepos) {\footnotesize\textbf{\status}};
        \else
            \node[value, text=customGreen] at (16,\basepos) {\footnotesize\textbf{\status}};
        \fi

        % 添加卡片
        \pgfmathsetmacro{\cardypos}{\basepos-0.5}
        \begin{scope}[shift={(0,\cardypos)}]
            % 卡片背景
            \pgfmathsetmacro{\cardheight}{
                \ifnum\pdfstrcmp{\status}{超标}=0
                    1.0  % 两行内容时的高度
                \else
                    0.6  % 一行内容时的高度
                \fi
            }

            \fill[rounded corners=5pt, customTeal!5, draw=gray!5]
                (0.3,-\cardheight) rectangle (17.3,0);

            % 菌群简介图标和内容
            \node[anchor=west] at (0.5,-0.3) {
                \textbf{\color{gray!90}\footnotesize \textcolor{customTeal}{\faInfoCircle}}
            };
            \node[anchor=west, text width=16cm] at (0.9,-0.3) {
                {\small\color{gray}\footnotesize \intro}
            };

            % 异常解读标题和内容
            \ifnum\pdfstrcmp{\status}{超标}=0
                \node[anchor=west] at (0.55,-0.8) {
                    \textbf{\color{customRed}\footnotesize \textcolor{customRed}{\faLightbulb}}
                };
                \node[anchor=west, text width=16cm] at (0.9,-0.8) {
                    {\small\color{gray}\footnotesize \suggestion}
                };
            \fi
        \end{scope}

        % 分割线
        \pgfmathsetmacro{\linepos}{
            \ifnum\pdfstrcmp{\status}{超标}=0
                \basepos-1.7  % 超标时的分割线位置
            \else
                \basepos-1.3  % 正常时的分割线位置
            \fi
        }
        \draw[gray!20] (0.2,\linepos) -- (\cardwidth-0.2,\linepos);

        % 根据当前行的状态计算下一行的位置增量
        \ifnum\pdfstrcmp{\status}{超标}=0
            \pgfmathsetmacro{\increment}{0.85}  % 超标行(两行内容)需要更大的增量
        \else
            \pgfmathsetmacro{\increment}{0.7}  % 正常行(一行内容)使用较小的增量
        \fi

        % 更新位置计数器
        \pgfmathsetmacro{\nextpos}{\currentpos+\increment}
        \xdef\currentpos{\nextpos}
    }

\end{tikzpicture}
\end{center}

\begin{tcolorbox}[
    enhanced,
    colback=gray!3,
    colframe=gray!3,
    arc=3mm,
    boxrule=0pt,
    width=\textwidth,
    top=8pt,
    bottom=8pt
]
{\small{\textcolor{green!85!orange}{\faLightbulb}}\quad 指标的评估值都在正常范围:IgA(黏膜防护者)和IgD(免疫调节者)都处于正常水平,说明目前身体的基础防御能力保持稳定。
}

\end{tcolorbox}

\newpage

\begin{tcolorbox}[
    enhanced,
    colback=blue!3,
    colframe=gray!3,
    arc=3mm,
    boxrule=0pt,
    width=\textwidth,
    top=8pt,
    bottom=8pt
]
{\large \textbf{非食源性氨基酸}}

\vspace{0.1cm}
{\small {\color{customTeal}\faInfoCircle} 非食源性氨基酸是指人体可以自身合成的氨基酸,共有11种。它们同样是构建人体蛋白质的基本单位,包括丙氨酸、天冬氨酸、天冬酰胺、谷氨酸、谷氨酰胺、甘氨酸、脯氨酸、丝氨酸、酪氨酸、半胱氨酸和精氨酸。其中,精氨酸、酪氨酸和半胱氨酸在特定生理状态(如生长发育期、疾病恢复期)下可能成为条件性必需氨基酸。虽然人体可以自行合成这些氨基酸,但均衡的饮食仍有助于维持其合成所需的原料供应。}
\end{tcolorbox}

\begin{center}
\begin{tikzpicture}[
    font=\small,
    title/.style={font=\small\bfseries\color{white}},
    value/.style={font=\small},
    reference/.style={font=\small},
    cell/.style={anchor=west, text width=4.2cm},
    note/.style={anchor=west, text width=4.5cm, align=left}
]
    \def\cardwidth{\textwidth}
    \def\cardheight{14}
    \def\barheight{0.25}
    \def\barwidth{1.5}
    \def\valuebarspace{0.4}

    % 容器和标题栏背景
    \draw[rounded corners=5, fill=white, draw=gray!20]
        (0,0) rectangle (\cardwidth,-\cardheight);
    \path[fill=customTeal]
        (0,0) [rounded corners=5] -- (\cardwidth,0) --
        (\cardwidth,0.8) -- (0,0.8) -- cycle;

    % 表头
    \node[title, anchor=west] at (0.5,0.4) {\textbf{检测项目}};
    \node[title] at (6, 0.4) {\textbf{正常范围}};
    \node[title] at (11, 0.4) {\textbf{检测丰度}};
    \node[title] at (16, 0.4) {\textbf{结果评价}};

    % 初始化位置计数器
    \def\currentpos{0.25}

    % 数据行和卡片
    \foreach \item/\enitem/\value/\range/\status/\intro/\suggestion/\index in {
        {丙氨酸}/{Alanine}/7.35/{350-700}/正常/{促进消化吸收,维持肠道功能。在氨基酸代谢和糖代谢中发挥重要作用。}/\currentpos,
        {谷氨酸}/{Glutamic acid}/1.77/{0.6-1.2}/超标/{合成蛋白质,修复肌肉组织。是重要的神经递质前体物质。}/{建议适当减少含谷氨酸较高的食物摄入,保持饮食均衡。}/\currentpos,
        {甘氨酸}/{Glycine}/1.77/{0.6-1.2}/超标/{促进生长发育,增强免疫力。参与胶原蛋白的合成。}/{注意调整饮食结构,避免过量摄入含甘氨酸较多的食物。}/\currentpos,
        {脯氨酸}/{Proline}/7.35/{350-700}/正常/{解毒护肝,促进毛发生长。对维持结缔组织健康很重要。}/{}/\currentpos,
        {丝氨酸}/{Serine}/0.48/{0.75-1.3}/缺乏/{合成神经递质,调节情绪。参与脂质代谢和免疫功能。}/{建议适量增加含丝氨酸的食物摄入,如大豆、花生等。}/\currentpos,
        {半胱氨酸}/{Cysteine}/1.77/{0.6-1.2}/超标/{改善睡眠,稳定情绪。具有抗氧化作用。}/{建议适当控制含半胱氨酸较多的食物摄入,注意营养均衡。}/\currentpos,
        {酪氨酸}/{Tyrosine}/1.77/{0.6-1.2}/超标/{促进肌肉代谢,提供能量。是多种激素的前体物质。}/{建议调整饮食结构,减少高酪氨酸食物的摄入。}/\currentpos
    }
    {
        % 计算当前行的基础位置
        \pgfmathsetmacro{\basepos}{-2.8*\currentpos}

        % 检测项目名称
        \node[cell, align=left] at (0.5,\basepos) {
            \small\textbf{\item}\\[-0.2em]
            {\color{gray}\small\enitem}
        };

        % 正常范围
        \node[reference] at (6,\basepos) {\footnotesize\range};

        % 进度条相关
        \pgfmathsetmacro{\barypos}{\basepos-\valuebarspace+0.1}
        \def\barstart{10.25}

        % 进度条背景
        \fill[gray!10, rounded corners=2] (\barstart,\barypos)
            rectangle (\barstart+\barwidth,\barypos+\barheight);

        % 检测丰度值
        \node[value] at (11, {\basepos-\valuebarspace+0.6}) {\footnotesize\value};

        % 解析范围并计算进度条长度
        \def\parserange#1-#2\endparse{\def\minval{#1}\def\maxval{#2}}
        \expandafter\parserange\range\endparse

        % 计算进度条长度和颜色
        \pgfmathsetmacro{\progress}{min(\value/\maxval, 1.0)}
        \pgfmathparse{\value > \maxval ? "customred" : (\value < \minval ? "customred" : "green!50")}
        \let\barcolor=\pgfmathresult

        % 进度条显示
        \ifnum\pdfstrcmp{\status}{超标}=0
            \fill[customred, rounded corners=2] (\barstart,\barypos)
                rectangle (\barstart+\barwidth,\barypos+\barheight);
        \else
            \fill[\barcolor, rounded corners=2] (\barstart,\barypos)
                rectangle (\barstart+\barwidth*\progress,\barypos+\barheight);
        \fi

        % 结果评价
        \ifnum\pdfstrcmp{\status}{超标}=0
            \node[value, text=customRed] at (16,\basepos) {\footnotesize\textbf{\status}};
        \else
            \node[value, text=customGreen] at (16,\basepos) {\footnotesize\textbf{\status}};
        \fi

        % 添加卡片
        \pgfmathsetmacro{\cardypos}{\basepos-0.5}
        \begin{scope}[shift={(0,\cardypos)}]
            % 卡片背景
            \pgfmathsetmacro{\cardheight}{
                \ifnum\pdfstrcmp{\status}{超标}=0
                    1.0  % 两行内容时的高度
                \else
                    0.6  % 一行内容时的高度
                \fi
            }

            \fill[rounded corners=5pt, customTeal!5, draw=gray!5]
                (0.3,-\cardheight) rectangle (17.3,0);

            % 菌群简介图标和内容
            \node[anchor=west] at (0.5,-0.3) {
                \textbf{\color{gray!90}\footnotesize \textcolor{customTeal}{\faInfoCircle}}
            };
            \node[anchor=west, text width=16cm] at (0.9,-0.3) {
                {\small\color{gray}\footnotesize \intro}
            };

            % 异常解读标题和内容
            \ifnum\pdfstrcmp{\status}{超标}=0
                \node[anchor=west] at (0.55,-0.8) {
                    \textbf{\color{customRed}\footnotesize \textcolor{customRed}{\faLightbulb}}
                };
                \node[anchor=west, text width=16cm] at (0.9,-0.8) {
                    {\small\color{gray}\footnotesize \suggestion}
                };
            \fi
        \end{scope}

        % 分割线
        \pgfmathsetmacro{\linepos}{
            \ifnum\pdfstrcmp{\status}{超标}=0
                \basepos-1.7  % 超标时的分割线位置
            \else
                \basepos-1.3  % 正常时的分割线位置
            \fi
        }
        \draw[gray!20] (0.2,\linepos) -- (\cardwidth-0.2,\linepos);

        % 根据当前行的状态计算下一行的位置增量
        \ifnum\pdfstrcmp{\status}{超标}=0
            \pgfmathsetmacro{\increment}{0.85}  % 超标行(两行内容)需要更大的增量
        \else
            \pgfmathsetmacro{\increment}{0.7}  % 正常行(一行内容)使用较小的增量
        \fi

        % 更新位置计数器
        \pgfmathsetmacro{\nextpos}{\currentpos+\increment}
        \xdef\currentpos{\nextpos}
    }

\end{tikzpicture}
\end{center}

\begin{tcolorbox}[
    enhanced,
    colback=gray!3,
    colframe=gray!3,
    arc=3mm,
    boxrule=0pt,
    width=\textwidth,
    top=8pt,
    bottom=8pt
]
{\small{\textcolor{green!85!orange}{\faLightbulb}}\quad 指标的评估值都在正常范围:IgA(黏膜防护者)和IgD(免疫调节者)都处于正常水平,说明目前身体的基础防御能力保持稳定。
}

\end{tcolorbox}

\newpage

% 检测结果标题

\begin{tcolorbox}[
    enhanced,
    colback=white,
    colframe=customTeal,
    arc=2mm,
    boxrule=1pt,
    left=20pt,
    right=20pt,
    top=12pt,
    bottom=12pt,
    width=\textwidth,
    fontupper=\sffamily,
    overlay={
    \draw[customTeal, line width=2pt]
    ([xshift=15pt]frame.south west) -- ([xshift=-15pt]frame.south east);
    }
]
{\Huge\bfseries\textcolor{customTeal}{慢病风险评估}}
\end{tcolorbox}

\begin{tcolorbox}[
    enhanced,
    colback=customTealBg,
    colframe=gray!5,
    arc=3mm,
    boxrule=0pt,
    width=\textwidth,
    top=8pt,
    bottom=8pt
]
{\small{\color{customTeal}\faInfoCircle}
肠道菌群是人体最大的微生物群落,在维持人体健康中发挥着关键作用。大量研究表明,肠道菌群的失衡与多种慢性疾病(包括但不限于消化系统疾病,免疫系统疾病,代谢性疾病,心脑血管疾病,神经系统疾病等)的发生发展密切相关。通过对肠道菌群的分析,我们可以及早发现疾病风险,实现疾病的预防和干预。\\

本报告评估基于肠道微生物组检测数据和机器学习方法,对以下慢性疾病风险进行分析预测:
\begin{itemize}
    \item \textbf{肠道相关疾病}:炎症性肠炎、肠易激综合征、感染性腹泻、肠道病毒感染和过敏性腹泻
    \item \textbf{器官功能疾病}:肝病、心脑血管疾病、甲状腺疾病和肺部疾病
    \item \textbf{神经系统疾病}:神经行为发育异常
\end{itemize}

%\item \footnotesize 慢病综合定期检测建议:
%    \begin{itemize}
%        \item \footnotesize 结直肠癌风险:每年一次
%        \item \footnotesize 其他疾病风险:每半年一次
%    \end{itemize}
%\item \footnotesize 检测结果说明:
%    \begin{itemize}
%        \item \footnotesize 如出现高风险提示,请务必重点关注
%        \item \footnotesize 建议按推荐周期进行追踪检测
%        \item \footnotesize 建议结合临床检查做进一步确认
%    \end{itemize}
%\end{itemize}

{\color{orange}\faExclamationTriangle} \textbf{特别提示}:本检测仅作为健康评估参考,不作为疾病诊断依据!}
\end{tcolorbox}

\begin{tcolorbox}[
    enhanced,
    colback=lightpurple!10, % 卡片底色
    colframe=lightpurple!10,  % 边框颜色
    arc=3mm,
    boxrule=0.5pt,
    width=\textwidth,
    top=8pt,
    bottom=8pt
]
{\small{\color{lightpurple}\faQuestionCircle}\quad \textbf{低风险就一定没有患病风险吗?}\\
{\color{orange!50}\faComments}\quad 如果评估结果显示慢病风险较低,这通常表明个体在某些生理或代谢指标上处于较为健康的状态。然而,这并不意味着完全没有患病风险。慢性病的发生往往是多因素共同作用的结果,包括遗传因素、生活方式、环境影响等。因此,肠道菌群数据虽然能提供有价值的参考信息,但不能单独作为排除病症风险的依据。
低风险并不等同于无风险,个体仍然可能受到其他未被检测的风险因素的影响。如果已经出现相关疾病症状,请立即就医。
}
\end{tcolorbox}

\begin{tcolorbox}[
    enhanced,
    colback=lightpurple!10, % 卡片底色
    colframe=lightpurple!10,  % 边框颜色
    arc=3mm,
    boxrule=0.5pt,
    width=\textwidth,
    top=8pt,
    bottom=8pt
]
{\small{\color{lightpurple}\faQuestionCircle}\quad \textbf{某个疾病提示我中高风险,但我并没有相关症状?}\\
{\color{orange!50}\faComments}\quad 疾病风险提示基于大数据分析和个体健康信息的综合评估,这可能涉及遗传因素、生活方式、甚至是肠道菌群数据等多方面的影响。因此,中高风险的评估不一定直接与您当前的症状相对应。许多慢性疾病在早期可能没有明显的症状,但依然可以在体内发展。因此,即便当前没有表现出疾病症状,仍然需要关注风险评估的结果。它可能是未来健康问题的预警信号。不同个体对疾病的反应和表现有所不同。有些人可能在早期阶段就显示出症状,而其他人则可能在同样的风险水平下保持良好状态。即使没有相关症状,建议定期进行健康检查和监测,以便及时发现任何潜在的问题。此外,维持健康的生活方式,如均衡饮食、适度运动和良好的心理健康,可以帮助降低未来发病的风险。如果风险提示令您感到担忧,或您希望更深入地了解情况,建议咨询医生或健康专业人士。他们可以根据您的具体健康状况、家族病史等因素提供更个性化的建议和指导。


}
\end{tcolorbox}




\begin{center}
\begin{tikzpicture}[scale=0.8]
    % 定义渐变色(降低饱和度)
    \definecolor{lowRisk}{RGB}{126,195,130}      % 更柔和的绿色
    \definecolor{medLowRisk}{RGB}{255,213,79}    % 更柔和的黄色
    \definecolor{medHighRisk}{RGB}{255,172,82}   % 更柔和的橙色
    \definecolor{highRisk}{RGB}{244,114,107}     % 更柔和的红色

    % 标准进度条(与原来相同)
    \node[anchor=east, font=\footnotesize] at (1,0.3) {\textbf{风险指数标准}};
    \fill[gray!15, rounded corners=2pt] (4,0) rectangle (14,0.4);

    % 四段彩色条(与原来相同)
    \fill[left color=lowRisk!70, right color=lowRisk, rounded corners=2pt]
        (2,0) rectangle (5.6,0.4);
    \fill[left color=medLowRisk!70, right color=medLowRisk, rounded corners=2pt]
        (5.6,0) rectangle (8,0.4);
    \fill[left color=medHighRisk!70, right color=medHighRisk, rounded corners=2pt]
        (8,0) rectangle (10.4,0.4);
    \fill[left color=highRisk!70, right color=highRisk, rounded corners=2pt]
        (10.4,0) rectangle (14,0.4);

    % 标准进度条的刻度和标签(与原来相同)
    \foreach \x/\label in {2/0, 5.6/0.3, 8/0.5, 10.4/0.7, 14/1.0} {
        \draw[gray!50, line width=0.5pt] (\x,0) -- (\x,-0.08);
        \node[below, font=\footnotesize] at (\x,-0.2) {\label};
    }

    % 风险等级标签(与原来相同)
    \node[above, font=\footnotesize\bfseries, text=lowRisk!90!black] at (3.8,0.6) {低风险};
    \node[above, font=\footnotesize\bfseries, text=medLowRisk!90!black] at (6.8,0.6) {注意};
    \node[above, font=\footnotesize\bfseries, text=medHighRisk!90!black] at (9.2,0.6) {中等风险};
    \node[above, font=\footnotesize\bfseries, text=highRisk!90!black] at (12.2,0.6) {高风险};

    % 添加疾病风险进度条
    % 示例疾病1(假设风险值为0.6)
    \node[anchor=east, font=\footnotesize] at (1,-1) {炎症性肠炎};
    \fill[gray!15, rounded corners=4pt] (2,-1.2) rectangle (14,-0.8);
    \fill[left color=lowRisk!70, right color=lowRisk, rounded corners=4pt]
        (2,-1.2) [rounded corners=4pt] -- (2,-0.8) [rounded corners=4pt] -- (5,-0.8)
        [rounded corners=4pt] -- (5,-1.2) [rounded corners=4pt] -- cycle;

    % 肠易激综合征(新增)
\node[anchor=east, font=\footnotesize] at (1,-2) {肠易激综合征};
\fill[gray!15, rounded corners=4pt] (2,-2.2) rectangle (14,-1.8);
\fill[left color=medLowRisk!70, right color=medLowRisk, rounded corners=4pt]
    (2,-2.2) [rounded corners=4pt] -- (2,-1.8) [rounded corners=4pt] -- (7.2,-1.8)
    [rounded corners=4pt] -- (7.2,-2.2) [rounded corners=4pt] -- cycle;  -1.2) [rounded corners=4pt] -- cycle;

    % 可以继续添加更多疾病...
    % 感染性腹泻(0.14)
\node[anchor=east, font=\footnotesize] at (1,-3) {感染性腹泻};
\fill[gray!15, rounded corners=4pt] (2,-3.2) rectangle (14,-2.8);
\fill[left color=lowRisk!70, right color=lowRisk, rounded corners=4pt]
    (2,-3.2) [rounded corners=4pt] -- (2,-2.8) [rounded corners=4pt] -- (3.8,-2.8)
    [rounded corners=4pt] -- (3.8,-3.2) [rounded corners=4pt] -- cycle;

% 肠道病毒感染(0.15)
\node[anchor=east, font=\footnotesize] at (1,-4) {肠道病毒感染};
\fill[gray!15, rounded corners=4pt] (2,-4.2) rectangle (14,-3.8);
\fill[left color=lowRisk!70, right color=lowRisk, rounded corners=4pt]
    (2,-4.2) [rounded corners=4pt] -- (2,-3.8) [rounded corners=4pt] -- (4,-3.8)
    [rounded corners=4pt] -- (4,-4.2) [rounded corners=4pt] -- cycle;

% 过敏性腹泻(0.03)
\node[anchor=east, font=\footnotesize] at (1,-5) {过敏性腹泻};
\fill[gray!15, rounded corners=4pt] (2,-5.2) rectangle (14,-4.8);
\fill[left color=lowRisk!70, right color=lowRisk, rounded corners=4pt]
    (2,-5.2) [rounded corners=4pt] -- (2,-4.8) [rounded corners=4pt] -- (2.6,-4.8)
    [rounded corners=4pt] -- (2.6,-5.2) [rounded corners=4pt] -- cycle;

% 肝病(0.30)
\node[anchor=east, font=\footnotesize] at (1,-6) {肝病};
\fill[gray!15, rounded corners=4pt] (2,-6.2) rectangle (14,-5.8);
\fill[left color=medLowRisk!70, right color=medLowRisk, rounded corners=4pt]
    (2,-6.2) [rounded corners=4pt] -- (2,-5.8) [rounded corners=4pt] -- (5.6,-5.8)
    [rounded corners=4pt] -- (5.6,-6.2) [rounded corners=4pt] -- cycle;

    % 心脑血管疾病(0.15)
\node[anchor=east, font=\footnotesize] at (1,-7) {心脑血管疾病};
\fill[gray!15, rounded corners=4pt] (2,-7.2) rectangle (14,-6.8);
\fill[left color=lowRisk!70, right color=lowRisk, rounded corners=4pt]
    (2,-7.2) [rounded corners=4pt] -- (2,-6.8) [rounded corners=4pt] -- (4,-6.8)
    [rounded corners=4pt] -- (4,-7.2) [rounded corners=4pt] -- cycle;

% 甲状腺疾病(0.19)
\node[anchor=east, font=\footnotesize] at (1,-8) {甲状腺疾病};
\fill[gray!15, rounded corners=4pt] (2,-8.2) rectangle (14,-7.8);
\fill[left color=lowRisk!70, right color=lowRisk, rounded corners=4pt]
    (2,-8.2) [rounded corners=4pt] -- (2,-7.8) [rounded corners=4pt] -- (4.4,-7.8)
    [rounded corners=4pt] -- (4.4,-8.2) [rounded corners=4pt] -- cycle;

% 肺部疾病(0.15)
\node[anchor=east, font=\footnotesize] at (1,-9) {肺部疾病};
\fill[gray!15, rounded corners=4pt] (2,-9.2) rectangle (14,-8.8);
\fill[left color=lowRisk!70, right color=lowRisk, rounded corners=4pt]
    (2,-9.2) [rounded corners=4pt] -- (2,-8.8) [rounded corners=4pt] -- (4,-8.8)
    [rounded corners=4pt] -- (4,-9.2) [rounded corners=4pt] -- cycle;

% 神经行为发育异常(0.39)
\node[anchor=east, font=\footnotesize] at (1,-10) {神经行为发育异常};
\fill[gray!15, rounded corners=4pt] (2,-10.2) rectangle (14,-9.8);
\fill[left color=medLowRisk!70, right color=medLowRisk, rounded corners=4pt]
    (2,-10.2) [rounded corners=4pt] -- (2,-9.8) [rounded corners=4pt] -- (6.8,-9.8)
    [rounded corners=4pt] -- (6.8,-10.2) [rounded corners=4pt] -- cycle;

\end{tikzpicture}
\end{center}
\vspace{1.5cm}
\begin{tcolorbox}[
    enhanced,
    colback=customTealBg,
    colframe=customTealBg,
    arc=3mm,
    boxrule=0pt,
    width=\textwidth,
    top=8pt,
    bottom=8pt
] {\large \textbf{炎症性肠炎}}
\hspace{0.1em} % 增加间距
\raisebox{-0.25ex}{
    \begin{tikzpicture}
    % 使用clip命令确保内容在圆角矩形内
    \begin{scope}
        \clip [rounded corners=4pt] (0,0) rectangle (4,0.3);

        % 绘制背景
        \fill[gray!10] (0,0) rectangle (4,0.3);

        % 绘制进度条(15%进度)带圆角
        \begin{scope}
            \clip [rounded corners=4pt] (0,0) rectangle (4*0.25,0.3);
            \shade[left color=customGreen,
                   right color={rgb,255: red,67; green,160; blue,71}]
                (0,0) rectangle (4*0.25,0.3);
        \end{scope}
    \end{scope}

    % 绘制圆角边框
    \draw[gray!10, rounded corners=4pt] (0,0) rectangle (4,0.3);
\end{tikzpicture}
}
\small 0.25
\hspace{0.1em}
\raisebox{0.25ex}{ % 调整此值实现垂直对齐
\tcbox[colback=customGreen, colframe=customGreen, arc=1.5mm,
       boxrule=0pt, left=1.5mm, right=1.5mm, top=0.1mm, bottom=0.1mm,
       rounded corners=south, on line]
{\scriptsize\textcolor{white}{\textbf{低风险}}}
}

{\footnotesize
{\color{customGreen}\faInfoCircle}
{\color{black}检测风险:}0.25,属于“低风险”级别。\\
{\scriptsize\textcolor{rgb,255:red,140;green,140;blue,140}{\faChartBar}} {\color{black}风险评估:}该风险提示目前患炎症性肠炎的风险相对较低,可能存在轻微的生活方式不当或其他健康问题。如若出现腹痛、腹部痉挛、腹泻或便秘等症状,同时伴有食欲下降和疲劳感,请及时就医。\\
{\scriptsize\textcolor{rgb,255:red,140;green,140;blue,140}{\faBookOpen}} {\color{black}疾病简介:}炎症性肠病是一种慢性、反复发作的肠道炎症性疾病,主要包括两种类型:克罗恩病和溃疡性结肠炎。其特点是肠道持续性炎症,可能导致肠壁损伤,并伴有全身性症状。
}
\end{tcolorbox}

\begin{tcolorbox}[
    enhanced,
    colback=customTealBg,
    colframe=customTealBg,
    arc=3mm,
    boxrule=0pt,
    width=\textwidth,
    top=8pt,
    bottom=8pt
] {\large \textbf{肠易激综合征}}
\hspace{0.1em} % 增加间距
\raisebox{-0.25ex}{
    \begin{tikzpicture}
    % 使用clip命令确保内容在圆角矩形内
    \begin{scope}
        \clip [rounded corners=4pt] (0,0) rectangle (4,0.3);

        % 绘制背景
        \fill[gray!10] (0,0) rectangle (4,0.3);

        % 绘制进度条(30%进度)带圆角
        \begin{scope}
            \clip [rounded corners=4pt] (0,0) rectangle (4*0.45,0.3);
            \shade[left color={rgb,255: red,255; green,215; blue,95},
                   right color={rgb,255: red,255; green,215; blue,95}]
                (0,0) rectangle (4*0.45,0.3);
        \end{scope}
    \end{scope}

    % 绘制圆角边框
    \draw[gray!10, rounded corners=4pt] (0,0) rectangle (4,0.3);
\end{tikzpicture}
}
{\small 0.45}
\raisebox{0.25ex}{ % 调整此值实现垂直对齐
\tcbox[colback=yellow!90!black, colframe=yellow!90!black, arc=1.5mm,
       boxrule=0pt, left=1.5mm, right=1.5mm, top=0.1mm, bottom=0.1mm,
       rounded corners=south, on line]
{\scriptsize\textcolor{white}{\textbf{注意}}}
}

{\footnotesize
{\color{yellow!90!black}
\faInfoCircle} \textbf{检测风险}:0.45,属于“注意”级别。\\
{\scriptsize\textcolor{rgb,255:red,140;green,140;blue,140}{\faChartBar}} \textbf{风险评估}:该风险值提示需要关注以下风险因素:长期压力和焦虑状态,不规律的饮食习惯,肠道感染史等。如若出现经常性腹痛或(因情绪波动引起的)腹泻与便秘交替等不适症状,建议及时就医。\\
{\scriptsize\textcolor{rgb,255:red,140;green,140;blue,140}{\faBookOpen}} \textbf{疾病简介}:肠易激综合征是一种常见的功能性胃肠道疾病,其特点是肠道功能紊乱,主要表现包括腹痛、腹胀、排便异常(如腹泻、便秘或两者交替),症状常因压力、饮食或某些外部因素诱发或加重。
}
\end{tcolorbox}

\begin{tcolorbox}[
    enhanced,
    colback=customTealBg,
    colframe=customTealBg,
    arc=3mm,
    boxrule=0pt,
    width=\textwidth,
    top=8pt,
    bottom=8pt
] {\large \textbf{感染性腹泻}}
\hspace{0.1em} % 增加间距
\raisebox{-0.25ex}{
    \begin{tikzpicture}
    % 使用clip命令确保内容在圆角矩形内
    \begin{scope}
        \clip [rounded corners=4pt] (0,0) rectangle (4,0.3);

        % 绘制背景
        \fill[gray!10] (0,0) rectangle (4,0.3);

        % 绘制进度条(15%进度)带圆角
        \begin{scope}
            \clip [rounded corners=4pt] (0,0) rectangle (4*0.14,0.3);
            \shade[left color=green!70!black,
                   right color={rgb,255: red,67; green,160; blue,71}]
                (0,0) rectangle (4*0.14,0.3);
        \end{scope}
    \end{scope}

    % 绘制圆角边框
    \draw[gray!10, rounded corners=4pt] (0,0) rectangle (4,0.3);
\end{tikzpicture}
}
\small 0.14
\hspace{0.1em}
\raisebox{0.25ex}{ % 调整此值实现垂直对齐
\tcbox[colback=green!70!black, colframe=green!70!black, arc=1.5mm,
       boxrule=0pt, left=1.5mm, right=1.5mm, top=0.1mm, bottom=0.1mm,
       rounded corners=south, on line]
{\scriptsize\textcolor{white}{\textbf{低风险}}}
}

{\footnotesize
{\color{green!50!black!80}\faInfoCircle}
检测风险值:0.14,属于“低风险”级别。\\
{\scriptsize\textcolor{rgb,255:red,140;green,140;blue,140}{\faChartBar}} 风险评估:该风险值提示您感染由病原微生物(细菌、病毒或寄生虫)感染引起的消化系统疾病的风险值相对较低。仍需警惕食品卫生安全状况,个人卫生习惯,饮用水质量,季节性传染病流行等。
}
\end{tcolorbox}


\begin{tcolorbox}[
    enhanced,
    colback=customTealBg,
    colframe=customTealBg,
    arc=3mm,
    boxrule=0pt,
    width=\textwidth,
    top=8pt,
    bottom=8pt
] {\large \textbf{肠道病毒感染}}
\hspace{0.1em} % 增加间距
\raisebox{-0.25ex}{
    \begin{tikzpicture}
    % 使用clip命令确保内容在圆角矩形内
    \begin{scope}
        \clip [rounded corners=4pt] (0,0) rectangle (4,0.3);

        % 绘制背景
        \fill[gray!10] (0,0) rectangle (4,0.3);

        % 绘制进度条(15%进度)带圆角
        \begin{scope}
            \clip [rounded corners=4pt] (0,0) rectangle (4*0.15,0.3);
            \shade[left color=green!70!black,
                   right color={rgb,255: red,67; green,160; blue,71}]
                (0,0) rectangle (4*0.15,0.3);
        \end{scope}
    \end{scope}

    % 绘制圆角边框
    \draw[gray!10, rounded corners=4pt] (0,0) rectangle (4,0.3);
\end{tikzpicture}
}
{\small 0.15}
\raisebox{0.25ex}{ % 调整此值实现垂直对齐
\tcbox[colback=green!70!black, colframe=green!70!black, arc=1.5mm,
       boxrule=0pt, left=1.5mm, right=1.5mm, top=0.1mm, bottom=0.1mm,
       rounded corners=south, on line]
{\scriptsize\textcolor{white}{\textbf{低风险}}}
}

{\footnotesize {\color{green!50!black!80}
\faInfoCircle} 检测风险值:0.15,属于“低风险”级别。\\
{\scriptsize\textcolor{rgb,255:red,140;green,140;blue,140}{\faChartBar}} 风险评估:虽然目前风险较低,但建议关注以下方面:保持良好的个人卫生习惯,勤洗手,尤其在进食前和如厕后,食用安全卫生的饮食;避免接触感染者;保持环境清洁和通风。\\
{\scriptsize\textcolor{rgb,255:red,140;green,140;blue,140}{\faBookOpen}} 肠道病毒感染是一类由多种肠道病毒引起的传染性疾病,可引起发热、手足口病、疱疹性咽峡炎等症状,多见于儿童。
}
\end{tcolorbox}

\begin{tcolorbox}[
    enhanced,
    colback=customTealBg,
    colframe=customTealBg,
    arc=3mm,
    boxrule=0pt,
    width=\textwidth,
    top=8pt,
    bottom=8pt
] {\large \textbf{过敏性腹泻}}
\hspace{0.1em} % 增加间距
\raisebox{-0.2ex}{
    \begin{tikzpicture}
    % 使用clip命令确保内容在圆角矩形内
    \begin{scope}
        \clip [rounded corners=4pt] (0,0) rectangle (4,0.3);

        % 绘制背景
        \fill[gray!10] (0,0) rectangle (4,0.3);

        % 绘制进度条(15%进度)带圆角
        \begin{scope}
            \clip [rounded corners=4pt] (0,0) rectangle (4*0.03,0.3);
            \shade[left color=green!70!black,
                   right color={rgb,255: red,67; green,160; blue,71}]
                (0,0) rectangle (4*0.03,0.3);
        \end{scope}
    \end{scope}

    % 绘制圆角边框
    \draw[gray!10, rounded corners=4pt] (0,0) rectangle (4,0.3);
\end{tikzpicture}
}
{\small 0.03}
\raisebox{0.35ex}{ % 调整此值实现垂直对齐
\tcbox[colback=green!70!black, colframe=green!70!black, arc=1.5mm,
       boxrule=0pt, left=1.5mm, right=1.5mm, top=0.1mm, bottom=0.1mm,
       rounded corners=south, on line]
{\scriptsize\textcolor{white}{\textbf{低风险}}}
}

{\footnotesize {\scriptsize\color{green!50!black!80}
\faInfoCircle} 检测风险值:0.03,属于“低风险”级别。\\
{\scriptsize\textcolor{rgb,255:red,140;green,140;blue,140}{\faChartBar}} 风险评估:尽管为低风险,您仍需注意在日常生活中的饮食和生活习惯注意识别并避免过敏食物;记录可能的过敏原,建立食物日志等。\\
{\scriptsize\textcolor{rgb,255:red,140;green,140;blue,140}{\faBookOpen}} 过敏性腹泻是一种由食物或其他过敏原引起的消化系统反应,可导致腹痛、腹泻等消化道症状,与个体免疫系统对特定物质的敏感性有关。}
\end{tcolorbox}

\begin{tcolorbox}[
    enhanced,
    colback=customTealBg,
    colframe=customTealBg,
    arc=3mm,
    boxrule=0pt,
    width=\textwidth,
    top=8pt,
    bottom=8pt
] {\large \textbf{肝病}}
\hspace{0.1em} % 增加间距
\raisebox{-0.2ex}{
    \begin{tikzpicture}
    % 使用clip命令确保内容在圆角矩形内
    \begin{scope}
        \clip [rounded corners=4pt] (0,0) rectangle (4,0.3);

        % 绘制背景
        \fill[gray!10] (0,0) rectangle (4,0.3);

        % 绘制进度条(30%进度)带圆角
        \begin{scope}
            \clip [rounded corners=4pt] (0,0) rectangle (4*0.30,0.3);
            \shade[left color={rgb,255: red,255; green,215; blue,95},
                   right color={rgb,255: red,255; green,215; blue,95}]
                (0,0) rectangle (4*0.30,0.3);
        \end{scope}
    \end{scope}

    % 绘制圆角边框
    \draw[gray!10, rounded corners=4pt] (0,0) rectangle (4,0.3);
\end{tikzpicture}
}
\small 0.30
\raisebox{0.25ex}{ % 调整此值实现垂直对齐
\tcbox[colback=yellow!90!black, colframe=yellow!90!black, arc=1.5mm,
       boxrule=0pt, left=1.5mm, right=1.5mm, top=0.1mm, bottom=0.1mm,
       rounded corners=south, on line]
{\scriptsize\textcolor{white}{\textbf{注意}}}
}

{\footnotesize
{\scriptsize{\color{yellow!90!black}
\faInfoCircle}} 检测风险值:0.30,属于“注意”级别。\\
{\scriptsize\textcolor{rgb,255:red,140;green,140;blue,140}{\faChartBar}} 风险评估:可能反映出近期生活方式不当或其他疾病导致的风险上升,也可能受到其他疾病的影响。此外,部分健康个体也可能在这个分值段。注意饮食卫生和营养均衡,避免饮酒等肝脏负担,出现不适症状及时就医进行常规肝功能检。\\
{\scriptsize\textcolor{rgb,255:red,140;green,140;blue,140}{\faBookOpen}}肝病是指各种影响肝脏功能和结构的疾病统称,包括病毒性肝炎(如乙肝、丙肝等)、脂肪肝、酒精性肝病、自身免疫性肝病和肝硬化等。
}
\end{tcolorbox}

\begin{tcolorbox}[
    enhanced,
    colback=customTealBg,
    colframe=customTealBg,
    arc=3mm,
    boxrule=0pt,
    width=\textwidth,
    top=8pt,
    bottom=8pt
] {\large \textbf{心脑血管疾病}}
\hspace{0.1em} % 增加间距
\raisebox{-0.2ex}{
    \begin{tikzpicture}
    % 使用clip命令确保内容在圆角矩形内
    \begin{scope}
        \clip [rounded corners=4pt] (0,0) rectangle (4,0.3);

        % 绘制背景
        \fill[gray!10] (0,0) rectangle (4,0.3);

        % 绘制进度条(15%进度)带圆角
        \begin{scope}
            \clip [rounded corners=4pt] (0,0) rectangle (4*0.15,0.3);
            \shade[left color=green!70!black,
                   right color={rgb,255: red,67; green,160; blue,71}]
                (0,0) rectangle (4*0.15,0.3);
        \end{scope}
    \end{scope}

    % 绘制圆角边框
    \draw[gray!10, rounded corners=4pt] (0,0) rectangle (4,0.3);
\end{tikzpicture}
%     \begin{tikzpicture}
%     % 底部阴影效果
%     \fill[gray!30, rounded corners=4pt] (0.05,-0.05) rectangle (4.05,0.25);

%     % 主体进度条容器
%     \begin{scope}
%         \clip [rounded corners=4pt] (0,0) rectangle (4,0.3);

%         % 背景渐变效果
%         \shade[top color=gray!5, bottom color=gray!20]
%             (0,0) rectangle (4,0.3);

%         % 进度条渐变效果(15%进度)
%         \shade[top color={rgb,255: red,87; green,180; blue,91},
%                bottom color={rgb,255: red,67; green,160; blue,71}]
%             (0,0) rectangle (4*0.15,0.3);

%         % 添加高光效果
%         \fill[white,opacity=0.3]
%             (0,0.2) rectangle (4,0.3);
%     \end{scope}

%     % 边框效果
%     \draw[gray!30, rounded corners=4pt] (0,0) rectangle (4,0.3);
% \end{tikzpicture}
% \begin{tikzpicture}
%     % 主体进度条背景
%     \begin{scope}
%         \clip [rounded corners=4pt] (0,0) rectangle (4,0.3);

%         % 背景 - 纯灰色,加入立体感
%         \fill[gray!15] (0,0) rectangle (4,0.3);
%         \fill[white,opacity=0.4] (0,0.2) rectangle (4,0.3);
%         \fill[black,opacity=0.1] (0,0) rectangle (4,0.1);

%         % 进度条 - 绿色渐变(15%进度)带立体感
%         \begin{scope}
%             \clip [rounded corners=4pt] (0,0) rectangle (4*0.15,0.3);
%             % 主体颜色
%             \shade[top color={rgb,255: red,97; green,190; blue,101},
%                    bottom color={rgb,255: red,67; green,160; blue,71}]
%                 (0,0) rectangle (4*0.15,0.3);
%             % 添加高光和阴影效果增加立体感
%             \fill[white,opacity=0.4] (0,0.2) rectangle (4*0.15,0.3);
%             \fill[black,opacity=0.1] (0,0) rectangle (4*0.15,0.1);
%         \end{scope}
%     \end{scope}

%     % 边框效果
%     \draw[gray!30,rounded corners=4pt] (0,0) rectangle (4,0.3);
% \end{tikzpicture}
% \begin{tikzpicture}
%     % 主体进度条背景
%     \begin{scope}
%         \clip [rounded corners=4pt] (0,0) rectangle (3,0.25);

%         % 背景 - 浅灰色
%         \fill[gray!12] (0,0) rectangle (3,0.25);

%         % 进度条 - 绿色(15%进度)
%         \begin{scope}
%             \clip [rounded corners=4pt] (0,0) rectangle (3*0.15,0.25);
%             % 主体颜色
%             \shade[top color={rgb,255: red,82; green,175; blue,86},
%                    bottom color={rgb,255: red,67; green,160; blue,71}]
%                 (0,0) rectangle (3*0.15,0.25);
%             % subtle highlight
%             \fill[white,opacity=0.2] (0,0.125) rectangle (3*0.15,0.25);
%         \end{scope}
%     \end{scope}

%     % 细边框
%     \draw[gray!25,rounded corners=4pt] (0,0) rectangle (3,0.25);

%     % 带尾巴的标识框
%     \begin{scope}[xshift=3.2cm, yshift=-0.025cm]
%         % 白色背景和尾巴
%         \fill[white] (0,0) -- (0.6,0) -- (0.6,0.3) -- (0,0.3) -- (0,0.2) -- (-0.1,0.15) -- (0,0.1) -- cycle;
%         % 灰色边框和尾巴
%         \draw[gray!25] (0,0) -- (0.6,0) -- (0.6,0.3) -- (0,0.3) -- (0,0.2) -- (-0.1,0.15) -- (0,0.1) -- cycle;
%         % 百分比文字
%         \node[text=gray!70] at (0.3,0.15) {\footnotesize 15\%};
%     \end{scope}
% \end{tikzpicture}
}
\small 0.15
\raisebox{0.25ex}{ % 调整此值实现垂直对齐
\tcbox[colback=green!70!black, colframe=green!70!black, arc=1.5mm,
       boxrule=0pt, left=1.5mm, right=1.5mm, top=0.1mm, bottom=0.1mm,
       rounded corners=south, on line]
{\scriptsize\textcolor{white}{低风险}}
}

{\footnotesize
{\color{green!50!black!80}
\faInfoCircle} 检测风险值:0.15,属于“低风险”级别。\\
{\scriptsize\textcolor{rgb,255:red,140;green,140;blue,140}{\faChartBar}} 风险评估:虽然目前检测显示为低风险,但心脑血管疾病的发生常与长期的生活方式习惯相关。\\
{\scriptsize\textcolor{rgb,255:red,140;green,140;blue,140}{\faBookOpen}} 心脑血管疾病是一类影响心脏和脑部血管系统的疾病,包括心肌梗塞、脑梗塞等严重疾病。}
\end{tcolorbox}

\begin{tcolorbox}[
    enhanced,
    colback=customTealBg,
    colframe=customTealBg,
    arc=3mm,
    boxrule=0pt,
    width=\textwidth,
    top=8pt,
    bottom=8pt
] {\large \textbf{甲状腺疾病}}
\hspace{0.1em} % 增加间距
\raisebox{-0.2ex}{
    \begin{tikzpicture}
    % 使用clip命令确保内容在圆角矩形内
    \begin{scope}
        \clip [rounded corners=4pt] (0,0) rectangle (4,0.3);

        % 绘制背景
        \fill[gray!10] (0,0) rectangle (4,0.3);

        % 绘制进度条(15%进度)带圆角
        \begin{scope}
            \clip [rounded corners=4pt] (0,0) rectangle (4*0.15,0.3);
            \shade[left color=green!70!black,
                   right color={rgb,255: red,67; green,160; blue,71}]
                (0,0) rectangle (4*0.19,0.3);
        \end{scope}
    \end{scope}

    % 绘制圆角边框
    \draw[gray!10, rounded corners=4pt] (0,0) rectangle (4,0.3);
\end{tikzpicture}
}
\small 0.19
\raisebox{0.25ex}{ % 调整此值实现垂直对齐
\tcbox[colback=green!70!black, colframe=green!70!black, arc=1.5mm,
       boxrule=0pt, left=1.5mm, right=1.5mm, top=0.1mm, bottom=0.1mm,
       rounded corners=south, on line]
{\scriptsize\textcolor{white}{低风险}}
}

{\footnotesize
{\color{green!50!black!80}
\faInfoCircle} 检测风险值:0.19,属于“低风险”级别。\\
{\scriptsize\textcolor{rgb,255:red,140;green,140;blue,140}{\faChartBar}} 风险评估:尽管为低风险,建议仍需在日常生活中:关注甲状腺功能指标,颈部状况以及基础代谢水平等,保持健康的生活方式,定期进行甲状腺功能检查,注意补充适量碘元素,避免过度劳累和压力,保持规律作息,充足睡眠等。\\
{\scriptsize\textcolor{rgb,255:red,140;green,140;blue,140}{\faBookOpen}} 甲状腺疾病是一类影响甲状腺功能的内分泌系统疾病,包括甲状腺功能亢进、甲状腺功能减退等。}
\end{tcolorbox}

\begin{tcolorbox}[
    enhanced,
    colback=customTealBg,
    colframe=customTealBg,
    arc=3mm,
    boxrule=0pt,
    width=\textwidth,
    top=8pt,
    bottom=8pt
] {\large \textbf{肺部疾病}}
\hspace{0.1em} % 增加间距
\raisebox{-0.2ex}{
    \begin{tikzpicture}
    % 使用clip命令确保内容在圆角矩形内
    \begin{scope}
        \clip [rounded corners=4pt] (0,0) rectangle (4,0.3);

        % 绘制背景
        \fill[gray!10] (0,0) rectangle (4,0.3);

        % 绘制进度条(15%进度)带圆角
        \begin{scope}
            \clip [rounded corners=4pt] (0,0) rectangle (4*0.15,0.3);
            \shade[left color=green!70!black,
                   right color={rgb,255: red,67; green,160; blue,71}]
                (0,0) rectangle (4*0.15,0.3);
        \end{scope}
    \end{scope}

    % 绘制圆角边框
    \draw[gray!10, rounded corners=4pt] (0,0) rectangle (4,0.3);
\end{tikzpicture}
}
\small 0.15
\raisebox{0.25ex}{ % 调整此值实现垂直对齐
\tcbox[colback=green!70!black, colframe=green!70!black, arc=1.5mm,
       boxrule=0pt, left=1.5mm, right=1.5mm, top=0.1mm, bottom=0.1mm,
       rounded corners=south, on line]
{\scriptsize\textcolor{white}{低风险}}
}

{\footnotesize
{\color{green!50!black!80}
\faInfoCircle} 检测风险值:0.15,属于“低风险”级别。\\
{\scriptsize\textcolor{rgb,255:red,140;green,140;blue,140}{\faChartBar}} 风险评估:虽然目前风险较低,但还是建议在日常生活中:保持健康的生活方式,避免吸烟及二手烟暴露,注意防护,减少空气污染暴露,保持规律运动,增强肺功能,保持室内通风,维持良好空气质量等。\\
{\scriptsize\textcolor{rgb,255:red,140;green,140;blue,140}{\faBookOpen}} 肺部疾病是一类影响呼吸系统功能的疾病,包括慢性支气管炎、肺炎、哮喘等。
}
\end{tcolorbox}

\begin{tcolorbox}[
    enhanced,
    colback=customTealBg,
    colframe=customTealBg,
    arc=3mm,
    boxrule=0pt,
    width=\textwidth,
    top=8pt,
    bottom=8pt
] {\large \textbf{神经行为发育异常}}
\hspace{0.1em} % 增加间距
\raisebox{-0.2ex}{
    \begin{tikzpicture}
    % 使用clip命令确保内容在圆角矩形内
    \begin{scope}
        \clip [rounded corners=4pt] (0,0) rectangle (4,0.3);

        % 绘制背景
        \fill[gray!10] (0,0) rectangle (4,0.3);

        % 绘制进度条(30%进度)带圆角
        \begin{scope}
            \clip [rounded corners=4pt] (0,0) rectangle (4*0.30,0.3);
            \shade[left color={rgb,255: red,255; green,215; blue,95},
                   right color={rgb,255: red,255; green,215; blue,95}]
                (0,0) rectangle (4*0.39,0.3);
        \end{scope}
    \end{scope}

    % 绘制圆角边框
    \draw[gray!10, rounded corners=4pt] (0,0) rectangle (4,0.3);
\end{tikzpicture}
}
\small 0.39
\raisebox{0.25ex}{ % 调整此值实现垂直对齐
\tcbox[colback=yellow!90!black, colframe=yellow!90!black, arc=1.5mm,
       boxrule=0pt, left=1.5mm, right=1.5mm, top=0.1mm, bottom=0.1mm,
       rounded corners=south, on line]
{\scriptsize\textcolor{white}{\textbf{注意}}}
}

{\footnotesize
{\scriptsize{\color{yellow!90!black}
\faInfoCircle}} 检测风险值:0.39,属于“注意”级别。\\
{\scriptsize\textcolor{rgb,255:red,140;green,140;blue,140}{\faChartBar}} 风险评估:可能反映出近期生活方式不当或其他疾病导致的风险上升,也可能受到其他疾病的影响。要注意保持充足的睡眠,健康饮食和规律运动等。若发现记忆力下降、注意力难以集中、情绪波动、冲动行为或认知功能受损等症状,请及时就医。\\
{\scriptsize\textcolor{rgb,255:red,140;green,140;blue,140}{\faBookOpen}} 神经行为异常通常是指由于中枢神经系统功能紊乱或损伤,所引发的行为和情绪上的异常表现,例如自闭症,抑郁症,帕金森病等。
}
\end{tcolorbox}

\newpage

\begin{tcolorbox}[
    enhanced,
    colback=white,
    colframe=customTeal,
    arc=2mm,
    boxrule=1pt,
    left=20pt,
    right=20pt,
    top=12pt,
    bottom=12pt,
    width=\textwidth,
    fontupper=\sffamily,
    overlay={
    \draw[customTeal, line width=2pt]
    ([xshift=15pt]frame.south west) -- ([xshift=-15pt]frame.south east);
    }
]
{\Large\bfseries\textcolor{customTeal}{\Huge 疾病相关菌群}}
\end{tcolorbox}

\begin{tcolorbox}[
    enhanced,
    colback=customTealBg,
    colframe=customTealBg,
    arc=3mm,
    boxrule=0pt,
    width=\textwidth,
    top=8pt,
    bottom=8pt
]
{\small{\color{customTeal}\faInfoCircle} 肠道菌群与疾病的关联性研究已成为现代医学研究的重要领域。通过大规模人群研究和临床观察,我们发现多种健康问题与肠道菌群的特定丰度水平直接相关。\\

{\color{orange}\faExclamationTriangle} \textbf{特别注意}:
\begin{itemize}
    \item 在本报告中,与疾病相关的某些肠道菌群可能会出现异常水平。然而,值得注意的是,这些菌群的丰度水平的异常并不一定会导致个体出现该疾病症状。这是因为疾病的发生通常涉及多种复杂因素,包括个体的代谢状态、遗传因素、饮食习惯、运动水平、睡眠质量以及整体的心理和身体健康状况。 \item 为了更好地规避或降低该疾病的潜在风险,您应结合以下菌群检测情况以及自身的生活习惯、病史等因素来综合评估该疾病风险。在出现该疾病症状或其他相关健康问题时,应及时咨询专业医生或营养师获取治疗建议。
\end{itemize}
}
\end{tcolorbox}

\begin{tcolorbox}[
    enhanced,
    colback=lightpurple!10, % 卡片底色
    colframe=lightpurple!10,  % 边框颜色
    arc=3mm,
    boxrule=0.5pt,
    width=\textwidth,
    top=8pt,
    bottom=8pt
]
{\small{\color{lightpurple}\faQuestionCircle}\quad \textbf{报告中会涉及到哪些疾病?}\\
{\color{orange!50}\faComments}\quad 以下是本报告中会涉及到的疾病:
\begin{itemize}
    \item \textbf{肥胖}:指体内脂肪过多,通常由不良饮食习惯和缺乏运动引起。
    \item \textbf{便秘}:指排便频率减少或排便困难,可能由饮食纤维摄入不足、缺乏运动或心理因素引起。
    \item \textbf{腹胀}:指腹部感到胀满和不适,通常由进食过快、饮食不当或肠道微生物失衡引起。
    \item \textbf{过敏}:指免疫系统对本无害物质(如花粉、食物)产生异常反应的情况。过敏反应可以是轻微的,也可能是严重的,
    \item \textbf{抑郁}:指情绪低落、兴趣缺乏等状态,可能由多种因素引起,包括生活压力、人际关系等。
    \item \textbf{失眠}:指难以入睡或维持睡眠的状态,常由压力、焦虑、生活方式影响产生。
\end{itemize}
}
\end{tcolorbox}

\begin{tcolorbox}[
    enhanced,
    colback=lightpurple!10, % 卡片底色
    colframe=lightpurple!10,  % 边框颜色
    arc=3mm,
    boxrule=0.5pt,
    width=\textwidth,
    top=8pt,
    bottom=8pt
]
{\small{\color{lightpurple}\faQuestionCircle}\quad \textbf{如何阅读以下的疾病相关菌的检测表格?}\\
{\color{orange!50}\faComments}\quad 以下是表格各列的解释
\begin{itemize}
    \item \textbf{菌种名称}:与疾病相关的肠道菌群的中文学术名和拉丁名。
    \item \textbf{正常范围}:该疾病相关菌种在健康人群中的丰度范围。
    \item \textbf{检测丰度}:实际检测到的该疾病相关菌种的数量,其中ND代表该菌种的数量过低未检测到。
    \item \textbf{结果评价}:根据检测丰度与正常范围的比较,给出的健康状态评估数值。
    \item \textbf{超过\%的人}:表示您肠道检测的该疾病相关菌属的丰度值比人群中\%的人要高。(-\%超标过高不予显示)
    \item \textbf{疾病相关性}:该疾病相关菌种和疾病的相关性,正相关指的是该菌种的丰度增加可能会加剧疾病症状表现,负相关指的是该菌种丰都的降低可能会加剧疾病的症状表现。
    \item \textbf{相关性强度}:该菌种和腹胀相关性的证据支持强度。相关性强度以{\small\color{lightgray}\faStar}到{\small\color{lightgray}\faStar \faStar \faStar}依次递增来评定。
    \begin{itemize}
        \item \small{\color{lightgray}\faStar}:来自单篇论文或人群数据差异统计。
        \item \small{\color{lightgray}\faStar\faStar}:大规模人群样本统计和至少单篇论文。
        \item \small{\color{lightgray}\faStar\faStar\faStar}:大量研究论文证实。
    \end{itemize}
\end{itemize}
建议在异常丰度的菌群中重点关注那些与疾病正相关的超标菌群和与疾病负相关的缺乏菌群,因为他们的增加超标和减少缺乏可能会增大疾病患病风险或者加剧现有疾病症状。
}
\end{tcolorbox}

\newpage

\definecolor{GoldenOrange}{RGB}{255, 200, 0}

\begin{tcolorbox}[
    enhanced,
    colback=white,
    colframe=white,
    boxrule=0pt,
    width=\textwidth,
    left=0pt,
    right=0pt,
    top=0pt,
    bottom=25pt
]
    \begin{tikzpicture}
        % 主背景(带圆角)
        \fill[customTeal, rounded corners=9pt] (0,0) rectangle (\textwidth,1.5);
        % 装饰色块(带圆角)
        \fill[customTeal!70, rounded corners=9pt] (0,0) rectangle (1,1.5);

        % 标题文字和图标在一起
        \node[anchor=west, text=white] at (1.5,0.75) {
            \fontsize{15}{31}\selectfont\textbf{肥胖相关菌} \Large\faBacteria
        };
    \end{tikzpicture}
\end{tcolorbox}
\vspace{-1.0cm}
\begin{tcolorbox}[
    enhanced,
    colback=customTealBg,
    colframe=customTealBg,
    arc=3mm,
    boxrule=0pt,
    width=\textwidth,
    top=8pt,
    bottom=8pt
]
{\small{\color{customTeal}\faInfoCircle}
肠道菌群通过影响能量吸收、脂肪代谢和免疫功能,在肥胖发生发展中扮演重要角色。\\

{\color{orange}\faExclamationTriangle} \textbf{特别注意}:
\begin{itemize}
    \item 在本报告中,与肥胖相关的某些肠道菌群可能会出现异常水平。然而,值得注意的是,这些菌群的丰度水平的异常并不一定会导致个体出现肥胖症状。这是因为肥胖的发生通常涉及多种复杂因素,包括个体的代谢状态、遗传因素、饮食习惯、运动水平、睡眠质量以及整体的心理和身体健康状况。 \item 为了更好地规避潜在的肥胖风险,您应结合以下菌群情况以及自身的生活习惯、病史等因素来综合评估肥胖风险。在出现体重增加、能量低下或其他相关健康问题时,应及时咨询专业医生或营养师,以获取准确的评估和进一步的治疗建议。
\end{itemize}
}
\end{tcolorbox}
\vspace{-0.7cm}
\begin{center}
\begin{tikzpicture}[
    font=\small,
    title/.style={font=\small\bfseries\color{white}},
    value/.style={font=\small},
    reference/.style={font=\small},
    cell/.style={anchor=west, text width=4.2cm},
    note/.style={anchor=west, text width=4.5cm, align=left}
]
    \def\cardwidth{\textwidth}
    \def\cardheight{14.7}
    \def\barheight{0.25}
    \def\barwidth{1.5}
    \def\valuebarspace{0.4}

    % 容器和标题栏背景
    \draw[rounded corners=5, fill=white, draw=gray!20]
        (0,0) rectangle (\cardwidth,-\cardheight);
    \path[fill=customTeal!70]
        (0,0) [rounded corners=5] -- (\cardwidth,0) --
        (\cardwidth,0.8) -- (0,0.8) -- cycle;

    % 表头
    \node[title, anchor=west] at (0.5,0.4) {\textbf{菌种名称}};
    \node[title] at (4.5,0.4) {\textbf{正常范围\%}};
    \node[title] at (7.5,0.4) {\textbf{检测丰度}};
    \node[title] at (9.5,0.4) {\textbf{检测结果}}; % 新增检测结果列名
    \node[title] at (11.5,0.4) {\textbf{超过\%的人}};
    \node[title] at (13.75,0.4) {\textbf{失眠相关性}};
    \node[title] at (16,0.4) {\textbf{相关性强度}};

    % 初始化位置计数器
    \def\currentpos{0.25}

    % 数据行和卡片
    \foreach \item/\enitem/\value/\range/\percentile/\detection/\status/\intro/\suggestion/\index in {
        {厚壁菌门}/{Firmicutes}/94.36370/{28.594-83.961}/99\%/正相关/超标/{Firmicutes菌门的数量在肥胖人群中较高,可以从食物中提取更多的能量,并将其储存为脂肪。}/{建议:碳水化合物、植物蛋白、豆制材、低能量饮食/热量限制、全麦、杂芝麻丝体}/{\color{GoldenOrange}\faStar \faStar \faStar}/\currentpos,
        {阿德勒氏菌属}/{Adlercreutzia}/0.00133/{0-0.74966}/99\%/正相关/正常/{能够代谢大豆异黄酮,促进女性肥胖。}/{保持健康的饮食习惯,适量摄入大豆制品。}/{\color{orange}\faStar \faStar \faStar}/\currentpos,
        {双歧杆菌属}/{Bifidobacterium}/0.03296/{1.7545796-35.500554}/24\%/正相关/偏低/{促进肠道内有益菌的生长,抑制有害菌的生长,减少肥胖相关的炎症反应代谢素乱。}/{建议增加益生菌和膳食纤维的摄入。}/{\color{orange}\faStar \faStar \faStar}/\currentpos,
        {嗜胆菌属}/{Bilophila}/0.73110/{0-0.1544794}/-\%/正相关/超标/{Bilophila wadsworthia是一种能够产生硫化物的细菌,其在肥胖人群中的丰度较高。}/{建议:鼠李糖乳杆菌、低聚果糖和菊粉、酵母β-葡聚糖、水果和蔬菜}/{\color{orange}\faStar \faStar \faStar}/\currentpos,
        {梭菌属}/{Clostridium}/1.90370/{0-4.4645924}/67\%/正相关/正常/{肥胖者的丰度较高,尤其Clostridium butyricum,可通过增强营养吸收促进肥胖的发展。}/{保持均衡饮食,控制热量摄入。}/{\color{orange}\faStar \faStar \faStar}/\currentpos,
        {埃希氏菌属}/{Escherichia}/0.14296/{0-3.83}/6\%/正相关/正常/{肥胖人群丰度明显增加,能够产生酒精,从而导致血液中酒精水平升高,也可能是肥胖人群易患非酒精性脂肪性肝病的原因之一。}/{建议保持健康的饮食习惯,控制热量摄入。}/{\color{orange}\faStar \faStar \faStar}/\currentpos,
        {霍尔德曼氏菌属}/{Holdemania}/0.10581/{0-0.28}/98\%/正相关/正常/{在肥胖人群中明显增加,与神经炎症、脂质和葡萄糖代谢障碍相关,与肥胖相关的肝硬化、糖尿病和代谢综合征等疾病有关。}/{建议调整饮食结构,增加膳食纤维摄入。}/{\color{orange}\faStar \faStar \faStar}/\currentpos
    }
    {
        % 计算当前行的基础位置
        \pgfmathsetmacro{\basepos}{-2.8*\currentpos}

        % 菌种名称
        \node[cell, align=left] at (0.5,\basepos) {
            \small\textbf{\item}\\[-0.2em]
            {\color{lightgray}\small\enitem}
        };

        % 正常范围
        \node[reference] at (4.5,\basepos) {\footnotesize\range};

        % 进度条相关
        \pgfmathsetmacro{\barypos}{\basepos-\valuebarspace+0.1}
        \def\barstart{6.75}

        % 进度条背景
        \fill[gray!10, rounded corners=2] (\barstart,\barypos)
            rectangle (\barstart+\barwidth,\barypos+\barheight);

        % 检测丰度值
        \node[value] at (7.5,{\basepos-\valuebarspace+0.6}) {\footnotesize\value};

        % 解析范围并计算进度条长度
        \def\parserange#1-#2\endparse{\def\minval{#1}\def\maxval{#2}}
        \expandafter\parserange\range\endparse

        % 计算进度条长度和颜色
        \pgfmathsetmacro{\progress}{min(\value/\maxval, 1.0)}
        \pgfmathparse{\value > \maxval ? "customred" : (\value < \minval ? "customred" : "green!50")}
        \let\barcolor=\pgfmathresult

        % 进度条显示
        \ifnum\pdfstrcmp{\status}{超标}=0
            \fill[customred, rounded corners=2] (\barstart,\barypos)
                rectangle (\barstart+\barwidth,\barypos+\barheight);
        \else
            \fill[\barcolor, rounded corners=2] (\barstart,\barypos)
                rectangle (\barstart+\barwidth*\progress,\barypos+\barheight);
        \fi

        % 检测结果
%        \node[value] at (9.5,\basepos) {\footnotesize\status}; % 检测结果列
        % 检测结果
        \ifnum\pdfstrcmp{\status}{正常}=0

        \node[value, text=customGreen] at (9.5,\basepos) {\footnotesize\textbf{\status}}; % 检测结果为正常
        \else

        \node[value, text=customRed] at (9.5,\basepos) {\footnotesize\textbf{\status}}; % 检测结果为其他状态
        \fi


        % 超过%的人
        \node[value] at (11.5,\basepos) {\footnotesize\percentile};

        % 肥胖相关性
        \node[value] at (13.75,\basepos) {\footnotesize\textbf{\detection}};

        % 相关性强度
        \node[value] at (16,\basepos) {\footnotesize\textbf{\index}};

        % 添加卡片
        \pgfmathsetmacro{\cardypos}{\basepos-0.5}
        \begin{scope}[shift={(0,\cardypos)}]
            % 卡片背景
            \pgfmathsetmacro{\cardheight}{
                \ifnum\pdfstrcmp{\status}{超标}=0
                    1.0  % 两行内容时的高度
                \else
                    0.6  % 一行内容时的高度
                \fi
            }

            \fill[rounded corners=5pt, customTeal!5, draw=gray!5]
                (0.3,-\cardheight) rectangle (17.3,0);

            % 菌群简介图标和内容
            \node[anchor=west] at (0.5,-0.3) {
                \textbf{\color{gray!90}\footnotesize \textcolor{customTeal}{\faInfoCircle}}
            };
            \node[anchor=west, text width=16cm] at (0.9,-0.3) {
                {\small\color{gray}\footnotesize \intro}
            };

            % 异常解读标题和内容
            \ifnum\pdfstrcmp{\status}{超标}=0
                \node[anchor=west] at (0.55,-0.8) {
                    \textbf{\color{customRed}\footnotesize \textcolor{customRed}{\faBell}}
                };
                \node[anchor=west, text width=16cm] at (0.9,-0.8) {
                    {\small\color{gray}\footnotesize \suggestion}
                };
            \fi
        \end{scope}

        % 分割线
        \pgfmathsetmacro{\linepos}{
            \ifnum\pdfstrcmp{\status}{超标}=0
                \basepos-1.7  % 超标时的分割线位置
            \else
                \basepos-1.3  % 正常时的分割线位置
            \fi
        }
        \draw[gray!20] (0.2,\linepos) -- (\cardwidth-0.2,\linepos);

        % 根据当前行的状态计算下一行的位置增量
        \ifnum\pdfstrcmp{\status}{超标}=0
            \pgfmathsetmacro{\increment}{0.85}  % 超标行(两行内容)需要更大的增量
        \else
            \pgfmathsetmacro{\increment}{0.7}  % 正常行(一行内容)使用较小的增量
        \fi

        % 更新位置计数器
        \pgfmathsetmacro{\nextpos}{\currentpos+\increment}
        \xdef\currentpos{\nextpos}
    }

    % 最后一行的处理,消除多余的空白
    \pgfmathsetmacro{\lastincrement}{2}  % 最后一行的增量
    \pgfmathsetmacro{\nextpos}{\currentpos+\lastincrement}
    \xdef\currentpos{\nextpos}

\end{tikzpicture}
\end{center}

\newpage

\begin{center}
\begin{tikzpicture}[
    font=\small,
    title/.style={font=\small\bfseries\color{white}},
    value/.style={font=\small},
    reference/.style={font=\small},
    cell/.style={anchor=west, text width=4.2cm},
    note/.style={anchor=west, text width=4.5cm, align=left}
]
    \def\cardwidth{\textwidth}
    \def\cardheight{22.5}
    \def\barheight{0.25}
    \def\barwidth{1.5}
    \def\valuebarspace{0.4}

    % 容器和标题栏背景
    \draw[rounded corners=5, fill=white, draw=gray!20]
        (0,0) rectangle (\cardwidth,-\cardheight);
    \path[fill=customTeal]
        (0,0) [rounded corners=5] -- (\cardwidth,0) --
        (\cardwidth,0.8) -- (0,0.8) -- cycle;

    % 表头
    \node[title, anchor=west] at (0.5,0.4) {\textbf{菌种名称}};
    \node[title] at (4.5,0.4) {\textbf{正常范围\%}};
    \node[title] at (7.5,0.4) {\textbf{检测丰度}};
    \node[title] at (9.5,0.4) {\textbf{检测结果}}; % 新增检测结果列名
    \node[title] at (11.5,0.4) {\textbf{超过\%的人}};
    \node[title] at (13.75,0.4) {\textbf{失眠相关性}};
    \node[title] at (16,0.4) {\textbf{相关性强度}};

    % 初始化位置计数器
    \def\currentpos{0.25}

    % 数据行和卡片
    \foreach \item/\enitem/\value/\range/\percentile/\detection/\status/\intro/\suggestion/\index in {
        {厚壁菌门}/{Firmicutes}/94.36370/{28.594-83.961}/99\%/正相关/超标/{Firmicutes菌门的数量在肥胖人群中较高,可以从食物中提取更多的能量,并将其储存为脂肪。}/{建议:碳水化合物、植物蛋白、豆制材、低能量饮食/热量限制、全麦、杂芝麻丝体}/{\color{orange}\faStar \faStar \faStar}/\currentpos,
        {阿德勒氏菌属}/{Adlercreutzia}/0.00133/{0-0.74966}/99\%/正相关/正常/{能够代谢大豆异黄酮,促进女性肥胖。}/{保持健康的饮食习惯,适量摄入大豆制品。}/{\color{orange}\faStar \faStar \faStar}/\currentpos,
        {双歧杆菌属}/{Bifidobacterium}/0.03296/{1.7545796-35.500554}/24\%/正相关/偏低/{促进肠道内有益菌的生长,抑制有害菌的生长,减少肥胖相关的炎症反应代谢素乱。}/{建议增加益生菌和膳食纤维的摄入。}/{\color{orange}\faStar \faStar \faStar}/\currentpos,
        {嗜胆菌属}/{Bilophila}/0.73110/{0-0.1544794}/-\%/正相关/超标/{Bilophila wadsworthia是一种能够产生硫化物的细菌,其在肥胖人群中的丰度较高。}/{建议:鼠李糖乳杆菌、低聚果糖和菊粉、酵母β-葡聚糖、水果和蔬菜}/{\color{orange}\faStar \faStar \faStar}/\currentpos,
        {梭菌属}/{Clostridium}/1.90370/{0-4.4645924}/67\%/正相关/正常/{肥胖者的丰度较高,尤其Clostridium butyricum,可通过增强营养吸收促进肥胖的发展。}/{保持均衡饮食,控制热量摄入。}/{\color{orange}\faStar \faStar \faStar}/\currentpos,
        {埃希氏菌属}/{Escherichia}/0.14296/{0-3.83}/6\%/正相关/正常/{肥胖人群丰度明显增加,能够产生酒精,从而导致血液中酒精水平升高,也可能是肥胖人群易患非酒精性脂肪性肝病的原因之一。}/{建议保持健康的饮食习惯,控制热量摄入。}/{\color{orange}\faStar \faStar \faStar}/\currentpos,
        {霍尔德曼氏菌属}/{Holdemania}/0.10581/{0-0.28}/98\%/正相关/正常/{在肥胖人群中明显增加,与神经炎症、脂质和葡萄糖代谢障碍相关,与肥胖相关的肝硬化、糖尿病和代谢综合征等疾病有关。}/{建议调整饮食结构,增加膳食纤维摄入。}/{\color{orange}\faStar \faStar \faStar}/\currentpos,
        {巨单胞菌属}/{Megamonas}/0.00049/{0-0.6915718}/58\%/正相关/正常/{肥胖人群丰度较高,与糖尿病、炎症反应等代谢性疾病密切相关。}/{建议增加运动量,控制饮食摄入。}/{\color{orange}\faStar \faStar \faStar}/\currentpos,
        {震颤杆菌属}/{Oscillibacter}/0.17179/{0-3.1958}/78\%/正相关/正常/{在肥胖症患者中的丰度明显增加,与肥胖症的发生和发展密切相关,增加可能导致肠道菌群失衡,从而影响能量代谢和肥胖症的发生。}/{建议调整饮食结构,增加益生菌摄入。}/{\color{orange}\faStar \faStar \faStar}/\currentpos,
        {丹毒丝菌科}/{Erysipelotrichaceae}/0.34971/{0-1.7163}/97\%/正相关/正常/{在肥胖人群中明显增加,与肥胖相关的代谢紊乱和炎症反应有关。}/{建议增加运动量,注意饮食均衡。}/{\color{orange}\faStar \faStar \faStar}/\currentpos,
        {红蝽菌科}/{Coriobacteriaceae}/0.01926/{0-10.241}/82\%/正相关/正常/{通过调节胆固醇吸收来影响能量代谢,与高脂饮食下的肠病和肥胖的抵抗力相关。}/{建议控制脂肪摄入,保持健康的生活方式。}/{\color{orange}\faStar \faStar \faStar}/\currentpos
    }
    {
        % 计算当前行的基础位置
        \pgfmathsetmacro{\basepos}{-2.8*\currentpos}

        % 菌种名称
        \node[cell, align=left] at (0.5,\basepos) {
            \small\textbf{\item}\\[-0.2em]
            {\color{lightgray}\small\enitem}
        };

        % 正常范围
        \node[reference] at (4.5,\basepos) {\footnotesize\range};

        % 进度条相关
        \pgfmathsetmacro{\barypos}{\basepos-\valuebarspace+0.1}
        \def\barstart{6.75}

        % 进度条背景
        \fill[gray!10, rounded corners=2] (\barstart,\barypos)
            rectangle (\barstart+\barwidth,\barypos+\barheight);

        % 检测丰度值
        \node[value] at (7.5,{\basepos-\valuebarspace+0.6}) {\footnotesize\value};

        % 解析范围并计算进度条长度
        \def\parserange#1-#2\endparse{\def\minval{#1}\def\maxval{#2}}
        \expandafter\parserange\range\endparse

        % 计算进度条长度和颜色
        \pgfmathsetmacro{\progress}{min(\value/\maxval, 1.0)}
        \pgfmathparse{\value > \maxval ? "customred" : (\value < \minval ? "customred" : "green!50")}
        \let\barcolor=\pgfmathresult

        % 进度条显示
        \ifnum\pdfstrcmp{\status}{超标}=0
            \fill[customred, rounded corners=2] (\barstart,\barypos)
                rectangle (\barstart+\barwidth,\barypos+\barheight);
        \else
            \fill[\barcolor, rounded corners=2] (\barstart,\barypos)
                rectangle (\barstart+\barwidth*\progress,\barypos+\barheight);
        \fi

        % 检测结果
        \node[value] at (9.5,\basepos) {\footnotesize\status}; % 检测结果列

        % 超过%的人
        \node[value] at (11.5,\basepos) {\footnotesize\percentile};

        % 失眠相关性
        \node[value] at (13.75,\basepos) {\footnotesize \detection};

        % 相关性强度
        \node[value] at (16,\basepos) {\footnotesize \index}};

        % 添加卡片
        \pgfmathsetmacro{\cardypos}{\basepos-0.5}
        \begin{scope}[shift={(0,\cardypos)}]
            % 卡片背景
            \pgfmathsetmacro{\cardheight}{
                \ifnum\pdfstrcmp{\status}{超标}=0
                    1.0  % 两行内容时的高度
                \else
                    0.6  % 一行内容时的高度
                \fi
            }

            \fill[rounded corners=5pt, customTeal!5, draw=gray!5]
                (0.3,-\cardheight) rectangle (17.3,0);

            % 菌群简介图标和内容
            \node[anchor=west] at (0.5,-0.3) {
                \textbf{\color{gray!90}\footnotesize \textcolor{customTeal}{\faInfoCircle}}
            };
            \node[anchor=west, text width=16cm] at (0.9,-0.3) {
                {\small\color{gray}\footnotesize \intro}
            };

            % 异常解读标题和内容
            \ifnum\pdfstrcmp{\status}{超标}=0
                \node[anchor=west] at (0.55,-0.8) {
                    \textbf{\color{customRed}\footnotesize \textcolor{customRed}{\faLightbulb}}
                };
                \node[anchor=west, text width=16cm] at (0.9,-0.8) {
                    {\small\color{gray}\footnotesize \suggestion}
                };
            \fi
        \end{scope}

        % 分割线
        \pgfmathsetmacro{\linepos}{
            \ifnum\pdfstrcmp{\status}{超标}=0
                \basepos-1.7  % 超标时的分割线位置
            \else
                \basepos-1.3  % 正常时的分割线位置
            \fi
        }
        \draw[gray!20] (0.2,\linepos) -- (\cardwidth-0.2,\linepos);

        % 根据当前行的状态计算下一行的位置增量
        \ifnum\pdfstrcmp{\status}{超标}=0
            \pgfmathsetmacro{\increment}{0.85}  % 超标行(两行内容)需要更大的增量
        \else
            \pgfmathsetmacro{\increment}{0.7}  % 正常行(一行内容)使用较小的增量
        \fi

        % 更新位置计数器
        \pgfmathsetmacro{\nextpos}{\currentpos+\increment}
        \xdef\currentpos{\nextpos}
    }

    % 最后一行的处理,消除多余的空白
    \pgfmathsetmacro{\lastincrement}{2}  % 最后一行的增量
    \pgfmathsetmacro{\nextpos}{\currentpos+\lastincrement}
    \xdef\currentpos{\nextpos}

\end{tikzpicture}
\end{center}

\newpage

\begin{center}
\begin{tikzpicture}[
    font=\small,
    title/.style={font=\small\bfseries\color{white}},
    value/.style={font=\small},
    reference/.style={font=\small},
    cell/.style={anchor=west, text width=4.2cm},
    note/.style={anchor=west, text width=4.5cm, align=left}
]
    \def\cardwidth{\textwidth}
    \def\cardheight{22.5}
    \def\barheight{0.25}
    \def\barwidth{1.5}
    \def\valuebarspace{0.4}

    % 容器和标题栏背景
    \draw[rounded corners=5, fill=white, draw=gray!20]
        (0,0) rectangle (\cardwidth,-\cardheight);
    \path[fill=customTeal]
        (0,0) [rounded corners=5] -- (\cardwidth,0) --
        (\cardwidth,0.8) -- (0,0.8) -- cycle;

    % 表头
    \node[title, anchor=west] at (0.5,0.4) {\textbf{菌种名称}};
    \node[title] at (4.5,0.4) {\textbf{正常范围\%}};
    \node[title] at (7.5,0.4) {\textbf{检测丰度}};
    \node[title] at (9.5,0.4) {\textbf{检测结果}}; % 新增检测结果列名
    \node[title] at (11.5,0.4) {\textbf{超过\%的人}};
    \node[title] at (13.75,0.4) {\textbf{失眠相关性}};
    \node[title] at (16,0.4) {\textbf{相关性强度}};

    % 初始化位置计数器
    \def\currentpos{0.25}

    % 数据行和卡片
    \foreach \item/\enitem/\value/\range/\percentile/\detection/\status/\intro/\suggestion/\index in {
        {脆弱拟杆菌}/{Bacteroides fragilis}/0.00145/{0-0.05}/8\%/正相关/正常/{与婴儿3周和26周时的BMI呈正相关,可能通过抑制乙酸水平来加速肥胖,与ALT呈负相关。}/{建议保持健康的饮食习惯,控制体重。}/{\color{lightgray}\faStar \faStar \faStar}/\currentpos,
        {牙龈卟啉单胞菌}/{Porphyromonas gingivalis}/ND/{0-0.05}/-\%/正相关/正常/{Porphyromonas gingivalis是一种牙周病原菌,会导致饮食性肥胖进一步增重和,改变棕色脂肪组织的内分泌功能影响肥胖。}/{注意口腔卫生,定期进行牙齿检查和清洁。}/{\color{lightgray}\faStar \faStar \faStar}/\currentpos,
        {活波瘤胃球菌}/{Ruminococcus gnavus}/0.22486/{0-0.05}/20\%/正相关/超标/{与短链脂肪酸的产生有关,在女性中与ICPP、NAFLD和肥胖等疾病有关。}/{建议:咖啡、乳杆菌补充、白藜芦醇、母乳低聚糖、槲皮素红维素。}/{\color{lightgray}\faStar \faStar \faStar}/\currentpos,
        {别样杆菌属}/{Alistipes}/0.00099/{0.080658-18.1198662}/15\%/正相关/偏低/{在肠道菌群中起重要作用,与代谢健康相关。}/{建议增加膳食纤维的摄入,保持饮食多样性。}/{\color{lightgray}\faStar \faStar}/\currentpos,
        {厌氧棒状菌属}/{Anaerotruncs}/0.08642/{0-0.13747}/55\%/正相关/正常/{在肠道菌群失调,饱和脂肪酸摄入量较高时丰度更高,与肥胖有关。}/{建议控制饱和脂肪酸的摄入,保持健康的饮食习惯。}/{\color{lightgray}\faStar \faStar}/\currentpos,
        {柯林斯氏菌属}/{Collinsella}/0.01763/{0-9.1528862}/82\%/正相关/正常/{增加导致肠道微生物群落的失衡,潜在的促炎症成分增加,如短链脂肪酸减少,可能会导致肥胖和代谢综合征的发生。}/{建议增加益生菌摄入,调节肠道菌群平衡。}/{\color{lightgray}\faStar \faStar}/\currentpos,
        {乳杆菌属}/{Lactobacillus}/0.00660/{0-0.4302374}/8\%/正相关/正常/{肥胖人群中丰度增加,但是一些菌株能够对肥胖产生有益影响,与其抑制脂肪酸合成酶基因表达、降低脂肪酸氧化酶活性有关。}/{建议适量补充益生菌,保持肠道菌群平衡。}/{\color{lightgray}\faStar \faStar}/\currentpos,
        {乳球菌属}/{Lactococcus}/0.00222/{0-0.058488}/99\%/正相关/正常/{肥胖人群的肠道微生物组中丰度较少,而正常体重人群中较多。参与调节肠道内的卡路里代谢和能量平衡,从而影响体重。}/{建议通过饮食调节提高有益菌群的丰度。}/{\color{lightgray}\faStar \faStar}/\currentpos,
        {颤螺菌属}/{Oscillospira}/ND/{0.033-5.346}/36\%/正相关/偏低/{Oscillospira过多或过少都与肥胖相关,益生菌或益生元可以增加Oscillospira的数量,从而减轻肥胖和代谢疾病的症状。}/{建议适量补充益生菌和益生元,调节肠道菌群平衡。}/{\color{lightgray}\faStar \faStar}/\currentpos,
        {副萨特氏菌属}/{Parasutterella}/0.33869/{0-0.8235}/99\%/正相关/正常/{与L-半胱氨酸和脂肪酸生物合成径有关,在肥胖人群中,丰度与BMI和2型糖尿病呈正相关。}/{建议控制饮食摄入,保持健康的生活方式,定期监测血糖水平。}/{\color{lightgray}\faStar \faStar}/\currentpos,
        {龙包沃氏菌属}/{Romboutsia}/0.01546/{0-0.021404}/73\%/正相关/正常/{肠道黏膜屏障功能下降、肠道炎症反应增加,葡萄糖代谢异常等现象有关。}/{建议改善饮食结构,增加膳食纤维摄入,注意调节血糖水平。}/{\color{lightgray}\faStar \faStar}/\currentpos
    }
    {
        % 计算当前行的基础位置
        \pgfmathsetmacro{\basepos}{-2.8*\currentpos}

        % 菌种名称
        \node[cell, align=left] at (0.5,\basepos) {
            \small\textbf{\item}\\[-0.2em]
            {\color{lightgray}\small\enitem}
        };

        % 正常范围
        \node[reference] at (4.5,\basepos) {\footnotesize\range};

        % 进度条相关
        \pgfmathsetmacro{\barypos}{\basepos-\valuebarspace+0.1}
        \def\barstart{6.75}

        % 进度条背景
        \fill[gray!10, rounded corners=2] (\barstart,\barypos)
            rectangle (\barstart+\barwidth,\barypos+\barheight);

        % 检测丰度值
        \node[value] at (7.5,{\basepos-\valuebarspace+0.6}) {\footnotesize\value};

        % 解析范围并计算进度条长度
        \def\parserange#1-#2\endparse{\def\minval{#1}\def\maxval{#2}}
        \expandafter\parserange\range\endparse

        % 计算进度条长度和颜色
        \pgfmathsetmacro{\progress}{min(\value/\maxval, 1.0)}
        \pgfmathparse{\value > \maxval ? "customred" : (\value < \minval ? "customred" : "green!50")}
        \let\barcolor=\pgfmathresult

        % 进度条显示
        \ifnum\pdfstrcmp{\status}{超标}=0
            \fill[customred, rounded corners=2] (\barstart,\barypos)
                rectangle (\barstart+\barwidth,\barypos+\barheight);
        \else
            \fill[\barcolor, rounded corners=2] (\barstart,\barypos)
                rectangle (\barstart+\barwidth*\progress,\barypos+\barheight);
        \fi

        % 检测结果
        \node[value] at (9.5,\basepos) {\footnotesize\status}; % 检测结果列

        % 超过%的人
        \node[value] at (11.5,\basepos) {\footnotesize\percentile};

        % 失眠相关性
        \node[value] at (13.75,\basepos) {\footnotesize\detection};

        % 相关性强度
        \node[value] at (16,\basepos) {\footnotesize\index};

        % 添加卡片
        \pgfmathsetmacro{\cardypos}{\basepos-0.5}
        \begin{scope}[shift={(0,\cardypos)}]
            % 卡片背景
            \pgfmathsetmacro{\cardheight}{
                \ifnum\pdfstrcmp{\status}{超标}=0
                    1.0  % 两行内容时的高度
                \else
                    0.6  % 一行内容时的高度
                \fi
            }

            \fill[rounded corners=5pt, customTeal!5, draw=gray!5]
                (0.3,-\cardheight) rectangle (17.3,0);

            % 菌群简介图标和内容
            \node[anchor=west] at (0.5,-0.3) {
                \textbf{\color{gray!90}\footnotesize \textcolor{customTeal}{\faInfoCircle}}
            };
            \node[anchor=west, text width=16cm] at (0.9,-0.3) {
                {\small\color{gray}\footnotesize \intro}
            };

            % 异常解读标题和内容
            \ifnum\pdfstrcmp{\status}{超标}=0
                \node[anchor=west] at (0.55,-0.8) {
                    \textbf{\color{customRed}\footnotesize \textcolor{customRed}{\faLightbulb}}
                };
                \node[anchor=west, text width=16cm] at (0.9,-0.8) {
                    {\small\color{gray}\footnotesize \suggestion}
                };
            \fi
        \end{scope}

        % 分割线
        \pgfmathsetmacro{\linepos}{
            \ifnum\pdfstrcmp{\status}{超标}=0
                \basepos-1.7  % 超标时的分割线位置
            \else
                \basepos-1.3  % 正常时的分割线位置
            \fi
        }
        \draw[gray!20] (0.2,\linepos) -- (\cardwidth-0.2,\linepos);

        % 根据当前行的状态计算下一行的位置增量
        \ifnum\pdfstrcmp{\status}{超标}=0
            \pgfmathsetmacro{\increment}{0.85}  % 超标行(两行内容)需要更大的增量
        \else
            \pgfmathsetmacro{\increment}{0.7}  % 正常行(一行内容)使用较小的增量
        \fi

        % 更新位置计数器
        \pgfmathsetmacro{\nextpos}{\currentpos+\increment}
        \xdef\currentpos{\nextpos}
    }

    % 最后一行的处理,消除多余的空白
    \pgfmathsetmacro{\lastincrement}{2}  % 最后一行的增量
    \pgfmathsetmacro{\nextpos}{\currentpos+\lastincrement}
    \xdef\currentpos{\nextpos}

\end{tikzpicture}
\end{center}

\newpage

\begin{center}
\begin{tikzpicture}[
    font=\small,
    title/.style={font=\small\bfseries\color{white}},
    value/.style={font=\small},
    reference/.style={font=\small},
    cell/.style={anchor=west, text width=4.2cm},
    note/.style={anchor=west, text width=4.5cm, align=left}
]
    \def\cardwidth{\textwidth}
    \def\cardheight{22.5}
    \def\barheight{0.25}
    \def\barwidth{1.5}
    \def\valuebarspace{0.4}

    % 容器和标题栏背景
    \draw[rounded corners=5, fill=white, draw=gray!20]
        (0,0) rectangle (\cardwidth,-\cardheight);
    \path[fill=customTeal]
        (0,0) [rounded corners=5] -- (\cardwidth,0) --
        (\cardwidth,0.8) -- (0,0.8) -- cycle;

    % 表头
    \node[title, anchor=west] at (0.5,0.4) {\textbf{菌种名称}};
    \node[title] at (4.5,0.4) {\textbf{正常范围\%}};
    \node[title] at (7.5,0.4) {\textbf{检测丰度}};
    \node[title] at (9.5,0.4) {\textbf{检测结果}}; % 新增检测结果列名
    \node[title] at (11.5,0.4) {\textbf{超过\%的人}};
    \node[title] at (13.75,0.4) {\textbf{失眠相关性}};
    \node[title] at (16,0.4) {\textbf{相关性强度}};

    % 初始化位置计数器
    \def\currentpos{0.25}

    % 数据行和卡片
    \foreach \item/\enitem/\value/\range/\percentile/\detection/\status/\intro/\suggestion/\index in {
        {瘤胃梭菌属}/{Ruminiclostridium}/0.00203/{0-0.05}/46\%/正相关/正常/{主要与肥胖发型呈正相关。}/{建议:乳杆菌补充、低聚果糖、没食子酸、芽孢杆菌补充。}/{\color{lightgray}\faStar \faStar}/\currentpos,
        {普雷沃氏菌科}/{Prevotellaceae}/0.04756/{0-56.537}/70\%/正相关/正常/{肥胖患者肠道中Prevotellaceae的丰度增加,女性中高丰度的Prevotellaceae与肥胖有关。}/{建议调整饮食结构,增加膳食纤维摄入。}/{\color{lightgray}\faStar \faStar}/\currentpos,
        {具核梭杆菌}/{Fusobacterium nucleatum}/0.00459/{0-0.05}/-\%/正相关/正常/{Fusobacterium nucleatum在肥胖人群中更为丰富,是一种机会性病原体,与牙周病的发生和发展密切相关。}/{建议注意口腔卫生,定期进行牙齿检查。}/{\color{lightgray}\faStar \faStar}/\currentpos,
        {扭链瘤胃球菌}/{Ruminococcus torques}/0.31451/{0-3.8364}/99\%/正相关/正常/{增加导致牛磺酸结合胆酸(TCA)和脱氧胆酸(DCA)水平升高,并激活脂肪组织的G蛋白偶联胆酸受体(GPBAR1,TGR5)。}/{建议控制饮食摄入,保持运动习惯。}/{\color{lightgray}\faStar \faStar}/\currentpos,
        {棒杆菌属}/{Corynebacterium}/0.00843/{0-0.05}/-\%/正相关/正常/{Corynebacterium jeddahense和C. massiliensis是从患有病态肥胖的人的粪便中分离出来的菌株。}/{建议保持健康的生活方式,控制体重。}/{\color{lightgray}\faStar}/\currentpos,
        {戴阿利斯特杆菌属}/{Dialister}/2.74325/{0-3.7365342}/79\%/正相关/正常/{Dialister属与高炎症指数相关,运动可以增加Dialister属的丰度,从而改善肥胖儿童的肠道菌群组成。}/{建议增加适度运动,改善肠道菌群。}/{\color{lightgray}\faStar}/\currentpos,
        {粪杆菌属}/{Faecalibacterium}/28.86665/{1.9350868-17.7942438}/99\%/正相关/超标/{肥胖人群中丰度明显降低,炎症反应增加,促进脂肪分解和吸收,减少脂肪的积累,促进肠道内废素的分泌,减少食欲和促进代谢。}/{建议:乳杆菌补充、亚麻籽、壳聚糖、柿子糖(甜栗)。}/{\color{lightgray}\faStar}/\currentpos,
        {纤毛菌属}/{Leptotrichia}/0.00146/{0-0}/95\%/正相关/超标/{在肥胖女性中的相对丰度较高,与糖尿病、中风等代谢性疾病有关联。}/{建议控制饮食,定期监测血糖。}/{\color{lightgray}\faStar}/\currentpos,
        {巨球形菌属}/{Megasphaera}/0.00399/{0-0.13046}/99\%/正相关/正常/{在肥胖人群中富集,促进脂肪沉积和代谢。}/{建议调整饮食结构,控制脂肪摄入。}/{\color{lightgray}\faStar}/\currentpos,
        {普雷沃氏菌属}/{Prevotella}/0.04290/{0-67.8009886}/50\%/正相关/正常/{丰度与体重、腰围、BMI、脂肪质量指数、甘油三酯和高敏C-反应蛋白水平呈正相关,而与高密度脂蛋白胆固醇水平呈负相关。}/{建议控制体重,注意饮食均衡。}/{\color{lightgray}\faStar}/\currentpos,
        {瘤胃球菌属}/{Ruminococcus}/9.84521/{0.0543588-19.7985354}/44\%/正相关/正常/{肥胖者的肠道菌群中明显增加,产生支链氨基酸和谷氨酸等代谢物可增加肥胖可能性。}/{建议调整饮食结构,控制热量摄入。}/{\color{lightgray}\faStar}/\currentpos
    }
    {
        % 计算当前行的基础位置
        \pgfmathsetmacro{\basepos}{-2.8*\currentpos}

        % 菌种名称
        \node[cell, align=left] at (0.5,\basepos) {
            \small\textbf{\item}\\[-0.2em]
            {\color{lightgray}\small\enitem}
        };

        % 正常范围
        \node[reference] at (4.5,\basepos) {\footnotesize\range};

        % 进度条相关
        \pgfmathsetmacro{\barypos}{\basepos-\valuebarspace+0.1}
        \def\barstart{6.75}

        % 进度条背景
        \fill[gray!10, rounded corners=2] (\barstart,\barypos)
            rectangle (\barstart+\barwidth,\barypos+\barheight);

        % 检测丰度值
        \node[value] at (7.5,{\basepos-\valuebarspace+0.6}) {\footnotesize\value};

        % 解析范围并计算进度条长度
        \def\parserange#1-#2\endparse{\def\minval{#1}\def\maxval{#2}}
        \expandafter\parserange\range\endparse

        % 计算进度条长度和颜色
        \pgfmathsetmacro{\progress}{min(\value/\maxval, 1.0)}
        \pgfmathparse{\value > \maxval ? "customred" : (\value < \minval ? "customred" : "green!50")}
        \let\barcolor=\pgfmathresult

        % 进度条显示
        \ifnum\pdfstrcmp{\status}{超标}=0
            \fill[customred, rounded corners=2] (\barstart,\barypos)
                rectangle (\barstart+\barwidth,\barypos+\barheight);
        \else
            \fill[\barcolor, rounded corners=2] (\barstart,\barypos)
                rectangle (\barstart+\barwidth*\progress,\barypos+\barheight);
        \fi

        % 检测结果
        \node[value] at (9.5,\basepos) {\footnotesize\status}; % 检测结果列

        % 超过%的人
        \node[value] at (11.5,\basepos) {\footnotesize\percentile};

        % 失眠相关性
        \node[value] at (13.75,\basepos) {\footnotesize\detection};

        % 相关性强度
        \node[value] at (16,\basepos) {\footnotesize\index};

        % 添加卡片
        \pgfmathsetmacro{\cardypos}{\basepos-0.5}
        \begin{scope}[shift={(0,\cardypos)}]
            % 卡片背景
            \pgfmathsetmacro{\cardheight}{
                \ifnum\pdfstrcmp{\status}{超标}=0
                    1.0  % 两行内容时的高度
                \else
                    0.6  % 一行内容时的高度
                \fi
            }

            \fill[rounded corners=5pt, customTeal!5, draw=gray!5]
                (0.3,-\cardheight) rectangle (17.3,0);

            % 菌群简介图标和内容
            \node[anchor=west] at (0.5,-0.3) {
                \textbf{\color{gray!90}\footnotesize \textcolor{customTeal}{\faInfoCircle}}
            };
            \node[anchor=west, text width=16cm] at (0.9,-0.3) {
                {\small\color{gray}\footnotesize \intro}
            };

            % 异常解读标题和内容
            \ifnum\pdfstrcmp{\status}{超标}=0
                \node[anchor=west] at (0.55,-0.8) {
                    \textbf{\color{customRed}\footnotesize \textcolor{customRed}{\faLightbulb}}
                };
                \node[anchor=west, text width=16cm] at (0.9,-0.8) {
                    {\small\color{gray}\footnotesize \suggestion}
                };
            \fi
        \end{scope}

        % 分割线
        \pgfmathsetmacro{\linepos}{
            \ifnum\pdfstrcmp{\status}{超标}=0
                \basepos-1.7  % 超标时的分割线位置
            \else
                \basepos-1.3  % 正常时的分割线位置
            \fi
        }
        \draw[gray!20] (0.2,\linepos) -- (\cardwidth-0.2,\linepos);

        % 根据当前行的状态计算下一行的位置增量
        \ifnum\pdfstrcmp{\status}{超标}=0
            \pgfmathsetmacro{\increment}{0.85}  % 超标行(两行内容)需要更大的增量
        \else
            \pgfmathsetmacro{\increment}{0.7}  % 正常行(一行内容)使用较小的增量
        \fi

        % 更新位置计数器
        \pgfmathsetmacro{\nextpos}{\currentpos+\increment}
        \xdef\currentpos{\nextpos}
    }

    % 最后一行的处理,消除多余的空白
    \pgfmathsetmacro{\lastincrement}{2}  % 最后一行的增量
    \pgfmathsetmacro{\nextpos}{\currentpos+\lastincrement}
    \xdef\currentpos{\nextpos}

\end{tikzpicture}
\end{center}

\newpage

\begin{center}
\begin{tikzpicture}[
    font=\small,
    title/.style={font=\small\bfseries\color{white}},
    value/.style={font=\small},
    reference/.style={font=\small},
    cell/.style={anchor=west, text width=4.2cm},
    note/.style={anchor=west, text width=4.5cm, align=left}
]
    \def\cardwidth{\textwidth}
    \def\cardheight{22.5}
    \def\barheight{0.25}
    \def\barwidth{1.5}
    \def\valuebarspace{0.4}

    % 容器和标题栏背景
    \draw[rounded corners=5, fill=white, draw=gray!20]
        (0,0) rectangle (\cardwidth,-\cardheight);
    \path[fill=customTeal]
        (0,0) [rounded corners=5] -- (\cardwidth,0) --
        (\cardwidth,0.8) -- (0,0.8) -- cycle;

    % 表头
    \node[title, anchor=west] at (0.5,0.4) {\textbf{菌种名称}};
    \node[title] at (4.5,0.4) {\textbf{正常范围\%}};
    \node[title] at (7.5,0.4) {\textbf{检测丰度}};
    \node[title] at (9.5,0.4) {\textbf{检测结果}}; % 新增检测结果列名
    \node[title] at (11.5,0.4) {\textbf{超过\%的人}};
    \node[title] at (13.75,0.4) {\textbf{失眠相关性}};
    \node[title] at (16,0.4) {\textbf{相关性强度}};

    % 初始化位置计数器
    \def\currentpos{0.25}

    % 数据行和卡片
    \foreach \item/\enitem/\value/\range/\percentile/\detection/\status/\intro/\suggestion/\index in {
        {产碱菌科}/{Alcaligenaceae}/0.01035/{0-0}/25\%/正相关/超标/{肥胖者肠道中相对丰度较高,可以产生内毒素LPS,引起慢性低度炎症,也是机会性病原菌,特别是在患有自闭症的儿童中。}/{建议增加益生菌摄入,改善肠道环境。}/{\color{lightgray}\faStar}/\currentpos,
        {赭色噬帽菌}/{Capnocytophaga ochracea}/ND/{0-0}/99\%/正相关/正常/{肥胖女性的Streptococcus sanguinis、Streptococcus oralis和Capnocytophaga ochracea的水平显著高于非肥胖女性。}/{建议注意口腔卫生,保持健康的生活方式。}/{\color{lightgray}\faStar}/\currentpos,
        {柯氏菌}/{Collinsella aerofaciens}/0.00117/{0-10.121}/30\%/正相关/正常/{Collinsella aerofaciens是一种与肥胖和代谢综合征相关的微生物生物标志物。}/{建议调整饮食结构,增加膳食纤维摄入。}/{\color{lightgray}\faStar}/\currentpos,
        {长链多尔氏菌}/{Dorea longicatena}/0.33368/{0-4.2889}/99\%/正相关/正常/{Dorea longicatena是肥胖的生物标志物之一。}/{建议保持健康的饮食习惯,控制热量摄入。}/{\color{lightgray}\faStar}/\currentpos,
        {史氏甲烷短杆菌}/{Methanobrevibacter smithii}/ND/{0-3.3301}/99\%/正相关/正常/{产生的甲烷与便秘、肠易激综合征和肥胖有关。}/{建议调节肠道菌群,改善肠道环境。}/{\color{lightgray}\faStar}/\currentpos,
        {血链球菌}/{Streptococcus sanguinis}/0.00203/{0-0.05}/-\%/正相关/正常/{研究发现,Streptococcus sanguinis、Streptococcus oralis和Capnocytophaga ochracea在肥胖女性的水平显著高于非肥胖女性。}/{建议注意口腔卫生,保持健康的生活方式。}/{\color{lightgray}\faStar}/\currentpos,
        {艾克曼菌}/{Akkermansia muciniphila}/0.00394/{0-6.6395}/45\%/负相关/正常/{降低肠道黏膜的炎症程度,增加肠道屏障功能,降低肥胖相关的代谢疾病。}/{建议增加益生菌摄入,改善肠道环境。}/{\color{lightgray}\faStar \faStar \faStar}/\currentpos,
        {粪杆菌属}/{Faecalibacterium}/28.86665/{1.9350868-17.7942438}/99\%/负相关/超标/{肥胖人群中丰度明显降低,炎症反应增加,促进脂肪分解和吸收,减少脂肪的积累,促进肠道内废素的分泌,减少食欲和促进代谢。}/{建议:乳杆菌补充、亚麻籽、壳聚糖、柿子糖(甜栗)。}/{\color{lightgray}\faStar \faStar \faStar}/\currentpos,
        {拟杆菌门}/{Bacteroidetes}/0.15639/{19.203-90.8}/17\%/负相关/偏低/{丰度与肥胖呈负相关,越多肥胖的风险越低。}/{建议增加膳食纤维摄入,调节肠道菌群。}/{\color{lightgray}\faStar \faStar \faStar}/\currentpos,
        {优杆菌属}/{Eubacterium}/5.75186/{0.1145944-9.4883306}/95\%/负相关/正常/{Eubacterium dolichum与代谢紊乱和肥胖有关其丰度与内脂肪质量呈正相关。}/{建议控制脂肪摄入,保持运动习惯。}/{\color{lightgray}\faStar \faStar}/\currentpos,
        {卵形拟杆菌}/{Bacteroides ovatus}/ND/{0-3.0624}/25\%/负相关/正常/{具有抗炎作用,在肥胖、2型糖尿病和胰腺炎硬化疾病患者中数量减少,具有保护作用,可以诱导肠道IgA的产生,有益于肠道稳态和免疫健康。}/{建议增加膳食纤维摄入,改善肠道环境。}/{\color{lightgray}\faStar \faStar}/\currentpos
    }
    {
        % 计算当前行的基础位置
        \pgfmathsetmacro{\basepos}{-2.8*\currentpos}

        % 菌种名称
        \node[cell, align=left] at (0.5,\basepos) {
            \small\textbf{\item}\\[-0.2em]
            {\color{lightgray}\small\enitem}
        };

        % 正常范围
        \node[reference] at (4.5,\basepos) {\footnotesize\range};

        % 进度条相关
        \pgfmathsetmacro{\barypos}{\basepos-\valuebarspace+0.1}
        \def\barstart{6.75}

        % 进度条背景
        \fill[gray!10, rounded corners=2] (\barstart,\barypos)
            rectangle (\barstart+\barwidth,\barypos+\barheight);

        % 检测丰度值
        \node[value] at (7.5,{\basepos-\valuebarspace+0.6}) {\footnotesize\value};

        % 解析范围并计算进度条长度
        \def\parserange#1-#2\endparse{\def\minval{#1}\def\maxval{#2}}
        \expandafter\parserange\range\endparse

        % 计算进度条长度和颜色
        \pgfmathsetmacro{\progress}{min(\value/\maxval, 1.0)}
        \pgfmathparse{\value > \maxval ? "customred" : (\value < \minval ? "customred" : "green!50")}
        \let\barcolor=\pgfmathresult

        % 进度条显示
        \ifnum\pdfstrcmp{\status}{超标}=0
            \fill[customred, rounded corners=2] (\barstart,\barypos)
                rectangle (\barstart+\barwidth,\barypos+\barheight);
        \else
            \fill[\barcolor, rounded corners=2] (\barstart,\barypos)
                rectangle (\barstart+\barwidth*\progress,\barypos+\barheight);
        \fi

        % 检测结果
        \node[value] at (9.5,\basepos) {\footnotesize\status}; % 检测结果列

        % 超过%的人
        \node[value] at (11.5,\basepos) {\footnotesize\percentile};

        % 失眠相关性
        \node[value] at (13.75,\basepos) {\footnotesize\detection};

        % 相关性强度
        \node[value] at (16,\basepos) {\footnotesize\index};

        % 添加卡片
        \pgfmathsetmacro{\cardypos}{\basepos-0.5}
        \begin{scope}[shift={(0,\cardypos)}]
            % 卡片背景
            \pgfmathsetmacro{\cardheight}{
                \ifnum\pdfstrcmp{\status}{超标}=0
                    1.0  % 两行内容时的高度
                \else
                    0.6  % 一行内容时的高度
                \fi
            }

            \fill[rounded corners=5pt, customTeal!5, draw=gray!5]
                (0.3,-\cardheight) rectangle (17.3,0);

            % 菌群简介图标和内容
            \node[anchor=west] at (0.5,-0.3) {
                \textbf{\color{gray!90}\footnotesize \textcolor{customTeal}{\faInfoCircle}}
            };
            \node[anchor=west, text width=16cm] at (0.9,-0.3) {
                {\small\color{gray}\footnotesize \intro}
            };

            % 异常解读标题和内容
            \ifnum\pdfstrcmp{\status}{超标}=0
                \node[anchor=west] at (0.55,-0.8) {
                    \textbf{\color{customRed}\footnotesize \textcolor{customRed}{\faLightbulb}}
                };
                \node[anchor=west, text width=16cm] at (0.9,-0.8) {
                    {\small\color{gray}\footnotesize \suggestion}
                };
            \fi
        \end{scope}

        % 分割线
        \pgfmathsetmacro{\linepos}{
            \ifnum\pdfstrcmp{\status}{超标}=0
                \basepos-1.7  % 超标时的分割线位置
            \else
                \basepos-1.3  % 正常时的分割线位置
            \fi
        }
        \draw[gray!20] (0.2,\linepos) -- (\cardwidth-0.2,\linepos);

        % 根据当前行的状态计算下一行的位置增量
        \ifnum\pdfstrcmp{\status}{超标}=0
            \pgfmathsetmacro{\increment}{0.85}  % 超标行(两行内容)需要更大的增量
        \else
            \pgfmathsetmacro{\increment}{0.7}  % 正常行(一行内容)使用较小的增量
        \fi

        % 更新位置计数器
        \pgfmathsetmacro{\nextpos}{\currentpos+\increment}
        \xdef\currentpos{\nextpos}
    }

    % 最后一行的处理,消除多余的空白
    \pgfmathsetmacro{\lastincrement}{2}  % 最后一行的增量
    \pgfmathsetmacro{\nextpos}{\currentpos+\lastincrement}
    \xdef\currentpos{\nextpos}

\end{tikzpicture}
\end{center}

\begin{center}
\begin{tikzpicture}[
    font=\small,
    title/.style={font=\small\bfseries\color{white}},
    value/.style={font=\small},
    reference/.style={font=\small},
    cell/.style={anchor=west, text width=4.2cm},
    note/.style={anchor=west, text width=4.5cm, align=left}
]
    \def\cardwidth{\textwidth}
    \def\cardheight{2}
    \def\barheight{0.25}
    \def\barwidth{1.5}
    \def\valuebarspace{0.4}

    % 容器和标题栏背景
    \draw[rounded corners=5, fill=white, draw=gray!20]
        (0,0) rectangle (\cardwidth,-\cardheight);
    \path[fill=customTeal]
        (0,0) [rounded corners=5] -- (\cardwidth,0) --
        (\cardwidth,0.8) -- (0,0.8) -- cycle;

    % 表头
    \node[title, anchor=west] at (0.5,0.4) {\textbf{菌种名称}};
    \node[title] at (4.5,0.4) {\textbf{正常范围\%}};
    \node[title] at (7.5,0.4) {\textbf{检测丰度}};
    \node[title] at (9.5,0.4) {\textbf{检测结果}}; % 新增检测结果列名
    \node[title] at (11.5,0.4) {\textbf{超过\%的人}};
    \node[title] at (13.75,0.4) {\textbf{失眠相关性}};
    \node[title] at (16,0.4) {\textbf{相关性强度}};

    % 初始化位置计数器
    \def\currentpos{0.25}

    % 数据行和卡片
    \foreach \item/\enitem/\value/\range/\percentile/\detection/\status/\intro/\suggestion/\index in {
        {颤螺菌属}/{Oscillospira}/ND/{0.033-5.346}/36\%/负相关/偏低/{Oscillospira过多或过少都与肥胖相关,益生菌或益生元可以增加Oscillospira的数量,从而减轻肥胖和代谢疾病的症状。}/{建议适量补充益生菌和益生元,调节肠道菌群平衡。}/{\color{lightgray}\faStar}/\currentpos
    }
    {
        % 计算当前行的基础位置
        \pgfmathsetmacro{\basepos}{-2.8*\currentpos}

        % 菌种名称
        \node[cell, align=left] at (0.5,\basepos) {
            \small\textbf{\item}\\[-0.2em]
            {\color{lightgray}\small\enitem}
        };

        % 正常范围
        \node[reference] at (4.5,\basepos) {\footnotesize\range};

        % 进度条相关
        \pgfmathsetmacro{\barypos}{\basepos-\valuebarspace+0.1}
        \def\barstart{6.75}

        % 进度条背景
        \fill[gray!10, rounded corners=2] (\barstart,\barypos)
            rectangle (\barstart+\barwidth,\barypos+\barheight);

        % 检测丰度值
        \node[value] at (7.5,{\basepos-\valuebarspace+0.6}) {\footnotesize\value};

        % 解析范围并计算进度条长度
        \def\parserange#1-#2\endparse{\def\minval{#1}\def\maxval{#2}}
        \expandafter\parserange\range\endparse

        % 计算进度条长度和颜色
        \pgfmathsetmacro{\progress}{min(\value/\maxval, 1.0)}
        \pgfmathparse{\value > \maxval ? "customred" : (\value < \minval ? "customred" : "green!50")}
        \let\barcolor=\pgfmathresult

        % 进度条显示
        \ifnum\pdfstrcmp{\status}{超标}=0
            \fill[customred, rounded corners=2] (\barstart,\barypos)
                rectangle (\barstart+\barwidth,\barypos+\barheight);
        \else
            \fill[\barcolor, rounded corners=2] (\barstart,\barypos)
                rectangle (\barstart+\barwidth*\progress,\barypos+\barheight);
        \fi

        % 检测结果
        \node[value] at (9.5,\basepos) {\footnotesize\status}; % 检测结果列

        % 超过%的人
        \node[value] at (11.5,\basepos) {\footnotesize\percentile};

        % 失眠相关性
        \node[value] at (13.75,\basepos) {\footnotesize\detection};

        % 相关性强度
        \node[value] at (16,\basepos) {\footnotesize\index};

        % 添加卡片
        \pgfmathsetmacro{\cardypos}{\basepos-0.5}
        \begin{scope}[shift={(0,\cardypos)}]
            % 卡片背景
            \pgfmathsetmacro{\cardheight}{
                \ifnum\pdfstrcmp{\status}{超标}=0
                    1.0  % 两行内容时的高度
                \else
                    0.6  % 一行内容时的高度
                \fi
            }

            \fill[rounded corners=5pt, customTeal!5, draw=gray!5]
                (0.3,-\cardheight) rectangle (17.3,0);

            % 菌群简介图标和内容
            \node[anchor=west] at (0.5,-0.3) {
                \textbf{\color{gray!90}\footnotesize \textcolor{customTeal}{\faInfoCircle}}
            };
            \node[anchor=west, text width=16cm] at (0.9,-0.3) {
                {\small\color{gray}\footnotesize \intro}
            };

            % 异常解读标题和内容
            \ifnum\pdfstrcmp{\status}{超标}=0
                \node[anchor=west] at (0.55,-0.8) {
                    \textbf{\color{customRed}\footnotesize \textcolor{customRed}{\faLightbulb}}
                };
                \node[anchor=west, text width=16cm] at (0.9,-0.8) {
                    {\small\color{gray}\footnotesize \suggestion}
                };
            \fi
        \end{scope}

        % 分割线
        \pgfmathsetmacro{\linepos}{
            \ifnum\pdfstrcmp{\status}{超标}=0
                \basepos-1.7  % 超标时的分割线位置
            \else
                \basepos-1.3  % 正常时的分割线位置
            \fi
        }
        \draw[gray!20] (0.2,\linepos) -- (\cardwidth-0.2,\linepos);

        % 根据当前行的状态计算下一行的位置增量
        \ifnum\pdfstrcmp{\status}{超标}=0
            \pgfmathsetmacro{\increment}{0.85}  % 超标行(两行内容)需要更大的增量
        \else
            \pgfmathsetmacro{\increment}{0.7}  % 正常行(一行内容)使用较小的增量
        \fi

        % 更新位置计数器
        \pgfmathsetmacro{\nextpos}{\currentpos+\increment}
        \xdef\currentpos{\nextpos}
    }

    % 最后一行的处理,消除多余的空白
    \pgfmathsetmacro{\lastincrement}{2}  % 最后一行的增量
    \pgfmathsetmacro{\nextpos}{\currentpos+\lastincrement}
    \xdef\currentpos{\nextpos}

\end{tikzpicture}
\end{center}


\newpage

\begin{tcolorbox}[
    enhanced,
    colback=white,
    colframe=white,
    arc=2mm,
    boxrule=0pt,
    width=\textwidth,
    left=15pt,
    right=15pt,
    top=10pt,
    bottom=10pt,
    drop shadow={
        opacity=0.2,
        color=customTeal
    },
    borderline west={5pt}{0pt}{customTeal}
]
\textcolor{customTeal}{\Large\textbf{便秘相关菌}}
\end{tcolorbox}

\begin{tcolorbox}[
    enhanced,
    colback=customTealBg,
    colframe=customTealBg,
    arc=3mm,
    boxrule=0pt,
    width=\textwidth,
    top=8pt,
    bottom=8pt
]
{\small{\color{customTeal}\faInfoCircle} 肠道菌群与便秘的关系研究日益受到重视。研究表明,特定肠道菌群的丰度水平与便秘的发生和发展密切相关。通过分析个体的肠道菌群,我们能够更好地理解便秘的风险及病因,以及如何通过调节肠道微生态来预防和改善症状。\\

{\color{orange}\faExclamationTriangle} \textbf{特别注意}:
\begin{itemize}
    \item 在本报告中,某些与便秘相关的肠道菌群可能会显示出异常水平。然而,需要指出的是,即使这些菌群的丰度不正常,也并不一定意味着个体会经历便秘症状。便秘的发生通常是多因素综合作用的结果,包括饮食结构、生活方式、情绪状态以及其他健康问题等。
    \item 为了有效预防和改善便秘,建议您综合考虑肠道菌群检测结果以及个人饮食习惯、运动情况、压力管理等因素。同时,若出现持续的便秘症状,应及时咨询专业医生或营养师以获取适当的干预和建议。
\end{itemize}
}
\end{tcolorbox}


\begin{tcolorbox}[
    enhanced,
    colback=lightpurple!10, % 卡片底色
    colframe=lightpurple!10,  % 边框颜色
    arc=3mm,
    boxrule=0.5pt,
    width=\textwidth,
    top=8pt,
    bottom=8pt
]
{\small{\color{lightpurple}\faQuestionCircle}\quad \textbf{肠道菌群是如何影响便秘的?}\\
{\color{orange!50}\faComments}\quad 肠道菌群与便秘之间的关系涉及多种机制和因素,以下是一些主要方面:
\begin{itemize}
    \item \textbf{菌群构成}:益生菌与有害菌:健康的肠道菌群中含有大量的益生菌(如双歧杆菌、乳酸菌),它们能促进肠道蠕动,帮助消化和排便。相反,有害菌(如某些致病菌)的过度繁殖可能抑制正常的肠道功能,导致便秘。
    \item \textbf{代谢产物}:短链脂肪酸(SCFAs):肠道细菌通过发酵膳食纤维产生 SCFAs(如醋酸、丁酸和丙酸),这些物质不仅是肠道上皮细胞的重要能量来源,还能促进肠道运动,增加肠道的蠕动,从而改善便秘情况。
    \item \textbf{腹腔神经和内分泌系统}:信号传递:肠道菌群的代谢物可以调节肠道神经系统和内分泌反应,影响肠道的运动模式。通过释放神经递质和激素(如肠促胰液素、胃动素等),这些信号可以增强肠道的蠕动。
    \item \textbf{免疫系统}:免疫调节:肠道菌群能够影响局部和全身免疫反应。健康的菌群平衡有助于维持肠道的免疫稳定性,预防炎症。如果炎症发生,可能会影响肠道的正常功能,导致便秘。
    \item \textbf{纤维素摄入与菌群交互}:膳食纤维:膳食纤维是维持肠道菌群平衡的重要因素。丰富的膳食纤维能促进有益菌的生长,同时为其提供发酵的底物,增加粪便体积,促进排便。
\end{itemize}
肠道菌群通过影响肠道动力、代谢产物的产生、免疫反应及综合的生活方式因素,对便秘的发生有显著影响。但是需要注意的是,不同个体的肠道菌群组成可能因遗传、饮食、生活方式和环境等因素而异,导致便秘的易感性也不同。一些人可能对特定的饮食或菌群变化更敏感。
}
\end{tcolorbox}

\newpage

\begin{center}
\begin{tikzpicture}[
    font=\small,
    title/.style={font=\small\bfseries\color{white}},
    value/.style={font=\small},
    reference/.style={font=\small},
    cell/.style={anchor=west, text width=4.2cm},
    note/.style={anchor=west, text width=4.5cm, align=left}
]
    \def\cardwidth{\textwidth}
    \def\cardheight{22.5}
    \def\barheight{0.25}
    \def\barwidth{1.5}
    \def\valuebarspace{0.4}

    % 容器和标题栏背景
    \draw[rounded corners=5, fill=white, draw=gray!20]
        (0,0) rectangle (\cardwidth,-\cardheight);
    \path[fill=customTeal]
        (0,0) [rounded corners=5] -- (\cardwidth,0) --
        (\cardwidth,0.8) -- (0,0.8) -- cycle;

    % 表头
    \node[title, anchor=west] at (0.5,0.4) {\textbf{菌种名称}};
    \node[title] at (4.5,0.4) {\textbf{正常范围\%}};
    \node[title] at (7.5,0.4) {\textbf{检测丰度}};
    \node[title] at (9.5,0.4) {\textbf{检测结果}}; % 新增检测结果列名
    \node[title] at (11.5,0.4) {\textbf{超过\%的人}};
    \node[title] at (13.75,0.4) {\textbf{失眠相关性}};
    \node[title] at (16,0.4) {\textbf{相关性强度}};

    % 初始化位置计数器
    \def\currentpos{0.25}

    % 数据行和卡片
    \foreach \item/\enitem/\value/\range/\percentile/\detection/\status/\intro/\suggestion/\index in {
        {史氏甲烷短杆菌}/{Methanobrevibacter smithii}/ND/{0-3.3301}/99\%/正相关/正常/{产生甲烷,而甲烷能够延迟肠道的传输时间,从而导致便秘。}/{建议调节肠道菌群,改善排便功能。}/{\color{lightgray}\faStar \faStar \faStar}/\currentpos,
        {粪球菌属}/{Coprococcus}/0.05595/{0.082336-9.4195178}/64\%/正相关/偏低/{Coprococcus的丰度过高或过低都可能与功能性便秘有关。}/{建议调整饮食结构,增加膳食纤维摄入。}/{\color{lightgray}\faStar \faStar \faStar}/\currentpos,
        {埃希氏菌属}/{Escherichia}/0.14296/{0-3.83}/6\%/正相关/正常/{过度生长会导致色氨酸转化为吲哚,从而增加脂肪组织产生的吲哚含量,与慢性便秘患者尿液成分的改变有关。}/{建议调节肠道菌群平衡,控制饮食。}/{\color{lightgray}\faStar \faStar \faStar}/\currentpos,
        {毛螺菌属}/{Lachnospira}/5.02851/{0.0346628-8.6596222}/70\%/正相关/正常/{一些菌株能够产生乳乳酸和醋酸,这些物质通过抑制炎症分泌导致便秘。}/{建议增加益生菌摄入,改善肠道环境。}/{\color{lightgray}\faStar \faStar \faStar}/\currentpos,
        {厌氧棒状菌属}/{Anaerotruncs}/0.08642/{0-0.13747}/55\%/正相关/正常/{便秘患者的肠道中Anaerotruncs的数量明显增加,可以产生一些酸和气体等代谢产物,可能会刺激肠道神经,进一步影响肠道蠕动和排便便功能。}/{建议调节肠道菌群,改善肠道环境。}/{\color{lightgray}\faStar \faStar}/\currentpos,
        {葡萄球菌属}/{Staphylococcus}/0.51158/{0-0.05}/-\%/正相关/超标/{感染是导致后感染性肠易激综合征(PI-IBS)的潜在病原因素之一,与肠道梗阻和肠穿孔疾病有关。}/{建议:乳杆菌补充、大蒜、绿茶、双歧杆菌补充。}/{\color{lightgray}\faStar \faStar}/\currentpos,
        {链球菌属}/{Streptococcus}/0.31690/{0-0.3495704}/13\%/正相关/正常/{在IBS患者中富集,与便秘的发生和发展有关。}/{建议调节肠道菌群,改善肠道功能。}/{\color{lightgray}\faStar \faStar}/\currentpos,
        {脆弱拟杆菌}/{Bacteroides fragilis}/0.00145/{0-0.05}/8\%/正相关/正常/{功能性便秘的儿童和老年患者中,含量增加。}/{建议增加运动量,保持规律作息。}/{\color{lightgray}\faStar \faStar}/\currentpos,
        {卵形拟杆菌}/{Bacteroides ovatus}/ND/{0-3.0624}/25\%/正相关/正常/{在老年便秘患者和儿童功能性便秘患者中,相对丰度增加。}/{建议增加膳食纤维摄入,保持充足水分。}/{\color{lightgray}\faStar \faStar}/\currentpos,
        {下水道菌属}/{Cloacibacillus}/0.00028/{0-0}/99\%/正相关/超标/{Cloacibacillus在便秘患者的肠道中更为丰富,过度生长时,可能会导致肠道菌群失衡,从而引起便秘。}/{建议调节肠道菌群,增加益生菌摄入。}/{\color{lightgray}\faStar}/\currentpos,
        {梭菌属}/{Clostridium}/1.90370/{0-4.4645924}/67\%/正相关/正常/{便秘患者的肠道中Lachnospira和Clostridium的水平明显升高,而Clostridium difficile则被证明增加了后感染性肠易激综合征的风险。}/{建议调节肠道菌群,控制饮食。}/{\color{lightgray}\faStar}/\currentpos
    }
    {
        % 计算当前行的基础位置
        \pgfmathsetmacro{\basepos}{-2.8*\currentpos}

        % 菌种名称
        \node[cell, align=left] at (0.5,\basepos) {
            \small\textbf{\item}\\[-0.2em]
            {\color{lightgray}\small\enitem}
        };

        % 正常范围
        \node[reference] at (4.5,\basepos) {\footnotesize\range};

        % 进度条相关
        \pgfmathsetmacro{\barypos}{\basepos-\valuebarspace+0.1}
        \def\barstart{6.75}

        % 进度条背景
        \fill[gray!10, rounded corners=2] (\barstart,\barypos)
            rectangle (\barstart+\barwidth,\barypos+\barheight);

        % 检测丰度值
        \node[value] at (7.5,{\basepos-\valuebarspace+0.6}) {\footnotesize\value};

        % 解析范围并计算进度条长度
        \def\parserange#1-#2\endparse{\def\minval{#1}\def\maxval{#2}}
        \expandafter\parserange\range\endparse

        % 计算进度条长度和颜色
        \pgfmathsetmacro{\progress}{min(\value/\maxval, 1.0)}
        \pgfmathparse{\value > \maxval ? "customred" : (\value < \minval ? "customred" : "green!50")}
        \let\barcolor=\pgfmathresult

        % 进度条显示
        \ifnum\pdfstrcmp{\status}{超标}=0
            \fill[customred, rounded corners=2] (\barstart,\barypos)
                rectangle (\barstart+\barwidth,\barypos+\barheight);
        \else
            \fill[\barcolor, rounded corners=2] (\barstart,\barypos)
                rectangle (\barstart+\barwidth*\progress,\barypos+\barheight);
        \fi

        % 检测结果
        \node[value] at (9.5,\basepos) {\footnotesize\status}; % 检测结果列

        % 超过%的人
        \node[value] at (11.5,\basepos) {\footnotesize\percentile};

        % 失眠相关性
        \node[value] at (13.75,\basepos) {\footnotesize\detection};

        % 相关性强度
        \node[value] at (16,\basepos) {\footnotesize\index};

        % 添加卡片
        \pgfmathsetmacro{\cardypos}{\basepos-0.5}
        \begin{scope}[shift={(0,\cardypos)}]
            % 卡片背景
            \pgfmathsetmacro{\cardheight}{
                \ifnum\pdfstrcmp{\status}{超标}=0
                    1.0  % 两行内容时的高度
                \else
                    0.6  % 一行内容时的高度
                \fi
            }

            \fill[rounded corners=5pt, customTeal!5, draw=gray!5]
                (0.3,-\cardheight) rectangle (17.3,0);

            % 菌群简介图标和内容
            \node[anchor=west] at (0.5,-0.3) {
                \textbf{\color{gray!90}\footnotesize \textcolor{customTeal}{\faInfoCircle}}
            };
            \node[anchor=west, text width=16cm] at (0.9,-0.3) {
                {\small\color{gray}\footnotesize \intro}
            };

            % 异常解读标题和内容
            \ifnum\pdfstrcmp{\status}{超标}=0
                \node[anchor=west] at (0.55,-0.8) {
                    \textbf{\color{customRed}\footnotesize \textcolor{customRed}{\faLightbulb}}
                };
                \node[anchor=west, text width=16cm] at (0.9,-0.8) {
                    {\small\color{gray}\footnotesize \suggestion}
                };
            \fi
        \end{scope}

        % 分割线
        \pgfmathsetmacro{\linepos}{
            \ifnum\pdfstrcmp{\status}{超标}=0
                \basepos-1.7  % 超标时的分割线位置
            \else
                \basepos-1.3  % 正常时的分割线位置
            \fi
        }
        \draw[gray!20] (0.2,\linepos) -- (\cardwidth-0.2,\linepos);

        % 根据当前行的状态计算下一行的位置增量
        \ifnum\pdfstrcmp{\status}{超标}=0
            \pgfmathsetmacro{\increment}{0.85}  % 超标行(两行内容)需要更大的增量
        \else
            \pgfmathsetmacro{\increment}{0.7}  % 正常行(一行内容)使用较小的增量
        \fi

        % 更新位置计数器
        \pgfmathsetmacro{\nextpos}{\currentpos+\increment}
        \xdef\currentpos{\nextpos}
    }

    % 最后一行的处理,消除多余的空白
    \pgfmathsetmacro{\lastincrement}{2}  % 最后一行的增量
    \pgfmathsetmacro{\nextpos}{\currentpos+\lastincrement}
    \xdef\currentpos{\nextpos}

\end{tikzpicture}
\end{center}

\newpaqge

\begin{center}
\begin{tikzpicture}[
    font=\small,
    title/.style={font=\small\bfseries\color{white}},
    value/.style={font=\small},
    reference/.style={font=\small},
    cell/.style={anchor=west, text width=4.2cm},
    note/.style={anchor=west, text width=4.5cm, align=left}
]
    \def\cardwidth{\textwidth}
    \def\cardheight{14}
    \def\barheight{0.25}
    \def\barwidth{1.5}
    \def\valuebarspace{0.4}

    % 容器和标题栏背景
    \draw[rounded corners=5, fill=white, draw=gray!20]
        (0,0) rectangle (\cardwidth,-\cardheight);
    \path[fill=customTeal]
        (0,0) [rounded corners=5] -- (\cardwidth,0) --
        (\cardwidth,0.8) -- (0,0.8) -- cycle;

    % 表头
    \node[title, anchor=west] at (0.5,0.4) {\textbf{菌种名称}};
    \node[title] at (4.5,0.4) {\textbf{正常范围\%}};
    \node[title] at (7.5,0.4) {\textbf{检测丰度}};
    \node[title] at (9.5,0.4) {\textbf{检测结果}}; % 新增检测结果列名
    \node[title] at (11.5,0.4) {\textbf{超过\%的人}};
    \node[title] at (13.75,0.4) {\textbf{失眠相关性}};
    \node[title] at (16,0.4) {\textbf{相关性强度}};

    % 初始化位置计数器
    \def\currentpos{0.25}

    % 数据行和卡片
    \foreach \item/\enitem/\value/\range/\percentile/\detection/\status/\intro/\suggestion/\index in {
        {双歧杆菌属}/{Bifidobacterium}/0.03296/{1.7545796-35.500554}/24\%/负相关/偏低/{可以与其他菌种合作转放短链脂肪酸和气体,促进肠道蠕动缓解便秘症状。可以增加乳酸菌的丰度,从而缓解便秘。但也可能会促进胆脂症的发生。}/{建议适量补充双歧杆菌,增加膳食纤维摄入。}/{\color{lightgray}\faStar \faStar \faStar}/\currentpos,
        {乳杆菌属}/{Lactobacillus}/0.00660/{0-0.4302374}/8\%/负相关/正常/{在便秘患者中含量降低,补充后可改善便秘症状。}/{建议适量补充乳酸菌,改善肠道环境。}/{\color{lightgray}\faStar \faStar \faStar}/\currentpos,
        {阿克曼氏菌属}/{Akkermansia}/0.00460/{0-5.7984096}/44\%/负相关/正常/{调节肠道黏膜屏障的功能,促进肠道蠕动和水分吸收,减少肠道炎症反应,从而改善便秘症状。}/{建议增加膳食纤维摄入,改善肠道环境。}/{\color{lightgray}\faStar \faStar}/\currentpos,
        {经黏液真杆菌属}/{Blautia}/10.17279/{0.0846204-6.9055608}/75\%/负相关/超标/{在便秘患者中明显降低。通过摄入亚麻籽等食物,可以增加Blautia的数量,从而改善便秘症状。能够产生丁酸可以促进肠道蠕动,从而改善便秘。}/{建议:高脂汁酸、红酒、啤酒。}/{\color{lightgray}\faStar \faStar}/\currentpos,
        {普雷沃氏菌属}/{Prevotella}/0.04290/{0-67.8009886}/50\%/负相关/正常/{可以分解纤维素和其他难以消化的食物成分,产生大量的短链脂肪酸,刺激肠道蠕动,促进排便。}/{建议增加膳食纤维摄入,保持饮食均衡。}/{\color{lightgray}\faStar \faStar}/\currentpos,
        {梭菌属}/{Clostridium}/1.90370/{0-4.4645924}/67\%/负相关/正常/{便秘患者的肠道中Lachnospira和Clostridium的水平明显升高。而Clostridium difficile则被证明增加了后感染性肠易激综合征的风险。}/{建议调节肠道菌群,控制饮食。}/{\color{lightgray}\faStar}/\currentpos,
        {瘤胃球菌属}/{Ruminococcus}/9.84521/{0.0543588-19.7985354}/44\%/负相关/正常/{其丰度过高与便秘的加重有关。其代谢产物可以影响肠道的运动和水分吸收,从而促进排便,因此缺乏也可能导致便秘。}/{建议保持适度运动,调节肠道菌群平衡。}/{\color{lightgray}\faStar}/\currentpos
    }
    {
        % 计算当前行的基础位置
        \pgfmathsetmacro{\basepos}{-2.8*\currentpos}

        % 菌种名称
        \node[cell, align=left] at (0.5,\basepos) {
            \small\textbf{\item}\\[-0.2em]
            {\color{lightgray}\small\enitem}
        };

        % 正常范围
        \node[reference] at (4.5,\basepos) {\footnotesize\range};

        % 进度条相关
        \pgfmathsetmacro{\barypos}{\basepos-\valuebarspace+0.1}
        \def\barstart{6.75}

        % 进度条背景
        \fill[gray!10, rounded corners=2] (\barstart,\barypos)
            rectangle (\barstart+\barwidth,\barypos+\barheight);

        % 检测丰度值
        \node[value] at (7.5,{\basepos-\valuebarspace+0.6}) {\footnotesize\value};

        % 解析范围并计算进度条长度
        \def\parserange#1-#2\endparse{\def\minval{#1}\def\maxval{#2}}
        \expandafter\parserange\range\endparse

        % 计算进度条长度和颜色
        \pgfmathsetmacro{\progress}{min(\value/\maxval, 1.0)}
        \pgfmathparse{\value > \maxval ? "customred" : (\value < \minval ? "customred" : "green!50")}
        \let\barcolor=\pgfmathresult

        % 进度条显示
        \ifnum\pdfstrcmp{\status}{超标}=0
            \fill[customred, rounded corners=2] (\barstart,\barypos)
                rectangle (\barstart+\barwidth,\barypos+\barheight);
        \else
            \fill[\barcolor, rounded corners=2] (\barstart,\barypos)
                rectangle (\barstart+\barwidth*\progress,\barypos+\barheight);
        \fi

        % 检测结果
        \node[value] at (9.5,\basepos) {\footnotesize\status}; % 检测结果列

        % 超过%的人
        \node[value] at (11.5,\basepos) {\footnotesize\percentile};

        % 失眠相关性
        \node[value] at (13.75,\basepos) {\footnotesize\detection};

        % 相关性强度
        \node[value] at (16,\basepos) {\footnotesize\index};

        % 添加卡片
        \pgfmathsetmacro{\cardypos}{\basepos-0.5}
        \begin{scope}[shift={(0,\cardypos)}]
            % 卡片背景
            \pgfmathsetmacro{\cardheight}{
                \ifnum\pdfstrcmp{\status}{超标}=0
                    1.0  % 两行内容时的高度
                \else
                    0.6  % 一行内容时的高度
                \fi
            }

            \fill[rounded corners=5pt, customTeal!5, draw=gray!5]
                (0.3,-\cardheight) rectangle (17.3,0);

            % 菌群简介图标和内容
            \node[anchor=west] at (0.5,-0.3) {
                \textbf{\color{gray!90}\footnotesize \textcolor{customTeal}{\faInfoCircle}}
            };
            \node[anchor=west, text width=16cm] at (0.9,-0.3) {
                {\small\color{gray}\footnotesize \intro}
            };

            % 异常解读标题和内容
            \ifnum\pdfstrcmp{\status}{超标}=0
                \node[anchor=west] at (0.55,-0.8) {
                    \textbf{\color{customRed}\footnotesize \textcolor{customRed}{\faLightbulb}}
                };
                \node[anchor=west, text width=16cm] at (0.9,-0.8) {
                    {\small\color{gray}\footnotesize \suggestion}
                };
            \fi
        \end{scope}

        % 分割线
        \pgfmathsetmacro{\linepos}{
            \ifnum\pdfstrcmp{\status}{超标}=0
                \basepos-1.7  % 超标时的分割线位置
            \else
                \basepos-1.3  % 正常时的分割线位置
            \fi
        }
        \draw[gray!20] (0.2,\linepos) -- (\cardwidth-0.2,\linepos);

        % 根据当前行的状态计算下一行的位置增量
        \ifnum\pdfstrcmp{\status}{超标}=0
            \pgfmathsetmacro{\increment}{0.85}  % 超标行(两行内容)需要更大的增量
        \else
            \pgfmathsetmacro{\increment}{0.7}  % 正常行(一行内容)使用较小的增量
        \fi

        % 更新位置计数器
        \pgfmathsetmacro{\nextpos}{\currentpos+\increment}
        \xdef\currentpos{\nextpos}
    }

    % 最后一行的处理,消除多余的空白
    \pgfmathsetmacro{\lastincrement}{2}  % 最后一行的增量
    \pgfmathsetmacro{\nextpos}{\currentpos+\lastincrement}
    \xdef\currentpos{\nextpos}

\end{tikzpicture}
\end{center}

\newpage

\begin{tcolorbox}[
    enhanced,
    colback=white,
    colframe=white,
    arc=2mm,
    boxrule=0pt,
    width=\textwidth,
    left=15pt,
    right=15pt,
    top=10pt,
    bottom=10pt,
    drop shadow={
        opacity=0.2,
        color=customTeal
    },
    borderline west={5pt}{0pt}{customTeal}
]
\textcolor{customTeal}{\Large\textbf{腹胀相关菌}}
\end{tcolorbox}

\begin{tcolorbox}[
    enhanced,
    colback=customTealBg,
    colframe=customTealBg,
    arc=3mm,
    boxrule=0pt,
    width=\textwidth,
    top=8pt,
    bottom=8pt
]
{\small{\color{customTeal}\faInfoCircle} 肠道菌群与腹胀的关系已逐渐引起医学研究的关注。大量研究表明,特定的肠道菌群组成与个体的腹胀症状密切相关,肠道微生态的失衡可能导致气体生成增加,从而引发腹胀感。\\

{\color{orange}\faExclamationTriangle} \textbf{特别注意}:
\begin{itemize}
    \item 在本报告中,与腹胀相关的某些肠道菌群可能会出现异常丰度。然而,需要强调的是,这些菌群的丰度变化并不一定会直接导致腹胀症状的出现。腹胀的表现通常与多种因素有关,包括个体的饮食选择、消化能力、生活方式、压力水平及其他健康状况等。
    \item 为了有效降低腹胀的发生风险,建议您综合考虑肠道菌群检测结果以及自身的饮食习惯、运动情况、心理状态等因素。在出现持续的腹胀症状时,应及时咨询专业医生或营养师,以获得合适的建议和对策。
\end{itemize}
}
\end{tcolorbox}


\begin{tcolorbox}[
    enhanced,
    colback=lightpurple!10, % 卡片底色
    colframe=lightpurple!10,  % 边框颜色
    arc=3mm,
    boxrule=0.5pt,
    width=\textwidth,
    top=8pt,
    bottom=8pt
]
{\small{\color{lightpurple}\faQuestionCircle}\quad \textbf{肠道菌群是如何影响腹胀的?}\\
{\color{orange!50}\faComments}\quad 肠道菌群与腹胀之间的关系涉及多个机制,以下是一些主要方面:
\begin{itemize}
    \item \textbf{菌群组成}:益生菌与有害菌:健康的肠道菌群中包含一些益生菌(如乳酸菌、双歧杆菌),这些细菌能有效分解食物并产生气体,适当的气体量是正常的消化过程。然而,过量的有害菌(如某些致病菌)的生长可能导致气体产生增加,从而引起腹胀。
    \item \textbf{发酵过程}:膳食纤维的发酵:肠道细菌会发酵膳食纤维,产生气体和短链脂肪酸。虽然短链脂肪酸对肠道健康有益,但如果发酵产生的气体超出肠道的处理能力,就可能导致腹胀。
    \item \textbf{肠道运动}:肠道动力和神经调节:肠道菌群的代谢产物可以影响肠道的运动状态,促进或抑制蠕动。如果肠道运动减缓,食物在肠道内停留时间过长,可能会导致气体积聚和腹胀。
    \item \textbf{免疫反应}:炎症与肠道健康:肠道微生物失调可能导致局部炎症反应。炎症可能影响肠道的消化能力和气体处理能力,进而引发腹胀。
    \item \textbf{饮食习惯与菌群的相互作用}:特定食物对微生物群落的影响:摄入某些食物(如豆类、洋葱等)会促进特定细菌的繁殖,这些细菌在发酵过程中产生大量气体,可能加重腹胀的程度。
\end{itemize}
肠道菌群通过影响气体生成、肠道运动、免疫反应和饮食习惯等方面,对腹胀的产生有显著影响。然而,不同个体的肠道菌群组成受遗传、饮食、生活方式和环境的影响,各人的腹胀敏感性也可能存在差异。
}
\end{tcolorbox}

\newpage

\begin{center}
\begin{tikzpicture}[
    font=\small,
    title/.style={font=\small\bfseries\color{white}},
    value/.style={font=\small},
    reference/.style={font=\small},
    cell/.style={anchor=west, text width=4.2cm},
    note/.style={anchor=west, text width=4.5cm, align=left}
]
    \def\cardwidth{\textwidth}
    \def\cardheight{22.5}
    \def\barheight{0.25}
    \def\barwidth{1.5}
    \def\valuebarspace{0.4}

    % 容器和标题栏背景
    \draw[rounded corners=5, fill=white, draw=gray!20]
        (0,0) rectangle (\cardwidth,-\cardheight);
    \path[fill=customTeal]
        (0,0) [rounded corners=5] -- (\cardwidth,0) --
        (\cardwidth,0.8) -- (0,0.8) -- cycle;

    % 表头
    \node[title, anchor=west] at (0.5,0.4) {\textbf{菌种名称}};
    \node[title] at (4.5,0.4) {\textbf{正常范围\%}};
    \node[title] at (7.5,0.4) {\textbf{检测丰度}};
    \node[title] at (9.5,0.4) {\textbf{检测结果}}; % 新增检测结果列名
    \node[title] at (11.5,0.4) {\textbf{超过\%的人}};
    \node[title] at (13.75,0.4) {\textbf{失眠相关性}};
    \node[title] at (16,0.4) {\textbf{相关性强度}};

    % 初始化位置计数器
    \def\currentpos{0.25}

    % 数据行和卡片
    \foreach \item/\enitem/\value/\range/\percentile/\detection/\status/\intro/\suggestion/\index in {
        {史氏甲烷短杆菌}/{M. smithii}/ND/{0-0.3301}/99\%/负相关/正常/{主要参与肠道内甲烷的产生。}/{保持健康的饮食习惯,增加膳食纤维摄入。}/{\color{lightgray}\faStar}/\currentpos,
        {拟杆菌属}/{Bacteroides}/0.08974/{1.0577624-47.3225368}/29\%/正相关/偏低/{参与碳水化合物代谢,维持肠道稳态,可能影响睡眠。}/{建议增加膳食纤维摄入,如全谷物、蔬菜水果等}/{\color{lightgray}\faStar \faStar}/\currentpos,
        {梭菌属}/{Clostridium}/1.90370/{0-4.4645924}/67\%/正相关/正常/{参与肠道健康和免疫调节。}/{保持均衡饮食,增加益生元摄入。}/{\color{lightgray}\faStar \faStar}/\currentpos,
        {瘤胃球菌属}/{Ruminococcus}/9.84521/{0.05438-19.7985354}/44\%/正相关/正常/{与肠道代谢相关。}/{保持健康的饮食习惯,增加膳食纤维摄入。}/{\color{lightgray}\faStar}/\currentpos,
        {变形菌属}/{Proteobacteria}/3.89401/{0-1.742}/20\%/正相关/正常/{与肠道微生物失衡相关。}/{建议增加益生元摄入。}/{\color{lightgray}\faStar}/\currentpos,
        {艰难梭菌}/{Clostridium difficile}/0.08647/{0-0.05}/94\%/正相关/超标/{可能导致肠道感染。}/{建议减少糖和油腻食物的摄入。}/{\color{lightgray}\faStar \faStar}/\currentpos,
        {肠球菌属}/{Enterococcus}/0.12259/{0-0.05}/52\%/正相关/正常/{与肠道健康密切相关。}/{保持均衡饮食,增加益生元摄入。}/{\color{lightgray}\faStar}/\currentpos,
        {普雷沃氏菌属}/{Prevotella}/0.04290/{0-67.8009886}/50\%/正相关/正常/{与碳水化合物代谢相关。}/{保持健康的饮食习惯,增加膳食纤维摄入。}/{\color{lightgray}\faStar}/\currentpos,
        {链球菌属}/{Streptococcus}/0.31690/{0-0.3495704}/13\%/正相关/正常/{可能影响免疫功能。}/{保持均衡饮食,增加益生元摄入。}/{\color{lightgray}\faStar}/\currentpos,
        {蓝藻门}/{Cyanobacteria}/0.02100/{0-0}/99\%/负相关/正常/{与肠道微生物失衡相关。}/{保持健康的饮食习惯,增加膳食纤维摄入。}/{\color{lightgray}\faStar}/\currentpos,
        {产气荚膜梭菌}/{Clostridium perfringens}/0.00419/{0-0.05}/-9\%/负相关/正常/{可能导致食物中毒。}/{保持健康的饮食习惯,增加膳食纤维摄入。}/{\color{lightgray}\faStar}/\currentpos
%        {乳杆菌}/{Lactobacillus}/0.00660/{0-0.4302374}/8\%/负相关/正常/{参与乳酸发酵,促进肠道健康。}/{建议适当补充含乳杆菌的益生菌制剂。}/{\color{lightgray}\faStar}/\currentpos,
%        {毛状双歧杆菌}/{Lachnospiraceae}/26.47123/{1.8245-46.849}/27\%/负相关/正常/{与肠道健康密切相关。}/{保持健康的饮食习惯,增加膳食纤维摄入。}/{\color{lightgray}\faStar}/\currentpos,
%        {双歧杆菌}/{Bifidobacterium}/0.03296/{1.7545796-35.500554}/24\%/负相关/正常/{与肠道健康密切相关。}/{保持健康的饮食习惯,增加膳食纤维摄入。}/{\color{lightgray}\faStar}/\currentpos
    }
    {
        % 计算当前行的基础位置
        \pgfmathsetmacro{\basepos}{-2.8*\currentpos}

        % 菌种名称
        \node[cell, align=left] at (0.5,\basepos) {
            \small\textbf{\item}\\[-0.2em]
            {\color{lightgray}\small\enitem}
        };

        % 正常范围
        \node[reference] at (4.5,\basepos) {\footnotesize\range};

        % 进度条相关
        \pgfmathsetmacro{\barypos}{\basepos-\valuebarspace+0.1}
        \def\barstart{6.75}

        % 进度条背景
        \fill[gray!10, rounded corners=2] (\barstart,\barypos)
            rectangle (\barstart+\barwidth,\barypos+\barheight);

        % 检测丰度值
        \node[value] at (7.5,{\basepos-\valuebarspace+0.6}) {\footnotesize\value};

        % 解析范围并计算进度条长度
        \def\parserange#1-#2\endparse{\def\minval{#1}\def\maxval{#2}}
        \expandafter\parserange\range\endparse

        % 计算进度条长度和颜色
        \pgfmathsetmacro{\progress}{min(\value/\maxval, 1.0)}
        \pgfmathparse{\value > \maxval ? "customred" : (\value < \minval ? "customred" : "green!50")}
        \let\barcolor=\pgfmathresult

        % 进度条显示
        \ifnum\pdfstrcmp{\status}{超标}=0
            \fill[customred, rounded corners=2] (\barstart,\barypos)
                rectangle (\barstart+\barwidth,\barypos+\barheight);
        \else
            \fill[\barcolor, rounded corners=2] (\barstart,\barypos)
                rectangle (\barstart+\barwidth*\progress,\barypos+\barheight);
        \fi

        % 检测结果
        \node[value] at (9.5,\basepos) {\footnotesize\status}; % 检测结果列

        % 超过%的人
        \node[value] at (11.5,\basepos) {\footnotesize\percentile};

        % 失眠相关性
        \node[value] at (13.75,\basepos) {\footnotesize\detection};

        % 相关性强度
        \node[value] at (16,\basepos) {\footnotesize\index};

        % 添加卡片
        \pgfmathsetmacro{\cardypos}{\basepos-0.5}
        \begin{scope}[shift={(0,\cardypos)}]
            % 卡片背景
            \pgfmathsetmacro{\cardheight}{
                \ifnum\pdfstrcmp{\status}{超标}=0
                    1.0  % 两行内容时的高度
                \else
                    0.6  % 一行内容时的高度
                \fi
            }

            \fill[rounded corners=5pt, customTeal!5, draw=gray!5]
                (0.3,-\cardheight) rectangle (17.3,0);

            % 菌群简介图标和内容
            \node[anchor=west] at (0.5,-0.3) {
                \textbf{\color{gray!90}\footnotesize \textcolor{customTeal}{\faInfoCircle}}
            };
            \node[anchor=west, text width=16cm] at (0.9,-0.3) {
                {\small\color{gray}\footnotesize \intro}
            };

            % 异常解读标题和内容
            \ifnum\pdfstrcmp{\status}{超标}=0
                \node[anchor=west] at (0.55,-0.8) {
                    \textbf{\color{customRed}\footnotesize \textcolor{customRed}{\faLightbulb}}
                };
                \node[anchor=west, text width=16cm] at (0.9,-0.8) {
                    {\small\color{gray}\footnotesize \suggestion}
                };
            \fi
        \end{scope}

        % 分割线
        \pgfmathsetmacro{\linepos}{
            \ifnum\pdfstrcmp{\status}{超标}=0
                \basepos-1.7  % 超标时的分割线位置
            \else
                \basepos-1.3  % 正常时的分割线位置
            \fi
        }
        \draw[gray!20] (0.2,\linepos) -- (\cardwidth-0.2,\linepos);

        % 根据当前行的状态计算下一行的位置增量
        \ifnum\pdfstrcmp{\status}{超标}=0
            \pgfmathsetmacro{\increment}{0.85}  % 超标行(两行内容)需要更大的增量
        \else
            \pgfmathsetmacro{\increment}{0.7}  % 正常行(一行内容)使用较小的增量
        \fi

        % 更新位置计数器
        \pgfmathsetmacro{\nextpos}{\currentpos+\increment}
        \xdef\currentpos{\nextpos}
    }

    % 最后一行的处理,消除多余的空白
    \pgfmathsetmacro{\lastincrement}{2}  % 最后一行的增量
    \pgfmathsetmacro{\nextpos}{\currentpos+\lastincrement}
    \xdef\currentpos{\nextpos}

\end{tikzpicture}
\end{center}

\newpage

\begin{center}
\begin{tikzpicture}[
    font=\small,
    title/.style={font=\small\bfseries\color{white}},
    value/.style={font=\small},
    reference/.style={font=\small},
    cell/.style={anchor=west, text width=4.2cm},
    note/.style={anchor=west, text width=4.5cm, align=left}
]
    \def\cardwidth{\textwidth}
    \def\cardheight{6}
    \def\barheight{0.25}
    \def\barwidth{1.5}
    \def\valuebarspace{0.4}

    % 容器和标题栏背景
    \draw[rounded corners=5, fill=white, draw=gray!20]
        (0,0) rectangle (\cardwidth,-\cardheight);
    \path[fill=customTeal]
        (0,0) [rounded corners=5] -- (\cardwidth,0) --
        (\cardwidth,0.8) -- (0,0.8) -- cycle;

    % 表头
    \node[title, anchor=west] at (0.5,0.4) {\textbf{菌种名称}};
    \node[title] at (4.5,0.4) {\textbf{正常范围\%}};
    \node[title] at (7.5,0.4) {\textbf{检测丰度}};
    \node[title] at (9.5,0.4) {\textbf{检测结果}}; % 新增检测结果列名
    \node[title] at (11.5,0.4) {\textbf{超过\%的人}};
    \node[title] at (13.75,0.4) {\textbf{失眠相关性}};
    \node[title] at (16,0.4) {\textbf{相关性强度}};

    % 初始化位置计数器
    \def\currentpos{0.25}

    % 数据行和卡片
    \foreach \item/\enitem/\value/\range/\percentile/\detection/\status/\intro/\suggestion/\index in {
        {乳杆菌属}/{Lactobacillus}/0.00660/{0-0.4302374}/8\%/负相关/正常/{在儿童肠易激综合征患者中具有显著的减轻腹痛、打嗝、腹胀和便秘的作用,与高纤维饮食结合可有效减少结肠慢室病患者的腹胀和长期腹痛。}/{建议适量补充乳酸菌,调节肠道菌群平衡。}/{\color{lightgray}\faStar \faStar \faStar}/\currentpos,
        {毛螺旋菌科}/{Lachnospiraceae}/26.47123/{1.8245-46.849}/27\%/负相关/正常/{减少与腹胀有关,可将复杂的植物多糖发酵成乙酸、丁酸和丙酸等短链脂肪酸。}/{建议增加膳食纤维摄入,改善肠道环境。}/{\color{lightgray}\faStar \faStar \faStar}/\currentpos,
        {双歧杆菌属}/{Bifidobacterium}/0.03296/{1.7545796-35.500554}/24\%/负相关/偏低/{与腹胀和腹痛等症状的减轻有关。口服VSL#3等益生菌可以增加有益菌的数量,从而改善便秘、腹泻等症状,并减轻腹胀的程度。}/{建议适量补充双歧杆菌,增加益生菌摄入。}/{\color{lightgray}\faStar \faStar}/\currentpos
    }
    {
        % 计算当前行的基础位置
        \pgfmathsetmacro{\basepos}{-2.8*\currentpos}

        % 菌种名称
        \node[cell, align=left] at (0.5,\basepos) {
            \small\textbf{\item}\\[-0.2em]
            {\color{lightgray}\small\enitem}
        };

        % 正常范围
        \node[reference] at (4.5,\basepos) {\footnotesize\range};

        % 进度条相关
        \pgfmathsetmacro{\barypos}{\basepos-\valuebarspace+0.1}
        \def\barstart{6.75}

        % 进度条背景
        \fill[gray!10, rounded corners=2] (\barstart,\barypos)
            rectangle (\barstart+\barwidth,\barypos+\barheight);

        % 检测丰度值
        \node[value] at (7.5,{\basepos-\valuebarspace+0.6}) {\footnotesize\value};

        % 解析范围并计算进度条长度
        \def\parserange#1-#2\endparse{\def\minval{#1}\def\maxval{#2}}
        \expandafter\parserange\range\endparse

        % 计算进度条长度和颜色
        \pgfmathsetmacro{\progress}{min(\value/\maxval, 1.0)}
        \pgfmathparse{\value > \maxval ? "customred" : (\value < \minval ? "customred" : "green!50")}
        \let\barcolor=\pgfmathresult

        % 进度条显示
        \ifnum\pdfstrcmp{\status}{超标}=0
            \fill[customred, rounded corners=2] (\barstart,\barypos)
                rectangle (\barstart+\barwidth,\barypos+\barheight);
        \else
            \fill[\barcolor, rounded corners=2] (\barstart,\barypos)
                rectangle (\barstart+\barwidth*\progress,\barypos+\barheight);
        \fi

        % 检测结果
        \node[value] at (9.5,\basepos) {\footnotesize\status}; % 检测结果列

        % 超过%的人
        \node[value] at (11.5,\basepos) {\footnotesize\percentile};

        % 失眠相关性
        \node[value] at (13.75,\basepos) {\footnotesize\detection};

        % 相关性强度
        \node[value] at (16,\basepos) {\footnotesize\index};

        % 添加卡片
        \pgfmathsetmacro{\cardypos}{\basepos-0.5}
        \begin{scope}[shift={(0,\cardypos)}]
            % 卡片背景
            \pgfmathsetmacro{\cardheight}{
                \ifnum\pdfstrcmp{\status}{超标}=0
                    1.0  % 两行内容时的高度
                \else
                    0.6  % 一行内容时的高度
                \fi
            }

            \fill[rounded corners=5pt, customTeal!5, draw=gray!5]
                (0.3,-\cardheight) rectangle (17.3,0);

            % 菌群简介图标和内容
            \node[anchor=west] at (0.5,-0.3) {
                \textbf{\color{gray!90}\footnotesize \textcolor{customTeal}{\faInfoCircle}}
            };
            \node[anchor=west, text width=16cm] at (0.9,-0.3) {
                {\small\color{gray}\footnotesize \intro}
            };

            % 异常解读标题和内容
            \ifnum\pdfstrcmp{\status}{超标}=0
                \node[anchor=west] at (0.55,-0.8) {
                    \textbf{\color{customRed}\footnotesize \textcolor{customRed}{\faLightbulb}}
                };
                \node[anchor=west, text width=16cm] at (0.9,-0.8) {
                    {\small\color{gray}\footnotesize \suggestion}
                };
            \fi
        \end{scope}

        % 分割线
        \pgfmathsetmacro{\linepos}{
            \ifnum\pdfstrcmp{\status}{超标}=0
                \basepos-1.7  % 超标时的分割线位置
            \else
                \basepos-1.3  % 正常时的分割线位置
            \fi
        }
        \draw[gray!20] (0.2,\linepos) -- (\cardwidth-0.2,\linepos);

        % 根据当前行的状态计算下一行的位置增量
        \ifnum\pdfstrcmp{\status}{超标}=0
            \pgfmathsetmacro{\increment}{0.85}  % 超标行(两行内容)需要更大的增量
        \else
            \pgfmathsetmacro{\increment}{0.7}  % 正常行(一行内容)使用较小的增量
        \fi

        % 更新位置计数器
        \pgfmathsetmacro{\nextpos}{\currentpos+\increment}
        \xdef\currentpos{\nextpos}
    }

    % 最后一行的处理,消除多余的空白
    \pgfmathsetmacro{\lastincrement}{2}  % 最后一行的增量
    \pgfmathsetmacro{\nextpos}{\currentpos+\lastincrement}
    \xdef\currentpos{\nextpos}

\end{tikzpicture}
\end{center}


\newpage

\begin{tcolorbox}[
    enhanced,
    colback=white,
    colframe=white,
    arc=2mm,
    boxrule=0pt,
    width=\textwidth,
    left=15pt,
    right=15pt,
    top=10pt,
    bottom=10pt,
    drop shadow={
        opacity=0.2,
        color=customTeal
    },
    borderline west={5pt}{0pt}{customTeal}
]
\textcolor{customTeal}{\Large\textbf{过敏相关菌}}
\end{tcolorbox}

\begin{tcolorbox}[
    enhanced,
    colback=customTealBg,
    colframe=customTealBg,
    arc=3mm,
    boxrule=0pt,
    width=\textwidth,
    top=8pt,
    bottom=8pt
]
{\small{\color{customTeal}\faInfoCircle} 肠道菌群与过敏反应之间的关系已成为现代医学研究的热点。越来越多的证据表明,肠道菌群的组成和多样性与个体的免疫系统密切相关,某些肠道菌群的失调可能会促使过敏反应的发生。\\

{\color{orange}\faExclamationTriangle} \textbf{特别注意}:
\begin{itemize}
    \item 在本报告中,与过敏相关的某些肠道菌群可能会显示出异常水平。然而,重要的是要注意,这些菌群的丰度异常并不一定会直接引发过敏症状。过敏的发生往往是由多种复杂因素共同作用的结果,包括遗传因素、环境影响、饮食习惯和个体特殊的免疫反应等。
    \item 为了有效降低过敏的发生风险,建议您结合肠道菌群检测结果以及自身的生活习惯、过敏史等因素进行综合评估。在出现过敏症状时,应及时咨询专业医生或过敏专家,以获取适当的诊断和治疗建议。
\end{itemize}
}
\end{tcolorbox}

\begin{tcolorbox}[
    enhanced,
    colback=lightpurple!10, % 卡片底色
    colframe=lightpurple!10,  % 边框颜色
    arc=3mm,
    boxrule=0.5pt,
    width=\textwidth,
    top=8pt,
    bottom=8pt
]
{\small{\color{lightpurple}\faQuestionCircle}\quad \textbf{肠道菌群是如何影响过敏的?}\\
{\color{orange!50}\faComments}\quad 肠道菌群与过敏之间的关系涉及多个机制,以下是一些主要方面:
\begin{itemize}
    \item \textbf{免疫调节}:肠道菌群在调节免疫反应方面发挥重要作用。健康的肠道菌群能够增强免疫耐受,帮助机体适应普通过敏原,而失调的菌群则可能使机体对某些物质产生过度敏感,导致过敏反应的发生。
    \item \textbf{细菌代谢产物}:短链脂肪酸(SCFAs):肠道细菌通过发酵纤维素产生的短链脂肪酸可以调节免疫细胞,促进免疫系统的平衡,降低过敏的发展风险。若SCFAs产量不足,可能会导致过敏症状的加重。
    \item \textbf{肠道屏障功能}:肠道菌群有助于维持肠道屏障的完整性,防止有害物质和过敏原进入血液循环。若肠道屏障受损,可能导致系统性炎症反应和过敏的发展。
    \item \textbf{菌群多样性}:多样性较高的肠道菌群通常与更好的免疫功能相关。而单一或减少的菌群多样性则可能与过敏反应的增加相关。
    \item \textbf{饮食和生活方式}:饮食结构、抗生素使用、环境暴露等生活方式因素对肠道菌群的组成产生持续影响,进而影响个体对过敏的易感性。
\end{itemize}
肠道菌群通过调节免疫系统、产生代谢产物、维护肠道屏障等方式对过敏的发生具有重要影响。然而,不同个体的肠道菌群组成受多种因素影响,包括遗传、饮食和环境,导致过敏的易感性存在差异。
}
\end{tcolorbox}

\newpage

\begin{center}
\begin{tikzpicture}[
    font=\small,
    title/.style={font=\small\bfseries\color{white}},
    value/.style={font=\small},
    reference/.style={font=\small},
    cell/.style={anchor=west, text width=4.2cm},
    note/.style={anchor=west, text width=4.5cm, align=left}
]
    \def\cardwidth{\textwidth}
    \def\cardheight{22.5}
    \def\barheight{0.25}
    \def\barwidth{1.5}
    \def\valuebarspace{0.4}

    % 容器和标题栏背景
    \draw[rounded corners=5, fill=white, draw=gray!20]
        (0,0) rectangle (\cardwidth,-\cardheight);
    \path[fill=customTeal]
        (0,0) [rounded corners=5] -- (\cardwidth,0) --
        (\cardwidth,0.8) -- (0,0.8) -- cycle;

    % 表头
    \node[title, anchor=west] at (0.5,0.4) {\textbf{菌种名称}};
    \node[title] at (4.5,0.4) {\textbf{正常范围\%}};
    \node[title] at (7.5,0.4) {\textbf{检测丰度}};
    \node[title] at (9.5,0.4) {\textbf{检测结果}}; % 新增检测结果列名
    \node[title] at (11.5,0.4) {\textbf{超过\%的人}};
    \node[title] at (13.75,0.4) {\textbf{失眠相关性}};
    \node[title] at (16,0.4) {\textbf{相关性强度}};

    % 初始化位置计数器
    \def\currentpos{0.25}

    % 数据行和卡片
    \foreach \item/\enitem/\value/\range/\percentile/\detection/\status/\intro/\suggestion/\index in {
        {脆弱拟杆菌}/{B. fragilis}/0.00145/{0-0.05}/8\%/正相关/正常/{花粉过敏成年人丰度上升,可以通过补充双歧杆菌来预防,会诱导更多的Th2细胞因子,与花生和坚果过敏存在相关。}/{建议适量补充双歧杆菌,调节免疫反应。}/{\color{lightgray}\faStar \faStar \faStar}/\currentpos,
        {构橘酸杆菌属}/{Citrobacter}/0.07731/{0-0.5}/-\%/正相关/正常/{Citrobacter在食物过敏的肠道中明显富集,会加重系统性过敏症状并减少肠道Th17细胞。}/{建议调节肠道菌群,改善过敏症状。}/{\color{lightgray}\faStar \faStar}/\currentpos,
        {克雷伯氏菌属}/{Klebsiella}/0.09652/{0-0.05}/-\%/正相关/超标/{过度生长可能会导致肠道微生物群落失衡,从而引发过敏反应。过敏婴儿肠道中Klebsiella的数量明显增加,有益菌数量减少。}/{建议:乳杆菌补充、双歧杆菌补充、大麦、低聚果糖}/{\color{lightgray}\faStar \faStar}/\currentpos,
        {莫拉氏菌属}/{Moraxella}/0.00228/{0-0}/99\%/正相关/超标/{与哮喘等过敏性疾病有关联。儿童早期感染Moraxella会增加呼吸道疾病的严重程度和发生急性哮喘的风险。}/{建议:低聚甘露糖、大蒜}/{\color{lightgray}\faStar \faStar}/\currentpos,
        {变形菌门}/{Proteobacteria}/3.89408/{0-1.742}/20\%/正相关/超标/{增加血液中的脂多糖内毒素,增加肠道通透性,导致慢性炎症反应。提入低聚果糖可调节肠道微生物,改善过敏反应。}/{建议:低脂饮食、牛磺酸、生酮饮食、聚甘露糖醛酸}/{\color{lightgray}\faStar \faStar}/\currentpos,
        {肠杆菌科}/{Enterobacteriaceae}/0.55581/{0-5.7271}/7\%/正相关/正常/{早期婴儿肠道微生物群落中,丰度的增加与食物敏感性的发展有关联。可以产生脂多糖(LPS),与多种代谢性疾病的炎症有关。}/{建议调节肠道菌群,改善免疫功能。}/{\color{lightgray}\faStar \faStar}/\currentpos,
        {活波瘤胃球菌}/{R. gnavus}/0.22486/{0-0.05}/20\%/正相关/超标/{与食物敏感性、哮喘、呼吸道过敏等疾病的发生有关,会导致肠道内炎症反应的加剧,引发过敏症状。降低肠道内纤维素降解酶的潜力。}/{建议:咖啡、乳杆菌补充、白藜芦醇、母乳低聚糖、羧甲基纤维素}/{\color{lightgray}\faStar \faStar}/\currentpos,
        {气单胞菌属}/{Aeromonas}/0.00200/{0-0.05}/-\%/正相关/正常/{感染会引起过敏反应,如Kounis综合征,同时过敏患者对某些抗生素也可能存在过敏反应,因此需要注意过敏史。}/{建议注意过敏史,谨慎用药。}/{\color{lightgray}\faStar}/\currentpos,
        {经黏液真杆菌属}/{Blautia}/10.17279/{0.0846204-6.9055608}/75\%/正相关/超标/{高丰度可能会导致过敏疾病的发展,如婴儿的食物过敏和儿童的1型糖尿病。}/{建议:高脂汁酸,红酒,啤酒}/{\color{lightgray}\faStar}/\currentpos,
        {肠球菌属}/{Enterococcus}/0.12259/{0-0.05}/52\%/正相关/超标/{通过激活巨噬细胞,促进炎症过程,从而影响过敏或变态反应的人类IgE抗体调节。}/{建议:乳杆菌补充、可可、母乳低聚糖、椰皮素}/{\color{lightgray}\faStar}/\currentpos
    }
    {
        % 计算当前行的基础位置
        \pgfmathsetmacro{\basepos}{-2.8*\currentpos}

        % 菌种名称
        \node[cell, align=left] at (0.5,\basepos) {
            \small\textbf{\item}\\[-0.2em]
            {\color{lightgray}\small\enitem}
        };

        % 正常范围
        \node[reference] at (4.5,\basepos) {\footnotesize\range};

        % 进度条相关
        \pgfmathsetmacro{\barypos}{\basepos-\valuebarspace+0.1}
        \def\barstart{6.75}

        % 进度条背景
        \fill[gray!10, rounded corners=2] (\barstart,\barypos)
            rectangle (\barstart+\barwidth,\barypos+\barheight);

        % 检测丰度值
        \node[value] at (7.5,{\basepos-\valuebarspace+0.6}) {\footnotesize\value};

        % 解析范围并计算进度条长度
        \def\parserange#1-#2\endparse{\def\minval{#1}\def\maxval{#2}}
        \expandafter\parserange\range\endparse

        % 计算进度条长度和颜色
        \pgfmathsetmacro{\progress}{min(\value/\maxval, 1.0)}
        \pgfmathparse{\value > \maxval ? "customred" : (\value < \minval ? "customred" : "green!50")}
        \let\barcolor=\pgfmathresult

        % 进度条显示
        \ifnum\pdfstrcmp{\status}{超标}=0
            \fill[customred, rounded corners=2] (\barstart,\barypos)
                rectangle (\barstart+\barwidth,\barypos+\barheight);
        \else
            \fill[\barcolor, rounded corners=2] (\barstart,\barypos)
                rectangle (\barstart+\barwidth*\progress,\barypos+\barheight);
        \fi

        % 检测结果
        \node[value] at (9.5,\basepos) {\footnotesize\status}; % 检测结果列

        % 超过%的人
        \node[value] at (11.5,\basepos) {\footnotesize\percentile};

        % 失眠相关性
        \node[value] at (13.75,\basepos) {\footnotesize\detection};

        % 相关性强度
        \node[value] at (16,\basepos) {\footnotesize\index};

        % 添加卡片
        \pgfmathsetmacro{\cardypos}{\basepos-0.5}
        \begin{scope}[shift={(0,\cardypos)}]
            % 卡片背景
            \pgfmathsetmacro{\cardheight}{
                \ifnum\pdfstrcmp{\status}{超标}=0
                    1.0  % 两行内容时的高度
                \else
                    0.6  % 一行内容时的高度
                \fi
            }

            \fill[rounded corners=5pt, customTeal!5, draw=gray!5]
                (0.3,-\cardheight) rectangle (17.3,0);

            % 菌群简介图标和内容
            \node[anchor=west] at (0.5,-0.3) {
                \textbf{\color{gray!90}\footnotesize \textcolor{customTeal}{\faInfoCircle}}
            };
            \node[anchor=west, text width=16cm] at (0.9,-0.3) {
                {\small\color{gray}\footnotesize \intro}
            };

            % 异常解读标题和内容
            \ifnum\pdfstrcmp{\status}{超标}=0
                \node[anchor=west] at (0.55,-0.8) {
                    \textbf{\color{customRed}\footnotesize \textcolor{customRed}{\faLightbulb}}
                };
                \node[anchor=west, text width=16cm] at (0.9,-0.8) {
                    {\small\color{gray}\footnotesize \suggestion}
                };
            \fi
        \end{scope}

        % 分割线
        \pgfmathsetmacro{\linepos}{
            \ifnum\pdfstrcmp{\status}{超标}=0
                \basepos-1.7  % 超标时的分割线位置
            \else
                \basepos-1.3  % 正常时的分割线位置
            \fi
        }
        \draw[gray!20] (0.2,\linepos) -- (\cardwidth-0.2,\linepos);

        % 根据当前行的状态计算下一行的位置增量
        \ifnum\pdfstrcmp{\status}{超标}=0
            \pgfmathsetmacro{\increment}{0.85}  % 超标行(两行内容)需要更大的增量
        \else
            \pgfmathsetmacro{\increment}{0.7}  % 正常行(一行内容)使用较小的增量
        \fi

        % 更新位置计数器
        \pgfmathsetmacro{\nextpos}{\currentpos+\increment}
        \xdef\currentpos{\nextpos}
    }

    % 最后一行的处理,消除多余的空白
    \pgfmathsetmacro{\lastincrement}{2}  % 最后一行的增量
    \pgfmathsetmacro{\nextpos}{\currentpos+\lastincrement}
    \xdef\currentpos{\nextpos}

\end{tikzpicture}
\end{center}

\newpage

\begin{center}
\begin{tikzpicture}[
    font=\small,
    title/.style={font=\small\bfseries\color{white}},
    value/.style={font=\small},
    reference/.style={font=\small},
    cell/.style={anchor=west, text width=4.2cm},
    note/.style={anchor=west, text width=4.5cm, align=left}
]
    \def\cardwidth{\textwidth}
    \def\cardheight{22.5}
    \def\barheight{0.25}
    \def\barwidth{1.5}
    \def\valuebarspace{0.4}

    % 容器和标题栏背景
    \draw[rounded corners=5, fill=white, draw=gray!20]
        (0,0) rectangle (\cardwidth,-\cardheight);
    \path[fill=customTeal]
        (0,0) [rounded corners=5] -- (\cardwidth,0) --
        (\cardwidth,0.8) -- (0,0.8) -- cycle;

    % 表头
    \node[title, anchor=west] at (0.5,0.4) {\textbf{菌种名称}};
    \node[title] at (4.5,0.4) {\textbf{正常范围\%}};
    \node[title] at (7.5,0.4) {\textbf{检测丰度}};
    \node[title] at (9.5,0.4) {\textbf{检测结果}}; % 新增检测结果列名
    \node[title] at (11.5,0.4) {\textbf{超过\%的人}};
    \node[title] at (13.75,0.4) {\textbf{失眠相关性}};
    \node[title] at (16,0.4) {\textbf{相关性强度}};

    % 初始化位置计数器
    \def\currentpos{0.25}

    % 数据行和卡片
    \foreach \item/\enitem/\value/\range/\percentile/\detection/\status/\intro/\suggestion/\index in {
        {粪杆菌属}/{Faecalibacterium}/28.86665/{1.9350868-17.7942438}/99\%/正相关/超标/{能够产生丰富的短链脂肪酸和其他代谢物,减少过敏原物质的进入和吸收,减少过度的免疫和炎症反应,降低过敏的严重程度和持续时间。}/{建议:乳杆菌补充、亚麻籽、壳聚糖、柿子糖(甜栗)}/{\color{lightgray}\faStar}/\currentpos,
        {丙酸杆菌属}/{Propionibacterium}/ND/{0-0}/99\%/正相关/正常/{能够产生丙酸和细菌素,可以抑制S. aureus或其他病原菌的增殖,被认为具有抗过敏的作用,包括对哮喘和AD的保护作用。}/{建议调节肠道菌群,改善免疫功能。}/{\color{lightgray}\faStar}/\currentpos,
        {罗氏菌属}/{Roseburia}/6.33667/{0.5829956-16.3580638}/10\%/正相关/正常/{可产生丁酸的益生菌,食物过敏患者的肠道中Roseburia的丰度较低,通过摄食干预或益生菌治疗,可以增加Roseburia的丰度,促进过敏的缓解。}/{建议增加膳食纤维摄入,改善肠道环境。}/{\color{lightgray}\faStar}/\currentpos,
        {链状双歧杆菌}/{Bifidobacterium catenulatum}/ND/{0-0.50838}/99\%/正相关/正常/{成人型哮喘可能会促进过敏炎症所特有的Th2偏向免疫反应。}/{建议适量补充双歧杆菌,调节免疫反应。}/{\color{lightgray}\faStar}/\currentpos,
        {铜绿假单胞菌}/{Pseudomonas aeruginosa}/0.04152/{0-0.05}/-\%/正相关/正常/{感染也可能会引起过敏反应,感染可能会导致过敏性支气管曲霉病(ABPA)的发生,增加宿主患上过敏性疾病的风险。}/{建议注意卫生,预防感染。}/{\color{lightgray}\faStar}/\currentpos,
        {双歧杆菌属}/{Bifidobacterium}/0.03296/{1.7545796-35.500554}/24\%/负相关/偏低/{可以调节肠道免疫系统,减少过敏反应。Bifidobacterium可以改善过敏症状,如哮喘和湿疹,可以进一步提高其对过敏的保护作用。}/{建议适量补充双歧杆菌,增加益生菌摄入。}/{\color{lightgray}\faStar \faStar \faStar}/\currentpos,
        {乳杆菌属}/{Lactobacillus}/0.00660/{0-0.4302374}/8\%/负相关/正常/{可以减轻食物过敏和湿疹等过敏症状,可以通过降低食物中过敏原的含量来减轻过敏症状。}/{建议适量补充乳酸菌,改善免疫功能。}/{\color{lightgray}\faStar \faStar \faStar}/\currentpos,
        {粪杆菌属}/{Faecalibacterium}/28.86665/{1.9350868-17.7942438}/99\%/负相关/超标/{产生丰富的短链脂肪酸,如丙酸、丁酸和乙酸等,维持肠道健康和免疫系统的正常功能,促进免疫系统的平衡,减少过敏症状的发生。}/{建议:乳杆菌补充、亚麻籽、壳聚糖、柿子糖(甜栗)}/{\color{lightgray}\faStar \faStar \faStar}/\currentpos,
        {艾克曼氏菌}/{Akkermansia muciniphila}/0.00394/{0-6.6395}/45\%/负相关/正常/{调节肠道微生物群和短链脂肪酸预防过敏,阻断免疫细胞的流入,减轻病理,降低了炎症水平。}/{建议增加膳食纤维摄入,改善肠道环境。}/{\color{lightgray}\faStar \faStar \faStar}/\currentpos,
        {阿克曼氏菌属}/{Akkermansia}/0.00460/{0-5.7984096}/44\%/负相关/正常/{减少多种炎症标志物,修复肠道,富含多糖的蔓越莓提取物和其他富含类黄酮的食物,包括绿茶和红茶可增高Akk水平。}/{建议增加富含多糖和类黄酮的食物摄入。}/{\color{lightgray}\faStar \faStar}/\currentpos,
        {拟杆菌属}/{Bacteroides}/0.08974/{1.0577624-47.3225368}/2\%/负相关/偏低/{在婴儿期和幼儿期,丰度降低与食物过敏或特应性皮炎的发生有关,过多也与过敏结果有关,包括哮喘、花粉症和食物过敏等。}/{建议调节肠道菌群平衡。}/{\color{lightgray}\faStar \faStar}/\currentpos
    }
    {
        % 计算当前行的基础位置
        \pgfmathsetmacro{\basepos}{-2.8*\currentpos}

        % 菌种名称
        \node[cell, align=left] at (0.5,\basepos) {
            \small\textbf{\item}\\[-0.2em]
            {\color{lightgray}\small\enitem}
        };

        % 正常范围
        \node[reference] at (4.5,\basepos) {\footnotesize\range};

        % 进度条相关
        \pgfmathsetmacro{\barypos}{\basepos-\valuebarspace+0.1}
        \def\barstart{6.75}

        % 进度条背景
        \fill[gray!10, rounded corners=2] (\barstart,\barypos)
            rectangle (\barstart+\barwidth,\barypos+\barheight);

        % 检测丰度值
        \node[value] at (7.5,{\basepos-\valuebarspace+0.6}) {\footnotesize\value};

        % 解析范围并计算进度条长度
        \def\parserange#1-#2\endparse{\def\minval{#1}\def\maxval{#2}}
        \expandafter\parserange\range\endparse

        % 计算进度条长度和颜色
        \pgfmathsetmacro{\progress}{min(\value/\maxval, 1.0)}
        \pgfmathparse{\value > \maxval ? "customred" : (\value < \minval ? "customred" : "green!50")}
        \let\barcolor=\pgfmathresult

        % 进度条显示
        \ifnum\pdfstrcmp{\status}{超标}=0
            \fill[customred, rounded corners=2] (\barstart,\barypos)
                rectangle (\barstart+\barwidth,\barypos+\barheight);
        \else
            \fill[\barcolor, rounded corners=2] (\barstart,\barypos)
                rectangle (\barstart+\barwidth*\progress,\barypos+\barheight);
        \fi

        % 检测结果
        \node[value] at (9.5,\basepos) {\footnotesize\status}; % 检测结果列

        % 超过%的人
        \node[value] at (11.5,\basepos) {\footnotesize\percentile};

        % 失眠相关性
        \node[value] at (13.75,\basepos) {\footnotesize\detection};

        % 相关性强度
        \node[value] at (16,\basepos) {\footnotesize\index};

        % 添加卡片
        \pgfmathsetmacro{\cardypos}{\basepos-0.5}
        \begin{scope}[shift={(0,\cardypos)}]
            % 卡片背景
            \pgfmathsetmacro{\cardheight}{
                \ifnum\pdfstrcmp{\status}{超标}=0
                    1.0  % 两行内容时的高度
                \else
                    0.6  % 一行内容时的高度
                \fi
            }

            \fill[rounded corners=5pt, customTeal!5, draw=gray!5]
                (0.3,-\cardheight) rectangle (17.3,0);

            % 菌群简介图标和内容
            \node[anchor=west] at (0.5,-0.3) {
                \textbf{\color{gray!90}\footnotesize \textcolor{customTeal}{\faInfoCircle}}
            };
            \node[anchor=west, text width=16cm] at (0.9,-0.3) {
                {\small\color{gray}\footnotesize \intro}
            };

            % 异常解读标题和内容
            \ifnum\pdfstrcmp{\status}{超标}=0
                \node[anchor=west] at (0.55,-0.8) {
                    \textbf{\color{customRed}\footnotesize \textcolor{customRed}{\faLightbulb}}
                };
                \node[anchor=west, text width=16cm] at (0.9,-0.8) {
                    {\small\color{gray}\footnotesize \suggestion}
                };
            \fi
        \end{scope}

        % 分割线
        \pgfmathsetmacro{\linepos}{
            \ifnum\pdfstrcmp{\status}{超标}=0
                \basepos-1.7  % 超标时的分割线位置
            \else
                \basepos-1.3  % 正常时的分割线位置
            \fi
        }
        \draw[gray!20] (0.2,\linepos) -- (\cardwidth-0.2,\linepos);

        % 根据当前行的状态计算下一行的位置增量
        \ifnum\pdfstrcmp{\status}{超标}=0
            \pgfmathsetmacro{\increment}{0.85}  % 超标行(两行内容)需要更大的增量
        \else
            \pgfmathsetmacro{\increment}{0.7}  % 正常行(一行内容)使用较小的增量
        \fi

        % 更新位置计数器
        \pgfmathsetmacro{\nextpos}{\currentpos+\increment}
        \xdef\currentpos{\nextpos}
    }

    % 最后一行的处理,消除多余的空白
    \pgfmathsetmacro{\lastincrement}{2}  % 最后一行的增量
    \pgfmathsetmacro{\nextpos}{\currentpos+\lastincrement}
    \xdef\currentpos{\nextpos}

\end{tikzpicture}
\end{center}

\newpage

\begin{center}
\begin{tikzpicture}[
    font=\small,
    title/.style={font=\small\bfseries\color{white}},
    value/.style={font=\small},
    reference/.style={font=\small},
    cell/.style={anchor=west, text width=4.2cm},
    note/.style={anchor=west, text width=4.5cm, align=left}
]
    \def\cardwidth{\textwidth}
    \def\cardheight{16}
    \def\barheight{0.25}
    \def\barwidth{1.5}
    \def\valuebarspace{0.4}

    % 容器和标题栏背景
    \draw[rounded corners=5, fill=white, draw=gray!20]
        (0,0) rectangle (\cardwidth,-\cardheight);
    \path[fill=customTeal]
        (0,0) [rounded corners=5] -- (\cardwidth,0) --
        (\cardwidth,0.8) -- (0,0.8) -- cycle;

    % 表头
    \node[title, anchor=west] at (0.5,0.4) {\textbf{菌种名称}};
    \node[title] at (4.5,0.4) {\textbf{正常范围\%}};
    \node[title] at (7.5,0.4) {\textbf{检测丰度}};
    \node[title] at (9.5,0.4) {\textbf{检测结果}}; % 新增检测结果列名
    \node[title] at (11.5,0.4) {\textbf{超过\%的人}};
    \node[title] at (13.75,0.4) {\textbf{失眠相关性}};
    \node[title] at (16,0.4) {\textbf{相关性强度}};

    % 初始化位置计数器
    \def\currentpos{0.25}

    % 数据行和卡片
    \foreach \item/\enitem/\value/\range/\percentile/\detection/\status/\intro/\suggestion/\index in {
        {经黏液真杆菌属}/{Blautia}/10.17279/{0.0846204-6.9055608}/75\%/负相关/超标/{高丰度可能会导致过敏疾病的发展,如婴儿的食物过敏和儿童的1型糖尿病。}/{建议:高脂汁酸、红酒、啤酒}/{\color{lightgray}\faStar \faStar}/\currentpos,
        {Dorea菌属}/{Dorea}/0.43735/{0.05807-5.2120164}/20\%/负相关/正常/{Dorea与食物过敏和免疫有关,有助于保护免疫系统免受食物敏感和食物过敏的影响。}/{建议保持均衡饮食,改善肠道环境。}/{\color{lightgray}\faStar \faStar}/\currentpos,
        {普雷沃氏菌属}/{Prevotella}/0.04290/{0-67.8009886}/50\%/负相关/正常/{孕期母亲肠道微生物中Prevotella的丰度增加可以预防食物过敏,在食物过敏患者中,Prevotella的丰度较低。}/{建议调节肠道菌群,改善免疫功能。}/{\color{lightgray}\faStar \faStar}/\currentpos,
        {罗氏菌属}/{Roseburia}/6.33667/{0.5829956-16.3580638}/10\%/负相关/正常/{可产生丁酸的益生菌,食物过敏患者的肠道中Roseburia的丰度较低,通过膳食干预或益生菌治疗,可以增加Roseburia的丰度,促进过敏的缓解。}/{建议增加膳食纤维摄入,改善肠道环境。}/{\color{lightgray}\faStar \faStar}/\currentpos,
        {青春双歧杆菌}/{Bifidobacterium adolescentis}/ND/{0-14.286}/99\%/负相关/正常/{过敏儿童的肠道中数量较少,缺乏可能导致免疫系统的失调,使食物过敏原更容易穿过肠道屏障,导致食物过敏性的增加。}/{建议适量补充双歧杆菌,改善免疫功能。}/{\color{lightgray}\faStar \faStar}/\currentpos,
        {Lachnoclostridium}/{Lachnoclostridium}/0.17516/{0-0.2086068}/8\%/负相关/正常/{Lachnoclostridium可产生短链脂肪酸,减少食物过敏的发生。}/{建议增加膳食纤维摄入,促进短链脂肪酸的产生。}/{\color{lightgray}\faStar}/\currentpos,
        {普通拟杆菌}/{Bacteroides vulgatus}/ND/{0-20.066}/4\%/负相关/正常/{食物过敏与Bacteroides vulgatus相对丰度呈负相关。}/{建议调节肠道菌群平衡,改善免疫功能。}/{\color{lightgray}\faStar}/\currentpos
    }
    {
        % 计算当前行的基础位置
        \pgfmathsetmacro{\basepos}{-2.8*\currentpos}

        % 菌种名称
        \node[cell, align=left] at (0.5,\basepos) {
            \small\textbf{\item}\\[-0.2em]
            {\color{lightgray}\small\enitem}
        };

        % 正常范围
        \node[reference] at (4.5,\basepos) {\footnotesize\range};

        % 进度条相关
        \pgfmathsetmacro{\barypos}{\basepos-\valuebarspace+0.1}
        \def\barstart{6.75}

        % 进度条背景
        \fill[gray!10, rounded corners=2] (\barstart,\barypos)
            rectangle (\barstart+\barwidth,\barypos+\barheight);

        % 检测丰度值
        \node[value] at (7.5,{\basepos-\valuebarspace+0.6}) {\footnotesize\value};

        % 解析范围并计算进度条长度
        \def\parserange#1-#2\endparse{\def\minval{#1}\def\maxval{#2}}
        \expandafter\parserange\range\endparse

        % 计算进度条长度和颜色
        \pgfmathsetmacro{\progress}{min(\value/\maxval, 1.0)}
        \pgfmathparse{\value > \maxval ? "customred" : (\value < \minval ? "customred" : "green!50")}
        \let\barcolor=\pgfmathresult

        % 进度条显示
        \ifnum\pdfstrcmp{\status}{超标}=0
            \fill[customred, rounded corners=2] (\barstart,\barypos)
                rectangle (\barstart+\barwidth,\barypos+\barheight);
        \else
            \fill[\barcolor, rounded corners=2] (\barstart,\barypos)
                rectangle (\barstart+\barwidth*\progress,\barypos+\barheight);
        \fi

        % 检测结果
        \node[value] at (9.5,\basepos) {\footnotesize\status}; % 检测结果列

        % 超过%的人
        \node[value] at (11.5,\basepos) {\footnotesize\percentile};

        % 失眠相关性
        \node[value] at (13.75,\basepos) {\footnotesize\detection};

        % 相关性强度
        \node[value] at (16,\basepos) {\footnotesize\index};

        % 添加卡片
        \pgfmathsetmacro{\cardypos}{\basepos-0.5}
        \begin{scope}[shift={(0,\cardypos)}]
            % 卡片背景
            \pgfmathsetmacro{\cardheight}{
                \ifnum\pdfstrcmp{\status}{超标}=0
                    1.0  % 两行内容时的高度
                \else
                    0.6  % 一行内容时的高度
                \fi
            }

            \fill[rounded corners=5pt, customTeal!5, draw=gray!5]
                (0.3,-\cardheight) rectangle (17.3,0);

            % 菌群简介图标和内容
            \node[anchor=west] at (0.5,-0.3) {
                \textbf{\color{gray!90}\footnotesize \textcolor{customTeal}{\faInfoCircle}}
            };
            \node[anchor=west, text width=16cm] at (0.9,-0.3) {
                {\small\color{gray}\footnotesize \intro}
            };

            % 异常解读标题和内容
            \ifnum\pdfstrcmp{\status}{超标}=0
                \node[anchor=west] at (0.55,-0.8) {
                    \textbf{\color{customRed}\footnotesize \textcolor{customRed}{\faLightbulb}}
                };
                \node[anchor=west, text width=16cm] at (0.9,-0.8) {
                    {\small\color{gray}\footnotesize \suggestion}
                };
            \fi
        \end{scope}

        % 分割线
        \pgfmathsetmacro{\linepos}{
            \ifnum\pdfstrcmp{\status}{超标}=0
                \basepos-1.7  % 超标时的分割线位置
            \else
                \basepos-1.3  % 正常时的分割线位置
            \fi
        }
        \draw[gray!20] (0.2,\linepos) -- (\cardwidth-0.2,\linepos);

        % 根据当前行的状态计算下一行的位置增量
        \ifnum\pdfstrcmp{\status}{超标}=0
            \pgfmathsetmacro{\increment}{0.85}  % 超标行(两行内容)需要更大的增量
        \else
            \pgfmathsetmacro{\increment}{0.7}  % 正常行(一行内容)使用较小的增量
        \fi

        % 更新位置计数器
        \pgfmathsetmacro{\nextpos}{\currentpos+\increment}
        \xdef\currentpos{\nextpos}
    }

    % 最后一行的处理,消除多余的空白
    \pgfmathsetmacro{\lastincrement}{2}  % 最后一行的增量
    \pgfmathsetmacro{\nextpos}{\currentpos+\lastincrement}
    \xdef\currentpos{\nextpos}

\end{tikzpicture}
\end{center}

\newpage

\begin{tcolorbox}[
    enhanced,
    colback=white,
    colframe=white,
    arc=2mm,
    boxrule=0pt,
    width=\textwidth,
    left=15pt,
    right=15pt,
    top=10pt,
    bottom=10pt,
    drop shadow={
        opacity=0.2,
        color=customTeal
    },
    borderline west={5pt}{0pt}{customTeal}
]
\textcolor{customTeal}{\Large\textbf{抑郁相关菌}}
\end{tcolorbox}

\begin{tcolorbox}[
    enhanced,
    colback=customTealBg,
    colframe=customTealBg,
    arc=3mm,
    boxrule=0pt,
    width=\textwidth,
    top=8pt,
    bottom=8pt
]
{\small{\color{customTeal}\faInfoCircle} 肠道菌群与抑郁症之间的关系正受到越来越多的关注。研究显示,肠道菌群的组成和多样性可能影响大脑功能和心理状态,某些肠道菌群失衡可能与抑郁症状的出现相关。通过调节肠道微生态,可能有助于改善抑郁症状。\\

{\color{orange}\faExclamationTriangle} \textbf{特别注意}:
\begin{itemize}
    \item 在本报告中,与抑郁症相关的某些肠道菌群可能会出现异常水平。但是,需要强调的是,这些菌群的丰度异常并不一定直接导致抑郁症状的发展。抑郁症的发生通常涉及遗传因素、环境压力、社会支持以及个体的生活方式等多重因素。
    \item 为了有效降低抑郁症的发生风险,建议您结合肠道菌群检测结果以及自身的心理状态、饮食习惯、锻炼情况等因素进行综合评估。如果出现持续的抑郁症状,应及时寻求专业心理医生或精神健康专家的建议和支持。
\end{itemize}
}
\end{tcolorbox}


\begin{tcolorbox}[
    enhanced,
    colback=lightpurple!10, % 卡片底色
    colframe=lightpurple!10,  % 边框颜色
    arc=3mm,
    boxrule=0.5pt,
    width=\textwidth,
    top=8pt,
    bottom=8pt
]
{\small{\color{lightpurple}\faQuestionCircle}\quad \textbf{肠道菌群是如何影响抑郁的?}\\
{\color{orange!50}\faComments}\quad 肠道菌群与抑郁之间的关系涉及多个机制,以下是一些主要方面:
\begin{itemize}
    \item \textbf{神经递质的合成}:肠道细菌能合成多种神经递质,如血清素。这些神经递质在调节情绪和行为中扮演重要角色,而肠道菌群的失衡可能会导致这些物质的产生不足,从而影响情绪。
    \item \textbf{炎症反应}:肠道菌群的失调可能导致局部和全身性炎症增加。炎症因子可以影响大脑功能,研究表明,慢性炎症与抑郁症的发生相关,肠道菌群的健康有助于维持低水平的炎症。
    \item \textbf{肠-脑轴}:肠道和大脑之间存在复杂的双向交流(肠-脑轴),肠道菌群通过代谢产物和细菌信号影响脑功能。健康的菌群能够促进良好的心理健康,而失调的菌群可能通过这一通路影响抑郁。
    \item \textbf{微生物群多样性}:研究发现,多样性较高的肠道菌群通常与较好的心理健康相关。相反,低多样性的菌群与抑郁症状存在关联,表明肠道微生物的多样性可能是情绪健康的重要指标。
    \item \textbf{生活方式因素}:饮食、运动和压力管理对肠道菌群的组成具有显著影响,而这些因素也与抑郁症的风险密切相关。健康的生活方式可以促进有益菌的增长,从而改善心理健康。
\end{itemize}
肠道菌群通过影响神经递质合成、炎症反应、肠-脑轴和生活方式等多个方面,对抑郁的发生具有重要影响。然而,个体差异(如遗传、生活习惯和环境)使得不同人群在抑郁症状出现时,肠道菌群的作用可能存在显著差异。
}
\end{tcolorbox}


\begin{center}
\begin{tikzpicture}[
    font=\small,
    title/.style={font=\small\bfseries\color{white}},
    value/.style={font=\small},
    reference/.style={font=\small},
    cell/.style={anchor=west, text width=4.2cm},
    note/.style={anchor=west, text width=4.5cm, align=left}
]
    \def\cardwidth{\textwidth}
    \def\cardheight{22.85}
    \def\barheight{0.25}
    \def\barwidth{1.5}
    \def\valuebarspace{0.4}

    % 容器和标题栏背景
    \draw[rounded corners=5, fill=white, draw=gray!20]
        (0,0) rectangle (\cardwidth,-\cardheight);
    \path[fill=customTeal]
        (0,0) [rounded corners=5] -- (\cardwidth,0) --
        (\cardwidth,0.8) -- (0,0.8) -- cycle;

    % 表头
    \node[title, anchor=west] at (0.5,0.4) {\textbf{菌种名称}};
    \node[title] at (4.5,0.4) {\textbf{正常范围\%}};
    \node[title] at (7.5,0.4) {\textbf{检测丰度}};
    \node[title] at (9.5,0.4) {\textbf{检测结果}}; % 新增检测结果列名
    \node[title] at (11.5,0.4) {\textbf{超过\%的人}};
    \node[title] at (13.75,0.4) {\textbf{抑郁相关性}};
    \node[title] at (16,0.4) {\textbf{相关性强度}};

    % 初始化位置计数器
    \def\currentpos{0.25}

    % 数据行和卡片
    \foreach \item/\enitem/\value/\range/\percentile/\detection/\status/\intro/\suggestion/\index in {
        {脱硫弧菌属}/{Desulfovibrio}/0.02325/{0-0.05}/-\%/正相关/正常/{脱硫弧菌通过其代谢产物(如硫化氢)引发肠道炎症、破坏肠-脑轴信号。文献支持见[Dsv1,Dsv2,Dsv3]。}/{}/{\color{lightgray}\faStar\faStar\faStar}/\currentpos,
        {别样杆菌属}/{Alistipes}/0.00099/{0.080658-18.1198662}/15\%/正相关/缺乏/{重要的肠道共生菌,参与碳水化合物代谢,维持肠道稳态,可能影响睡眠。}/{建议增加膳食纤维摄入,如全谷物、蔬菜水果等}/{\color{lightgray}\faStar\faStar}/\currentpos,
        {拟杆菌属}/{Bacteroides}/0.08974/{1.0577624-47.3225368}/2\%/正相关/缺乏/{梭菌目是一类革兰氏阳性厌氧菌,通过肠-脑轴影响神经递质和炎症因子分泌,其失衡可直接导致睡眠质量下降和失眠症状。}/{梭菌目超标,建议多吃富含膳食纤维的绿色蔬菜和全麦食品,搭配酸奶、泡菜等发酵食品,同时减少糖和油腻食物的摄入。}/{\color{lightgray}\faStar\faStar}/\currentpos,
        {埃希氏菌属}/{Escherichia}/0.14296/{0-3.83}/6\%/正相关/正常/{产生短链脂肪酸,参与胆固醇代谢,具有抗炎作用,可能影响睡眠。}/{}/{\color{lightgray}\faStar \faStar}/\currentpos,
        {解黄酮菌属}/{Flavonifractor}/2.44087/{0-0.7285098}/57\%/正相关/超标/{经黏液真杆菌属是肠道关键菌种,通过神经递质和炎症调节,可直接影响褪黑激素分泌,导致失眠和睡眠质量下降。}/{推荐食用如洋葱、大蒜、香蕉等益生元食物,搭配酸奶、泡菜等发酵食品,并选择如苹果、芦笋、莲藕等纤维丰富的蔬果。}/{\color{lightgray}\faStar \faStar}/\currentpos,
        {震颤杆菌属}/{Oscillibacter}/0.17179/{0-3.1958}/78\%/正相关/正常/{具有益生功能,可产生多种酶类和抗菌物质,增强免疫力,可能影响睡眠。}/{建议适当补充含芽孢杆菌的益生菌制剂}/{\color{lightgray}\faStar \faStar}/\currentpos,
        {副拟杆菌属}/{Parabacteroides}/0.00675/{0.1095448-8.7303488}/9\%/正相关/超标/{厌氧菌,与肠道健康密切相关,参与多糖代谢,可能影响睡眠。}/{摄入富含益生元的绿色蔬菜和水果,如菊苣、大蒜、洋葱,同时选择发酵食品如酸奶和泡菜,并适度减少精制碳水化合物的摄入。}/{\color{lightgray}\faStar\faStar}/\currentpos,
        {放线菌属}/{Actinomyces}/0.01470/{0-0.053911}/52\%/正相关/正常/{重要的益生菌,产生丁酸,具有抗炎作用,维持肠道健康,可能与睡眠质量相关。}/{建议适当调整饮食结构,增加膳食纤维摄入}/{\color{lightgray}\faStar}/\currentpos,
        {厌氧棒状菌属}/{Anaerostipes}/0.39135/{0.093342-9.5549}/78\%/正相关/正常/{代谢碳水化合物,产生短链脂肪酸,与多种肠道疾病相关,可能影响睡眠。}/{}/{\color{lightgray}\faStar}/\currentpos,
        {经黏液真杆菌属}/{Blautia}/10.17279/{0.0846204-6.9055608}/75\%/正相关/超标/{龙包茨氏菌属是一类重要的革兰氏阴性厌氧菌,其失衡可能与炎症反应、免疫调节和代谢紊乱密切相关。}/{}/{\color{lightgray}\faStar}/\currentpos,
        {梭菌属}/{Clostridium}/1.90370/{0-4.4645924}/67\%/正相关/正常/{产生丁酸,抗炎,维持肠道健康,可能与睡眠质量相关。}/{}/{\color{lightgray}\faStar}/\currentpos
    }
    {
        % 计算当前行的基础位置
        \pgfmathsetmacro{\basepos}{-2.8*\currentpos}

        % 菌种名称
        \node[cell, align=left] at (0.5,\basepos) {
            \small\textbf{\item}\\[-0.2em]
            {\color{lightgray}\small\enitem}
        };

        % 正常范围
        \node[reference] at (4.5,\basepos) {\footnotesize\range};

        % 进度条相关
        \pgfmathsetmacro{\barypos}{\basepos-\valuebarspace+0.1}
        \def\barstart{6.75}

        % 进度条背景
        \fill[gray!10, rounded corners=2] (\barstart,\barypos)
            rectangle (\barstart+\barwidth,\barypos+\barheight);

        % 检测丰度值
        \node[value] at (7.5,{\basepos-\valuebarspace+0.6}) {\footnotesize\value};

        % 解析范围并计算进度条长度
        \def\parserange#1-#2\endparse{\def\minval{#1}\def\maxval{#2}}
        \expandafter\parserange\range\endparse

        % 计算进度条长度和颜色
        \pgfmathsetmacro{\progress}{min(\value/\maxval, 1.0)}
        \pgfmathparse{\value > \maxval ? "customred" : (\value < \minval ? "customred" : "green!50")}
        \let\barcolor=\pgfmathresult

        % 进度条显示
        \ifnum\pdfstrcmp{\status}{超标}=0
            \fill[customred, rounded corners=2] (\barstart,\barypos)
                rectangle (\barstart+\barwidth,\barypos+\barheight);
        \else
            \fill[\barcolor, rounded corners=2] (\barstart,\barypos)
                rectangle (\barstart+\barwidth*\progress,\barypos+\barheight);
        \fi

        % 检测结果
        \node[value] at (9.5,\basepos) {\footnotesize\status}; % 检测结果列

        % 超过%的人
        \node[value] at (11.5,\basepos) {\footnotesize\percentile};

        % 失眠相关性
        \node[value] at (13.75,\basepos) {\footnotesize\detection};

        % 相关性强度
        \node[value] at (16,\basepos) {\footnotesize\index};

        % 添加卡片
        \pgfmathsetmacro{\cardypos}{\basepos-0.5}
        \begin{scope}[shift={(0,\cardypos)}]
            % 卡片背景
            \pgfmathsetmacro{\cardheight}{
                \ifnum\pdfstrcmp{\status}{超标}=0
                    1.0  % 两行内容时的高度
                \else
                    0.6  % 一行内容时的高度
                \fi
            }

            \fill[rounded corners=5pt, customTeal!5, draw=gray!5]
                (0.3,-\cardheight) rectangle (17.3,0);

            % 菌群简介图标和内容
            \node[anchor=west] at (0.5,-0.3) {
                \textbf{\color{gray!90}\footnotesize \textcolor{customTeal}{\faInfoCircle}}
            };
            \node[anchor=west, text width=16cm] at (0.9,-0.3) {
                {\small\color{gray}\footnotesize \intro}
            };

            % 异常解读标题和内容
            \ifnum\pdfstrcmp{\status}{超标}=0
                \node[anchor=west] at (0.55,-0.8) {
                    \textbf{\color{customRed}\footnotesize \textcolor{customRed}{\faLightbulb}}
                };
                \node[anchor=west, text width=16cm] at (0.9,-0.8) {
                    {\small\color{gray}\footnotesize \suggestion}
                };
            \fi
        \end{scope}

        % 分割线
        \pgfmathsetmacro{\linepos}{
            \ifnum\pdfstrcmp{\status}{超标}=0
                \basepos-1.7  % 超标时的分割线位置
            \else
                \basepos-1.3  % 正常时的分割线位置
            \fi
        }
        \draw[gray!20] (0.2,\linepos) -- (\cardwidth-0.2,\linepos);

        % 根据当前行的状态计算下一行的位置增量
        \ifnum\pdfstrcmp{\status}{超标}=0
            \pgfmathsetmacro{\increment}{0.85}  % 超标行(两行内容)需要更大的增量
        \else
            \pgfmathsetmacro{\increment}{0.7}  % 正常行(一行内容)使用较小的增量
        \fi

        % 更新位置计数器
        \pgfmathsetmacro{\nextpos}{\currentpos+\increment}
        \xdef\currentpos{\nextpos}
    }

    % 最后一行的处理,消除多余的空白
    \pgfmathsetmacro{\lastincrement}{2}  % 最后一行的增量
    \pgfmathsetmacro{\nextpos}{\currentpos+\lastincrement}
    \xdef\currentpos{\nextpos}

\end{tikzpicture}
\end{center}

\newpage

\begin{center}
\begin{tikzpicture}[
    font=\small,
    title/.style={font=\small\bfseries\color{white}},
    value/.style={font=\small},
    reference/.style={font=\small},
    cell/.style={anchor=west, text width=4.2cm},
    note/.style={anchor=west, text width=4.5cm, align=left}
]
    \def\cardwidth{\textwidth}
    \def\cardheight{20.5}
    \def\barheight{0.25}
    \def\barwidth{1.5}
    \def\valuebarspace{0.4}

    % 容器和标题栏背景
    \draw[rounded corners=5, fill=white, draw=gray!20]
        (0,0) rectangle (\cardwidth,-\cardheight);
    \path[fill=customTeal]
        (0,0) [rounded corners=5] -- (\cardwidth,0) --
        (\cardwidth,0.8) -- (0,0.8) -- cycle;

    % 表头
    \node[title, anchor=west] at (0.5,0.4) {\textbf{菌种名称}};
    \node[title] at (4.5,0.4) {\textbf{正常范围\%}};
    \node[title] at (7.5,0.4) {\textbf{检测丰度}};
    \node[title] at (9.5,0.4) {\textbf{检测结果}}; % 新增检测结果列名
    \node[title] at (11.5,0.4) {\textbf{超过\%的人}};
    \node[title] at (13.75,0.4) {\textbf{抑郁相关性}};
    \node[title] at (16,0.4) {\textbf{相关性强度}};

    % 初始化位置计数器
    \def\currentpos{0.25}

    % 数据行和卡片
    \foreach \item/\enitem/\value/\range/\percentile/\detection/\status/\intro/\suggestion/\index in {
        {爱格氏菌属}/{Eggerthella}/0.00343/{0.0-0.04271}/99\%/正相关/正常/{抑制肠道炎症,产生短链脂肪酸(SCFAs)消耗有关。}/{}/{\color{lightgray}\faStar}/\currentpos,
        {嗜血杆菌属}/{Haemophilus}/0.07296/{0.0-0.09788}/-\%/正相关/正常/{与肠道健康、认知相关。}/{}/{\color{lightgray}\faStar}/\currentpos,
        {霍尔德曼氏菌属}/{Holdemania}/0.10581/{0.0-0.28}/98\%/正相关/正常/{与肠道健康相关,可能影响代谢。}/{}/{\color{lightgray}\faStar}/\currentpos,
        {克雷伯氏菌属}/{Klebsiella}/0.096521/{0.0-0.05}/-\%/正相关/超标/{与肠道健康相关,可能影响免疫。}/{}/{\color{lightgray}\faStar}/\currentpos,
        {副萨特氏菌属}/{Paraprevotella}/ND/{0.0-0.76066}/76\%/正相关/缺乏/{与肠道健康密切相关。}/{建议增加膳食纤维摄入}/{\color{lightgray}\faStar}/\currentpos,
        {副拟杆菌属}/{Parasutterella}/0.33869/{0.0-0.8235}/99\%/正相关/缺乏/{与肠道健康密切相关。}/{}/{\color{lightgray}\faStar}/\currentpos,
        {普雷沃氏菌属}/{Prevotella}/0.04290/{0.0-67.800886}/50\%/正相关/正常/{与肠道健康相关,可能影响心理健康。}/{}/{\color{lightgray}\faStar}/\currentpos,
        {韦荣菌属}/{Veillonella}/0.383267/{0.0-0.008568}/7\%/正相关/正常/{与肠道健康相关,可能影响睡眠。}/{}/{\color{lightgray}\faStar}/\currentpos,
        {单形拟杆菌}/{B. uniformis}/ND/{0.0-17.741}/4\%/负相关/缺乏/{与肠道健康密切相关。}/{}/{\color{lightgray}\faStar}/\currentpos,
        {普氏栖粪杆菌}/{F. prausnitzii}/26.23325/{0.18043-14.003}/99\%/负相关/超标/{高脂肪饮食(如动物脂肪)}/{建议增加膳食纤维摄入}/{\color{lightgray}\faStar}/\currentpos
%        {食葡糖罗斯拜瑞氏菌}/{Roseburia inulinivorans}/5.12041/{0.0-2.7653}/16\%/负相关/正常/{促进肠道健康,可能影响代谢。}/{}/{\color{lightgray}\faStar}/\currentpos
%        {双歧杆菌属}/{Bifidobacterium}/0.03296/{1.7545796-35.500554}/24\%/正相关/正常/{与肠道健康密切相关。}/{}/{\color{lightgray}\faStar}/\currentpos
    }
    {
        % 计算当前行的基础位置
        \pgfmathsetmacro{\basepos}{-2.8*\currentpos}

        % 菌种名称
        \node[cell, align=left] at (0.5,\basepos) {
            \small\textbf{\item}\\[-0.2em]
            {\color{lightgray}\small\enitem}
        };

        % 正常范围
        \node[reference] at (4.5,\basepos) {\footnotesize\range};

        % 进度条相关
        \pgfmathsetmacro{\barypos}{\basepos-\valuebarspace+0.1}
        \def\barstart{6.75}

        % 进度条背景
        \fill[gray!10, rounded corners=2] (\barstart,\barypos)
            rectangle (\barstart+\barwidth,\barypos+\barheight);

        % 检测丰度值
        \node[value] at (7.5,{\basepos-\valuebarspace+0.6}) {\footnotesize\value};

        % 解析范围并计算进度条长度
        \def\parserange#1-#2\endparse{\def\minval{#1}\def\maxval{#2}}
        \expandafter\parserange\range\endparse

        % 计算进度条长度和颜色
        \pgfmathsetmacro{\progress}{min(\value/\maxval, 1.0)}
        \pgfmathparse{\value > \maxval ? "customred" : (\value < \minval ? "customred" : "green!50")}
        \let\barcolor=\pgfmathresult

        % 进度条显示
        \ifnum\pdfstrcmp{\status}{超标}=0
            \fill[customred, rounded corners=2] (\barstart,\barypos)
                rectangle (\barstart+\barwidth,\barypos+\barheight);
        \else
            \fill[\barcolor, rounded corners=2] (\barstart,\barypos)
                rectangle (\barstart+\barwidth*\progress,\barypos+\barheight);
        \fi

        % 检测结果
        \node[value] at (9.5,\basepos) {\footnotesize\status}; % 检测结果列

        % 超过%的人
        \node[value] at (11.5,\basepos) {\footnotesize\percentile};

        % 失眠相关性
        \node[value] at (13.75,\basepos) {\footnotesize\detection};

        % 相关性强度
        \node[value] at (16,\basepos) {\footnotesize\index};

        % 添加卡片
        \pgfmathsetmacro{\cardypos}{\basepos-0.5}
        \begin{scope}[shift={(0,\cardypos)}]
            % 卡片背景
            \pgfmathsetmacro{\cardheight}{
                \ifnum\pdfstrcmp{\status}{超标}=0
                    1.0  % 两行内容时的高度
                \else
                    0.6  % 一行内容时的高度
                \fi
            }

            \fill[rounded corners=5pt, customTeal!5, draw=gray!5]
                (0.3,-\cardheight) rectangle (17.3,0);

            % 菌群简介图标和内容
            \node[anchor=west] at (0.5,-0.3) {
                \textbf{\color{gray!90}\footnotesize \textcolor{customTeal}{\faInfoCircle}}
            };
            \node[anchor=west, text width=16cm] at (0.9,-0.3) {
                {\small\color{gray}\footnotesize \intro}
            };

            % 异常解读标题和内容
            \ifnum\pdfstrcmp{\status}{超标}=0
                \node[anchor=west] at (0.55,-0.8) {
                    \textbf{\color{customRed}\footnotesize \textcolor{customRed}{\faLightbulb}}
                };
                \node[anchor=west, text width=16cm] at (0.9,-0.8) {
                    {\small\color{gray}\footnotesize \suggestion}
                };
            \fi
        \end{scope}

        % 分割线
        \pgfmathsetmacro{\linepos}{
            \ifnum\pdfstrcmp{\status}{超标}=0
                \basepos-1.7  % 超标时的分割线位置
            \else
                \basepos-1.3  % 正常时的分割线位置
            \fi
        }
        \draw[gray!20] (0.2,\linepos) -- (\cardwidth-0.2,\linepos);

        % 根据当前行的状态计算下一行的位置增量
        \ifnum\pdfstrcmp{\status}{超标}=0
            \pgfmathsetmacro{\increment}{0.85}  % 超标行(两行内容)需要更大的增量
        \else
            \pgfmathsetmacro{\increment}{0.7}  % 正常行(一行内容)使用较小的增量
        \fi

        % 更新位置计数器
        \pgfmathsetmacro{\nextpos}{\currentpos+\increment}
        \xdef\currentpos{\nextpos}
    }

    % 最后一行的处理,消除多余的空白
    \pgfmathsetmacro{\lastincrement}{2}  % 最后一行的增量
    \pgfmathsetmacro{\nextpos}{\currentpos+\lastincrement}
    \xdef\currentpos{\nextpos}

\end{tikzpicture}
\end{center}

\newpage

\begin{center}
\begin{tikzpicture}[
    font=\small,
    title/.style={font=\small\bfseries\color{white}},
    value/.style={font=\small},
    reference/.style={font=\small},
    cell/.style={anchor=west, text width=4.2cm},
    note/.style={anchor=west, text width=4.5cm, align=left}
]
    \def\cardwidth{\textwidth}
    \def\cardheight{14.65}
    \def\barheight{0.25}
    \def\barwidth{1.5}
    \def\valuebarspace{0.4}

    % 容器和标题栏背景
    \draw[rounded corners=5, fill=white, draw=gray!20]
        (0,0) rectangle (\cardwidth,-\cardheight);
    \path[fill=customTeal]
        (0,0) [rounded corners=5] -- (\cardwidth,0) --
        (\cardwidth,0.8) -- (0,0.8) -- cycle;

    % 表头
    \node[title, anchor=west] at (0.5,0.4) {\textbf{菌种名称}};
    \node[title] at (4.5,0.4) {\textbf{正常范围\%}};
    \node[title] at (7.5,0.4) {\textbf{检测丰度}};
    \node[title] at (9.5,0.4) {\textbf{检测结果}}; % 新增检测结果列名
    \node[title] at (11.5,0.4) {\textbf{超过\%的人}};
    \node[title] at (13.75,0.4) {\textbf{抑郁相关性}};
    \node[title] at (16,0.4) {\textbf{相关性强度}};

    % 初始化位置计数器
    \def\currentpos{0.25}

    % 数据行和卡片
    \foreach \item/\enitem/\value/\range/\percentile/\detection/\status/\intro/\suggestion/\index in {
        {食葡糖罗斯拜瑞氏菌}/{R. inulinivorans}/5.12041/{0.0-2.7653}/16\%/负相关/超标/{促进肠道健康,可能影响代谢。}/{}/{\color{lightgray}\faStar\faStar\faStar}/\currentpos
        {双歧杆菌属}/{Bifidobacterium}/0.03296/{1.7545796-35.500554}/24\%/负相关/缺乏/{与肠道健康密切相关。}/{}/{\color{lightgray}\faStar\faStar}/\currentpos,
        {粪杆菌属}/{Faecalibacterium}/28.86665/{1.9350868-17.7942438}/99\%/负相关/超标/{具有双相情感障碍和重度抑郁症的患者肠道中Faecalibacterium水平较低。}/{}/{\color{lightgray}\faStar\faStar}/\currentpos,
        {乳杆菌属}/{Lactobacillus}/0.00660/{0.0-0.4302374}/8\%/负相关/正常/{通过上调海马体中的GABA来缓解抑郁症。}/{}/{\color{lightgray}\faStar\faStar}/\currentpos,
        {罗氏菌属}/{Roseburia}/6.33667/{0.5829956-16.3580638}/10\%/负相关/正常/{能帮助产生短链脂肪酸和脑部-肠道信号。}/{}/{\color{lightgray}\faStar\faStar}/\currentpos,
        {拟杆菌属}/{Bacteroides}/0.08974/{1.0577624-47.3225368}/2\%/负相关/缺乏/{产生益生元和短链脂肪酸,减轻抑郁和焦虑症状。}/{建议增加膳食纤维摄入,如全谷物、蔬菜水果等}/{\color{lightgray}\faStar}/\currentpos,
        {优杆菌属}/{Eubacterium}/5.75186/{0.1145944-9.488306}/95\%/负相关/正常/{与降低血清清蛋白有关。}/{}/{\color{lightgray}\faStar}/\currentpos,
        {普雷沃氏菌属}/{Prevotella}/0.04290/{0.0-67.800896}/50\%/负相关/正常/{Prevotella与多种神经递质的产生有关,如γ-氨基丁酸(GABA)和色氨酸等。}/{}/{\color{lightgray}\faStar}/\currentpos
    }
    {
        % 计算当前行的基础位置
        \pgfmathsetmacro{\basepos}{-2.8*\currentpos}

        % 菌种名称
        \node[cell, align=left] at (0.5,\basepos) {
            \small\textbf{\item}\\[-0.2em]
            {\color{lightgray}\small\enitem}
        };

        % 正常范围
        \node[reference] at (4.5,\basepos) {\footnotesize\range};

        % 进度条相关
        \pgfmathsetmacro{\barypos}{\basepos-\valuebarspace+0.1}
        \def\barstart{6.75}

        % 进度条背景
        \fill[gray!10, rounded corners=2] (\barstart,\barypos)
            rectangle (\barstart+\barwidth,\barypos+\barheight);

        % 检测丰度值
        \node[value] at (7.5,{\basepos-\valuebarspace+0.6}) {\footnotesize\value};

        % 解析范围并计算进度条长度
        \def\parserange#1-#2\endparse{\def\minval{#1}\def\maxval{#2}}
        \expandafter\parserange\range\endparse

        % 计算进度条长度和颜色
        \pgfmathsetmacro{\progress}{min(\value/\maxval, 1.0)}
        \pgfmathparse{\value > \maxval ? "customred" : (\value < \minval ? "customred" : "green!50")}
        \let\barcolor=\pgfmathresult

        % 进度条显示
        \ifnum\pdfstrcmp{\status}{超标}=0
            \fill[customred, rounded corners=2] (\barstart,\barypos)
                rectangle (\barstart+\barwidth,\barypos+\barheight);
        \else
            \fill[\barcolor, rounded corners=2] (\barstart,\barypos)
                rectangle (\barstart+\barwidth*\progress,\barypos+\barheight);
        \fi

        % 检测结果
        \node[value] at (9.5,\basepos) {\footnotesize\status}; % 检测结果列

        % 超过%的人
        \node[value] at (11.5,\basepos) {\footnotesize\percentile};

        % 失眠相关性
        \node[value] at (13.75,\basepos) {\footnotesize\detection};

        % 相关性强度
        \node[value] at (16,\basepos) {\footnotesize\index};

        % 添加卡片
        \pgfmathsetmacro{\cardypos}{\basepos-0.5}
        \begin{scope}[shift={(0,\cardypos)}]
            % 卡片背景
            \pgfmathsetmacro{\cardheight}{
                \ifnum\pdfstrcmp{\status}{超标}=0
                    1.0  % 两行内容时的高度
                \else
                    0.6  % 一行内容时的高度
                \fi
            }

            \fill[rounded corners=5pt, customTeal!5, draw=gray!5]
                (0.3,-\cardheight) rectangle (17.3,0);

            % 菌群简介图标和内容
            \node[anchor=west] at (0.5,-0.3) {
                \textbf{\color{gray!90}\footnotesize \textcolor{customTeal}{\faInfoCircle}}
            };
            \node[anchor=west, text width=16cm] at (0.9,-0.3) {
                {\small\color{gray}\footnotesize \intro}
            };

            % 异常解读标题和内容
            \ifnum\pdfstrcmp{\status}{超标}=0
                \node[anchor=west] at (0.55,-0.8) {
                    \textbf{\color{customRed}\footnotesize \textcolor{customRed}{\faLightbulb}}
                };
                \node[anchor=west, text width=16cm] at (0.9,-0.8) {
                    {\small\color{gray}\footnotesize \suggestion}
                };
            \fi
        \end{scope}

        % 分割线
        \pgfmathsetmacro{\linepos}{
            \ifnum\pdfstrcmp{\status}{超标}=0
                \basepos-1.7  % 超标时的分割线位置
            \else
                \basepos-1.3  % 正常时的分割线位置
            \fi
        }
        \draw[gray!20] (0.2,\linepos) -- (\cardwidth-0.2,\linepos);

        % 根据当前行的状态计算下一行的位置增量
        \ifnum\pdfstrcmp{\status}{超标}=0
            \pgfmathsetmacro{\increment}{0.85}  % 超标行(两行内容)需要更大的增量
        \else
            \pgfmathsetmacro{\increment}{0.7}  % 正常行(一行内容)使用较小的增量
        \fi

        % 更新位置计数器
        \pgfmathsetmacro{\nextpos}{\currentpos+\increment}
        \xdef\currentpos{\nextpos}
    }

    % 最后一行的处理,消除多余的空白
    \pgfmathsetmacro{\lastincrement}{2}  % 最后一行的增量
    \pgfmathsetmacro{\nextpos}{\currentpos+\lastincrement}
    \xdef\currentpos{\nextpos}

\end{tikzpicture}
\end{center}

\begin{tcolorbox}[
    enhanced,
    colback=gray!3,
    colframe=gray!3,
    arc=3mm,
    boxrule=0pt,
    width=\textwidth,
    top=8pt,
    bottom=8pt
]

{\small \textcolor{green!85!orange}{\faBell}\quad 解读:
}
\end{tcolorbox}

\newpage

\Large \textbf{\textcolor{customTeal}{文献支持}}
\vspace{0.5cm}

\reference{1}{Exposure to concentrated ambient PM2.5 alters the composition of gut microbiota in a murine model.}{Wang, W. et al.}{Part Fibre Toxicol \textbf{15}, 17}{2018}

\reference{2}{The gut microbiome and depression: a systematic review.}{Dinan, T. G. et al.}{Psychosom Med \textbf{80}, 1-10}{2018}
\reference{3}{Faecalibacterium prausnitzii: a key player in gut health and disease.}{Miquel, S. et al.}{Front Microbiol \textbf{7}, 1-10}{2016}
\reference{4}{Probiotics and prebiotics in mental health: a systematic review.}{Sarris, J. et al.}{Nutritional Neuroscience \textbf{20}, 1-10}{2017}



\newpage

% 自定义命令来绘制星星
\newcommand{\starred}[1]{%
    \begin{tikzpicture}[scale=0.5]
        % 绘制三颗星
        \foreach \i in {1, 2, 3} {
            \node at (\i * 1.5, 0) {\faStar}; % 绘制星星
        }
        % 点亮第一颗星
        \fill[red] (\faStar) circle (0.5); % 用红色填充第一颗星
    \end{tikzpicture}%
}

\begin{tcolorbox}[
    enhanced,
    colback=white,
    colframe=white,
    arc=2mm,
    boxrule=0pt,
    width=\textwidth,
    left=15pt,
    right=15pt,
    top=10pt,
    bottom=10pt,
    drop shadow={
        opacity=0.2,
        color=customTeal
    },
    borderline west={5pt}{0pt}{customTeal}
]
\textcolor{customTeal}{\Large\textbf{失眠相关菌}}
\end{tcolorbox}

\begin{tcolorbox}[
    enhanced,
    colback=customTealBg,
    colframe=customTealBg,
    arc=3mm,
    boxrule=0pt,
    width=\textwidth,
    top=8pt,
    bottom=8pt
]
{\small{\color{customTeal}\faInfoCircle} 肠道菌群与失眠之间的关系正在受到越来越多的关注。研究表明,肠道微生态的健康状况与睡眠质量密切相关,特定的肠道菌群失衡可能会导致睡眠问题,从而导致失眠的发生。调节肠道菌群可能为改善睡眠提供新的途径。\\

{\color{orange}\faExclamationTriangle} \textbf{特别注意}:
\begin{itemize}
    \item 在本报告中,与失眠相关的某些肠道菌群可能会出现异常水平。然而,这些菌群的丰富度异常并不一定直接导致失眠的发生。失眠通常是由多重因素引起的,包括心理压力、生活方式、环境因素和生理健康等。
    \item 为了改善睡眠质量,建议您结合肠道菌群检测结果和自身的生活习惯、作息规律、压力水平等因素进行综合评估。如果出现持续的失眠问题,应及时咨询专业医生或睡眠专家以获取有效的支持和建议。
\end{itemize}
}
\end{tcolorbox}

\begin{tcolorbox}[
    enhanced,
    colback=lightpurple!10, % 卡片底色
    colframe=lightpurple!10,  % 边框颜色
    arc=3mm,
    boxrule=0.5pt,
    width=\textwidth,
    top=8pt,
    bottom=8pt
]
{\small{\color{lightpurple}\faQuestionCircle}\quad \textbf{肠道菌群是如何影响失眠的?}\\
{\color{orange!50}\faComments}\quad 肠道菌群与失眠之间的关系涉及多个机制,以下是一些主要方面:
\begin{itemize}
    \item \textbf{神经递质的合成}:肠道细菌能合成多种神经递质,如血清素和γ-氨基丁酸(GABA),这些物质在调节情绪与睡眠中起到重要作用。肠道菌群的失衡可能导致这些神经递质的合成不足,进而影响睡眠质量。
    \item \textbf{炎症反应}:肠道菌群的失调可能导致局部和全身性炎症的增加,炎症因子能够影响中枢神经系统的功能,进而影响睡眠模式。慢性炎症与失眠密切相关。
    \item \textbf{微生物群多样性}:多样性较高的肠道菌群通常与更好的心理与生理健康相关,低多样性的菌群则可能与失眠症状的发生有关,显示肠道微生物的多样性可能是睡眠健康的一个指标。
    \item \textbf{肠-脑轴}:肠道与大脑之间的双向联系(肠-脑轴)使得肠道菌群通过代谢产物和信号分子影响脑功能和情绪,改变睡眠模式。
    \item \textbf{生活方式因素}:饮食、运动及压力管理等生活方式因素显著影响肠道菌群的组成,从而间接影响睡眠质量。健康的生活方式有助于促进睡眠。
\end{itemize}
肠道菌群通过影响神经递质合成、炎症反应、肠-脑轴及生活方式等多个方面,对失眠的发生起到重要作用。然而,不同个体的肠道菌群组成受遗传、环境和生活习惯等多种因素影响,因而每个人在失眠问题上可能表现出不同的敏感性。
}
\end{tcolorbox}
\vspace{-0.7cm}
\begin{center}
\begin{tikzpicture}[
    font=\small,
    title/.style={font=\small\bfseries\color{white}},
    value/.style={font=\small},
    reference/.style={font=\small},
    cell/.style={anchor=west, text width=4.2cm},
    note/.style={anchor=west, text width=4.5cm, align=left}
]
    \def\cardwidth{\textwidth}
    \def\cardheight{7}
    \def\barheight{0.25}
    \def\barwidth{1.5}
    \def\valuebarspace{0.4}

    % 容器和标题栏背景
    \draw[rounded corners=5, fill=white, draw=gray!20]
        (0,0) rectangle (\cardwidth,-\cardheight);
    \path[fill=customTeal]
        (0,0) [rounded corners=5] -- (\cardwidth,0) --
        (\cardwidth,0.8) -- (0,0.8) -- cycle;

    % 表头
    \node[title, anchor=west] at (0.5,0.4) {\textbf{菌种名称}};
    \node[title] at (4.5,0.4) {\textbf{正常范围\%}};
    \node[title] at (7.5,0.4) {\textbf{检测丰度}};
    \node[title] at (9.5,0.4) {\textbf{检测结果}}; % 新增检测结果列名
    \node[title] at (11.5,0.4) {\textbf{超过\%的人}};
    \node[title] at (13.75,0.4) {\textbf{失眠相关性}};
    \node[title] at (16,0.4) {\textbf{相关性强度}};

    % 初始化位置计数器
    \def\currentpos{0.25}

    % 数据行和卡片
    \foreach \item/\enitem/\value/\range/\percentile/\detection/\status/\intro/\suggestion/\index in {
        {拟杆菌属}/{Bacteroides}/0.08974/{1.0578-47.3225}/2\%/正相关/偏低/{重要的肠道共生菌,参与碳水化合物代谢,维持肠道稳态,可能影响睡眠。}/{建议增加膳食纤维摄入,如全谷物、蔬菜水果等}/{\color{lightgray}\faStar \faStar \faStar}/\currentpos,
        {梭菌目}/{Clostridiales}/87.70725/{6.2649-69.865}/90\%/正相关/超标/{梭菌目是一类革兰氏阳性厌氧菌,通过肠-脑轴影响神经递质和炎症因子分泌,其失衡可直接导致睡眠质量下降和失眠症状。}/{梭菌目超标,建议多吃富含膳食纤维的绿色蔬菜和全麦食品,搭配酸奶、泡菜等发酵食品,同时减少糖和油腻食物的摄入。}/{\color{lightgray}\faStar \faStar \faStar}/\currentpos,
        {经黏液真杆菌属}/{Blautia}/10.17279/{0.0846204-6.9055608}/75\%/正相关/超标/{经黏液真杆菌属是肠道关键菌种,通过神经递质和炎症调节,可直接影响褪黑激素分泌,导致失眠和睡眠质量下降。}/{推荐食用如洋葱、大蒜、香蕉等益生元食物,搭配酸奶、泡菜等发酵食品,并选择如苹果、芦笋、莲藕等纤维丰富的蔬果。}/{\color{lightgray}\faStar \faStar}/\currentpos
    }
    {
        % 计算当前行的基础位置
        \pgfmathsetmacro{\basepos}{-2.8*\currentpos}

        % 菌种名称
        \node[cell, align=left] at (0.5,\basepos) {
            \small\textbf{\item}\\[-0.2em]
            {\color{lightgray}\small\enitem}
        };

        % 正常范围
        \node[reference] at (4.5,\basepos) {\footnotesize\range};

        % 进度条相关
        \pgfmathsetmacro{\barypos}{\basepos-\valuebarspace+0.1}
        \def\barstart{6.75}

        % 进度条背景
        \fill[gray!10, rounded corners=2] (\barstart,\barypos)
            rectangle (\barstart+\barwidth,\barypos+\barheight);

        % 检测丰度值
        \node[value] at (7.5,{\basepos-\valuebarspace+0.6}) {\footnotesize\value};

        % 解析范围并计算进度条长度
        \def\parserange#1-#2\endparse{\def\minval{#1}\def\maxval{#2}}
        \expandafter\parserange\range\endparse

        % 计算进度条长度和颜色
        \pgfmathsetmacro{\progress}{min(\value/\maxval, 1.0)}
        \pgfmathparse{\value > \maxval ? "customred" : (\value < \minval ? "customred" : "green!50")}
        \let\barcolor=\pgfmathresult

        % 进度条显示
        \ifnum\pdfstrcmp{\status}{超标}=0
            \fill[customred, rounded corners=2] (\barstart,\barypos)
                rectangle (\barstart+\barwidth,\barypos+\barheight);
        \else
            \fill[\barcolor, rounded corners=2] (\barstart,\barypos)
                rectangle (\barstart+\barwidth*\progress,\barypos+\barheight);
        \fi

        % 检测结果
        \ifnum\pdfstrcmp{\status}{正常}=0

        \node[value, text=customGreen] at (9.5,\basepos) {\footnotesize\textbf{\status}}; % 检测结果为正常
        \else

        \node[value, text=customRed] at (9.5,\basepos) {\footnotesize\textbf{\status}}; % 检测结果为其他状态
        \fi

        % 超过%的人
        \node[value] at (11.5,\basepos) {\footnotesize\percentile};

        % 失眠相关性
        \node[value] at (13.75,\basepos) {\footnotesize\textbf{\detection}};

        % 相关性强度
        \node[value] at (16,\basepos) {\footnotesize\index};

        % 添加卡片
        \pgfmathsetmacro{\cardypos}{\basepos-0.5}
        \begin{scope}[shift={(0,\cardypos)}]
            % 卡片背景
            \pgfmathsetmacro{\cardheight}{
                \ifnum\pdfstrcmp{\status}{超标}=0
                    1.0  % 两行内容时的高度
                \else
                    0.6  % 一行内容时的高度
                \fi
            }

            \fill[rounded corners=5pt, customTeal!5, draw=gray!5]
                (0.3,-\cardheight) rectangle (17.3,0);

            % 菌群简介图标和内容
            \node[anchor=west] at (0.5,-0.3) {
                \textbf{\color{gray!90}\footnotesize \textcolor{customTeal}{\faInfoCircle}}
            };
            \node[anchor=west, text width=16cm] at (0.9,-0.3) {
                {\small\color{gray}\footnotesize \intro}
            };

            % 异常解读标题和内容
            \ifnum\pdfstrcmp{\status}{超标}=0
                \node[anchor=west] at (0.55,-0.8) {
                    \textbf{\color{customRed}\footnotesize \textcolor{customRed}{\faBell}}
                };
                \node[anchor=west, text width=16cm] at (0.9,-0.8) {
                    {\small\color{gray}\footnotesize \suggestion}
                };
            \fi
        \end{scope}

        % 分割线
        \pgfmathsetmacro{\linepos}{
            \ifnum\pdfstrcmp{\status}{超标}=0
                \basepos-1.7  % 超标时的分割线位置
            \else
                \basepos-1.3  % 正常时的分割线位置
            \fi
        }
        \draw[gray!20] (0.2,\linepos) -- (\cardwidth-0.2,\linepos);

        % 根据当前行的状态计算下一行的位置增量
        \ifnum\pdfstrcmp{\status}{超标}=0
            \pgfmathsetmacro{\increment}{0.85}  % 超标行(两行内容)需要更大的增量
        \else
            \pgfmathsetmacro{\increment}{0.7}  % 正常行(一行内容)使用较小的增量
        \fi

        % 更新位置计数器
        \pgfmathsetmacro{\nextpos}{\currentpos+\increment}
        \xdef\currentpos{\nextpos}
    }

    % 最后一行的处理,消除多余的空白
    \pgfmathsetmacro{\lastincrement}{2}  % 最后一行的增量
    \pgfmathsetmacro{\nextpos}{\currentpos+\lastincrement}
    \xdef\currentpos{\nextpos}

\end{tikzpicture}
\end{center}

\begin{center}
\begin{tikzpicture}[
    font=\small,
    title/.style={font=\small\bfseries\color{white}},
    value/.style={font=\small},
    reference/.style={font=\small},
    cell/.style={anchor=west, text width=4.2cm},
    note/.style={anchor=west, text width=4.5cm, align=left}
]
    \def\cardwidth{\textwidth}
    \def\cardheight{16.25}
    \def\barheight{0.25}
    \def\barwidth{1.5}
    \def\valuebarspace{0.4}

    % 容器和标题栏背景
    \draw[rounded corners=5, fill=white, draw=gray!20]
        (0,0) rectangle (\cardwidth,-\cardheight);
    \path[fill=customTeal]
        (0,0) [rounded corners=5] -- (\cardwidth,0) --
        (\cardwidth,0.8) -- (0,0.8) -- cycle;

    % 表头
    \node[title, anchor=west] at (0.5,0.4) {\textbf{菌种名称}};
    \node[title] at (4.5,0.4) {\textbf{正常范围\%}};
    \node[title] at (7.5,0.4) {\textbf{检测丰度}};
    \node[title] at (9.5,0.4) {\textbf{检测结果}}; % 新增检测结果列名
    \node[title] at (11.5,0.4) {\textbf{超过\%的人}};
    \node[title] at (13.75,0.4) {\textbf{失眠相关性}};
    \node[title] at (16,0.4) {\textbf{相关性强度}};

    % 初始化位置计数器
    \def\currentpos{0.25}

    % 数据行和卡片
    \foreach \item/\enitem/\value/\range/\percentile/\detection/\status/\intro/\suggestion/\index in {
        {梭菌属}/{Clostridium}/1.90370/{0-4.4645924}/67\%/正相关/正常/{具有益生功能,可产生多种酶类和抗菌物质,增强免疫力,可能影响睡眠。}/{建议适当补充含芽孢杆菌的益生菌制剂}/{\color{lightgray}\faStar \faStar}/\currentpos,
        {链球菌属}/{Streptococcus}/0.31690/{0-0.3495704}/13\%/正相关/正常/{产生丁酸,抗炎,维持肠道健康,可能与睡眠质量相关。}/{}/{\color{lightgray}\faStar \faStar}/\currentpos,
        {龙包茨氏菌属}/{Romboutsia}/0.01546/{0-0.021404}/73\%/正相关/正常/{龙包茨氏菌属是一类重要的革兰氏阴性厌氧菌,其失衡可能与炎症反应、免疫调节和代谢紊乱密切相关。}/{}/{\color{lightgray}\faStar}/\currentpos,
        {普雷沃氏菌属}/{Prevotella}/0.04290/{0-67.8009886}/50\%/正相关/正常/{代谢碳水化合物,产生短链脂肪酸,与多种肠道疾病相关,可能影响睡眠。}/{}/{\color{lightgray}\faStar}/\currentpos,
        {粪杆菌属}/{Faecalibacterium}/28.86665/{1.9350868-17.7942438}/99\%/负相关/超标/{厌氧菌,与肠道健康密切相关,参与多糖代谢,可能影响睡眠。}/{摄入富含益生元的绿色蔬菜和水果,如菊苣、大蒜、洋葱,同时选择发酵食品如酸奶和泡菜,并适度减少精制碳水化合物的摄入。}/{\color{lightgray}\faStar \faStar \faStar}/\currentpos,
        {阿克曼氏菌属}/{Akkermansia}/0.00460/{0-5.7984096}/44\%/负相关/正常/{参与肠道代谢,改善肠道健康,可能与睡眠质量相关。}/{}/{\color{lightgray}\faStar \faStar}/\currentpos,
        {长双歧杆菌}/{B. longum}/0.01861/{0-6.854}/99\%/负相关/正常/{产生短链脂肪酸,参与胆固醇代谢,具有抗炎作用,可能影响睡眠。}/{}/{\color{lightgray}\faStar \faStar}/\currentpos,
        {Lachnoclostridium}/{Lachnoclostridium}/0.17516/{0-0.2086068}/8\%/负相关/正常/{重要的益生菌,产生丁酸,具有抗炎作用,维持肠道健康,可能与睡眠质量相关。}/{建议适当调整饮食结构,增加膳食纤维摄入}/{\color{lightgray}\faStar \faStar}/\currentpos
    }
    {
        % 计算当前行的基础位置
        \pgfmathsetmacro{\basepos}{-2.8*\currentpos}

        % 菌种名称
        \node[cell, align=left] at (0.5,\basepos) {
            \small\textbf{\item}\\[-0.2em]
            {\color{lightgray}\small\enitem}
        };

        % 正常范围
        \node[reference] at (4.5,\basepos) {\footnotesize\range};

        % 进度条相关
        \pgfmathsetmacro{\barypos}{\basepos-\valuebarspace+0.1}
        \def\barstart{6.75}

        % 进度条背景
        \fill[gray!10, rounded corners=2] (\barstart,\barypos)
            rectangle (\barstart+\barwidth,\barypos+\barheight);

        % 检测丰度值
        \node[value] at (7.5,{\basepos-\valuebarspace+0.6}) {\footnotesize\value};

        % 解析范围并计算进度条长度
        \def\parserange#1-#2\endparse{\def\minval{#1}\def\maxval{#2}}
        \expandafter\parserange\range\endparse

        % 计算进度条长度和颜色
        \pgfmathsetmacro{\progress}{min(\value/\maxval, 1.0)}
        \pgfmathparse{\value > \maxval ? "customred" : (\value < \minval ? "customred" : "green!50")}
        \let\barcolor=\pgfmathresult

        % 进度条显示
        \ifnum\pdfstrcmp{\status}{超标}=0
            \fill[customred, rounded corners=2] (\barstart,\barypos)
                rectangle (\barstart+\barwidth,\barypos+\barheight);
        \else
            \fill[\barcolor, rounded corners=2] (\barstart,\barypos)
                rectangle (\barstart+\barwidth*\progress,\barypos+\barheight);
        \fi

        % 检测结果
        \node[value] at (9.5,\basepos) {\footnotesize\status}; % 检测结果列

        % 超过%的人
        \node[value] at (11.5,\basepos) {\footnotesize\percentile};

        % 失眠相关性
        \node[value] at (13.75,\basepos) {\footnotesize\detection};

        % 相关性强度
        \node[value] at (16,\basepos) {\footnotesize\index};

        % 添加卡片
        \pgfmathsetmacro{\cardypos}{\basepos-0.5}
        \begin{scope}[shift={(0,\cardypos)}]
            % 卡片背景
            \pgfmathsetmacro{\cardheight}{
                \ifnum\pdfstrcmp{\status}{超标}=0
                    1.0  % 两行内容时的高度
                \else
                    0.6  % 一行内容时的高度
                \fi
            }

            \fill[rounded corners=5pt, customTeal!5, draw=gray!5]
                (0.3,-\cardheight) rectangle (17.3,0);

            % 菌群简介图标和内容
            \node[anchor=west] at (0.5,-0.3) {
                \textbf{\color{gray!90}\footnotesize \textcolor{customTeal}{\faInfoCircle}}
            };
            \node[anchor=west, text width=16cm] at (0.9,-0.3) {
                {\small\color{gray}\footnotesize \intro}
            };

            % 异常解读标题和内容
            \ifnum\pdfstrcmp{\status}{超标}=0
                \node[anchor=west] at (0.55,-0.8) {
                    \textbf{\color{customRed}\footnotesize \textcolor{customRed}{\faBell}}
                };
                \node[anchor=west, text width=16cm] at (0.9,-0.8) {
                    {\small\color{gray}\footnotesize \suggestion}
                };
            \fi
        \end{scope}

        % 分割线
        \pgfmathsetmacro{\linepos}{
            \ifnum\pdfstrcmp{\status}{超标}=0
                \basepos-1.7  % 超标时的分割线位置
            \else
                \basepos-1.3  % 正常时的分割线位置
            \fi
        }
        \draw[gray!20] (0.2,\linepos) -- (\cardwidth-0.2,\linepos);

        % 根据当前行的状态计算下一行的位置增量
        \ifnum\pdfstrcmp{\status}{超标}=0
            \pgfmathsetmacro{\increment}{0.85}  % 超标行(两行内容)需要更大的增量
        \else
            \pgfmathsetmacro{\increment}{0.7}  % 正常行(一行内容)使用较小的增量
        \fi

        % 更新位置计数器
        \pgfmathsetmacro{\nextpos}{\currentpos+\increment}
        \xdef\currentpos{\nextpos}
    }

    % 最后一行的处理,消除多余的空白
    \pgfmathsetmacro{\lastincrement}{2}  % 最后一行的增量
    \pgfmathsetmacro{\nextpos}{\currentpos+\lastincrement}
    \xdef\currentpos{\nextpos}

\end{tikzpicture}
\end{center}

\begin{tcolorbox}[
    enhanced,
    colback=gray!5,
    colframe=gray!5,
    arc=3mm,
    boxrule=0pt,
    width=\textwidth,
    top=8pt,
    bottom=8pt
]
{\small \textbf{解读}:
}
\end{tcolorbox}

\newpage

\large \textbf{文献支持}
\vspace{0.5cm}

\reference{1}{Exposure to concentrated ambient PM2.5 alters the composition of gut microbiota in a murine model.}{Wang, W. et al.}{Part Fibre Toxicol \textbf{15}, 17}{2018}

\newpage

\begin{tcolorbox}[
    enhanced,
    colback=white,
    colframe=customTeal,
    arc=2mm,
    boxrule=1pt,
    left=20pt,
    right=20pt,
    top=12pt,
    bottom=12pt,
    width=\textwidth,
    fontupper=\sffamily,
    overlay={
    \draw[customTeal, line width=2pt]
    ([xshift=15pt]frame.south west) -- ([xshift=-15pt]frame.south east);
    }
]
{\Large\bfseries\textcolor{customTeal}{\Huge 个性化食物推荐}}
\end{tcolorbox}

\begin{tcolorbox}[
    enhanced,
    colback=customTealBg,
    colframe=gray!3,
    arc=3mm,
    boxrule=0pt,
    width=\textwidth,
    top=8pt,
    bottom=8pt
]
{\small{\color{customTeal}\faInfoCircle} 我们为您提供了200多种常见食物的个性化推荐指数,评分范围从-100到100,并附有每种食物的详细营养成分构成(基于每100g计算)。\\

\textbf{推荐原理}
\begin{itemize}
    \item 基于您的肠道菌群构成、营养状况和疾病风险进行综合评估
    \item 计算每种食物的营养构成与您当前营养状况的匹配程度
    \item 考虑特定疾病需要避免的食物
    \item 正分值表示建议增加摄入,负分值表示建议减少摄入
    \item 对于营养缺乏的成分,含该营养较高的食物会获得更高的推荐分值
\end{itemize}
\textbf{使用说明}
\begin{itemize}
    \item 本推荐适用于成年人,不适合2岁以下婴幼儿
    \item 母乳喂养期间,可作为母亲的膳食参考
    \item 建议2个月后重新检测更新建议,以适应健康状况变化
\end{itemize}
\textbf{数据来源}
\begin{itemize}
    \item 基于大规模人群营养饮食调查
    \item 参考2017中国居民膳食指南
    \item 采用机器学习和统计方法计算
    \item 营养成分数据来自USDA食品成分数据库
\end{itemize}

}
\end{tcolorbox}

\newpage

\begin{center}
    \begin{tikzpicture}[scale=1.75]
        % 主进度条
        % 左侧不推荐食物部分(红色)
        \fill[red!90] (-4,0) rectangle (-3.2,0.3); % 避免摄入 (-100 到 -60)
        \fill[red!70] (-3.2,0) rectangle (-2.4,0.3);
        \fill[orange!60] (-2.4,0) rectangle (-1.8,0.3); % 控制摄入 (-60 到 -30)
        \fill[orange!40] (-1.8,0) rectangle (-1.2,0.3);
        \fill[yellow!60] (-1.2,0) rectangle (-0.6,0.3);
        \fill[yellow!30] (-0.6,0) rectangle (0,0.3);  % 适当摄入 (-30 到 0)

        % 右侧推荐食物部分(绿色)
        \fill[yellow!30] (0,0) rectangle (0.6,0.3);
        \fill[green!30] (0.6,0) rectangle (1.2,0.3);  % 建议食用 (0 到 30)
        \fill[green!40] (1.2,0) rectangle (1.8,0.3);
        \fill[green!50] (1.8,0) rectangle (2.4,0.3);  % 建议多食用 (30 到 60)
        \fill[green!60] (2.4,0) rectangle (3.2,0.3);
        \fill[green!70] (3.2,0) rectangle (4,0.3);  % 强烈推荐 (60 到 100)

        % 标签(放在进度条上方)
        \node at (-3.2, 0.5) {\textbf{避免摄入}};
        \node at (-1.8, 0.5) {\textbf{控制摄入}};
        \node at (-0.6, 0.5) {\textbf{适当摄入}};
        \node at (0.6, 0.5) {\textbf{建议食用}};
        \node at (1.8, 0.5) {\textbf{建议多食用}};
        \node at (3.2, 0.5) {\textbf{强烈推荐}};

        % 数字范围(放在进度条下方)
        \node at (-4, -0.2) {-100};
        \node at (-2.4, -0.2) {-60};
        \node at (-1.2, -0.2) {-30};
        \node at (0, -0.2) {0};
        \node at (1.2, -0.2) {30};
        \node at (2.4, -0.2) {60};
        \node at (4, -0.2) {100};

        % 右侧分数为68的进度条(大麦)
        \fill[green!50] (0, -0.8) rectangle (4, -0.5); % 大麦
        \node at (4 - 0.2, -0.65) {100};
        \node at (4 + 0.4, -0.65) {藜麦};

        % 左侧分数为-40的进度条(奶酪)
        \fill[red!50] (-4, -0.8) rectangle (0, -0.5); % 奶酪
        \node at (-4 - 0.4, -0.65) {黄油};

        % 添加藜麦(98)
        \fill[green!70] (0, -1.2) rectangle (3.0, -0.9); % 藜麦
        \node at (3 - 0.2, -1.05) {80};
        \node at (3.0 + 0.4, -1.05) {牛蒡根};

        % 添加猪培根(-92)
        \fill[red!60] (-3.9, -1.2) rectangle (0, -0.9); % 猪培根
        \node at (-3.9 - 0.4, -1.05) {猪培根};

    \end{tikzpicture}
\end{center}

\newpage

\begin{tcolorbox}[
    enhanced,
    colback=white,
    colframe=white,
    arc=2mm,
    boxrule=0pt,
    width=\textwidth,
    left=15pt,
    right=15pt,
    top=10pt,
    bottom=10pt,
    drop shadow={
        opacity=0.2,
        color=customTeal
    },
    borderline west={5pt}{0pt}{customTeal}
]
\textcolor{customTeal}{\Large\textbf{谷物类主食}}
\end{tcolorbox}

\begin{tcolorbox}[
    enhanced,
    colback=customTealBg,
    colframe=gray!3,
    arc=3mm,
    boxrule=0pt,
    width=\textwidth,
    top=8pt,
    bottom=8pt
]
{\small{\color{customTeal}\faInfoCircle}
谷物类主食
}
\end{tcolorbox}


\begin{center}\vspace{-10pt}
\begin{tikzpicture}[
    font=\small,
    title/.style={font=\small\bfseries\color{white}},
    value/.style={font=\small},
    reference/.style={font=\small},
    cell/.style={anchor=west, text width=8cm}  % 增加宽度以适应两行文本
]
    \def\cardwidth{\textwidth}
    \def\cardheight{12.7}
    \def\barheight{0.25}
    \def\rowspace{0.9}

    % 创建容器
    \draw[rounded corners=5, fill=white, draw=gray!20]
        (0,0) rectangle (\cardwidth,-\cardheight);

    % 标题栏背景
    \path[fill=customTeal]
        (0,0) [rounded corners=5] -- (\cardwidth,0) --
        (\cardwidth,0.8) -- (0,0.8) -- cycle;

    % 调整表头位置和宽度
    \node[title, anchor=west] at (0.5,0.4) {\textbf{食物名称}};
    \node[title] at (3.25, 0.4) {\textbf{推荐指数}};
    \node[title] at (5, 0.4) {\textbf{能量 (KJ)}};
    \node[title] at (6.5, 0.4) {\textbf{蛋白 (g)}};
    \node[title] at (8, 0.4) {\textbf{脂肪 (g)}};
    \node[title] at (10, 0.4) {\textbf{碳水化合物 (g)}};
    \node[title] at (12, 0.4) {\textbf{淀粉 (g)}};
    \node[title] at (13.75, 0.4) {\textbf{膳食纤维 (g)}};
    \node[title] at (16.0, 0.4) {\textbf{胆固醇 (mg)}};

    % 数据行
    \foreach \item/\recommend/\energy/\protein/\fat/\carbs/\starch/\fiber/\cholesterol/\index in {
        {玉米粒}/{72}/{298}/{1}/{0}/{14}/{14}/{0}/{0}/0.5,
        {燕麦}/{71}/{297}/{2}/{1}/{12}/{0}/{0}/{1}/1.5,
        {大麦}/{68}/{1481}/{12}/{2}/{73}/{0}/{0}/{17}/2.5,
        {小麦}/{58}/{1423}/{10}/{1}/{75}/{0}/{0}/{12}/3.5,
        {玉米饼}/{44}/{912}/{5}/{2}/{44}/{0}/{0}/{6}/4.5,
        {小米}/{41}/{1582}/{11}/{4}/{72}/{0}/{0}/{8}/5.5,
        {意大利面}/{30}/{386}/{1}/{3}/{13}/{0}/{0}/{2}/6.5,
        {芽麦粉}/{26}/{1402}/{12}/{3}/{70}/{0}/{0}/{10}/7.5,
        {黑麦面包}/{21}/{1188}/{9}/{3}/{53}/{0}/{0}/{6}/8.5,
        {鸡蛋面包}/{2}/{1201}/{9}/{6}/{47}/{0}/{0}/{2}/9.5,
        {葡萄干浆\\即食谷物}/{0}/{1354}/{7}/{1}/{78}/{0}/{0}/{13}/10.5,
        {米饭}/{-1}/{1527}/{7}/{0}/{79}/{0}/{0}/{1}/11.5,
        {面条}/{-7}/{1609}/{14}/{4}/{71}/{0}/{0}/{3}/12.5,
        {小麦面包}/{-12}/{1116}/{10}/{3}/{48}/{36}/{0}/{4}/13.5
    }
    {
        % 计算每行的y位置
        \pgfmathsetmacro{\ypos}{-\rowspace*\index}

        % 项目名
        \node[cell, align=left] at (0.5,\ypos) {\small\textbf{\item}};

        % 计算进度条垂直居中的位置
        \pgfmathsetmacro{\barypos}{\ypos-0.5*\barheight}

        % 进度条背景
        \fill[gray!10, rounded corners=2] (2.5,\barypos) rectangle (4.00,\barypos+\barheight); % 修改背景宽度

        % 计算推荐指数
        \pgfmathsetmacro{\recommendvalue}{\recommend/100}

        % 进度条和数值显示
        \ifnum\recommend<0
            \fill[red!70, rounded corners=2] (2.5,\barypos) rectangle (2.5-\recommendvalue*2,\barypos+\barheight); % 修改这里
            \node[value, text=red!70] at (3.25,\barypos+\barheight+0.15) {\textbf{\footnotesize\recommend}}; % 调整位置
        \else
            \fill[green!90!black, rounded corners=2] (2.5,\barypos) rectangle (2.5+\recommendvalue*2,\barypos+\barheight); % 修改这里
            \node[value, text=green!90!black] at (3.25,\barypos+\barheight+0.15) {\textbf{\footnotesize\recommend}}; % 调整位置
        \fi

        % 其他数值显示
        \node[value] at (5.0,\ypos) {\footnotesize\energy};
        \node[value] at (6.5,\ypos) {\footnotesize\protein};
        \node[value] at (8.0,\ypos) {\footnotesize\fat};
        \node[value] at (10.0,\ypos) {\footnotesize\carbs};
        \node[value] at (12.0,\ypos) {\footnotesize\starch};
        \node[value] at (13.75,\ypos) {\footnotesize\fiber};
        \node[value] at (16.0,\ypos) {\footnotesize\cholesterol};
    }

    % 分隔线
    \foreach \i in {1,...,13} {
        \pgfmathsetmacro{\y}{-\rowspace*\i}
        \draw[gray!20] (0.2,\y) -- (\cardwidth-0.2,\y);
    }

\end{tikzpicture}
\end{center}

\newpage

\begin{tcolorbox}[
    enhanced,
    colback=white,
    colframe=white,
    arc=2mm,
    boxrule=0pt,
    width=\textwidth,
    left=15pt,
    right=15pt,
    top=10pt,
    bottom=10pt,
    drop shadow={
        opacity=0.2,
        color=customTeal
    },
    borderline west={5pt}{0pt}{customTeal}
]
\textcolor{customTeal}{\Large\textbf{乳制品}}
\end{tcolorbox}

\begin{tcolorbox}[
    enhanced,
    colback=customTealBg,
    colframe=gray!3,
    arc=3mm,
    boxrule=0pt,
    width=\textwidth,
    top=8pt,
    bottom=8pt
]
{\small{\color{customTeal}\faInfoCircle}
以下是6种乳制品的推荐指数,分别按推荐指数从高到低递减排列(从强烈推荐到避免食用)
}
\end{tcolorbox}

\begin{center}\vspace{-0.5cm}
\begin{tikzpicture}[
    font=\small,
    title/.style={font=\small\bfseries\color{white}},
    value/.style={font=\small},
    reference/.style={font=\small},
    cell/.style={anchor=west, text width=8cm}  % 增加宽度以适应两行文本
]
    \def\cardwidth{\textwidth}
    \def\cardheight{5.4}
    \def\barheight{0.25}
    \def\rowspace{0.9}

    % 创建容器
    \draw[rounded corners=5, fill=white, draw=gray!20]
        (0,0) rectangle (\cardwidth,-\cardheight);

    % 标题栏背景
    \path[fill=customTeal]
        (0,0) [rounded corners=5] -- (\cardwidth,0) --
        (\cardwidth,0.8) -- (0,0.8) -- cycle;

    % 调整表头位置和宽度
    \node[title, anchor=west] at (0.5,0.4) {\textbf{食物名称}};
    \node[title] at (3.25, 0.4) {\textbf{推荐指数}};
    \node[title] at (5, 0.4) {\textbf{能量 (KJ)}};
    \node[title] at (6.5, 0.4) {\textbf{蛋白 (g)}};
    \node[title] at (8, 0.4) {\textbf{脂肪 (g)}};
    \node[title] at (10, 0.4) {\textbf{碳水化合物 (g)}};
    \node[title] at (12, 0.4) {\textbf{淀粉 (g)}};
    \node[title] at (13.75, 0.4) {\textbf{膳食纤维 (g)}};
    \node[title] at (16.0, 0.4) {\textbf{胆固醇 (mg)}};

    % 数据行
    \foreach \item/\recommend/\energy/\protein/\fat/\carbs/\starch/\fiber/\cholesterol/\index in {
        {脱脂牛奶}/{42}/{142}/{3}/{0}/{4}/{0}/{0}/{2}/0.5,
        {牛奶}/{25}/{268}/{3}/{3}/{4}/{0}/{0}/{14}/1.5,
        {冰淇淋}/{1}/{690}/{1}/{3}/{32}/{0}/{0}/{8}/2.5,
        {奶油}/{-1}/{515}/{3}/{10}/{4}/{0}/{0}/{35}/3.5,
        {奶酪}/{-40}/{1552}/{23}/{29}/{2}/{0}/{0}/{94}/4.5,
        {黄油}/{-100}/{2999}/{0}/{81}/{0}/{0}/{215}/5.5
    }
    {
        % 计算每行的y位置
        \pgfmathsetmacro{\ypos}{-\rowspace*\index}

        % 项目名
        \node[cell, align=left] at (0.5,\ypos) {\small\textbf{\item}};

        % 计算进度条垂直居中的位置
        \pgfmathsetmacro{\barypos}{\ypos-0.5*\barheight}

        % 进度条背景
        \fill[gray!10, rounded corners=2] (2.5,\barypos) rectangle (4.00,\barypos+\barheight); % 修改背景宽度

        % 计算推荐指数
        \pgfmathsetmacro{\recommendvalue}{\recommend/100}

        % 进度条和数值显示
        \ifnum\recommend<0
            \fill[red!70, rounded corners=2] (2.5,\barypos) rectangle (2.5-\recommendvalue*2,\barypos+\barheight); % 修改这里
            \node[value, text=red!70] at (3.25,\barypos+\barheight+0.15) {\textbf{\footnotesize\recommend}}; % 调整位置
        \else
            \fill[green!90!black, rounded corners=2] (2.5,\barypos) rectangle (2.5+\recommendvalue*2,\barypos+\barheight); % 修改这里
            \node[value, text=green!90!black] at (3.25,\barypos+\barheight+0.15) {\textbf{\footnotesize\recommend}}; % 调整位置
        \fi

        % 其他数值显示
        \node[value] at (5.0,\ypos) {\footnotesize\energy};
        \node[value] at (6.5,\ypos) {\footnotesize\protein};
        \node[value] at (8.0,\ypos) {\footnotesize\fat};
        \node[value] at (10.0,\ypos) {\footnotesize\carbs};
        \node[value] at (12.0,\ypos) {\footnotesize\starch};
        \node[value] at (13.75,\ypos) {\footnotesize\fiber};
        \node[value] at (16.0,\ypos) {\footnotesize\cholesterol};
    }

    % 分隔线
    \foreach \i in {1,...,5} {
        \pgfmathsetmacro{\y}{-\rowspace*\i}
        \draw[gray!20] (0.2,\y) -- (\cardwidth-0.2,\y);
    }

\end{tikzpicture}
\end{center}

\newpage

\begin{tcolorbox}[
    enhanced,
    colback=white,
    colframe=white,
    arc=2mm,
    boxrule=0pt,
    width=\textwidth,
    left=15pt,
    right=15pt,
    top=10pt,
    bottom=10pt,
    drop shadow={
        opacity=0.2,
        color=customTeal
    },
    borderline west={5pt}{0pt}{customTeal}
]
\textcolor{customTeal}{\Large\textbf{干果}}
\end{tcolorbox}

\begin{tcolorbox}[
    enhanced,
    colback=customTealBg,
    colframe=gray!3,
    arc=3mm,
    boxrule=0pt,
    width=\textwidth,
    top=8pt,
    bottom=8pt
]
{\small{\color{customTeal}\faInfoCircle}
干果
}
\end{tcolorbox}

\begin{center}\vspace{-10pt}
\begin{tikzpicture}[
    font=\small,
    title/.style={font=\small\bfseries\color{white}},
    value/.style={font=\small},
    reference/.style={font=\small},
    cell/.style={anchor=west, text width=8cm}  % 增加宽度以适应两行文本
]
    \def\cardwidth{\textwidth}
    \def\cardheight{13.5}
    \def\barheight{0.25}
    \def\rowspace{0.9}

    % 创建容器
    \draw[rounded corners=5, fill=white, draw=gray!20]
        (0,0) rectangle (\cardwidth,-\cardheight);

    % 标题栏背景
    \path[fill=customTeal]
        (0,0) [rounded corners=5] -- (\cardwidth,0) --
        (\cardwidth,0.8) -- (0,0.8) -- cycle;

    % 调整表头位置和宽度
    \node[title, anchor=west] at (0.5,0.4) {\textbf{食物名称}};
    \node[title] at (3.25, 0.4) {\textbf{推荐指数}};
    \node[title] at (5, 0.4) {\textbf{能量 (KJ)}};
    \node[title] at (6.5, 0.4) {\textbf{蛋白 (g)}};
    \node[title] at (8, 0.4) {\textbf{脂肪 (g)}};
    \node[title] at (10, 0.4) {\textbf{碳水化合物 (g)}};
    \node[title] at (12, 0.4) {\textbf{淀粉 (g)}};
    \node[title] at (13.75, 0.4) {\textbf{膳食纤维 (g)}};
    \node[title] at (16.0, 0.4) {\textbf{胆固醇 (mg)}};

    % 数据行
    \foreach \item/\recommend/\energy/\protein/\fat/\carbs/\starch/\fiber/\cholesterol/\index in {
        {莲子}/{77}/{372}/{4}/{0}/{17}/{0}/{0}/{0}/0.5,
        {栗子}/{53}/{1519}/{6}/{1}/{79}/{0}/{0}/{0}/1.5,
        {榛子}/{37}/{2629}/{14}/{60}/{16}/{0}/{9}/{0}/2.5,
        {芝麻酱}/{8}/{2454}/{18}/{50}/{24}/{0}/{5}/{0}/3.5,
        {橡子}/{8}/{1619}/{6}/{23}/{40}/{0}/{0}/{0}/4.5,
        {葵花子}/{3}/{2445}/{20}/{51}/{20}/{0}/{8}/{0}/5.5,
        {椰肉}/{0}/{1481}/{33}/{15}/{0}/{0}/{9}/{0}/6.5,
        {杏仁}/{-1}/{2423}/{21}/{49}/{21}/{0}/{12}/{0}/7.5,
        {银杏坚果}/{-2}/{1456}/{10}/{2}/{72}/{0}/{0}/{0}/8.5,
        {山核桃干}/{-9}/{2749}/{12}/{64}/{18}/{0}/{6}/{0}/9.5,
        {腰果}/{-12}/{2402}/{15}/{46}/{32}/{0}/{3}/{0}/10.5,
        {核桃}/{-27}/{2738}/{15}/{65}/{13}/{0}/{6}/{0}/11.5,
        {开心果}/{-27}/{2392}/{21}/{45}/{28}/{1}/{10}/{0}/12.5,
        {山核桃}/{-31}/{2889}/{9}/{71}/{13}/{0}/{9}/{0}/13.5,
        {松子}/{-59}/{2816}/{13}/{68}/{13}/{1}/{3}/{0}/14.5
    }
    {
        % 计算每行的y位置
        \pgfmathsetmacro{\ypos}{-\rowspace*\index}

        % 项目名
        \node[cell, align=left] at (0.5,\ypos) {\small\textbf{\item}};

        % 计算进度条垂直居中的位置
        \pgfmathsetmacro{\barypos}{\ypos-0.5*\barheight}

        % 进度条背景
        \fill[gray!10, rounded corners=2] (2.5,\barypos) rectangle (4.00,\barypos+\barheight); % 修改背景宽度

        % 计算推荐指数
        \pgfmathsetmacro{\recommendvalue}{\recommend/100}

        % 进度条和数值显示
        \ifnum\recommend<0
            \fill[red!70, rounded corners=2] (2.5,\barypos) rectangle (2.5-\recommendvalue*2,\barypos+\barheight); % 修改这里
            \node[value, text=red!70] at (3.25,\barypos+\barheight+0.15) {\textbf{\footnotesize\recommend}}; % 调整位置
        \else
            \fill[green!90!black, rounded corners=2] (2.5,\barypos) rectangle (2.5+\recommendvalue*2,\barypos+\barheight); % 修改这里
            \node[value, text=green!90!black] at (3.25,\barypos+\barheight+0.15) {\textbf{\footnotesize\recommend}}; % 调整位置
        \fi

        % 其他数值显示
        \node[value] at (5.0,\ypos) {\footnotesize\energy};
        \node[value] at (6.5,\ypos) {\footnotesize\protein};
        \node[value] at (8.0,\ypos) {\footnotesize\fat};
        \node[value] at (10.0,\ypos) {\footnotesize\carbs};
        \node[value] at (12.0,\ypos) {\footnotesize\starch};
        \node[value] at (13.75,\ypos) {\footnotesize\fiber};
        \node[value] at (16.0,\ypos) {\footnotesize\cholesterol};
    }

    % 分隔线
    \foreach \i in {1,...,14} {
        \pgfmathsetmacro{\y}{-\rowspace*\i}
        \draw[gray!20] (0.2,\y) -- (\cardwidth-0.2,\y);
    }

\end{tikzpicture}
\end{center}

\newpage

\begin{tcolorbox}[
    enhanced,
    colback=white,
    colframe=white,
    arc=2mm,
    boxrule=0pt,
    width=\textwidth,
    left=15pt,
    right=15pt,
    top=10pt,
    bottom=10pt,
    drop shadow={
        opacity=0.2,
        color=customTeal
    },
    borderline west={5pt}{0pt}{customTeal}
]
\textcolor{customTeal}{\Large\textbf{快餐}}
\end{tcolorbox}

\begin{tcolorbox}[
    enhanced,
    colback=customTealBg,
    colframe=gray!3,
    arc=3mm,
    boxrule=0pt,
    width=\textwidth,
    top=8pt,
    bottom=8pt
]
{\small{\color{customTeal}\faInfoCircle}
快餐
}
\end{tcolorbox}

\begin{center}\vspace{-10pt}
\begin{tikzpicture}[
    font=\small,
    title/.style={font=\small\bfseries\color{white}},
    value/.style={font=\small},
    reference/.style={font=\small},
    cell/.style={anchor=west, text width=8cm}  % 增加宽度以适应两行文本
]
    \def\cardwidth{\textwidth}
    \def\cardheight{2.75}
    \def\barheight{0.25}
    \def\rowspace{0.9}

    % 创建容器
    \draw[rounded corners=5, fill=white, draw=gray!20]
        (0,0) rectangle (\cardwidth,-\cardheight);

    % 标题栏背景
    \path[fill=customTeal]
        (0,0) [rounded corners=5] -- (\cardwidth,0) --
        (\cardwidth,0.8) -- (0,0.8) -- cycle;

    % 调整表头位置和宽度
    \node[title, anchor=west] at (0.5,0.4) {\textbf{食物名称}};
    \node[title] at (3.25, 0.4) {\textbf{推荐指数}};
    \node[title] at (5, 0.4) {\textbf{能量 (KJ)}};
    \node[title] at (6.5, 0.4) {\textbf{蛋白 (g)}};
    \node[title] at (8, 0.4) {\textbf{脂肪 (g)}};
    \node[title] at (10, 0.4) {\textbf{碳水化合物 (g)}};
    \node[title] at (12, 0.4) {\textbf{淀粉 (g)}};
    \node[title] at (13.75, 0.4) {\textbf{膳食纤维 (g)}};
    \node[title] at (16.0, 0.4) {\textbf{胆固醇 (mg)}};

    % 数据行
    \foreach \item/\recommend/\energy/\protein/\fat/\carbs/\starch/\fiber/\cholesterol/\index in {
        {比萨}/{-27}/{1121}/{10}/{12}/{29}/{18}/{2}/{14}/0.5,
        {热狗}/{-53}/{1167}/{9}/{3}/{50}/{36}/{1}/{0}/1.5,
        {鸡米花}/{-61}/{1469}/{17}/{21}/{18}/{1}/{1}/{40}/2.5
    }
    {
        % 计算每行的y位置
        \pgfmathsetmacro{\ypos}{-\rowspace*\index}

        % 项目名
        \node[cell, align=left] at (0.5,\ypos) {\small\textbf{\item}};

        % 计算进度条垂直居中的位置
        \pgfmathsetmacro{\barypos}{\ypos-0.5*\barheight}

        % 进度条背景
        \fill[gray!10, rounded corners=2] (2.5,\barypos) rectangle (4.00,\barypos+\barheight); % 修改背景宽度

        % 计算推荐指数
        \pgfmathsetmacro{\recommendvalue}{\recommend/100}

        % 进度条和数值显示
        \ifnum\recommend<0
            \fill[red!70, rounded corners=2] (2.5,\barypos) rectangle (2.5-\recommendvalue*2,\barypos+\barheight); % 修改这里
            \node[value, text=red!70] at (3.25,\barypos+\barheight+0.15) {\textbf{\footnotesize\recommend}}; % 调整位置
        \else
            \fill[green!90!black, rounded corners=2] (2.5,\barypos) rectangle (2.5+\recommendvalue*2,\barypos+\barheight); % 修改这里
            \node[value, text=green!90!black] at (3.25,\barypos+\barheight+0.15) {\textbf{\footnotesize\recommend}}; % 调整位置
        \fi

        % 其他数值显示
        \node[value] at (5.0,\ypos) {\footnotesize\energy};
        \node[value] at (6.5,\ypos) {\footnotesize\protein};
        \node[value] at (8.0,\ypos) {\footnotesize\fat};
        \node[value] at (10.0,\ypos) {\footnotesize\carbs};
        \node[value] at (12.0,\ypos) {\footnotesize\starch};
        \node[value] at (13.75,\ypos) {\footnotesize\fiber};
        \node[value] at (16.0,\ypos) {\footnotesize\cholesterol};
    }

    % 分隔线
    \foreach \i in {1,...,2} {
        \pgfmathsetmacro{\y}{-\rowspace*\i}
        \draw[gray!20] (0.2,\y) -- (\cardwidth-0.2,\y);
    }

\end{tikzpicture}
\end{center}

\newpage

\begin{tcolorbox}[
    enhanced,
    colback=white,
    colframe=white,
    arc=2mm,
    boxrule=0pt,
    width=\textwidth,
    left=15pt,
    right=15pt,
    top=10pt,
    bottom=10pt,
    drop shadow={
        opacity=0.2,
        color=customTeal
    },
    borderline west={5pt}{0pt}{customTeal}
]
\textcolor{customTeal}{\Large\textbf{水产品}}
\end{tcolorbox}

\begin{tcolorbox}[
    enhanced,
    colback=customTealBg,
    colframe=gray!3,
    arc=3mm,
    boxrule=0pt,
    width=\textwidth,
    top=8pt,
    bottom=8pt
]
{\small{\color{customTeal}\faInfoCircle}
水产品
}
\end{tcolorbox}

\begin{center}\vspace{-10pt}
\begin{tikzpicture}[
    font=\small,
    title/.style={font=\small\bfseries\color{white}},
    value/.style={font=\small},
    reference/.style={font=\small},
    cell/.style={anchor=west, text width=8cm}  % 增加宽度以适应两行文本
]
    \def\cardwidth{\textwidth}
    \def\cardheight{19}
    \def\barheight{0.25}
    \def\rowspace{0.9}

    % 创建容器
    \draw[rounded corners=5, fill=white, draw=gray!20]
        (0,0) rectangle (\cardwidth,-\cardheight);

    % 标题栏背景
    \path[fill=customTeal]
        (0,0) [rounded corners=5] -- (\cardwidth,0) --
        (\cardwidth,0.8) -- (0,0.8) -- cycle;

    % 调整表头位置和宽度
    \node[title, anchor=west] at (0.5,0.4) {\textbf{食物名称}};
    \node[title] at (3.25, 0.4) {\textbf{推荐指数}};
    \node[title] at (5, 0.4) {\textbf{能量 (KJ)}};
    \node[title] at (6.5, 0.4) {\textbf{蛋白 (g)}};
    \node[title] at (8, 0.4) {\textbf{脂肪 (g)}};
    \node[title] at (10, 0.4) {\textbf{碳水化合物 (g)}};
    \node[title] at (12, 0.4) {\textbf{淀粉 (g)}};
    \node[title] at (13.75, 0.4) {\textbf{膳食纤维 (g)}};
    \node[title] at (16.0, 0.4) {\textbf{胆固醇 (mg)}};

    % 数据行
    \foreach \item/\recommend/\energy/\protein/\fat/\carbs/\starch/\fiber/\cholesterol/\index in {
        {三文鱼}/{23}/{594}/{19}/{6}/{0}/{0}/{0}/{55}/0.5,
        {凤尾鱼}/{20}/{548}/{20}/{4}/{0}/{0}/{0}/{60}/1.5,
        {墨鱼}/{23}/{516}/{16}/{0}/{0}/{0}/{0}/{112}/2.5,
        {大比目鱼}/{36}/{343}/{18}/{1}/{0}/{0}/{0}/{49}/3.5,
        {大西洋鳕鱼}/{35}/{322}/{15}/{0}/{0}/{0}/{0}/{114}/4.5,
        {小龙虾}/{16}/{360}/{14}/{2}/{0}/{0}/{0}/{24}/5.5,
        {扇贝}/{33}/{406}/{17}/{2}/{0}/{0}/{0}/{80}/6.5,
        {条纹鲈鱼 }/{9}/{364}/{9}/{2}/{4}/{0}/{0}/{52}/7.5,
        {海鲈鱼}/{37}/{385}/{19}/{1}/{0}/{0}/{0}/{37}/8.5,
        {牡蛎}/{41}/{343}/{14}/{2}/{0}/{0}/{0}/{48}/9.5,
        {白鲑}/{9}/{360}/{14}/{0}/{3}/{1}/{0}/{30}/10.5,
        {石斑鱼}/{15}/{364}/{18}/{1}/{0}/{0}/{0}/{78}/11.5,
        {章鱼}/{32}/{602}/{23}/{4}/{0}/{0}/{0}/{38}/12.5,
        {虾}/{32}/{602}/{23}/{4}/{0}/{0}/{0}/{38}/13.5,
        {蛤蜊}/{-26}/{1105}/{24}/{17}/{4}/{0}/{0}/{588}/14.5,
        {蟹}/{26}/{381}/{15}/{1}/{3}/{0}/{0}/{233}/15.5,
        {贻贝}/{35}/{381}/{19}/{0}/{0}/{0}/{0}/{90}/16.5,
        {金枪鱼}/{19}/{439}/{17}/{0}/{6}/{0}/{0}/{85}/17.5,
        {鱼子酱}/{28}/{439}/{16}/{4}/{0}/{0}/{0}/{60}/18.5,
        {鱿鱼}/{26}/{531}/{17}/{5}/{0}/{0}/{0}/{66}/19.5,
        {鲈鱼}/{35}/{381}/{19}/{0}/{0}/{0}/{0}/{90}/20.5
    }
    {
        % 计算每行的y位置
        \pgfmathsetmacro{\ypos}{-\rowspace*\index}

        % 项目名
        \node[cell, align=left] at (0.5,\ypos) {\small\textbf{\item}};

        % 计算进度条垂直居中的位置
        \pgfmathsetmacro{\barypos}{\ypos-0.5*\barheight}

        % 进度条背景
        \fill[gray!10, rounded corners=2] (2.5,\barypos) rectangle (4.00,\barypos+\barheight); % 修改背景宽度

        % 计算推荐指数
        \pgfmathsetmacro{\recommendvalue}{\recommend/100}

        % 进度条和数值显示
        \ifnum\recommend<0
            \fill[red!70, rounded corners=2] (2.5,\barypos) rectangle (2.5-\recommendvalue*2,\barypos+\barheight); % 修改这里
            \node[value, text=red!70] at (3.25,\barypos+\barheight+0.15) {\textbf{\footnotesize\recommend}}; % 调整位置
        \else
            \fill[green!90!black, rounded corners=2] (2.5,\barypos) rectangle (2.5+\recommendvalue*2,\barypos+\barheight); % 修改这里
            \node[value, text=green!90!black] at (3.25,\barypos+\barheight+0.15) {\textbf{\footnotesize\recommend}}; % 调整位置
        \fi

        % 其他数值显示
        \node[value] at (5.0,\ypos) {\footnotesize\energy};
        \node[value] at (6.5,\ypos) {\footnotesize\protein};
        \node[value] at (8.0,\ypos) {\footnotesize\fat};
        \node[value] at (10.0,\ypos) {\footnotesize\carbs};
        \node[value] at (12.0,\ypos) {\footnotesize\starch};
        \node[value] at (13.75,\ypos) {\footnotesize\fiber};
        \node[value] at (16.0,\ypos) {\footnotesize\cholesterol};
    }

    % 分隔线
    \foreach \i in {1,...,20} {
        \pgfmathsetmacro{\y}{-\rowspace*\i}
        \draw[gray!20] (0.2,\y) -- (\cardwidth-0.2,\y);
    }

\end{tikzpicture}
\end{center}


\newpage

\begin{center}\vspace{-10pt}
\begin{tikzpicture}[
    font=\small,
    title/.style={font=\small\bfseries\color{white}},
    value/.style={font=\small},
    reference/.style={font=\small},
    cell/.style={anchor=west, text width=8cm}  % 增加宽度以适应两行文本
]
    \def\cardwidth{\textwidth}
    \def\cardheight{15}
    \def\barheight{0.25}
    \def\rowspace{0.9}

    % 创建容器
    \draw[rounded corners=5, fill=white, draw=gray!20]
        (0,0) rectangle (\cardwidth,-\cardheight);

    % 标题栏背景
    \path[fill=customTeal]
        (0,0) [rounded corners=5] -- (\cardwidth,0) --
        (\cardwidth,0.8) -- (0,0.8) -- cycle;

    % 调整表头位置和宽度
    \node[title, anchor=west] at (0.5,0.4) {\textbf{食物名称}};
    \node[title] at (3.25, 0.4) {\textbf{推荐指数}};
    \node[title] at (5, 0.4) {\textbf{能量 (KJ)}};
    \node[title] at (6.5, 0.4) {\textbf{蛋白 (g)}};
    \node[title] at (8, 0.4) {\textbf{脂肪 (g)}};
    \node[title] at (10, 0.4) {\textbf{碳水化合物 (g)}};
    \node[title] at (12, 0.4) {\textbf{淀粉 (g)}};
    \node[title] at (13.75, 0.4) {\textbf{膳食纤维 (g)}};
    \node[title] at (16.0, 0.4) {\textbf{胆固醇 (mg)}};

    % 数据行
    \foreach \item/\recommend/\energy/\protein/\fat/\carbs/\starch/\fiber/\cholesterol/\index in {
        {鲍鱼}/{0}/{661}/{17}/{9}/{0}/{0}/{0}/{60}/0.5,
        {鲟鱼}/{39}/{418}/{20}/{1}/{0}/{0}/{0}/{37}/1.5,
        {鲤鱼}/{16}/{294}/{12}/{1}/{0}/{0}/{0}/{45}/2.5,
        {鲭鱼}/{37}/{364}/{18}/{0}/{0}/{0}/{0}/{41}/3.5,
        {鲱鱼}/{9}/{770}/{18}/{11}/{0}/{0}/{0}/{126}/4.5,
        {鲶鱼}/{62}/{347}/{0}/{1}/{19}/{0}/{5}/{0}/5.5,
        {鲷鱼}/{17}/{611}/{23}/{5}/{0}/{0}/{0}/{55}/6.5,
        {鲽鱼}/{19}/{324}/{16}/{0}/{0}/{0}/{127}/7.5,
        {鳕鱼}/{0}/{661}/{17}/{9}/{0}/{0}/{0}/{60}/8.5,
        {鳗鱼}/{39}/{418}/{20}/{1}/{0}/{0}/{0}/{37}/9.5,
        {鳟鱼}/{16}/{294}/{12}/{1}/{0}/{0}/{0}/{45}/10.5,
        {沙丁鱼}/{37}/{364}/{18}/{0}/{0}/{0}/{0}/{41}/11.5,
        {黄尾}/{9}/{770}/{18}/{11}/{0}/{0}/{0}/{126}/12.5,
        {龙虾}/{62}/{347}/{0}/{1}/{19}/{0}/{5}/{0}/13.5
    }
    {
        % 计算每行的y位置
        \pgfmathsetmacro{\ypos}{-\rowspace*\index}

        % 项目名
        \node[cell, align=left] at (0.5,\ypos) {\small\textbf{\item}};

        % 计算进度条垂直居中的位置
        \pgfmathsetmacro{\barypos}{\ypos-0.5*\barheight}

        % 进度条背景
        \fill[gray!10, rounded corners=2] (2.5,\barypos) rectangle (4.00,\barypos+\barheight); % 修改背景宽度

        % 计算推荐指数
        \pgfmathsetmacro{\recommendvalue}{\recommend/100}

        % 进度条和数值显示
        \ifnum\recommend<0
            \fill[red!70, rounded corners=2] (2.5,\barypos) rectangle (2.5-\recommendvalue*2,\barypos+\barheight); % 修改这里
            \node[value, text=red!70] at (3.25,\barypos+\barheight+0.15) {\textbf{\footnotesize\recommend}}; % 调整位置
        \else
            \fill[green!90!black, rounded corners=2] (2.5,\barypos) rectangle (2.5+\recommendvalue*2,\barypos+\barheight); % 修改这里
            \node[value, text=green!90!black] at (3.25,\barypos+\barheight+0.15) {\textbf{\footnotesize\recommend}}; % 调整位置
        \fi

        % 其他数值显示
        \node[value] at (5.0,\ypos) {\footnotesize\energy};
        \node[value] at (6.5,\ypos) {\footnotesize\protein};
        \node[value] at (8.0,\ypos) {\footnotesize\fat};
        \node[value] at (10.0,\ypos) {\footnotesize\carbs};
        \node[value] at (12.0,\ypos) {\footnotesize\starch};
        \node[value] at (13.75,\ypos) {\footnotesize\fiber};
        \node[value] at (16.0,\ypos) {\footnotesize\cholesterol};
    }

    % 分隔线
    \foreach \i in {1,...,20} {
        \pgfmathsetmacro{\y}{-\rowspace*\i}
        \draw[gray!20] (0.2,\y) -- (\cardwidth-0.2,\y);
    }

\end{tikzpicture}
\end{center}

\newpage

\begin{tcolorbox}[
    enhanced,
    colback=white,
    colframe=white,
    arc=2mm,
    boxrule=0pt,
    width=\textwidth,
    left=15pt,
    right=15pt,
    top=10pt,
    bottom=10pt,
    drop shadow={
        opacity=0.2,
        color=customTeal
    },
    borderline west={5pt}{0pt}{customTeal}
]
\textcolor{customTeal}{\Large\textbf{水果}}
\end{tcolorbox}

\begin{tcolorbox}[
    enhanced,
    colback=customTealBg,
    colframe=gray!3,
    arc=3mm,
    boxrule=0pt,
    width=\textwidth,
    top=8pt,
    bottom=8pt
]
{\small{\color{customTeal}\faInfoCircle}
水果
}
\end{tcolorbox}

\begin{center}\vspace{-10pt}
\begin{tikzpicture}[
    font=\small,
    title/.style={font=\small\bfseries\color{white}},
    value/.style={font=\small},
    reference/.style={font=\small},
    cell/.style={anchor=west, text width=8cm}  % 增加宽度以适应两行文本
]
    \def\cardwidth{\textwidth}
    \def\cardheight{12.7}
    \def\barheight{0.25}
    \def\rowspace{0.9}

    % 创建容器
    \draw[rounded corners=5, fill=white, draw=gray!20]
        (0,0) rectangle (\cardwidth,-\cardheight);

    % 标题栏背景
    \path[fill=customTeal]
        (0,0) [rounded corners=5] -- (\cardwidth,0) --
        (\cardwidth,0.8) -- (0,0.8) -- cycle;

    % 调整表头位置和宽度
    \node[title, anchor=west] at (0.5,0.4) {\textbf{食物名称}};
    \node[title] at (3.25, 0.4) {\textbf{推荐指数}};
    \node[title] at (5, 0.4) {\textbf{能量 (KJ)}};
    \node[title] at (6.5, 0.4) {\textbf{蛋白 (g)}};
    \node[title] at (8, 0.4) {\textbf{脂肪 (g)}};
    \node[title] at (10, 0.4) {\textbf{碳水化合物 (g)}};
    \node[title] at (12, 0.4) {\textbf{淀粉 (g)}};
    \node[title] at (13.75, 0.4) {\textbf{膳食纤维 (g)}};
    \node[title] at (16.0, 0.4) {\textbf{胆固醇 (mg)}};

    % 数据行
    \foreach \item/\recommend/\energy/\protein/\fat/\carbs/\starch/\fiber/\cholesterol/\index in {
        {李子}/{-12}/{192}/{0}/{0}/{11}/{0}/{1}/{0}/1,
        {杏}/{-48}/{201}/{0}/{0}/{11}/{0}/{2}/{0}/2,
        {桃}/{80}/{128}/{1}/{0}/{6}/{0}/{2}/{0}/3,
        {梅}/{-76}/{117}/{4}/{1}/{72}/{0}/{6}/{0}/4,
        {枇杷}/{82}/{121}/{1}/{0}/{9}/{0}/{2}/{0}/5,
        {杏子}/{-11}/{165}/{0}/{0}/{18}/{0}/{3}/{0}/6,
        {樱桃}/{88}/{239}/{0}/{0}/{9}/{0}/{1}/{0}/7,
        {梨}/{79}/{615}/{1}/{5}/{27}/{0}/{3}/{0}/8,
        {榴莲}/{14}/{1395}/{1}/{0}/{80}/{0}/{2}/{0}/9,
        {哈密瓜}/{-38}/{140}/{0}/{0}/{8}/{0}/{0}/{0}/10,
        {草莓}/{-49}/{76}/{0}/{0}/{10}/{0}/{2}/{0}/11,
        {橙}/{76}/{188}/{0}/{0}/{10}/{0}/{0}/{0}/12,
        {柚子}/{75}/{405}/{1}/{0}/{25}/{0}/{10}/{0}/13,
        {榴莲}/{41}/{185}/{0}/{0}/{12}/{0}/{0}/{0}/14,
        {油桃}/{-41}/{255}/{1}/{0}/{14}/{0}/{3}/{0}/15,
        {猕猴桃}/{37}/{255}/{1}/{0}/{14}/{0}/{3}/{0}/16,
        {葡萄}/{-34}/{197}/{0}/{0}/{9}/{0}/{0}/{0}/17,
        {西瓜}/{40}/{146}/{0}/{0}/{10}/{0}/{0}/{0}/18,
        {石榴}/{89}/{346}/{1}/{1}/{18}/{0}/{4}/{0}/19,
        {红苹果}/{-2}/{247}/{0}/{0}/{14}/{0}/{2}/{0}/20,
        {芒果}/{-70}/{250}/{0}/{0}/{14}/{0}/{1}/{0}/21,
        {苹果}/{-2}/{218}/{0}/{0}/{13}/{0}/{2}/{0}/22,
        {苹果汁}/{-64}/{191}/{0}/{0}/{11}/{0}/{0}/{0}/23,
        {草莓}/{47}/{136}/{0}/{0}/{7}/{0}/{2}/{0}/24,
        {猕猴桃}/{58}/{276}/{0}/{0}/{16}/{0}/{1}/{0}/25,
        {蓝莓}/{-50}/{67}/{0}/{0}/{13}/{0}/{1}/{0}/26,
        {黑莓}/{82}/{181}/{1}/{0}/{9}/{0}/{5}/{0}/27,
        {龙眼}/{78}/{251}/{1}/{0}/{15}/{0}/{1}/{0}/28
    }
    {
        % 计算每行的y位置
        \pgfmathsetmacro{\ypos}{-\rowspace*\index}

        % 项目名
        \node[cell, align=left] at (0.5,\ypos) {\small\textbf{\item}};

        % 计算进度条垂直居中的位置
        \pgfmathsetmacro{\barypos}{\ypos-0.5*\barheight}

        % 进度条背景
        \fill[gray!10, rounded corners=2] (2.5,\barypos) rectangle (4.00,\barypos+\barheight); % 修改背景宽度

        % 计算推荐指数
        \pgfmathsetmacro{\recommendvalue}{\recommend/100}

        % 进度条和数值显示
        \ifnum\recommend<0
            \fill[red!70, rounded corners=2] (2.5,\barypos) rectangle (2.5-\recommendvalue*2,\barypos+\barheight); % 修改这里
            \node[value, text=red!70] at (3.25,\barypos+\barheight+0.15) {\textbf{\footnotesize\recommend}}; % 调整位置
        \else
            \fill[green!90!black, rounded corners=2] (2.5,\barypos) rectangle (2.5+\recommendvalue*2,\barypos+\barheight); % 修改这里
            \node[value, text=green!90!black] at (3.25,\barypos+\barheight+0.15) {\textbf{\footnotesize\recommend}}; % 调整位置
        \fi

        % 其他数值显示
        \node[value] at (5.0,\ypos) {\footnotesize\energy};
        \node[value] at (6.5,\ypos) {\footnotesize\protein};
        \node[value] at (8.0,\ypos) {\footnotesize\fat};
        \node[value] at (10.0,\ypos) {\footnotesize\carbs};
        \node[value] at (12.0,\ypos) {\footnotesize\starch};
        \node[value] at (13.75,\ypos) {\footnotesize\fiber};
        \node[value] at (16.0,\ypos) {\footnotesize\cholesterol};
    }

    % 分隔线
    \foreach \i in {1,...,28} {
        \pgfmathsetmacro{\y}{-\rowspace*\i}
        \draw[gray!20] (0.2,\y) -- (\cardwidth-0.2,\y);
    }

\end{tikzpicture}
\end{center}

\newpage

\begin{tcolorbox}[
    enhanced,
    colback=white,
    colframe=white,
    arc=2mm,
    boxrule=0pt,
    width=\textwidth,
    left=15pt,
    right=15pt,
    top=10pt,
    bottom=10pt,
    drop shadow={
        opacity=0.2,
        color=customTeal
    },
    borderline west={5pt}{0pt}{customTeal}
]
\textcolor{customTeal}{\Large\textbf{汤}}
\end{tcolorbox}

\begin{tcolorbox}[
    enhanced,
    colback=customTealBg,
    colframe=gray!3,
    arc=3mm,
    boxrule=0pt,
    width=\textwidth,
    top=8pt,
    bottom=8pt
]
{\small{\color{customTeal}\faInfoCircle}
汤
}
\end{tcolorbox}

\begin{center}\vspace{-10pt}
\begin{tikzpicture}[
    font=\small,
    title/.style={font=\small\bfseries\color{white}},
    value/.style={font=\small},
    reference/.style={font=\small},
    cell/.style={anchor=west, text width=8cm}  % 增加宽度以适应两行文本
]
    \def\cardwidth{\textwidth}
    \def\cardheight{2.75}
    \def\barheight{0.25}
    \def\rowspace{0.9}

    % 创建容器
    \draw[rounded corners=5, fill=white, draw=gray!20]
        (0,0) rectangle (\cardwidth,-\cardheight);

    % 标题栏背景
    \path[fill=customTeal]
        (0,0) [rounded corners=5] -- (\cardwidth,0) --
        (\cardwidth,0.8) -- (0,0.8) -- cycle;

    % 调整表头位置和宽度
    \node[title, anchor=west] at (0.5,0.4) {\textbf{食物名称}};
    \node[title] at (3.25, 0.4) {\textbf{推荐指数}};
    \node[title] at (5, 0.4) {\textbf{能量 (KJ)}};
    \node[title] at (6.5, 0.4) {\textbf{蛋白 (g)}};
    \node[title] at (8, 0.4) {\textbf{脂肪 (g)}};
    \node[title] at (10, 0.4) {\textbf{碳水化合物 (g)}};
    \node[title] at (12, 0.4) {\textbf{淀粉 (g)}};
    \node[title] at (13.75, 0.4) {\textbf{膳食纤维 (g)}};
    \node[title] at (16.0, 0.4) {\textbf{胆固醇 (mg)}};

    % 数据行
    \foreach \item/\recommend/\energy/\protein/\fat/\carbs/\starch/\fiber/\cholesterol/\index in {
        {番茄汤}/{38}/{165}/{0}/{0}/{9}/{0}/{0}/{0}/0.5,
        {土豆蔬菜汤}/{7}/{126}/{1}/{1}/{3}/{0}/{0}/{1}/1.5,
        {素食蔬菜汤}/{9}/{119}/{0}/{0}/{4}/{0}/{0}/{0}/2.5
    }
    {
        % 计算每行的y位置
        \pgfmathsetmacro{\ypos}{-\rowspace*\index}

        % 项目名
        \node[cell, align=left] at (0.5,\ypos) {\small\textbf{\item}};

        % 计算进度条垂直居中的位置
        \pgfmathsetmacro{\barypos}{\ypos-0.5*\barheight}

        % 进度条背景
        \fill[gray!10, rounded corners=2] (2.5,\barypos) rectangle (4.00,\barypos+\barheight); % 修改背景宽度

        % 计算推荐指数
        \pgfmathsetmacro{\recommendvalue}{\recommend/100}

        % 进度条和数值显示
        \ifnum\recommend<0
            \fill[red!70, rounded corners=2] (2.5,\barypos) rectangle (2.5-\recommendvalue*2,\barypos+\barheight); % 修改这里
            \node[value, text=red!70] at (3.25,\barypos+\barheight+0.15) {\textbf{\footnotesize\recommend}}; % 调整位置
        \else
            \fill[green!90!black, rounded corners=2] (2.5,\barypos) rectangle (2.5+\recommendvalue*2,\barypos+\barheight); % 修改这里
            \node[value, text=green!90!black] at (3.25,\barypos+\barheight+0.15) {\textbf{\footnotesize\recommend}}; % 调整位置
        \fi

        % 其他数值显示
        \node[value] at (5.0,\ypos) {\footnotesize\energy};
        \node[value] at (6.5,\ypos) {\footnotesize\protein};
        \node[value] at (8.0,\ypos) {\footnotesize\fat};
        \node[value] at (10.0,\ypos) {\footnotesize\carbs};
        \node[value] at (12.0,\ypos) {\footnotesize\starch};
        \node[value] at (13.75,\ypos) {\footnotesize\fiber};
        \node[value] at (16.0,\ypos) {\footnotesize\cholesterol};
    }

    % 分隔线
    \foreach \i in {1,...,2} {
        \pgfmathsetmacro{\y}{-\rowspace*\i}
        \draw[gray!20] (0.2,\y) -- (\cardwidth-0.2,\y);
    }

\end{tikzpicture}
\end{center}

\newpage

\begin{tcolorbox}[
    enhanced,
    colback=white,
    colframe=white,
    arc=2mm,
    boxrule=0pt,
    width=\textwidth,
    left=15pt,
    right=15pt,
    top=10pt,
    bottom=10pt,
    drop shadow={
        opacity=0.2,
        color=customTeal
    },
    borderline west={5pt}{0pt}{customTeal}
]
\textcolor{customTeal}{\Large\textbf{肉类}}
\end{tcolorbox}

\begin{tcolorbox}[
    enhanced,
    colback=customTealBg,
    colframe=gray!3,
    arc=3mm,
    boxrule=0pt,
    width=\textwidth,
    top=8pt,
    bottom=8pt
]
{\small{\color{customTeal}\faInfoCircle}
肉类
}
\end{tcolorbox}

\begin{center}\vspace{-10pt}
\begin{tikzpicture}[
    font=\small,
    title/.style={font=\small\bfseries\color{white}},
    value/.style={font=\small},
    reference/.style={font=\small},
    cell/.style={anchor=west, text width=8cm}  % 增加宽度以适应两行文本
]
    \def\cardwidth{\textwidth}
    \def\cardheight{22}
    \def\barheight{0.25}
    \def\rowspace{0.9}

    % 创建容器
    \draw[rounded corners=5, fill=white, draw=gray!20]
        (0,0) rectangle (\cardwidth,-\cardheight);

    % 标题栏背景
    \path[fill=customTeal]
        (0,0) [rounded corners=5] -- (\cardwidth,0) --
        (\cardwidth,0.8) -- (0,0.8) -- cycle;

    % 调整表头位置和宽度
    \node[title, anchor=west] at (0.5,0.4) {\textbf{食物名称}};
    \node[title] at (3.25, 0.4) {\textbf{推荐指数}};
    \node[title] at (5, 0.4) {\textbf{能量 (KJ)}};
    \node[title] at (6.5, 0.4) {\textbf{蛋白 (g)}};
    \node[title] at (8, 0.4) {\textbf{脂肪 (g)}};
    \node[title] at (10, 0.4) {\textbf{碳水化合物 (g)}};
    \node[title] at (12, 0.4) {\textbf{淀粉 (g)}};
    \node[title] at (13.75, 0.4) {\textbf{膳食纤维 (g)}};
    \node[title] at (16.0, 0.4) {\textbf{胆固醇 (mg)}};

    % 数据行
    \foreach \item/\recommend/\energy/\protein/\fat/\carbs/\starch/\fiber/\cholesterol/\index in {
        {鹅肝}/{49}/{556}/{16}/{4}/{6}/{0}/{0}/{515}/0.5,
        {鸡肝}/{40}/{496}/{16}/{4}/{0}/{0}/{0}/{345}/1.5,
        {猪肝}/{35}/{690}/{26}/{4}/{3}/{0}/{0}/{355}/2.5,
        {牛蛙}/{33}/{305}/{16}/{0}/{0}/{0}/{0}/{50}/3.5,
        {牛肉瘦}/{32}/{488}/{23}/{2}/{0}/{0}/{0}/{55}/4.5,
        {猪瘦肉}/{24}/{562}/{21}/{4}/{0}/{0}/{0}/{64}/5.5,
        {鸡心}/{19}/{640}/{15}/{9}/{0}/{0}/{0}/{136}/6.5,
        {火鸡}/{10}/{790}/{28}/{7}/{0}/{0}/{0}/{109}/7.5,
        {瘦羊肉}/{9}/{862}/{28}/{9}/{0}/{0}/{0}/{92}/8.5,
        {牛肉汤}/{9}/{25}/{1}/{0}/{0}/{0}/{0}/{0}/9.5,
        {鹌鹑}/{6}/{803}/{19}/{12}/{0}/{0}/{0}/{76}/10.5,
        {鸡汤}/{2}/{26}/{0}/{0}/{0}/{0}/{0}/{2}/11.5,
        {火腿}/{-8}/{683}/{16}/{8}/{3}/{0}/{1}/{57}/12.5,
        {猪蹄}/{-13}/{889}/{23}/{12}/{0}/{0}/{0}/{88}/13.5,
        {烧鹅}/{-20}/{1276}/{25}/{21}/{0}/{0}/{0}/{91}/14.5,
        {鸡肉}/{-28}/{604}/{28}/{3}/{0}/{0}/{0}/{86}/15.5,
        {烤肉}/{-29}/{1512}/{20}/{30}/{0}/{0}/{0}/{105}/16.5,
        {烟熏火腿}/{-34}/{591}/{18}/{2}/{10}/{0}/{0}/{50}/17.5,
        {肉丸}/{-39}/{1196}/{14}/{22}/{8}/{2}/{2}/{66}/18.5,
        {猪头肉}/{-42}/{658}/{13}/{10}/{0}/{0}/{0}/{69}/19.5,
        {烤鸭}/{-45}/{1410}/{18}/{28}/{0}/{0}/{0}/{84}/20.5,
        {肥猪肉}/{-92}/{2449}/{10}/{60}/{0}/{0}/{0}/{81}/21.5,
        {猪培根}/{-92}/{1744}/{12}/{39}/{1}/{0}/{0}/{66}/22.5,
        {肥羊肉}/{-100}/{2782}/{6}/{70}/{0}/{0}/{0}/{90}/23.5
    }
    {
        % 计算每行的y位置
        \pgfmathsetmacro{\ypos}{-\rowspace*\index}

        % 项目名
        \node[cell, align=left] at (0.5,\ypos) {\small\textbf{\item}};

        % 计算进度条垂直居中的位置
        \pgfmathsetmacro{\barypos}{\ypos-0.5*\barheight}

        % 进度条背景
        \fill[gray!10, rounded corners=2] (2.5,\barypos) rectangle (4.00,\barypos+\barheight); % 修改背景宽度

        % 计算推荐指数
        \pgfmathsetmacro{\recommendvalue}{\recommend/100}

        % 进度条和数值显示
        \ifnum\recommend<0
            \fill[red!70, rounded corners=2] (2.5,\barypos) rectangle (2.5-\recommendvalue*1.5,\barypos+\barheight); % 修改这里
            \node[value, text=red!70] at (3.25,\barypos+\barheight+0.15) {\textbf{\footnotesize\recommend}}; % 调整位置
        \else
            \fill[green!90!black, rounded corners=2] (2.5,\barypos) rectangle (2.5+\recommendvalue*1.5,\barypos+\barheight); % 修改这里
            \node[value, text=green!90!black] at (3.25,\barypos+\barheight+0.15) {\textbf{\footnotesize\recommend}}; % 调整位置
        \fi

        % 其他数值显示
        \node[value] at (5.0,\ypos) {\footnotesize\energy};
        \node[value] at (6.5,\ypos) {\footnotesize\protein};
        \node[value] at (8.0,\ypos) {\footnotesize\fat};
        \node[value] at (10.0,\ypos) {\footnotesize\carbs};
        \node[value] at (12.0,\ypos) {\footnotesize\starch};
        \node[value] at (13.75,\ypos) {\footnotesize\fiber};
        \node[value] at (16.0,\ypos) {\footnotesize\cholesterol};
    }

    % 分隔线
    \foreach \i in {1,...,23} {
        \pgfmathsetmacro{\y}{-\rowspace*\i}
        \draw[gray!20] (0.2,\y) -- (\cardwidth-0.2,\y);
    }

\end{tikzpicture}
\end{center}

\newpage

\begin{tcolorbox}[
    enhanced,
    colback=white,
    colframe=white,
    arc=2mm,
    boxrule=0pt,
    width=\textwidth,
    left=15pt,
    right=15pt,
    top=10pt,
    bottom=10pt,
    drop shadow={
        opacity=0.2,
        color=customTeal
    },
    borderline west={5pt}{0pt}{customTeal}
]
\textcolor{customTeal}{\Large\textbf{蔬菜}}
\end{tcolorbox}

\begin{tcolorbox}[
    enhanced,
    colback=customTealBg,
    colframe=gray!3,
    arc=3mm,
    boxrule=0pt,
    width=\textwidth,
    top=8pt,
    bottom=8pt
]
{\small{\color{customTeal}\faInfoCircle}
蔬菜
}
\end{tcolorbox}

\begin{center}\vspace{-10pt}
\begin{tikzpicture}[
    font=\small,
    title/.style={font=\small\bfseries\color{white}},
    value/.style={font=\small},
    reference/.style={font=\small},
    cell/.style={anchor=west, text width=8cm}  % 增加宽度以适应两行文本
]
    \def\cardwidth{\textwidth}
    \def\cardheight{20}
    \def\barheight{0.25}
    \def\rowspace{0.9}

    % 创建容器
    \draw[rounded corners=5, fill=white, draw=gray!20]
        (0,0) rectangle (\cardwidth,-\cardheight);

    % 标题栏背景
    \path[fill=customTeal]
        (0,0) [rounded corners=5] -- (\cardwidth,0) --
        (\cardwidth,0.8) -- (0,0.8) -- cycle;

    % 调整表头位置和宽度
    \node[title, anchor=west] at (0.5,0.4) {\textbf{食物名称}};
    \node[title] at (3.25, 0.4) {\textbf{推荐指数}};
    \node[title] at (5, 0.4) {\textbf{能量 (KJ)}};
    \node[title] at (6.5, 0.4) {\textbf{蛋白 (g)}};
    \node[title] at (8, 0.4) {\textbf{脂肪 (g)}};
    \node[title] at (10, 0.4) {\textbf{碳水化合物 (g)}};
    \node[title] at (12, 0.4) {\textbf{淀粉 (g)}};
    \node[title] at (13.75, 0.4) {\textbf{膳食纤维 (g)}};
    \node[title] at (16.0, 0.4) {\textbf{胆固醇 (mg)}};

    % 数据行
    \foreach \item/\recommend/\energy/\protein/\fat/\carbs/\starch/\fiber/\cholesterol/\index in {
        {南瓜}/{87}/{109}/{1}/{0}/{6}/{0}/{0}/{0}/0.5,
        {卷心菜}/{70}/{103}/{1}/{0}/{5}/{0}/{0}/{0}/1.5,
        {四季豆}/{86}/{131}/{1}/{0}/{6}/{0}/{0}/{0}/2.5,
        {土豆}/{81}/{322}/{2}/{0}/{17}/{15}/{0}/{0}/3.5,
        {土豆面粉}/{89}/{1493}/{6}/{0}/{83}/{0}/{0}/{0}/4.5,
        {大白菜}/{91}/{55}/{1}/{0}/{2}/{0}/{0}/{0}/5.5,
        {大蒜}/{85}/{623}/{6}/{0}/{33}/{0}/{0}/{0}/6.5,
        {大豆}/{80}/{614}/{12}/{6}/{11}/{0}/{0}/{0}/7.5,
        {小南瓜}/{0}/{69}/{1}/{0}/{3}/{0}/{0}/{0}/8.5,
        {小萝卜}/{93}/{76}/{0}/{0}/{4}/{0}/{0}/{0}/9.5,
        {山药}/{75}/{343}/{1}/{0}/{20}/{0}/{0}/{0}/10.5,
        {扁豆}/{77}/{1473}/{24}/{1}/{63}/{10}/{0}/{0}/11.5,
        {木薯}/{67}/{667}/{1}/{0}/{38}/{0}/{0}/{0}/12.5,
        {洋葱}/{49}/{166}/{1}/{0}/{9}/{0}/{0}/{1}/13.5,
        {炒蘑菇}/{69}/{110}/{3}/{0}/{4}/{0}/{0}/{1}/14.5,
        {炒香菇}/{64}/{162}/{3}/{0}/{7}/{0}/{0}/{3}/15.5,
        {牛蒡根}/{98}/{302}/{1}/{0}/{17}/{0}/{0}/{3}/16.5,
        {甘蓝}/{38}/{359}/{1}/{0}/{20}/{12}/{0}/{3}/17.5,
        {甜椒}/{84}/{84}/{0}/{0}/{4}/{0}/{0}/{1}/18.5,
        {甜玉米}/{45}/{360}/{3}/{1}/{18}/{5}/{2}/{0}/19.5,
        {甜菜}/{87}/{180}/{1}/{0}/{9}/{0}/{2}/{0}/20.5
    }
    {
        % 计算每行的y位置
        \pgfmathsetmacro{\ypos}{-\rowspace*\index}

        % 项目名
        \node[cell, align=left] at (0.5,\ypos) {\small\textbf{\item}};

        % 计算进度条垂直居中的位置
        \pgfmathsetmacro{\barypos}{\ypos-0.5*\barheight}

        % 进度条背景
        \fill[gray!10, rounded corners=2] (2.5,\barypos) rectangle (4.00,\barypos+\barheight); % 修改背景宽度

        % 计算推荐指数
        \pgfmathsetmacro{\recommendvalue}{\recommend/100}

        % 进度条和数值显示
        \ifnum\recommend<0
            \fill[red!70, rounded corners=2] (2.5,\barypos) rectangle (2.5-\recommendvalue*2,\barypos+\barheight); % 修改这里
            \node[value, text=red!70] at (3.25,\barypos+\barheight+0.15) {\textbf{\footnotesize\recommend}}; % 调整位置
        \else
            \fill[green!90!black, rounded corners=2] (2.5,\barypos) rectangle (2.5+\recommendvalue*2,\barypos+\barheight); % 修改这里
            \node[value, text=green!90!black] at (3.25,\barypos+\barheight+0.15) {\textbf{\footnotesize\recommend}}; % 调整位置
        \fi

        % 其他数值显示
        \node[value] at (5.0,\ypos) {\footnotesize\energy};
        \node[value] at (6.5,\ypos) {\footnotesize\protein};
        \node[value] at (8.0,\ypos) {\footnotesize\fat};
        \node[value] at (10.0,\ypos) {\footnotesize\carbs};
        \node[value] at (12.0,\ypos) {\footnotesize\starch};
        \node[value] at (13.75,\ypos) {\footnotesize\fiber};
        \node[value] at (16.0,\ypos) {\footnotesize\cholesterol};
    }

    % 分隔线
    \foreach \i in {1,...,20} {
        \pgfmathsetmacro{\y}{-\rowspace*\i}
        \draw[gray!20] (0.2,\y) -- (\cardwidth-0.2,\y);
    }

\end{tikzpicture}
\end{center}

\newpage

\begin{center}\vspace{-10pt}
\begin{tikzpicture}[
    font=\small,
    title/.style={font=\small\bfseries\color{white}},
    value/.style={font=\small},
    reference/.style={font=\small},
    cell/.style={anchor=west, text width=8cm}  % 增加宽度以适应两行文本
]
    \def\cardwidth{\textwidth}
    \def\cardheight{22}
    \def\barheight{0.25}
    \def\rowspace{0.9}

    % 创建容器
    \draw[rounded corners=5, fill=white, draw=gray!20]
        (0,0) rectangle (\cardwidth,-\cardheight);

    % 标题栏背景
    \path[fill=customTeal]
        (0,0) [rounded corners=5] -- (\cardwidth,0) --
        (\cardwidth,0.8) -- (0,0.8) -- cycle;

    % 调整表头位置和宽度
    \node[title, anchor=west] at (0.5,0.4) {\textbf{食物名称}};
    \node[title] at (3.25, 0.4) {\textbf{推荐指数}};
    \node[title] at (5, 0.4) {\textbf{能量 (KJ)}};
    \node[title] at (6.5, 0.4) {\textbf{蛋白 (g)}};
    \node[title] at (8, 0.4) {\textbf{脂肪 (g)}};
    \node[title] at (10, 0.4) {\textbf{碳水化合物 (g)}};
    \node[title] at (12, 0.4) {\textbf{淀粉 (g)}};
    \node[title] at (13.75, 0.4) {\textbf{膳食纤维 (g)}};
    \node[title] at (16.0, 0.4) {\textbf{胆固醇 (mg)}};

    % 数据行
    \foreach \item/\recommend/\energy/\protein/\fat/\carbs/\starch/\fiber/\cholesterol/\index in {
        {甜菜叶}/{90}/{92}/{2}/{0}/{4}/{0}/{3}/{0}/0.5,
        {生菜}/{100}/{75}/{0}/{0}/{3}/{0}/{1}/{0}/1.5,
        {生菜}/{78}/{62}/{1}/{0}/{2}/{0}/{1}/{0}/2.5,
        {番茄汁}/{10}/{72}/{0}/{0}/{3}/{0}/{0}/{0}/3.5,
        {白菜}/{72}/{48}/{1}/{0}/{1}/{0}/{1}/{0}/4.5,
        {白菜心}/{94}/{59}/{1}/{0}/{2}/{0}/{2}/{0}/5.5,
        {白萝卜}/{85}/{93}/{3}/{0}/{3}/{0}/{1}/{0}/6.5,
        {秋葵}/{88}/{138}/{1}/{0}/{7}/{0}/{3}/{0}/7.5,
        {竹笋}/{96}/{115}/{2}/{0}/{5}/{0}/{2}/{0}/8.5,
        {红心萝卜}/{92}/{132}/{1}/{0}/{7}/{0}/{3}/{0}/9.5,
        {红豆}/{65}/{121}/{4}/{0}/{4}/{0}/{0}/{0}/10.5,
        {绿豆}/{79}/{126}/{3}/{0}/{5}/{0}/{1}/{0}/11.5,
        {羽衣甘蓝}/{100}/{207}/{4}/{0}/{8}/{0}/{3}/{0}/12.5,
        {胡萝卜}/{22}/{173}/{0}/{0}/{9}/{1}/{2}/{1}/13.5,
        {芽头}/{85}/{469}/{1}/{0}/{26}/{0}/{4}/{0}/14.5,
        {芝麻}/{74}/{2397}/{17}/{49}/{23}/{0}/{11}/{0}/15.5,
        {芥末}/{100}/{114}/{2}/{0}/{4}/{0}/{3}/{0}/16.5,
        {芦笋}/{75}/{85}/{2}/{0}/{3}/{0}/{2}/{0}/17.5,
        {花椒}/{98}/{80}/{0}/{0}/{4}/{0}/{0}/{0}/18.5,
        {花椰菜}/{74}/{104}/{1}/{0}/{4}/{0}/{0}/{0}/19.5,
        {芹菜}/{83}/{67}/{0}/{0}/{2}/{0}/{0}/{0}/20.5,
        {苦瓜}/{100}/{126}/{5}/{0}/{3}/{0}/{0}/{0}/21.5,
        {茄子}/{77}/{104}/{0}/{0}/{5}/{0}/{0}/{0}/22.5,
        {苋菜叶}/{86}/{97}/{2}/{0}/{4}/{0}/{0}/{0}/23.5
    }
    {
        % 计算每行的y位置
        \pgfmathsetmacro{\ypos}{-\rowspace*\index}

        % 项目名
        \node[cell, align=left] at (0.5,\ypos) {\small\textbf{\item}};

        % 计算进度条垂直居中的位置
        \pgfmathsetmacro{\barypos}{\ypos-0.5*\barheight}

        % 进度条背景
        \fill[gray!10, rounded corners=2] (2.5,\barypos) rectangle (4.00,\barypos+\barheight); % 修改背景宽度

        % 计算推荐指数
        \pgfmathsetmacro{\recommendvalue}{\recommend/100}

        % 进度条和数值显示
        \ifnum\recommend<0
            \fill[red!70, rounded corners=2] (2.5,\barypos) rectangle (2.5-\recommendvalue*2,\barypos+\barheight); % 修改这里
            \node[value, text=red!70] at (3.25,\barypos+\barheight+0.15) {\textbf{\footnotesize\recommend}}; % 调整位置
        \else
            \fill[green!90!black, rounded corners=2] (2.5,\barypos) rectangle (2.5+\recommendvalue*2,\barypos+\barheight); % 修改这里
            \node[value, text=green!90!black] at (3.25,\barypos+\barheight+0.15) {\textbf{\footnotesize\recommend}}; % 调整位置
        \fi

        % 其他数值显示
        \node[value] at (5.0,\ypos) {\footnotesize\energy};
        \node[value] at (6.5,\ypos) {\footnotesize\protein};
        \node[value] at (8.0,\ypos) {\footnotesize\fat};
        \node[value] at (10.0,\ypos) {\footnotesize\carbs};
        \node[value] at (12.0,\ypos) {\footnotesize\starch};
        \node[value] at (13.75,\ypos) {\footnotesize\fiber};
        \node[value] at (16.0,\ypos) {\footnotesize\cholesterol};
    }

    % 分隔线
    \foreach \i in {1,...,23} {
        \pgfmathsetmacro{\y}{-\rowspace*\i}
        \draw[gray!20] (0.2,\y) -- (\cardwidth-0.2,\y);
    }

\end{tikzpicture}
\end{center}

\newpage

\begin{center}\vspace{-10pt}
\begin{tikzpicture}[
    font=\small,
    title/.style={font=\small\bfseries\color{white}},
    value/.style={font=\small},
    reference/.style={font=\small},
    cell/.style={anchor=west, text width=8cm}  % 增加宽度以适应两行文本
]
    \def\cardwidth{\textwidth}
    \def\cardheight{9}
    \def\barheight{0.25}
    \def\rowspace{0.9}

    % 创建容器
    \draw[rounded corners=5, fill=white, draw=gray!20]
        (0,0) rectangle (\cardwidth,-\cardheight);

    % 标题栏背景
    \path[fill=customTeal]
        (0,0) [rounded corners=5] -- (\cardwidth,0) --
        (\cardwidth,0.8) -- (0,0.8) -- cycle;

    % 调整表头位置和宽度
    \node[title, anchor=west] at (0.5,0.4) {\textbf{食物名称}};
    \node[title] at (3.25, 0.4) {\textbf{推荐指数}};
    \node[title] at (5, 0.4) {\textbf{能量 (KJ)}};
    \node[title] at (6.5, 0.4) {\textbf{蛋白 (g)}};
    \node[title] at (8, 0.4) {\textbf{脂肪 (g)}};
    \node[title] at (10, 0.4) {\textbf{碳水化合物 (g)}};
    \node[title] at (12, 0.4) {\textbf{淀粉 (g)}};
    \node[title] at (13.75, 0.4) {\textbf{膳食纤维 (g)}};
    \node[title] at (16.0, 0.4) {\textbf{胆固醇 (mg)}};

    % 数据行
    \foreach \item/\recommend/\energy/\protein/\fat/\carbs/\starch/\fiber/\cholesterol/\index in {
        {菊苣}/{100}/{71}/{0}/{0}/{4}/{0}/{0}/{0}/0.5,
        {菜豆}/{81}/{280}/{6}/{0}/{13}/{0}/{0}/{0}/1.5,
        {波菜}/{85}/{97}/{2}/{0}/{3}/{0}/{0}/{0}/2.5,
        {莴苣}/{95}/{143}/{4}/{0}/{5}/{0}/{0}/{0}/3.5,
        {莱}/{100}/{180}/{4}/{0}/{7}/{0}/{0}/{0}/4.5,
        {茄子}/{73}/{141}/{2}/{0}/{6}/{0}/{0}/{0}/5.5,
        {西兰花}/{98}/{141}/{2}/{0}/{6}/{0}/{0}/{0}/6.5,
        {西红柿}/{73}/{74}/{0}/{0}/{3}/{0}/{0}/{0}/7.5,
        {豆芽}/{85}/{510}/{13}/{6}/{9}/{0}/{1}/{0}/8.5,
        {豌豆}/{97}/{376}/{2}/{0}/{18}/{0}/{5}/{0}/9.5
    }
    {
        % 计算每行的y位置
        \pgfmathsetmacro{\ypos}{-\rowspace*\index}

        % 项目名
        \node[cell, align=left] at (0.5,\ypos) {\small\textbf{\item}};

        % 计算进度条垂直居中的位置
        \pgfmathsetmacro{\barypos}{\ypos-0.5*\barheight}

        % 进度条背景
        \fill[gray!10, rounded corners=2] (2.5,\barypos) rectangle (4.00,\barypos+\barheight); % 修改背景宽度

        % 计算推荐指数
        \pgfmathsetmacro{\recommendvalue}{\recommend/100}

        % 进度条和数值显示
        \ifnum\recommend<0
            \fill[red!70, rounded corners=2] (2.5,\barypos) rectangle (2.5-\recommendvalue*2,\barypos+\barheight); % 修改这里
            \node[value, text=red!70] at (3.25,\barypos+\barheight+0.15) {\textbf{\footnotesize\recommend}}; % 调整位置
        \else
            \fill[green!90!black, rounded corners=2] (2.5,\barypos) rectangle (2.5+\recommendvalue*2,\barypos+\barheight); % 修改这里
            \node[value, text=green!90!black] at (3.25,\barypos+\barheight+0.15) {\textbf{\footnotesize\recommend}}; % 调整位置
        \fi

        % 其他数值显示
        \node[value] at (5.0,\ypos) {\footnotesize\energy};
        \node[value] at (6.5,\ypos) {\footnotesize\protein};
        \node[value] at (8.0,\ypos) {\footnotesize\fat};
        \node[value] at (10.0,\ypos) {\footnotesize\carbs};
        \node[value] at (12.0,\ypos) {\footnotesize\starch};
        \node[value] at (13.75,\ypos) {\footnotesize\fiber};
        \node[value] at (16.0,\ypos) {\footnotesize\cholesterol};
    }

    % 分隔线
    \foreach \i in {1,...,9} {
        \pgfmathsetmacro{\y}{-\rowspace*\i}
        \draw[gray!20] (0.2,\y) -- (\cardwidth-0.2,\y);
    }

\end{tikzpicture}
\end{center}

\newpage

\begin{tcolorbox}[
    enhanced,
    colback=white,
    colframe=customTeal,
    arc=2mm,
    boxrule=1pt,
    left=20pt,
    right=20pt,
    top=12pt,
    bottom=12pt,
    width=\textwidth,
    fontupper=\sffamily,
    overlay={
    \draw[customTeal, line width=2pt]
    ([xshift=15pt]frame.south west) -- ([xshift=-15pt]frame.south east);
    }
]
{\Large\bfseries\textcolor{customTeal}{\Huge 检出肠道菌群}}
\end{tcolorbox}

\begin{tcolorbox}[
    enhanced,
    colback=customTealBg,
    colframe=customTealBg,
    arc=3mm,
    boxrule=0pt,
    width=\textwidth,
    top=8pt,
    bottom=8pt
]
{\small{\color{customTeal}\faInfoCircle} 本部分会列出在您肠道中实际检测出的肠道细菌,真菌和真核生物。
}
\end{tcolorbox}

\newpage

\begin{center}
\begin{tikzpicture}[
    font=\small,
    title/.style={font=\small\bfseries\color{white}},
    value/.style={font=\small},
    reference/.style={font=\small},
    cell/.style={anchor=west, text width=4.2cm},
    note/.style={anchor=west, text width=4.5cm, align=left}
]
    \def\cardwidth{\textwidth}
    \def\cardheight{15.75}
    \def\barheight{0.25}
    \def\barwidth{1.5}
    \def\valuebarspace{0.4}

    % 容器和标题栏背景
    \draw[rounded corners=5, fill=white, draw=gray!20]
        (0,0) rectangle (\cardwidth,-\cardheight);
    \path[fill=customTeal]
        (0,0) [rounded corners=5] -- (\cardwidth,0) --
        (\cardwidth,0.8) -- (0,0.8) -- cycle;

    % 修改后的表头(移除最后两列)
    \node[title, anchor=west] at (0.5,0.4) {\textbf{菌种名称}};
    \node[title] at (4.5,0.4) {\textbf{正常范围}};
    \node[title] at (7.5,0.4) {\textbf{检测丰度}};
    \node[title] at (10.5,0.4) {\textbf{结果评价}};

    % 初始化位置计数器
    \def\currentpos{0.25}

    % 数据行和卡片
    \foreach \item/\enitem/\value/\range/\percentile/\detection/\status/\intro/\suggestion/\index in {
        {梭菌属}/{Clostridium}/1.90370/{0-4.464}/67\%/99.52\%/正常/{厌氧菌,能产生丁酸盐,参与胆汁酸代谢和色氨酸代谢,对维持肠道屏障功能和免疫系统调节具有重要作用}/{}/\currentpos,
        {普雷沃氏菌属}/{Prevotella}/0.04290/{0-67.8009}/50\%/99.52\%/正常/{革兰氏阴性厌氧菌,专门降解植物多糖和黏蛋白的菌群,产生琥珀酸和乙酸,与植物性饮食密切相关}/{}/\currentpos,
        {瘤胃球菌属}/{Ruminococcus}/9.84521/{0.0544-19.7985}/44\%/99.52\%/正常/{专性厌氧菌,是肠道主要的纤维素降解菌,能产生乙酸和丁酸,对维持结肠上皮细胞健康至关重要}/{}/\currentpos,
        {拟杆菌属}/{Bacteroides}/0.08974/{1.0578-47.3225}/2\%/99.04\%/偏低/{革兰氏阴性厌氧菌,能降解复杂碳水化合物,产生丙酸盐,参与胆固醇代谢,调节宿主免疫系统}/{建议增加全谷物、豆类等膳食纤维摄入}/\currentpos,
        {真杆菌属}/{Eubacterium}/5.75186/{0.1146-9.4883}/95\%/98.56\%/正常/{专性厌氧菌,主要产生丁酸盐,具有抗炎作用,参与胆固醇转化和胆汁酸代谢,维持肠道屏障完整性}/{}/\currentpos,
        {乳酸杆菌属}/{Lactobacillus}/0.00660/{0-0.4302}/8\%/91.83\%/正常/{革兰氏阳性兼性厌氧菌,产生乳酸和抗菌物质,增强肠道屏障功能,调节免疫系统,抑制有害菌生长}/{}/\currentpos,
        {芽孢杆菌属}/{Bacillus}/ND/{0.0001-0.5535}/27\%/77.88\%/未检出/{革兰氏阳性需氧菌,能形成芽孢,产生多种水解酶和抗菌肽,增强肠道免疫功能,改善肠道微生态平衡}/{建议适当补充含芽孢杆菌的活性益生菌制剂}/\currentpos,
        {Lachnoclostridium}/{Lachnoclostridium}/0.17516/{0-0.2086}/8\%/98.56\%/正常/{厌氧产丁酸菌,能降解复杂碳水化合物,产生短链脂肪酸,参与结肠上皮细胞能量代谢,维持肠道健康}/{}/\currentpos
    }
    {
        % 计算当前行的基础位置
        \pgfmathsetmacro{\basepos}{-2.8*\currentpos}

        % 菌种名称
        \node[cell, align=left] at (0.5,\basepos) {
            \small\textbf{\item}\\[-0.2em]
            {\color{lightgray}\small\enitem}
        };

        % 正常范围
        \node[reference] at (4.5,\basepos) {\footnotesize\range};

        % 进度条相关
        \pgfmathsetmacro{\barypos}{\basepos-\valuebarspace+0.1}
        \def\barstart{6.75}

        % 进度条背景
        \fill[gray!10, rounded corners=2] (\barstart,\barypos)
            rectangle (\barstart+\barwidth,\barypos+\barheight);

        % 检测丰度值
        \node[value] at (7.5,{\basepos-\valuebarspace+0.6}) {\footnotesize\value};

        % 解析范围并计算进度条长度
        \def\parserange#1-#2\endparse{\def\minval{#1}\def\maxval{#2}}
        \expandafter\parserange\range\endparse

        % 计算进度条长度和颜色
        \pgfmathsetmacro{\progress}{min(\value/\maxval, 1.0)}
        \pgfmathparse{\value > \maxval ? "customred" : (\value < \minval ? "customred" : "green!50")}
        \let\barcolor=\pgfmathresult

        % 进度条显示
        \ifnum\pdfstrcmp{\status}{超标}=0
            \fill[customred, rounded corners=2] (\barstart,\barypos)
                rectangle (\barstart+\barwidth,\barypos+\barheight);
        \else
            \fill[\barcolor, rounded corners=2] (\barstart,\barypos)
                rectangle (\barstart+\barwidth*\progress,\barypos+\barheight);
        \fi

        % 结果评价
        \ifnum\pdfstrcmp{\status}{超标}=0
            \node[value, text=customred] at (10.5,\basepos) {\footnotesize\textbf{\status}};
        \else
            \node[value, text=customGreen] at (10.5,\basepos) {\footnotesize\textbf{\status}};
        \fi

        % 添加卡片
        \pgfmathsetmacro{\cardypos}{\basepos-0.5}
        \begin{scope}[shift={(0,\cardypos)}]
            % 卡片背景
            \pgfmathsetmacro{\cardheight}{
                \ifnum\pdfstrcmp{\status}{超标}=0
                    1.0  % 两行内容时的高度
                \else
                    0.6  % 一行内容时的高度
                \fi
            }

            \fill[rounded corners=5pt, customTeal!5, draw=gray!5]
                (0.3,-\cardheight) rectangle (17.3,0);

            % 菌群简介图标和内容
            \node[anchor=west] at (0.5,-0.3) {
                \textbf{\color{gray!90}\footnotesize \textcolor{customTeal}{\faInfoCircle}}
            };
            \node[anchor=west, text width=16cm] at (0.9,-0.3) {
                {\small\color{gray}\footnotesize \intro}
            };

            % 异常解读标题和内容
            \ifnum\pdfstrcmp{\status}{超标}=0
                \node[anchor=west] at (0.55,-0.8) {
                    \textbf{\color{customRed}\footnotesize \textcolor{customRed}{\faBell}}
                };
                \node[anchor=west, text width=16cm] at (0.9,-0.8) {
                    {\small\color{gray}\footnotesize \suggestion}
                };
            \fi
        \end{scope}

        % 分割线
        \pgfmathsetmacro{\linepos}{
            \ifnum\pdfstrcmp{\status}{超标}=0
                \basepos-1.7  % 超标时的分割线位置
            \else
                \basepos-1.3  % 正常时的分割线位置
            \fi
        }
        \draw[gray!20] (0.2,\linepos) -- (\cardwidth-0.2,\linepos);

        % 根据当前行的状态计算下一行的位置增量
        \ifnum\pdfstrcmp{\status}{超标}=0
            \pgfmathsetmacro{\increment}{0.85}  % 超标行(两行内容)需要更大的增量
        \else
            \pgfmathsetmacro{\increment}{0.7}  % 正常行(一行内容)使用较小的增量
        \fi

        % 更新位置计数器
        \pgfmathsetmacro{\nextpos}{\currentpos+\increment}
        \xdef\currentpos{\nextpos}
    }

    % 最后一行的处理,消除多余的空白
    \pgfmathsetmacro{\lastincrement}{2}  % 最后一行的增量
    \pgfmathsetmacro{\nextpos}{\currentpos+\lastincrement}
    \xdef\currentpos{\nextpos}

\end{tikzpicture}
\end{center}

\newpage

\begin{tcolorbox}[
    enhanced,
    colback=white,
    colframe=white,
    arc=2mm,
    boxrule=0pt,
    width=\textwidth,
    left=15pt,
    right=15pt,
    top=10pt,
    bottom=10pt,
    drop shadow={
        opacity=0.2,
        color=customTeal
    },
    borderline west={5pt}{0pt}{customTeal}
]
\textcolor{customTeal}{\large\textbf{检出细菌}}
\end{tcolorbox}

\begin{tcolorbox}[
    enhanced,
    colback=customTealBg,
    colframe=customTealBg,
    arc=3mm,
    boxrule=0pt,
    width=\textwidth,
    top=8pt,
    bottom=8pt
]
{\small{\color{customTeal}\faInfoCircle} 肠道微生物以细菌为主要组成部分,在人体肠道中约有数千种不同的细菌物种,总数量高达数十万亿个。健康成年人肠道中的细菌主要由厚壁菌门(Firmicutes)、拟杆菌门(Bacteroidetes)、放线菌门(Actinobacteria)和变形菌门(Proteobacteria)等构成,其中厚壁菌门和拟杆菌门的数量占比最高,可达90\%以上。
}
\end{tcolorbox}

\begin{center}\vspace{-10pt}
\begin{tikzpicture}[
    font=\small,
    title/.style={font=\small\bfseries\color{white}},
    value/.style={font=\small},
    reference/.style={font=\footnotesize},
    cell/.style={anchor=west, text width=8cm}
]
    \def\cardwidth{\textwidth}
    \def\cardheight{18}
    \def\barheight{0.2}
    \def\rowspace{1.2}
    \def\valueoffset{0.25}

    % 创建容器
    \draw[rounded corners=5, fill=white, draw=gray!20]
        (0,0) rectangle (\cardwidth,-\cardheight);

    % 标题栏背景
    \path[fill=customTeal]
        (0,0) [rounded corners=5] -- (\cardwidth,0) --
        (\cardwidth,0.8) -- (0,0.8) -- cycle;

    % 调整表头位置和宽度
    \node[title, anchor=west] at (0.5,0.4) {\normalsize \textbf{物种名称}};
    \node[title] at (8.9,0.4) {\normalsize \textbf{丰度}\%};
    \node[title] at (13.0,0.4) {\normalsize \textbf{检测结果}};
    \node[title] at (16.0,0.4) {\normalsize \textbf{正常范围}};

    % 数据行
    \foreach \latin/\chinese/\value/\min/\max/\index in {
        {Faecalibacterium prausnitzii}/{普氏粪类杆菌}/26.23325/0.18043/14.003/0.5,
        {Roseburia inulinivorans}/{食糖端孢斯氏菌}/5.12041/0/2.7653/1.5,
        {Lachnospira eligens}/{挑剔毛螺菌}/4.98592/0/4.489/2.5,
        {uncultured Eubacterium sp.}/{uncultured Eubacterium sp.}/{4.98092}/0/5.2054/3.5,
        {uncultured Eubacteriales bacterium}/{uncultured Eubacteriales bacterium}/{4.93946}/0/5.2541/4.5,
        {Flavonifractor plautii}/{珀氏黄素菌}/2.27540/0/0.82177/5.5,
        {[Eubacterium] rectale}/{直肠真杆菌}/2.14287/0/13.029/6.5,
        {Dialister invisus}/{混迹潜胞利斯特菌}/2.11822/0/0.61156/7.5,
        {Ruminococcus sp. AM28-13}/{Ruminococcus sp. AM28-13}/{1.67121}/0/0.08/8.5,
        {uncultured Ruminococcus sp.}/{uncultured Ruminococcus sp.}/{1.57275}/0/0.08/9.5,
        {Blautia sp. OM06-15AC}/{Blautia sp. OM06-15AC}/1.09259/0/0.08/10.5,
        {Negativibacillus massiliensis}/{马赛阴杆菌}/0.91079/0/0.08/11.5,
        {Subdoligranulum sp. APC924/74}/{Subdoligranulum sp. APC924/74}/0.83406/0/2.1055/12.5,
        {Blautia sp. TF10-30}/{Blautia sp. TF10-30}/0.81246/0/0.08/13.5,
        {Faecalibacterium sp. Marseille-P9312}/{Faecalibacterium sp. Marseille-P9312}/0.69125/0/0.08/14.5
%        {Blautia sp. AF32-4BH}/{Blautia sp. AF32-4BH}/0.67370/0/0.08/15.5,
%        {Bilophila wadsworthia}/{沃氏嗜胆菌}/0.67064/0/0.1702/16.5,
%        {Blautia sp. AM28-27}/{Blautia sp. AM28-27}/0.66662/0/0.08/17.5,
%        {uncultured Dialister sp.}/{uncultured Dialister sp.}/0.61900/0/0.08/18.5,
%        {Roseburia hominis}/{人罗斯伯氏菌}/0.60582/0/1.3625/19.5
    }
    {
        % 计算每行的y位置
        \pgfmathsetmacro{\ypos}{-\rowspace*\index}

        % 项目名(两行显示)
        \node[cell, align=left] at (0.5,\ypos) {
            \textbf{\chinese}\\[-0.2em]
            {\color{gray}\latin}
        };

        % 计算进度条位置
        \pgfmathsetmacro{\barypos}{\ypos-0.25}

        % 进度条背景
        \fill[gray!10, rounded corners=2] (7.9,\barypos-\barheight/2) rectangle (9.9,\barypos+\barheight/2);

        % 判断是否超过最大值
        \pgfmathparse{\value > \max ? 1 : 0}
        \ifnum\pgfmathresult=1
            % 超过最大值:红色满格进度条
            \fill[customred, rounded corners=2] (7.9,\barypos-\barheight/2) rectangle (9.9,\barypos+\barheight/2);
            % 数值(红色,放在进度条上方)
            \node[value, text=customred] at (8.9,\barypos+\valueoffset) {\footnotesize \textbf{\value}};
            % 计算超标倍数(保留两位小数)
            \pgfmathsetmacro{\ratio}{round(\value/\max*100)/100}
            % 检测结果显示超标倍数
            \node[value, text=customred] at (13.0,\ypos) {超标\ratio 倍};
        \else
            % 正常范围内:绿色进度条
            \pgfmathsetmacro{\progresswidth}{2*\value/\max}
            \fill[green!90!black, rounded corners=2] (7.9,\barypos-\barheight/2) rectangle (7.9+\progresswidth,\barypos+\barheight/2);
            % 数值(绿色,放在进度条上方)
            \node[value, text=green!90!black] at (8.9,\barypos+\valueoffset) {\footnotesize \textbf{\value}};
            % 检测结果
            \node[value, text=green!90!black] at (13.0,\ypos) {正常};
        \fi

        % 参考范围
        \node[reference] at (16.0,\ypos) {\min-\max};
    }

    % 分隔线
    \foreach \i in {1,...,14} {
        \pgfmathsetmacro{\y}{-\rowspace*\i}
        \draw[gray!20] (0.2,\y) -- (\cardwidth-0.2,\y);
    }

\end{tikzpicture}
\end{center}

%\newpage

\begin{center}\vspace{-10pt}
\begin{tikzpicture}[
    font=\small,
    title/.style={font=\small\bfseries\color{white}},
    value/.style={font=\small},
    reference/.style={font=\footnotesize},
    cell/.style={anchor=west, text width=8cm}
]
    \def\cardwidth{\textwidth}
    \def\cardheight{23}
    \def\barheight{0.2}
    \def\rowspace{1.2}
    \def\valueoffset{0.25}

    % 创建容器
    \draw[rounded corners=5, fill=white, draw=gray!20]
        (0,0) rectangle (\cardwidth,-\cardheight);

    % 标题栏背景
    \path[fill=customTeal]
        (0,0) [rounded corners=5] -- (\cardwidth,0) --
        (\cardwidth,0.8) -- (0,0.8) -- cycle;

    % 调整表头位置和宽度
    \node[title, anchor=west] at (0.5,0.4) {\textbf{物种名称}};
    \node[title] at (8.9,0.4) {\textbf{丰度}\%};
    \node[title] at (13.0,0.4) {\textbf{检测结果}};
    \node[title] at (16.0,0.4) {\textbf{正常范围}};

    % 数据行
    \foreach \latin/\chinese/\value/\min/\max/\index in {
        {Blautia sp. AF32-4BH}/{Blautia sp. AF32-4BH}/0.67370/0/0.08/0.5,
        {Bilophila wadsworthia}/{沃氏嗜胆菌}/0.67064/0/0.1702/1.5,
        {Blautia sp. AM28-27}/{Blautia sp. AM28-27}/0.66662/0/0.08/2.5,
        {uncultured Dialister sp.}/{uncultured Dialister sp.}/0.61900/0/0.08/3.5,
        {Roseburia hominis}/{人罗斯伯氏菌}/0.60582/0/1.3625/4.5,
        {Faecalibacterium sp. BIOML-A3}/{Faecalibacterium sp. BIOML-A3}/0.58051/0/0.08/5.5,
        {Faecalibacterium sp. AM43-5AT}/{Faecalibacterium sp. AM43-5AT}/0.56939/0/0.08/6.5,
        {Clostridiaceae bacterium}/{Clostridiaceae bacterium}/0.56004/0/0.08/7.5,
        {Eubacterium ramulus}/{细枝真杆菌}/0.55039/0/0.52999/8.5,
        {Enterocloster clostridiiformis}/{小梭形肠道梭状菌}/0.53859/0/0.08/9.5,
        {Luxibacter massiliensis}/{Luxibacter massiliensis}/0.52793/0/0.08/10.5,
        {Staphylococcus aureus}/{金黄色葡萄球菌}/0.50389/0/0.08/11.5,
        {Anaerotignum lactatifermentans}/{乳酸发酵厌氧杆菌}/0.48855/0/0.08/12.5,
        {Ruminococcus bicirculans}/{Ruminococcus bicirculans}/0.47817/0/3.9413/13.5,
        {Ruminococcus sp. AM42-11}/{Ruminococcus sp. AM42-11}/0.46941/0/0.08/14.5,
        {[Clostridium] leptum}/{[Clostridium] leptum}/0.41506/0/0.18319/15.5,
        {[Ruminococcus] lactaris}/{[Ruminococcus] lactaris}/0.41433/0/1.6609/16.5,
        {Ruminococcus sp. OM07-17}/{Ruminococcus sp. OM07-17}/0.40664/0/0.08/17.5,
        {Clostridiales bacterium CCNA10}/{Clostridiales bacterium CCNA10}/0.40084/0/0.08/18.5
    }
    {
        % 计算每行的y位置
        \pgfmathsetmacro{\ypos}{-\rowspace*\index}

        % 项目名(两行显示)
        \node[cell, align=left] at (0.5,\ypos) {
            \textbf{\chinese}\\[-0.2em]
            {\color{gray}\latin}
        };

        % 计算进度条位置
        \pgfmathsetmacro{\barypos}{\ypos-0.25}

        % 进度条背景
        \fill[gray!10, rounded corners=2] (7.9,\barypos-\barheight/2) rectangle (9.9,\barypos+\barheight/2);

        % 判断是否超过最大值
        \pgfmathparse{\value > \max ? 1 : 0}
        \ifnum\pgfmathresult=1
            % 超过最大值:红色满格进度条
            \fill[customred, rounded corners=2] (7.9,\barypos-\barheight/2) rectangle (9.9,\barypos+\barheight/2);
            % 数值(红色,放在进度条上方)
            \node[value, text=customred] at (8.9,\barypos+\valueoffset) {\footnotesize \textbf{\value}};
            % 计算超标倍数(保留两位小数)
            \pgfmathsetmacro{\ratio}{round(\value/\max*100)/100}
            % 检测结果显示超标倍数
            \node[value, text=customred] at (13.0,\ypos) {超标\ratio 倍};
        \else
            % 正常范围内:绿色进度条
            \pgfmathsetmacro{\progresswidth}{2*\value/\max}
            \fill[green!90!black, rounded corners=2] (7.9,\barypos-\barheight/2) rectangle (7.9+\progresswidth,\barypos+\barheight/2);
            % 数值(绿色,放在进度条上方)
            \node[value, text=green!90!black] at (8.9,\barypos+\valueoffset) {\footnotesize \textbf{\value}};
            % 检测结果
            \node[value, text=green!90!black] at (13.0,\ypos) {正常};
        \fi

        % 参考范围
        \node[reference, align=right] at (16.0,\ypos) {\min-\max};
    }

    % 分隔线
    \foreach \i in {1,...,18} {
        \pgfmathsetmacro{\y}{-\rowspace*\i}
        \draw[gray!20] (0.2,\y) -- (\cardwidth-0.2,\y);
    }

\end{tikzpicture}
\end{center}

\newpage

\begin{tcolorbox}[
    enhanced,
    colback=white,
    colframe=white,
    arc=2mm,
    boxrule=0pt,
    width=\textwidth,
    left=15pt,
    right=15pt,
    top=10pt,
    bottom=10pt,
    drop shadow={
        opacity=0.2,
        color=customTeal
    },
    borderline west={5pt}{0pt}{customTeal}
]
\textcolor{customTeal}{\large\textbf{检出真菌}}
\end{tcolorbox}

\begin{tcolorbox}[
    enhanced,
    colback=customTealBg,
    colframe=customTealBg,
    arc=3mm,
    boxrule=0pt,
    width=\textwidth,
    top=8pt,
    bottom=8pt
]
{\small{\color{customTeal}\faInfoCircle} 在肠道菌群中,除主要的细菌外,还存在少量真核微生物(包括寄生虫等)。以下表格列出了在您肠道中丰度最高的35种肠道真菌,按检测丰度从高到低排序。
}
\end{tcolorbox}


\begin{center}\vspace{-10pt}
\begin{tikzpicture}[
font=\small,
title/.style={font=\small\bfseries\color{white}},
value/.style={font=\small},
reference/.style={font=\footnotesize},
cell/.style={anchor=west, text width=8cm}
]
\def\cardwidth{\textwidth}
\def\cardheight{18}
\def\barheight{0.2}
\def\rowspace{1.2}
\def\valueoffset{0.35}

% 创建容器
\draw[rounded corners=5, fill=white, draw=gray!20]
    (0,0) rectangle (\cardwidth,-\cardheight);

% 标题栏背景
\path[fill=customTeal]
    (0,0) [rounded corners=5] -- (\cardwidth,0) --
    (\cardwidth,0.8) -- (0,0.8) -- cycle;

% 调整表头位置和宽度
\node[title, anchor=west] at (0.5,0.4) {\normalsize \textbf{物种名称}};
\node[title] at (8.9,0.4) {\normalsize \textbf{丰度}\%};
\node[title] at (13.0,0.4) {\normalsize \textbf{检测结果}};
\node[title] at (16.0,0.4) {\normalsize \textbf{正常范围}};

% 数据行
\foreach \latin/\chinese/\value/\min/\max/\index in {
    {Neocallimatix californiae}/{新美鞭菌}/0.01807/0/0.05/0.5,
    {Anaeromyces robustus}/{Anaeromyces robustus}/{0.01413}/0/0.05/1.5,
    {Caeconomyces churrovis}/{偏胃厌氧真菌}/0.01267/0/0.05/2.5,
    {Orpinomyces sp.}/{Orpinomyces sp.}/{0.01134}/0/0.05/3.5,
    {Gigaspora rosea}/{粉红巨孢囊霉}/{0.00909}/0/0.05/4.5,
    {Rhizophagus irregularis}/{异形根孢霉}/0.00770/0/0.05/5.5,
    {Zoophthora radicans}/{根状毁蝇菌}/{0.00700}/0/0.05/6.5,
    {Phakopsora pachyrhizi}/{大豆锈菌}/{0.00411}/0/0.05/7.5,
    {Zopfia rhizophila}/{Zopfia rhizophila}/{0.00390}/0/0.05/8.5,
    {Piromyces finnis}/{Piromyces finnis}/{0.00360}/0/0.05/9.5,
    {Piromyces sp. E2}/{皮罗菌E2菌株}/{0.00344}/0/0.05/10.5,
    {Leucoagaricus gongylophorus}/{切叶蚁共生白蘑菇}/{0.00328}/0/0.05/11.5,
    {Smittium culicis}/{Smittium culicis}/{0.00212}/0/0.05/12.5,
    {Trichoderma reesei}/{里氏木霉}/0.00174/0/0.05/13.5,
    {Puccinia graminis}/{禾锈菌}/0.00152/0/0.05/14.5
}
{
    % 计算每行的y位置
    \pgfmathsetmacro{\ypos}{-\rowspace*\index}

    % 项目名(两行显示)
    \node[cell, align=left] at (0.5,\ypos) {
        \textbf{\chinese}\\[-0.2em]
        {\color{gray}\latin}
    };

    % 计算进度条位置
    \pgfmathsetmacro{\barypos}{\ypos-0.25}

    % 进度条背景
    \fill[gray!10, rounded corners=2] (7.9,\barypos-\barheight/2) rectangle (9.9,\barypos+\barheight/2);

    % 进度条(所有值都在正常范围内,使用绿色)
    \pgfmathsetmacro{\progresswidth}{2*\value/\max}
    \fill[green!90!black, rounded corners=2] (7.9,\barypos-\barheight/2) rectangle (7.9+\progresswidth,\barypos+\barheight/2);

    % 数值(绿色)
    \node[value, text=green!90!black] at (8.9,\barypos+\valueoffset) {\footnotesize \textbf{\value}};

    % 检测结果
    \node[value, text=green!90!black] at (13.0,\ypos) {正常};

    % 参考范围
    \node[reference] at (16.0,\ypos) {\min-\max};
}

% 分隔线
\foreach \i in {1,...,14} {
    \pgfmathsetmacro{\y}{-\rowspace*\i}
    \draw[gray!20] (0.2,\y) -- (\cardwidth-0.2,\y);
}




\end{tikzpicture}
\end{center}

\newpage

\begin{tcolorbox}[
    enhanced,
    colback=white,
    colframe=white,
    arc=2mm,
    boxrule=0pt,
    width=\textwidth,
    left=15pt,
    right=15pt,
    top=10pt,
    bottom=10pt,
    drop shadow={
        opacity=0.2,
        color=customTeal
    },
    borderline west={5pt}{0pt}{customTeal}
]
\textcolor{customTeal}{\large\textbf{检出真核生物}}
\end{tcolorbox}

\begin{tcolorbox}[
    enhanced,
    colback=customTealBg,
    colframe=customTealBg,
    arc=3mm,
    boxrule=0pt,
    width=\textwidth,
    top=8pt,
    bottom=8pt
]
{\small{\color{customTeal}\faInfoCircle} 在肠道菌群中,除主要的细菌外,还存在少量真核微生物(包括寄生虫等)。以下表格列出了在您肠道中丰度最高的25种真核生物,按检测丰度从高到低排序。
}
\end{tcolorbox}

\begin{center}\vspace{-10pt}
\begin{tikzpicture}[
    font=\small,
    title/.style={font=\small\bfseries\color{white}},
    value/.style={font=\small},
    reference/.style={font=\small},
    cell/.style={anchor=west, text width=8cm}  % 增加宽度以适应两行文本
]
    \def\cardwidth{\textwidth}
    \def\cardheight{19.2}
    \def\barheight{0.25}
    \def\rowspace{1.2}

    % 创建容器
    \draw[rounded corners=5, fill=white, draw=gray!20]
        (0,0) rectangle (\cardwidth,-\cardheight);

    % 标题栏背景
    \path[fill=customTeal]
        (0,0) [rounded corners=5] -- (\cardwidth,0) --
        (\cardwidth,0.8) -- (0,0.8) -- cycle;

    % 调整表头位置和宽度
    \node[title, anchor=west] at (0.5,0.4) {\textbf{物种名称}};
    \node[title] at (12.0,0.4) {\textbf{丰度}\%};
    \node[title] at (16.0,0.4) {\textbf{正常范围}};

    % 数据行
    \foreach \latin/\chinese/\value/\range/\index in {
        {Trypanosoma cruzi}/{克氏锥虫}/0.00859/{0-0.05}/0.5,
        {Toxoplasma gondii}/{刚地弓形虫}/0.00767/{0-0.05}/1.5,
        {Trichomonas vaginalis}/{阴道毛滴虫}/0.00693/{0-0.05}/2.5,
        {Cyclospora cayetanensis}/{圆孢子虫}/0.00635/{0-0.05}/3.5,
        {Enterobius vermicularis}/{蠕形住肠线虫}/0.00456/{0-0.05}/4.5,
        {Ichthyophthirius multifiliis}/{多子小瓜虫}/0.00455/{0-0.05}/5.5,
        {Plasmodium ovale}/{疟原虫}/0.00389/{0-0.05}/6.5,
        {Ascaris lumbricoides}/{似蚓蛔线虫}/0.00358/{0-0.05}/7.5,
        {Schistosoma mansoni}/{曼氏血吸虫}/0.00349/{0-0.05}/8.5,
        {Schistosoma japonicum}/{日本血吸虫}/0.00343/{0-0.05}/9.5,
        {Tetrahymena thermophila}/{热原四膜虫}/0.00331/{0-0.05}/10.5,
        {Plasmodium vivax}/{间日疟原虫}/0.00318/{0-0.05}/11.5,
        {Schistosoma haematotium}/{埃及血吸虫}/0.00308/{0-0.05}/12.5,
        {Dictyostelium purpureum}/{紫色网柄菌}/0.00274/{0-0.05}/13.5,
        {Dictyostelium discoideum}/{盘基网柄菌}/0.00265/{0-0.05}/14.5,
        {Paragonimus westermani}/{卫氏并殖吸虫}/0.00256/{0-0.05}/15.5
    }
    {
        % 计算每行的y位置
        \pgfmathsetmacro{\ypos}{-\rowspace*\index}

        % 项目名(两行显示)
        \node[cell, align=left] at (0.5,\ypos) {
            \small\textbf{\chinese}\\[-0.2em]
            {\color{lightgray}\small\latin}
        };

        % 计算进度条垂直居中的位置
        \pgfmathsetmacro{\barypos}{\ypos-0.5*\barheight}

        % 进度条背景
        \fill[gray!10, rounded corners=2] (10.5,\barypos) rectangle (12.5,\barypos+\barheight);

        % 进度条
        \pgfmathsetmacro{\progresswidth}{2*\value/0.05}
        \fill[green!90!black, rounded corners=2] (10.5,\barypos) rectangle (10.5+\progresswidth,\barypos+\barheight);

        % 数值
        \node[value, text=green!90!black] at (13.5,\ypos) {\value};

        % 参考范围
        \node[reference] at (16.0,\ypos) {\footnotesize\range};
    }

    % 分隔线
    \foreach \i in {1,...,15} {
        \pgfmathsetmacro{\y}{-\rowspace*\i}
        \draw[gray!20] (0.2,\y) -- (\cardwidth-0.2,\y);
    }

\end{tikzpicture}
\end{center}

\newpage

\begin{center}\vspace{-10pt}
\begin{tikzpicture}[
    font=\small,
    title/.style={font=\small\bfseries\color{white}},
    value/.style={font=\small},
    reference/.style={font=\small},
    cell/.style={anchor=west, text width=8cm}  % 增加宽度以适应两行文本
]
    \def\cardwidth{\textwidth}
    \def\cardheight{4.8}
    \def\barheight{0.25}
    \def\rowspace{1.2}

    % 创建容器
    \draw[rounded corners=5, fill=white, draw=gray!20]
        (0,0) rectangle (\cardwidth,-\cardheight);

    % 标题栏背景
    \path[fill=customTeal]
        (0,0) [rounded corners=5] -- (\cardwidth,0) --
        (\cardwidth,0.8) -- (0,0.8) -- cycle;

    % 调整表头位置和宽度
    \node[title, anchor=west] at (0.5,0.4) {\textbf{物种名称}};
    \node[title] at (9.3,0.4) {\textbf{丰度}\%};
    \node[title] at (15.0,0.4) {\textbf{正常范围}};

    % 数据行
    \foreach \latin/\chinese/\value/\range/\index in {
        {Necator americanus}/{美洲钩口线虫}/0.00252/{0-0.05}/0.5,
        {Plasmodium gallinaceum}/{鸡疟原虫}/0.00216/{0-0.05}/1.5,
        {Fasciolopsis buski}/{布氏姜片吸虫}/0.00207/{0-0.05}/2.5,
        {Acanthamoeba culbertsoni}/{卡氏棘阿米巴}/0.00181/{0-0.05}/3.5
    }
    {
        % 计算每行的y位置
        \pgfmathsetmacro{\ypos}{-\rowspace*\index}

        % 项目名(两行显示)
        \node[cell, align=left] at (0.5,\ypos) {
            \small\textbf{\chinese}\\[-0.2em]
            {\color{lightgray}\small\latin}
        };

        % 计算进度条垂直居中的位置
        \pgfmathsetmacro{\barypos}{\ypos-0.5*\barheight}

        % 进度条背景
        \fill[gray!10, rounded corners=2] (7.5,\barypos) rectangle (9.5,\barypos+\barheight);

        % 进度条
        \pgfmathsetmacro{\progresswidth}{2*\value/0.05}
        \fill[green!90!black, rounded corners=2] (7.5,\barypos) rectangle (7.5+\progresswidth,\barypos+\barheight);

        % 数值
        \node[value, text=green!90!black] at (10.5,\ypos) {\textbf{\value}};

        % 参考范围
        \node[reference] at (15.0,\ypos) {\footnotesize\range};
    }

    % 分隔线
    \foreach \i in {1,...,3} {
        \pgfmathsetmacro{\y}{-\rowspace*\i}
        \draw[gray!20] (0.2,\y) -- (\cardwidth-0.2,\y);
    }

\end{tikzpicture}
\end{center}

\newpage

\begin{tcolorbox}[
enhanced,
colback=white,
colframe=customTeal,
arc=2mm,
boxrule=1pt,
left=20pt,
right=20pt,
top=12pt,
bottom=12pt,
width=\textwidth,
fontupper=\sffamily,
overlay={
\draw[customTeal, line width=2pt]
([xshift=15pt]frame.south west) -- ([xshift=-15pt]frame.south east);
}
]
{\Large\bfseries\textcolor{customTeal}{\Huge 个性化建议}}
\end{tcolorbox}

\newpage

\newpage

\newpage

\begin{tcolorbox}[
enhanced,
colback=white,
colframe=customTeal,
arc=2mm,
boxrule=1pt,
left=20pt,
right=20pt,
top=12pt,
bottom=12pt,
width=\textwidth,
fontupper=\sffamily,
overlay={
\draw[customTeal, line width=2pt]
([xshift=15pt]frame.south west) -- ([xshift=-15pt]frame.south east);
}
]
{\Large\bfseries\textcolor{customTeal}{\Huge 参考文献}}
\end{tcolorbox}

\begin{tcolorbox}[
    enhanced,
    colback=gray!3,
    colframe=gray!3,
    arc=3mm,
    boxrule=0pt,
    width=\textwidth,
    top=8pt,
    bottom=8pt
]
{\small{\color{customTeal}\faInfoCircle} 以下为参考文献。
}
\end{tcolorbox}


% 使用示例:
\reference{1}{Exposure to concentrated ambient PM2.5 alters the composition of gut microbiota in a murine model.}{Wang, W. et al.}{Part Fibre Toxicol \textbf{15}, 17}{2018}
\reference{2}{Gut Dysbiosis in Animals Due to Environmental Chemical Exposures.}{Rosenfeld, C. S.}{Front Cell Infect Microbiol \textbf{7}, 396}{2017}

\reference{3}{Gut Microbiota Richness and Composition and Dietary Intake of Overweight Pregnant Women Are Related to Serum Zonulin Concentration, a Marker for Intestinal Permeability.}{Mokkala, K. et al.}{J Nutr \textbf{146}, 1694--1700}{2016}

\reference{4}{Gut microbiota, dietary intakes and intestinal permeability reflected by serum zonulin in women.}{Mörkl, S. et al.}{Eur J Nutr \textbf{57}, 2985--2997}{2018}

\reference{5}{The neuroactive potential of the human gut microbiota in quality of life and depression.}{Valles-Colomer, M. et al.}{Nature Microbiology \textbf{4}, 623}{2019}

\reference{6}{Impact of the Gut Microbiota on Intestinal Immunity Mediated by Tryptophan Metabolism.}{Gao, J. et al.}{Front Cell Infect Microbiol \textbf{8}}{2018}

\reference{7}{Linking the gut microbiome to metabolism through endocrine hormones.}{Brubaker, P. L.}{Endocrinology}{2018}

\reference{8}{Microbial endocrinology: host–bacteria communication within the gut microbiome.}{Sandrini, S., Aldriwesh, M., Alruways, M. \& Freestone, P.}{Journal of Endocrinology \textbf{225}, R21--R34}{2015}

\reference{9}{A High Salt Diet Modulates the Gut Microbiota and Short Chain Fatty Acids Production in a Salt-Sensitive Hypertension Rat Model.}{Bier, A. et al.}{Nutrients \textbf{10}}{2018}

\reference{10}{Altered gut microbiome composition in children with refractory epilepsy after ketogenic diet.}{Zhang, Y. et al.}{Epilepsy Res. \textbf{145}, 163--168}{2018}

\reference{11}{Association analysis of dietary habits with gut microbiota of a native Chinese community.}{Qian, L. et al.}{Exp Ther Med \textbf{16}, 856--866}{2018}

\reference{12}{Effect of changes in food groups intake on magnesium, zinc, copper, and selenium serum levels during 2 years of dietary intervention.}{Paz-Tal, O. et al.}{J Am Coll Nutr \textbf{34}, 1--14}{2015}

\reference{13}{Intersection of salt- and immune-mediated mechanisms of hypertension in the gut microbiome.}{Wyatt, C. M. \& Crowley, S. D.}{Kidney Int. \textbf{93}, 532--534}{2018}

\reference{14}{Salt-responsive gut commensal modulates TH17 axis and disease.}{Wilck, N. et al.}{Nature \textbf{551}, 585--589}{2017}

\reference{15}{The Virtual Metabolic Human database: integrating human and gut microbiome metabolism with nutrition and disease.}{Noronha, A. et al.}{bioRxiv 321331}{2018}

\reference{16}{Fecal concentrations of bacterially derived vitamin K forms are associated with gut microbiota composition but not plasma or fecal cytokine concentrations in healthy adults.}{Karl, J. P. et al.}{Am. J. Clin. Nutr. \textbf{106}, 1052--1061}{2017}

\reference{17}{A novel ultra high-throughput 16S rRNA gene amplicon sequencing library preparation method for the Illumina HiSeq platform.}{de Muinck, E. J., Trosvik, P., Gilfillan, G. D., Hov, J. R. \& Sundaram, A. Y. M.}{Microbiome \textbf{5}, 68}{2017}

\reference{18}{Role of Neurochemicals in the Interaction between the Microbiota and the Immune and the Nervous System of the Host Organism.}{Oleskin, A. V., Shenderov, B. A. \& Rogovsky, V. S.}{Probiotics Antimicrob Proteins \textbf{9}, 215--234}{2017}

\reference{19}{Linking the Gut Microbiota to a Brain Neurotransmitter.}{Jameson, K. G. \& Hsiao, E. Y.}{Trends Neurosci. \textbf{41}, 413--414}{2018}

\newpage

\begin{tcolorbox}[
    enhanced,
    colback=white,
    colframe=white,
    boxrule=0pt,
    width=\textwidth,
    left=0pt,
    right=0pt,
    top=0pt,
    bottom=25pt
]
    \begin{tcolorbox}[
        enhanced,
        colback=customTeal,
        colframe=customTeal,
        arc=0mm,
        boxrule=0pt,
        width=\textwidth,
        left=40pt,
        right=40pt,
        top=25pt,
        bottom=25pt,
        frame code={
            \path[left color=customTeal,
                  right color=customTeal!80]
            (frame.south west) rectangle (frame.north east);
        }
    ]
        \begin{minipage}{0.8\textwidth}
            {\fontsize{26}{31}\selectfont\textcolor{white}{\textbf{肠道基础功能评估}}}

            \vspace{5pt}
            {\small\textcolor{white!90}{Intestinal Basic Function Assessment}}
        \end{minipage}
        \hfill
        \textcolor{white}{\fontsize{32}{38}\selectfont\faMicroscope}
    \end{tcolorbox}
\end{tcolorbox}

\begin{tcolorbox}[
    enhanced,
    colback=customTeal,
    colframe=customTeal,
    arc=0mm,
    boxrule=0pt,
    width=\textwidth,
    left=0pt,
    right=0pt,
    top=0pt,
    bottom=25pt
]
    \begin{tcolorbox}[
        enhanced,
        colback=customTeal,
        colframe=customTeal!90,
        boxrule=0pt,
        left=40pt,
        right=40pt,
        top=20pt,
        bottom=20pt,
    ]
        \begin{minipage}{0.7\textwidth}
            {\fontsize{28}{33}\selectfont\textcolor{white}{\textbf{肠道基础功能评估}}}

            \vspace{3pt}
            {\small\textcolor{white!80}{Microbiome Analysis Report}}
        \end{minipage}
        \hfill
        \begin{minipage}{0.25\textwidth}
            \raggedleft
            \textcolor{white}{\huge\faVial}
        \end{minipage}
    \end{tcolorbox}
\end{tcolorbox}

\begin{tcolorbox}[
    enhanced,
    colback=customTeal,
    colframe=customTeal,
    arc=0mm,
    boxrule=0pt,
    width=\textwidth,
    left=0pt,
    right=0pt,
    top=0pt,
    bottom=25pt
]
    \begin{tikzpicture}
        % 装饰线条
        \draw[white, line width=2pt] (0.5,0.2) -- (0.5,1.8);
        \draw[white!50, line width=1pt] (0.8,0.2) -- (0.8,1.8);

        % 标题文字
        \node[anchor=west, text=white] at (1.5,1) {
            \fontsize{26}{31}\selectfont\textbf{肠道基础功能评估}
        };

        % 右侧装饰
        \draw[white!30, line width=1pt] (\textwidth-2,0.2) -- (\textwidth-2,1.8);
        \node[anchor=east, text=white] at (\textwidth-0.5,1) {
            \Large\faMicroscope
        };
    \end{tikzpicture}
\end{tcolorbox}

\begin{tcolorbox}[
    enhanced,
    colback=customTeal!90,
    colframe=customTeal,
    arc=0mm,
    boxrule=1pt,
    width=\textwidth,
    left=0pt,
    right=0pt,
    top=0pt,
    bottom=25pt
]
    \begin{tcolorbox}[
        enhanced,
        colback=customTeal!90,
        colframe=white,
        boxrule=0pt,
        left=40pt,
        right=40pt,
        top=20pt,
        bottom=20pt,
    ]
        \begin{minipage}{\textwidth}
            {\fontsize{26}{31}\selectfont\textcolor{white}{\textbf{肠道基础功能评估}}}
            \hfill
            \textcolor{white}{\Large\faFlask}

            \vspace{5pt}
            \textcolor{white!80}{\rule{\linewidth}{0.5pt}}
        \end{minipage}
    \end{tcolorbox}
\end{tcolorbox}

\begin{tcolorbox}[
    enhanced,
    colback=white,
    colframe=white,
    boxrule=0pt,
    width=\textwidth,
    left=0pt,
    right=0pt,
    top=0pt,
    bottom=25pt
]
    \begin{tikzpicture}
        % 主背景
        \fill[customTeal, rounded corners=9pt] (0,0) rectangle (\textwidth,1.5);
        % 左侧装饰
        \fill[customTeal!70, rounded corners=9pt] (0,0) rectangle (0.3,1.5);
        \fill[customTeal!60, rounded corners=9pt] (0.4,0) rectangle (0.6,1.5);

        % 标题文字和图标
        \node[anchor=west, text=white] at (1.2,0.75) {
            \fontsize{15}{31}\selectfont\textbf{肥胖相关菌} \Large\faBacteria
        };
    \end{tikzpicture}
\end{tcolorbox}















\end{document}


